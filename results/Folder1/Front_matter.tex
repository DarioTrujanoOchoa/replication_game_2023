% \documentclass[12pt,legalpaper]{article}
% % \documentclass[12pt]{report}
% \usepackage[a4paper, total={6in, 8in}]{geometry}
% \large
% \usepackage{booktabs}
% \usepackage{setspace}
% % \usepackage{hyperref}
% \usepackage[hidelinks]{hyperref}
% \usepackage{graphicx}
% \usepackage{float}
% \usepackage{xcolor}
% \documentclass[12pt]{article}
% % \documentclass[12pt]{report}
% \usepackage[a4paper, total={6in, 8in}]{geometry}
% \large
% \usepackage{booktabs}
% \usepackage{setspace}
% \usepackage{hyperref}
% \usepackage{graphicx}
% \usepackage{float}
% \usepackage{xcolor}
% \usepackage{lscape}

% \begin{document}
\thispagestyle{empty} % no page number for title page
\doublespacing

\section*{\centering Abstract}


This dissertation utilizes empirical and experimental approaches to investigate the multifaceted factors that affect decision-making in economic contexts. The three chapters place a particular focus on group affiliation, perceived discrimination, and the gender gap in entrepreneurship.

The first chapter explores the impact of team diversity on economic decision-making within a dynamic context. Through a series of laboratory experiments involving participants randomly assigned to teams, the study examines how team diversity influences individual behavior and performance. Results indicate that team diversity significantly affects individual behavior, with newcomers being particularly influenced by the diversity of the team they join. The findings underscore the importance of understanding the effects of team composition on decision-making processes and performance outcomes. 

The second chapter delves into the implications of subjective evaluation systems for both supervisors and employees. Through experiments conducted with participants from two U.S. universities, the research examines how group affiliation influences perceived discrimination in economic decision-making. By uncovering disparities in anticipation of bias based on race and gender, the study highlights the importance of addressing systemic biases and fostering inclusive workplace environments.


The third chapter investigates the dynamics of fundraising in early stage startups. Despite dominance in certain female-dominated industries, women entrepreneurs struggle to secure funding. The analysis suggests potential industry-based sorting bias, with women-led ventures performing on par or better in high-funding industries.

Overall, these chapters deepen our understanding of team dynamics, job evaluations, and discrimination perceptions, while also offering insights for creating more inclusive work environments.

\thispagestyle{empty} % no page number for title page
\clearpage


\section*{\centering Acknowledgments}
I thank my Ph.D. Committee, Dr. Sherry Li, Dr. Peter J. McGee, Dr. Andy Brownback, and Dr. Md Amzad Hossain. I am grateful for their guidance and immense support. My main supervisor, Dr. Sherry Li, provided thoughtful advice, challenged my research rigor, and trained me to become an independent researcher. Dr. Peter J. McGee as the PhD program advisor and a committee member offered constructive comments on my research ideas. Dr. Andy Brownback read my draft papers with patience and provided illuminating feedback. I am also grateful to Dr. Md Amzad Hossain for his guidance on navigating the dissertation and the job market.  I am grateful for having the four of them as my mentors and thank their thoughtful training and genuine support.
 
I appreciate the mentoring I received from the Faculty of Economics at Sam M. Walton College of Business, the AEA Mentoring Program and Dr. Manisha Shah of UCLA. 

My Ph.D. program was sponsored the University of Arkansas. I would like to thank the University the generous financial support and for facilitating an enriching academic experience. 

Many people inspired me in this journey. My family's patience with my frequent disconnections due to the tremendous demanding nature of my Ph.D. program was instrumental. My PhD cohort mates Miguel Cuellar, Eric Neyou and Allan "Braziel" Hatch for all the fun times and struggles together.  



\thispagestyle{empty} % no page number for title page
\clearpage



\section*{\centering \textbf{Dedication}}

I dedicate this work to my beloved family. Firstly, to my mother, Comfort Asele, whose brilliant mind, though constrained by a lack of formal education, continuously inspired me to excel in every circumstance. Secondly, to my father, Stephen Atanga Agyeah, for his unwavering financial, emotional, and all-around support. I extend this dedication to my dear siblings, Janet Agyeah and Stephen Agyeah. The times and struggles we have been through together help me to stay rooted and resilient in all situations. Lastly and more significantly, to my daughter, Thelma Talata Agyeah, my cherished wife, Uzoma Vivian Igwe, and my second child, who serve as my enduring motivation to strive for excellence.


% Chapter 1: The National Science Foundation, the University of Arkansas and Behavioral Business Research Lab under the leadership of Dr. Sherry Li. I also extend my gratitude to my research assistants Vera Adabrah, Kaitlyn Pearson and Aashvi Dahiya for their hard work. 


% Chapter 2: Dr. Yufei Ren secured funding for part of this project.  

% Chapter 3: I am grateful to TechCrunch (now Crunchbase) for allowing me to use their data. 
% \thispagestyle{empty} % no page number for title page
% \clearpage
% % \doublespacing
% \thispagestyle{empty} % no page number for title page
% \tableofcontents
% % \clearpage

% % \documentclass[12pt]{article}
% % \documentclass[12pt]{report}
% \usepackage[a4paper, total={6in, 8in}]{geometry}
% \large
% \usepackage{booktabs}
% \usepackage{setspace}
% \usepackage{hyperref}
% \usepackage{graphicx}
% \usepackage{float}
% \usepackage{xcolor}
% \usepackage{lscape}

% \begin{document}
\thispagestyle{empty} % no page number for title page
\doublespacing

\begin{center}
\section*{Introduction}\label{sec:intro}
\end{center}

Fairness and equity are paramount in economic decision-making, impacting both individual opportunities and overall economic well-being.  Understanding the intricacies of team composition, perceptions of discrimination and the impact of team composition on fundraising success is essential for fostering inclusive and effective work environments. This dissertation delves into these areas, aiming to uncover insights that can inform organizational practices and policies.

\textbf{Chapter 1: Team Diversity and Economic Decision-Making}

Teams have become integral to various environments, offering diverse perspectives and skills that enhance decision-making processes. However, challenges such as coordination issues and delays in decision-making can arise, particularly in diverse teams. This chapter investigates the impact of team diversity on economic decision-making through a series of experiments. By exploring how team diversity influences individual behavior and performance, the research aims to provide valuable insights into optimizing team composition for efficient outcomes.

\textbf{Chapter 2: Group bias, anticipation and performance }

In the second chapter, the research delves into the role of subjectivity in job evaluations and its implications for supervisors and employees. Combining insights from economic literature and experimental studies conducted with participants from two U.S. universities, the study elucidates concerns such as discrimination and favoritism in evaluation processes. It investigates how group affiliation influences both the anticipation of bias and individuals' behaviors regarding performance evaluations. The results underscore significant disparities in the perception of discrimination based on race and gender, shedding light on the necessity for organizations to address systemic biases and cultivate inclusive environments. Furthermore, the research unveils a stronger emphasis on achieving positive outcomes from in-group membership compared to concerns about potential bias from out-group evaluators.

\textbf{Chapter 3: Analysis of Venture funding at the Seed Stage}

The third chapter explores the persistent gender gap in entrepreneurship, particularly within Science, Technology, Engineering, and Mathematics (STEM) fields. The study leverages data on startup teams to investigate how gender composition influences their ability to attract investment, especially during the crucial seed funding stage. The findings reveal a concerning disparity: while startups with women founders attract higher venture capital interest rates, they tend to struggle where it matters most.Women$-$Led startups are concentrated in areas where the area amount is often low and they less represented in high evaluation sectors of science and technology. The analysis suggests potential industry-based sorting bias, with women-led ventures performing on par or better in high-funding industries.

Collectively, these chapters contribute to a deeper understanding of team dynamics,perceptions of discrimination in the workplace and startup teams. By uncovering insights and implications for organizational practices, policies, and interventions, this dissertation aims to facilitate the creation of more inclusive and effective work environments. 



% \end{document}
% \clearpage

% \part{ Racial and Gender Diversity, Newcomers and Team Performance in a Dynamic Setting}
% 
% \bibliographystyle{aer}


% % BIBLIOGRAPHY %%%%%%%%%%%%%%
% % \usepackage[natbibapa]{apacite}  % to enable '\citet' and '\citep' macros
% % \bibliographystyle{apacite}
% % %%%%%%%%%%%%%%%%%%%%%%%%%%%%

% \title{
% {Racial and Gender Diversity, Newcomers and Team Performance in a Dynamic Setting}\\
% {\large University of Arkansas}\\
% % {\includegraphics{university.jpg}}
% }
% \author{ George Agyeah\thanks{{gagyeah@walton.uark.edu} University of Arkansas 
% \newline
% Financial support from the National Science Foundation (SES\#2215203),\href{https://bbrl.uark.edu/}{Behavioral Business Research Lab} and the \href{https://walton.uark.edu/diversity/} {Dr. Barbara A. Lofton Office of Diversity \& Inclusion} are acknowledged. The experiment was pre-registered with the AEA’s RCT registry under number \href{https://www.socialscienceregistry.org/trials/9048}{9048}.  } } 


% \begin{document}
% \setlength{\topmargin}{1in} % Set top margin
% \setlength{\oddsidemargin}{1in}  % Set left margin (odd pages)
% \setlength{\evensidemargin}{1in}  % Set left margin (even pages)
% \setlength{\textwidth}{6.5in}    % Adjust text width (considering margins)
% \setlength{\textheight}{9in}     % Adjust text height (considering margins)

% \title{
% {Exploring Diversity, Discrimination, and Performance Dynamics} }\

% \maketitle 
% \begin{center}
%     \href{https://wordpress.com/block-editor/page/ag-yeah.com/1254}{ \textcolor{black}{Click here for the latest version}} 
% \end{center}
% \begin{abstract}
% \small \noindent Teams play a crucial role in shaping decision-making processes within organizations and institutions by capitalizing on a diverse range of perspectives and skills. However, teams frequently encounter challenges, such as coordination issues and suboptimal cooperation, which may be amplified by the varied backgrounds and identities of team members. This study employs a lab experiment to explore the impact of both racial and gender diversity on individual economic decision-making within teams operating in a dynamic context. Across 111 independent sessions, 444 participants make cooperation and coordination decisions in the public goods provision and minimum effort activities. Video and audio recordings of participant interactions are then coded to capture participant communication. Incumbents in a team cooperate and coordinate at levels that are about 14\% higher than newcomers. Newcomers significantly increase their cooperation and coordination choices after joining a team. Further evidence shows that the existing incumbent team diversity affects the cooperation choices of newcomers, but not incumbents. The findings underscore the significance of bolstering team identity in team design and highlight the crucial consideration of the substantial influence that the specific team a newcomer joins has on their decision-making in organizations. \\

% \noindent\tiny\textbf{Keywords: Diversity, gender, race, public goods, experiment, teams }\\
% \noindent\textbf{JEL Codes: M14, J15, J16, H41, C91} \\
% \end{abstract}
% \setcounter{page}{0}
% \thispagestyle{empty}
% \pagebreak \newpage

\pagestyle{plain} 

\doublespacing

\section{Introduction} \label{sec:introduction}
 

Teams have become integral to the functioning of diverse environments ranging from corporate board rooms to academic research teams \citep{bjl14,lml}. Numerous reasons contribute to the significance of teams. Firstly, teams bring together a diverse range of perspectives and skills, enabling enhanced decision-making by leveraging the collective knowledge and experiences of each member. Secondly, teams facilitate workload sharing, particularly for complex or time-consuming decisions. This collaborative effort ensures that tasks are effectively handled. Thirdly, teams can foster a supportive and encouraging environment, promoting stress reduction and bolstering team morale. However, teams can also encounter challenges, such as coordination issues and potential delays in decision-making processes. Optimal outcomes require team members to cooperate, a process that can sometimes slow down decision-making in various settings. This effect can be further exacerbated by social category diversity if differences are magnified \citep{lau}. Social category diversity encompasses differences in race, gender, age, or cultural background among team members. By acknowledging and addressing these challenges, organizations can harness the benefits of team-based decision-making while effectively managing the complexities arising from social category diversity among team members. 

 % \hspace *{0mm} Team diversity can shape team behavior and overall performance. According to social identity theory, individuals tend to more readily identify with others who share similar social categories like them \citep{t78}. This can pose challenges when individuals from different backgrounds are brought together in a team. On the contrary, research has demonstrated that teams composed of individuals with diverse backgrounds and experiences can yield remarkable improvements in various aspects of team performance \citep{labbzs14,pln08}. Consequently, recognizing and comprehending the impact of diversity within teams becomes imperative for achieving efficient outcomes. 
 
 The public goods provision and the minimum effort activities are utilized to investigate the impact of team diversity and changes in the team dynamics on the performance of teams. The public goods provision activity and the minimum effort activity are social dilemma games that are useful in measuring behavior where the interest of the individual conflicts with the needs of the group. The dilemma lies in the fact that the pareto optimal outcome for the group is achieved when members of the group give at the highest levels of both activities. The dominant strategy for an individual in the public goods provision is to contribute at the lowest level to the group account. Cooperating in a social dilemma is essential for the functioning of teams while failure to cooperate has been linked to inefficiencies in teams \citep{stallen2023}. Contributions to the group in the public goods provision is analyzed as the level of cooperation in the team. Choices of the level of effort in the minimum effort activity are analyzed as the coordination choices within the group.  There is no dominant strategy but higher coordination choices reflect higher efficiency within the group. 
 
 I conduct a series of experiments to explore the impact of the level of team diversity on economic decision-making in a dynamic context.  Participants of varying racial and gender identities are randomly assigned to teams of three people. A newcomer is added to the team in the middle of the experiment to examine the evolution of cooperation and coordination among the members. The results indicate that team diversity has a significant effect on individual behavior. Additionally, the existing composition impacts contributions of incumbents prior to group composition changes. Furthermore, newcomers are affected by the diversity of the team they join but not incumbents. 

\hspace *{0mm} This work makes several contributions to the literature. Firstly, it extends our understanding of teams by investigating the influence of newcomers' identities on team dynamics and the impact of team diversity on both incumbents and newcomers. This approach enables me to adopt a functional perspective that closely mirrors real-world organizational settings. Secondly, my findings provide compelling evidence of the dual effect of racial and gender diversity on team performance within diverse teams. My design effectively highlights the gender and racial identity of each individual allowing individuals to perceive an individuals racial and gender racial identities based on interactions within the team. This enables me to more accurately account for the dual effects of racial identity and gender identity. Lastly, while prior studies have found mixed results regarding the impact of diversity on performance, my study tackles endogeneity and selection concerns that commonly arise in observational studies, thus offering more robust evidence.

\hspace *{0mm} The rest of the paper is organized as follows. Section \hyperref[sec:literature]{2} presents the literature review. Section \hyperref[sec:Design]{3} details the experimental design. Section \hyperref[sec:Hypotheses]{4} introduces hypotheses. Section \hyperref[sec:Analysis]{5} discusses the empirical analyses and results in the study. Section \hyperref[sec:Conclusion]{6} concludes.
 

\section{Literature Review} \label{sec:literature}
 
While numerous forms of diversity play a crucial role in influencing economic behavior, the extensive literature on the impact of diversity on economic behavior is too vast to comprehensively cover in this context. Therefore, I narrow the focus to a specific subset of literature that delves into the impact of gender and racial identities within teams. I specifically delve into the literature addressing teams, focusing on the impact of team diversity on both cooperation in the public goods provision activity and coordination in the minimum effort activity.

\subsection{Teams}
Research exploring the relationship between diversity and team behavior highlights the advantages and challenges in getting diverse teams to function efficiently  \citep{po15}. On one hand, a diverse team can offer a wealth of perspectives and experiences, fostering a climate conducive to creative and innovative solutions \citep{rvhv13}. For instance, diverse teams may generate novel ideas that would not emerge within a homogeneous group as diverse individuals are more likely to have different backgrounds and experiences that can provide different perspectives for the team. This has been shown in the literature to affect decision-making. For instance, according to \cite{ks16}, a diverse team is better positioned to utilize the diversity of skills in the team to make better-informed decisions. On the other hand, a diverse team may encounter obstacles such as increased conflict and coordination difficulties. These challenges can arise due to variations in styles and thinking processes among individuals from diverse backgrounds, leading to potential misunderstandings.  

\hspace *{0mm} Organizations have the potential to reap the benefits of diverse teams by effectively managing these challenges \citep{j23}. Furthermore, the demographic trends evident in census data highlight the ongoing surge in diversity within organizational settings within the US labor force, particularly along the dimensions of race and gender. Between 2010 and 2020, the proportion of the US population that was white decreased by 8.6\%.  Conversely, the percentage of the population identifying as black increased slightly, the Hispanic or Latino population grew a notable 23\%, and the population of individuals identifying as belonging to other racial categories surged by 129\% \citep{b20}. These shifts in population distribution underscore the diversification occurring along racial lines within the US population. Coupled with changes in labor force participation across gender identities, this indicates a rapid transformation in the composition of the labor force. As such, organizations that can effectively harness the potential of diverse teams stand to gain a competitive advantage in the evolving workforce landscape. Even organizations that currently lack diversity will increasingly witness the introduction of individuals from diverse demographic backgrounds, as the nature of the labor force more closely mirrors the melting pot that is the United States. 

\hspace *{0mm}Irrespective of the current diversity of work teams within an organization, if managed effectively, diversity can bring together diverse skill sets that enhance performance in an organization despite the potential for interpersonal tension \citep{rvhv13}. Evidence from research suggests that the benefits of diversity are driven by diverse teams having a diversity in expertise. For example, a study by \cite{ks16} finds that diverse teams are more likely to have a diversity of expertise. On the contrary, issues of conflicts and poor coordination common in teams can be further exacerbated in diverse teams. The predictions of similarity-attraction theory suggest that people prefer similarity in their interactions \citep{b71,bw62}. Hence, demographically distinct newcomers may disrupt established processes and struggle to gain acceptance in the new team if they are not well integrated \citep{c05,rkev13}.

\hspace *{0mm} Recent studies have explored the interplay between demographic diversity and team behavior. Notable studies highlight the impact of group diversity on aspects of team behavior. For instance, \cite{bjd13} find that the presence of a man in a fund management team increases the probability of selecting a higher risk investment, even though all-male teams do not inherently exhibit the highest risk-seeking tendencies. Generally, one can reasonably expect that individuals with dissimilar demographic backgrounds often have different cultural perspectives, have different educational experiences, and approach problems in a different way. Consequently, increasing the level of demographic diversity in a team could impact the level of skill diversity. For example, a Walmart store that serves a diverse community will benefit from having a diverse staff. Specifically, a diverse staff is more likely to include a staff member who distinctly understands the needs of a customer in search of an afro-pick (or any other item that is mostly used by a segment of the population). This dynamic applies in many settings including corporate America, where a culturally inappropriate product could be costly, both in terms of reputation and lost income.  Increasing the level of diversity in a team serves to increase the skill diversity available for decision-making \citep{jnn99}. However, reaping the benefits of skill diversity is contingent upon the ability of the team to coordinate and cooperate on the diverse ideas members bring forth from their respective backgrounds.

\hspace *{0mm} The positive performance of diverse teams can be impeded if team diversity negatively affects the behavior of individual team members. Substantial evidence indicates that individual behavior is impacted by the diversity of the team. For example, \cite{ch19} find that women in mixed-gender groups are twice as likely as women in single-gender groups to suffer from the gender stereotype effect, resulting in hesitancy to assume leadership roles or contribute ideas in gender-incongruent tasks. These findings are also corroborated by the work of \cite{brs22}. They find that women are more inclined to lead teams with a majority of females compared to teams with a majority of males. Diversity within teams not only influences the behavior of team members but also has effects on how individuals are treated within the team dynamics by others. The work of \cite{cfs21} sheds light on the impact of gender stereotypes on the selection of individuals to answer questions on behalf of the team, showing a preference for gender-congruent individuals in topics traditionally associated with specific genders. Their findings demonstrate that individuals who belong to a gender minority within a team are less inclined to engage in self-promotion. Similarly, \cite{skp21} provide compelling evidence that token women, standalone women in otherwise all male teams, tend to be less influential and receive less credit for their contributions. Furthermore, \cite{s17}has documented the role of gender in credit attribution within a team, highlighting its impact within academic research. It is worth noting that beyond the economic literature, \cite{bcn21} uncover that interns who share demographic similarities with senior managers tend to have more positive experiences and are more likely to receive job offers. These studies collectively emphasize the intricate dynamics of diversity in teams and underscore the need to address the biases and challenges that can arise from such diversity, ultimately fostering more equitable and inclusive environments. 

\hspace *{0mm} Exploring diversity within team environments presents a complex challenge. Teams can encompass diversity across various dimensions, such as age, gender, race, ethnicity, educational background, and work experience. Research investigating the impact of diversity reveal that the multifaceted nature of diversity can yield diverse behavioral outcomes across different contexts \citep{clls14}. Consequently, the impact of team diversity on performance can manifest in unique ways. The intricacy of studying diversity in team environments therefore necessitates grappling with the complex interplay between individual identity and its implications for team identification and diversity as well as the different dimensions of diversity. Research has demonstrated that individual identity significantly influences decision-making processes \citep{ ak08}. Recent evidence sheds light on the role of identity in shaping trust dynamics \citep{cde22}, anticipating instances of discrimination \citep{ack23}, and influencing income redistribution \citep{fghz23}. Moreover, team membership has also been shown to impact behavior across various contexts, encompassing phenomena such as shirking and free riding \citep{eg05}, preferences for outcomes \citep{crrbcdfggllmrwy07}, considerations of charitable acts and punishment \citep{chen2009group} and judicial claims \citep{sz11}. These studies underscore the significant influence of individual and group identities on behavior within distinct environments. Recognizing the diverse facets of identity becomes crucial when examining diversity within team settings, as different facets can yield distinct behavioral implications.

\hspace *{0mm} A growing body of literature delves into the implications of the different dimensions of diversity and team performance in economic decision-making. Much of this literature focuses on the effect of gender diversity on behavior. For instance, \cite{aai12} study the effect of the dynamics of gender diversity within teams. They utilize administrative data from endogenously formed teams with fixed compositions in the L'Oreal e-Strat Challenge. The findings reveal that all-women teams perform comparatively worse than all-men and heterogeneous gender teams, exhibiting less aggressive pricing strategies and placing a higher emphasis on social sustainability initiatives rather than research and development. While the study provides valuable insights into decision-making among homogeneous gender teams, it is limited in disentangling the specific mechanisms driving these decisions. The paper points to potential factors such as skill differences, sorting behaviors, and heterogeneous team dynamics. In another vein, \cite{am18} explore the influence of educational diversity within teams. In a randomized control trial, they investigate the significance of newcomers' educational backgrounds on team receptivity. Interestingly, the results demonstrate that "old-timers," incumbents in this study, are less accepting of newcomers with different educational backgrounds, particularly when they perceive the new arrivals as a threat to collective representation. Conversely, it is important to note that negative social categorization may still occur for newcomers in teams without the same concerns.

\subsection{Diversity and Cooperation }
\hspace *{0mm} Extensive research has demonstrated the complexities that arise from race and gender differences within teams and how they affect interactions among team members \citep{ha12}. According to Social Identity Theory \citep{t78,tbt79}, teams composed of individuals with diverse backgrounds and values may encounter challenges in integrating their unique perspectives and collaborating effectively. Furthermore, prior research widely agrees that people tend to feel more at ease working with others or groups they identify with. In contrast, the economics literature has found mixed results when exploring the impact of team member heterogeneity on cooperation in experimental settings utilizing public goods games. For instance, \cite{nt94} find that all-female groups demonstrate higher levels of cooperation compared to mixed-gender and all-male groups. In contrast, \cite{pbm19} find that mixed-gender groups are the most effective in cooperative collective action. For a comprehensive review of the literature, refer to \cite{blmv11} and \cite{cg09}. A growing body of studies seeks to reconcile these discrepancies, often explaining them through factors such as framing \citep{blmv11, ejmm13} and differences in conditional cooperativeness \citep{fkmmw21}.

\hspace *{0mm} While a significant body of work has investigated the impact of gender group diversity on contributions and cooperation in the provision of public goods, it is unclear how the race of individuals might impact cooperation within teams. Even less evident is how the combined racial and gender identities of an individual impact cooperation in the provision of public goods. It is important to note that an individual’s race is a significant component of their identity, and evidence from prior studies indicates that identities and common goals in teams can be influenced by racial identity \citep{burns2015}. The gender identity of an individual has also been shown to impact shirking behavior among team members. It is therefore particularly crucial for us to understand how diversity along racial and gender identities can affect efficiency of team production. The need to enhance cooperation can also be influenced by changes in group diversity, where existing members may leave, and new members may join. However, the impact of group dynamics, such as changes in the diversity of work teams, on the decisions of both existing and new team members, as well as the overall team performance, remains understudied. As aptly stated by \cite{ght18}, “[c]hanging group compositions over time, however, may alter a group in three different ways ceteris paribus: First, it divides up a group according to the entry of its individual members. Second, it implies the new group members’ adaption to the group. And third, there is also old group members’ adjustment to the new situation.” To address these dynamic aspects of group composition, my experimental design allows me to analyze and measure the impact of team diversity on team performance by distinguishing between these different mechanisms---gender, race and the addition of new members using the public goods provision activity. 

\subsection{Diversity and Coordination }

\hspace{0mm} Effective teamwork relies on coordination among group members to achieve efficient outcomes. A team can leverage its members' knowledge to facilitate such coordination, particularly when the group is homogeneous and shares certain demographic similarities. Hence, teams lacking such commonalities may encounter challenges in establishing connections that enable effective coordination \citep{pk12}. 

\hspace {0mm} While research on diversity and coordination predominantly focuses on gender, findings from \cite{g} indicate that smaller groups tend to coordinate more effectively than larger ones. Notably, they did not find significant gender differences in coordination. These results align with the findings of \cite{dg05}, who observed minor disparities in initial stages but not in the final stages of the repeated coordination activity. On a contrasting note, \cite{h00} finds that participants exhibited more aggressive behavior toward female co-players in a battle of sexes study . Surprisingly, this heterogeneous attitude to different genders actually facilitated coordination and increased earnings in mixed-gender groups compared to homogeneous pairs. Many of the arguments discussed above regarding the impact of group dynamic changes on cooperation can also affect the coordination among teams when group composition changes. In summary, newcomers could disturb existing channels due to their dissimilarity or their presence, changing the balance of the group. How incumbents and newcomers adjust to the group membership changes could impact choices within the group. 

 
\section{Experimental Design} \label{sec:Design}

To investigate the impact of group diversity on group and newcomer performances, I introduce a newcomer to a three-person pre-existing group, referred to as the incumbents.  111 sessions of 4 participants—444 undergraduate students overall—participated in the study conducted at the Behavioral Business Research Lab (BBRL, https://bbrl.uark.edu/) of the University of Arkansas between Fall 2022 and Spring 2023. A session in this study refers to the four-person participants that make decisions in the different parts of the study. Participants are recruited by gender and exogenously assigned gender teams in the laboratory when they arrive. In situations where possible, individuals are randomized into mixed-race teams or allwhite teams. Tables 1 and 2 below show the distribution of the treatments.Table 1 presents summary of sessions by gender composition and newcomer gender. Table 2 shows that there are 80 participants exogeneously assigned to incumbent all-men teams, 184 participants are assigned to incumbent all-women teams and 180 participants are assigned to mixed-gender teams. 


\begin{table}[H]
 \captionsetup{justification=raggedright,singlelinecheck=false}
\caption{Summary of Sessions by Gender Composition} \label{tab:table1}
\centering
\begin{table}[htbp]
\begin{tabular}{c c c c}
\toprule
                   &\multicolumn{3}{c}{Newcomer Gender}  \\
                   &      Male  &   Female &    Total    \\
\midrule
All-men            &      44 &         36 &       80     \\
\midrule
All-women          &      88 &         96 &      184   \\
\midrule
Mixed gender       &      88 &         92 &      180  \\
\midrule
Total              &      220 &       224 &      444  \\
\bottomrule
\end{tabular}
\end{table}

\end{table}

 

\begin{table}[H]
 \captionsetup{justification=raggedright,singlelinecheck=false}
\caption{Summary of Sessions by Racial Composition} \label{tab:table2}
\centering
\begin{table}[htbp]
\begin{tabular}{c c c c}
\toprule
                   &\multicolumn{3}{c}{Newcomer Race}  \\
                   &      Non-White  &   White &    Total    \\
\midrule
All-white            &      60 &         88 &       148     \\
\midrule
Mixed-race          &      184 &         112 &      296   \\
\midrule
Total              &      244 &       200 &      444  \\
\bottomrule
\end{tabular}
\end{table}
\end{table}

\noindent \textbf{ \textit{Part I}} 
 \newline
Each session involves four participants. When the participants arrive at the laboratory, they are randomly assigned IDs A to D conditional on their gender. A summary of the experimental design is presented in figure 1 below.  Subjects A, B, and C are sent to Lab 1, and D (referred to as "the standalone" hereafter) is sent to Lab 2. In Lab 1, participants are seated around a table with three chairs in the center of the room.  These team members are thus able to observe the gender and race of the other participants, a feature often lacking in other studies.  Participants are notified that the first part of the experiment consists of two stages. In the first stage, the group plays a triangle puzzle game designed to enhance group identity \citep{eg05}. Each group member is given an envelope containing four cut pieces of cardboard. The group is then told to make triangles similar to the sample on the table. The four pieces in each envelope are not enough to make a triangle, and group members are encouraged to communicate and trade pieces to be able to form the triangles. Additionally, participants are informed that interactions during the first stage (the puzzle stage) are being video and audio recorded. Group members are paid 10 Experimental Currency Units (ECUs) for each piece correctly placed by any group member. All teams correctly solved the puzzle. Earnings in ECUs from this part and other parts of the study are exchanged for dollars at the end of the study . The average payment per participant is 20 US dollars. It is public information that participants are informed about their payoff at the end of Stage 2 of Part II. Participants are given 10 minutes to work on the puzzle task. Simultaneously, the standalone in Lab 2 is instructed to wait for further instructions in 10 minutes. 

\begin{figure}[H]
\captionsetup{justification=raggedright,singlelinecheck=false}

\caption{Flowchart of Experimental Procedure}
\includegraphics[scale=0.6]{Figures/Design.png} 
\end{figure}
 
 
 \hspace  *{0mm} In Stage 2 of Part I, A, B, and C play economic decision games on their designated computers in Lab 1.  Each computer is located at a different corner of the lab separated by tall dividers for participants’ privacy as shown in figure 8 in the appendix. A, B, and C play the games on their own but as a group, and D---the standalone---plays the games with two computer robots in Lab 2. The participants play two games: a public goods provision game that involves a voluntary contribution mechanism (VCM) and a minimum effort game. The participants are given instructions at the beginning of each game and informed that they will play five rounds of each game. The marginal per capita return (MPCR) in Part I of the public provision game is 0.5.  Payoff of individual participants in the public goods provision game -voluntary contribution mechanism (VCM) are calculated as follows: 
 
 $ \pi_{i} = 100 $–$ c_{i} + \frac{M}{k}\sum_{j=1}^{k}(c_j )  $ \space \space \space  \space \space \space  \space  (1)
\newline
\noindent\textit{where $\pi_{i}$ represents individual \textit{i}'s payoff, $c_{i}$ represents individual $i’s$ contributions to the group account, and $c_{j}$ represents the individual contributions of all players to the group account. k is the number of participants in the group (i.e., three for the three-person group and four for the four-person group). Finally, M represents the multiplier, the constant by which contributions to the group are multiplied by.}

Payoffs of individual participants in the minimum effort are calculated as follows:

$\pi_{i}  =  85+min(H_{j} )- \frac{3}{4} h_{i}  $   \space \space \space  \space \space \space  \space \space  (2)   
\newline
\noindent\textit{where $h_i$ is the number of hours contributed by individual i toward the group activity, and min($H_j$) is the minimum hours contributed among all the individuals within group j. }

The participants are paid based on their cumulative payoffs of 5 rounds of one randomly chosen game out of the two games in accordance with incentive compatibility prescribed in \cite{ach18}. Additionally, participants are not provided feedback on choices of other participants until the end of Part I. 

\noindent \textbf{ \textit{Part II}}
\newline
Following the completion of Part I, participant D (the standalone) is brought into Lab 1 to join the three-person group of A, B, and C, and Part II of the experiment begins. The physical entrance of the standalone makes the individual's gender and race observable to the other participants, and vice versa. A, B, and C are invited back to their seats at the round table where they played the puzzle game in Part I, and an additional chair is added to the table for D. Participants are informed that in Part I, A, B, and C played two decision-making games as a group, and D played the same two decision-making games with two computer robots. 
  
The experimenter then reads instructions for Stage 1 of Part II, which is another puzzle game. The participants are then each given a new envelope containing six pieces. The participants are told that their individual task is to make a triangle similar to the sample shown. As in Part I, the six pieces in each envelope cannot form a triangle, and group members are encouraged to communicate and trade pieces to form triangles. Similar to Part I, participants are told that their interaction in the puzzle-solving stage is being video and audio recorded. Each group member is required to make their own triangle, and participants are paid 10 ECUs per correctly placed piece by each group member.  Once again, participants are given 10 minutes to complete the task. 65\% of teams correctly finished the puzzle in the allotted time. This activity is designed to enhance the group identity of the newly formed four-person group. The puzzle is purposefully more challenging than the one in Stage 1 of Part I. Since participant D (the standalone) did not participate in the similar puzzle in Part I, the pre-existing group members, A, B, and C, may help D complete the task.

 \hspace  *{0mm} After finishing the puzzle game, the four participants are again invited to their designated computers at the four corners of the room to play the two economic decision games—public goods provision game -voluntary contribution mechanism (VCM) and minimum effort —as in Stage 2 of Part I. The marginal per capita return (MPCR) in Part II is adjusted to 0.438 to avoid amplifying cooperation behavior. In addition, I elicit their risk preferences using a lottery mechanism prescribed by \cite{eg02}. The participants complete a post-experimental survey before seeing their Part II earnings and total earnings. Their Part II earnings include their payoffs in the puzzle game, the 5-round cumulative payoffs of a randomly chosen computerized decision game, and payoffs of the lottery. The total earnings for the experiment are the sum of earnings in Parts I and II and \$7 show-up fee.

  \hspace  *{0mm} The study occurs in two parts each consisting of 2 stages as summarized in flow chart diagram shown in figure 1 above. It is worth noting that in each part, participants engage in a group building activity in stage 1. The activity involves attempting a puzzle of varying difficulty. Communication and interactions during stage 1 are recorded from two angles. These recordings are coded by research assistants and included as chat controls in the regression analysis of decision making. In Part I, stage 1, the participants start with an incomplete set of pieces and need to communicate and trade pieces to get all required pieces as show in figure 4 in the appendix. Each individual participant is required to complete the puzzle by putting the pieces together to form triangle as shown figure 5 in the appendix. In Part II, stage 1, participants solve a different, more difficult puzzle requiring each participant to trade pieces to obtain all the required pieces as shown in figure 6. Similar to Part I, each participant in Part II is expected to complete the puzzle as shown in figure 6 in the appendix. Once participants finish the puzzle or the allotted 10 minutes for the puzzle is exhausted, participants proceed to stage 2 in both parts where decisions are made on the computer. Figure 8 in the appendix shows the environment of the lab. In the first and third images, pilot participants are shown working on the group building activity in stage I. The image in the middle of of figure 8 shows participants seated at their individual computers in the second stage. 

\section{Hypotheses} \label{sec:Hypotheses}
In many organizations, it is beneficial for employees to cooperate and coordinate. Many situations arise in which cooperation or coordination does not benefit the individual but is beneficial to the group or organization. In this study, I analyze group decision-making using a public goods activity and a minimum effort activity. The public goods game models a group production environment in which the provision of a good requires the contributions of a proportion of the group \citep{c10,stoop2012lab,kagel2020handbook}. The minimum effort activity is an economics decision-making activity modelled to represent a team environment in which the production of a good depends on the effort of the weakest link \citep{vbb91}.

 \hspace  *{0mm} In my study, individual decisions are made simultaneously under uncertainty. Willingness to cooperate is impacted by the behavioral interactions that the individual has with the group, their perceptions of the other people in the group, and their understanding of what is culturally expected of them. Findings in the economics literature shed light on how team composition affects cooperation and coordination tendencies. For example, \cite{nt94} find that all-women groups are more cooperative than all-men and mixed-gender groups. In contrast, \cite{pbm19} find that mixed-gender groups cooperate most effectively. Expanding on this extant literature, I propose my first hypothesis:
 
 \textit{Hypothesis 1: The diversity of a group affects the cooperation and coordination decisions of individual group members in a public goods provision and a minimum effort activity.}

\hspace  *{0mm} Evidence in the group identity literature shows that individuals who are part of a group are more cooperative. Members of teams build group identity and contribute more to the group compared to standalone individuals. Hence, the overall contributions to the group activity are likely to differ depending on the status of the individual as a newcomer or an incumbent. Additional evidence in the laboratory establishes that replacing established team members with newcomers yields a reduction in overall team performance driven partly by a breakdown in trust \citep{m2013}. Hence, newcomers are expected to exhibit different cooperation and coordination tendencies as compared to incumbents. This concept gives rise to my second hypothesis: 

  \textit{Hypothesis 2: Incumbents are expected to cooperate and coordinate better overall than newcomers.}

\hspace  *{0mm} Behavioral economics research focusing on team dynamics reveals that the behavior of established team members undergoes shifts based on the characteristics of newly joined individuals. An insightful study illustrates that existing members exhibit reduced openness towards newcomers who possess distinctive qualities, perceiving them as lacking in cooperation and competence \citep{am18}. Further evidence show partners and strangers in a team react differently when working together in a team \citep{ght18}. Expanding upon this literature, I postulate the following hypothesis: 

    \textit{Hypothesis 3: The cooperation and coordination of incumbents and newcomers change after newcomers join the team.}

\hspace  *{0mm} A wealth of research in the field of behavioral economics underscores the impact of group identity on the collective behavior of teams. These group dynamics can elevate performance across various contexts, even within diverse teams \citep{eg05}. This phenomenon is echoed in the findings of \citep{crrbcdfggllmrwy07}, who find that individuals who align their identity with a group exhibit distinct behavioral patterns compared to those who perceive themselves as isolated individuals within the same group. An extensive body of work in behavioral economics also underscores the influence of individual identities within a group on behavioral patterns \citep{har2009,chen11}. However, it is reasonable to expect a different effect in an environment where diverse individuals are joining the labor force. Diverse teams are likely to better assimilate newcomers than homogeneous teams as individuals are able to connect with members of the existing team. The congruence of an individual’s identity along racial and gender dimensions with the existing team could have an impact on newcomer behavior. Based on insights from these interconnected lines of research, I formulate my next hypothesis:

    \textit{Hypothesis 4: The diversity of the incumbent teams could impact cooperation and coordination choices of newcomers to the team.}

\hspace  *{0mm} Just as incumbents can influence the contributions of newcomers within a group, the identity of a newcomer can influence how incumbents contribute to the group. When an incumbent identifies more closely with a newcomer than other incumbents in the team, it can affect their subsequent contributions to the group. Conversely, the existing group identity among the incumbents may moderate the influence of an individual's identity on their contributions to the group. This argument forms the basis for my final hypothesis, which posits that the congruence between the identity of incumbents and the identity of newcomers influences the contributions of incumbents to the group.

    \textit{Hypothesis 5: The contributions of incumbents are influenced by the congruence of their identity and the identity of the newcomer.}


\section{Analysis} \label{sec:Analysis}
The analysis is organized into five main sections for clarity. The initial segment provides an overview of the study participants in the summary statistics section. Details of coding for the interactions in the group building activities, as described in the experimental design section, are presented in the second section. Section three provides information on the empirical specifications. Following this, the focus shifts to the results of cooperation within the public goods provision activity in the fourth section. Finally, the analysis delves into coordination choices within the minimum effort activity in the fifth section.

My analysis of cooperation in the public goods provision and coordination in the minimum effort activity proceed as follows. I first explore cooperation and coordination among incumbents in the first part of the study. I then broaden the scope of the analysis to look at the overall cooperation and coordination of both newcomers and incumbents. The evolution of actions in the second part of the study is then examined.

\subsection{Summary Statistics}
A total of 444 participants are recruited from the University of Arkansas, primarily sourced through the Walton College Behavioral Business Research Laboratory Sona System, along with targeted recruitment posters placed strategically around the Fayetteville campus. Among the participants, 178 individuals, accounting for 40\% of the sample, self-identified as men, while 261 participants identify as women. Additionally, four participants specify another gender identity, and one person opts not to disclose their gender identity.

Regarding racial identification, a notable majority, comprising 68\% of the total 444 participants, identify as white. Five percent identify as black/African American, 12\% as Hispanic, 10\% as Asian, 1\% as Middle Eastern, and 4\% as belonging to some other ethnicity. Within the participant pool, 184 individuals are purposefully assigned to all-women teams, 80 to all-men teams, and 180 to mixed-gender groups through an exogenous assignment process based on their gender. Furthermore, 33\% of the sample is placed into all-white homogeneous racial groups, while the remaining 67\% are assigned to other racial groups.


\subsection{Communication}
To examine how communication during the group building activity affects actions in the  public goods provision and minimum effort activities, we hire three English-fluent research assistants to code the content of video interactions. The coders receive a description of the experiment and experimental tasks but are not informed about the purpose of the study. The coders are told to code the messages independently using their own best judgement based on the pre-defined coding criteria. Video and audio interactions are divided into 30-seconds of video conversations called conversation segments. 

A conversation segment can be classified into multiple categories including: (1) frustration, (2) confusion, (3) talk in agreement, (4) talk in disagreement, (5) confident, (6) assertive, (7) excitement or satisfaction (8) comfortable as detailed in Table 25 in the appendix. Coders are tasked with assessing the presence or absence of frustration, with 1 denoting its presence and 0 indicating its absence. Similarly, expressions of confusion are assessed with a score of 1 during the 30-second segment if a coder assesses expressions of confusion. Participants engagement in conversations are also assigned binary values of whether they are affirming agreement (1 = present, 0 = absent) or expressing disagreement (1 = present, 0 = absent) with others during the puzzle-solving process. Participants confidence during the puzzle solving process is rated on a scale from 1 (not confident at all) to 5 (very confident). Expressions of excitement or satisfaction related to the puzzle-solving process are noted with 1, while their assertiveness in communication with others is measured on a scale from 1 (not assertive at all) to 5 (very assertive). Additionally, participants' comfort is evaluated based on language or nonverbal cues, ranging from 1 (not showing at all) to 5 (shows very clear signs). Each research assistant codes the interactions by 30-second segments for stage 1 of Part I and Part II of the study. The assessments of coders are summarized using majority voting for the binary assessments. The assessments that are on a likert scale are averaged across the three coders. The segments are then averaged per individual participant. 

It is important to note that, in addition to the decisions made by participants in stage 2, coded interactions of research assistants are included as chat controls. Chat controls are categorized into three main blocks using factorial analysis. "Positive chat" refers to the level of positive interactions a participant has with the other participants in the session and it is based on scores on assessments for assertiveness, excitement or satisfaction and comfort within the team. "Negative chat" assigns a value to the level of negative interactions a participant has with other participants. Variables included in the negative chat analysis include assessments of frustration, confusion and "talk in disagreement". Finally, the level of engagement of an individual in the puzzle-solving phase is assigned an engagement metric based on their speech and overall engagement in the task. 


\subsection{Empirical Specification} \label{subsec:Specification}
\subsubsection{Incumbent Teams in Part I} 
I use the tobit regression model to estimate the causal impact of team diversity on individual cooperation and coordination choices. First, I consider economic decision-making during the first part of the study before the newcomer is added to the team. Incumbent teams (3-person teams) are classified by levels of diversity. I start by analyzing the impact of the interaction between the gender diversity within a team and the gender identity of an individual on their cooperation and coordination choices in the activities. Gender-diverse teams consist of members with more than one gender identity, while gender-homogeneous teams consist of members sharing the same gender. Actions are modeled using equation 3 outlined below. The excluded group is a man in a gender-homogeneous team. 
\begin{center}      
 $ Y_{i}= \beta_1 Gender$-$diverse_{i} + \beta_2Notmale_{i}+\beta_3 Gender$-$diverse_{i}*Notmale_{i} + \theta X_{i} + \epsilon_{i} $ (3)
\end{center}

\noindent where $Y_{i}$ is the contribution of participant $i$ in either the public goods provision or the minimum effort activity. $Gender$-$diverse_{i}$ is a dummy variable indicating whether participant $i$ is in a gender diverse team. $NotMale_{i}$ is an indicator for participant gender. $Gender$-$diverse_{i}*Notmale_{i}$ is an interaction of non-male identifying individual in a gender diverse team. $X_{i}$ is a set of individual characteristics such as age, major, income, individual interactions with the group, parents' socioeconomic background and other personal characteristics. $\epsilon_{i}$ is the residual term. As the default inference method in the tobit regression, I specify upper and lower bounds. I also cluster standard errors at the session level.

Team assignment is randomly assigned. The coefficient, $\beta_1$ identifies the causal impact of a man in a gender diverse team in comparison to a man in the homogeneous gender team, $\beta_2$ identifies the effect of non-male identifying individual in homogeneous gender teams. The combined effects of $\beta_1$ , $\beta_2$ and $\beta_3$ identify the causal impact of a non-male identifying individual in a gender diverse team in comparison with a man in the homogeneous gender team. Under the null hypothesis of no treatment effects, there is no difference in contributions of individuals in different teams of varying gender diversity. 

\hspace  *{0mm} Next, I consider how the interaction of the racial diversity of the team and the racial identity of the individual affects the cooperation and coordination choices of the individual in the team. An individual's racial identity is categorized into white versus non-whites (minorities). A racially diverse team has more than one race while members of a racially homogeneous team share the same race. I utilize equation 4 below to examine how cooperation and coordination choices of an individual is impacted by the interaction of the team level of racial diversity and individual racial identity. In the equation below, the omitted category is a white person in a racially homogeneous team.

\begin{center}
 $ Y_{i} = \beta_1 Race$-$diverse_{i} + \beta_2 Notwhite_{i} + \beta_3 Race$-$diverse_{i}*Notwhite_{i} + \theta X_{i} + \epsilon_{i}  $    (4)
\end{center}

\noindent where $Y_{i}$ is the contribution of participant $i$ in either the public goods provision or the minimum effort activity. $Race$-$diverse_{i}$ is a dummy variable indicating whether participant $i$ is in a racially diverse team. $Notwhite_{i}$ is an indicator variable for a non-white person. $Race$-$diverse_{i}*Notwhite_{i}$ is an interaction variable for a non-white individual in a racially diverse team. $X_{i}$ is a set of individual characteristics such as age, major, income, individual interactions with the group, parents' socioeconomic background and other personal characteristics. $\epsilon_{i}$ is the residual term. As the default inference method, I specify upper and lower bounds as well as cluster standard errors at the session level.

The coefficient, $\beta_1$ identifies the causal impact of whites in a racially diverse team in comparison with whites in the racially homogeneous team, $\beta_2$ identifies the effect of non-white individuals in the homogeneous racial teams. The combined effects of $\beta_1$ , $\beta_2$ and $\beta_3$ identify the causal impact of a non-white individual in a racially diverse team in comparison with a white individual in the homogeneous racial team. Again, under the null hypothesis of no treatment effects, there is no difference in contributions of individuals in different teams of varying racial diversity. 

\hspace  *{0mm} Finally, I consider how the interaction of the diversity (racial and gender) of the team and the joint racial and gender identity of the individual affects the cooperation and coordination choices of the individual in the team. The experimental design reveals the salience of both identities. Teams that are homogeneous on both gender and race are classified as the least diverse. Teams that have heterogeneity on either race or gender are classified as "$Moderate$-$D$" diverse teams. Finally, teams that have heterogeneity on both race and gender are classified as the "$Most$-$D$" diverse. In addition to the classification of team diversity, I classify individuals into four main types - an interaction of the racial and gender identities of the individual. In the tobit regression model presented in equation 5 below, the omitted category is a white man in the least diverse team.  

\begin{center}
 $Y_{i}=\beta_1Moderate$-$D_{i}+\beta_2Most$-$D_{i}+\beta_3WhiteNonmale_{i}+\beta_4NonwhiteNonmale_{i}+\beta_5 NonwhiteMale_{i}+\beta_6Moderate$-$D_{i}*WhiteNonmale_{i}+\beta_7Moderate$-$D_{i}*NonwhiteNonmale_{i}+ \beta_8Moderate$-$D_{i}*NonwhiteMale_{i}+\beta_9Most$-$D_{i}*WhiteNonmale_{i}+\beta_{10}Most$-$D_{i}*NonwhiteNonmale_{i}+\beta_{11}Most$-$D_{i}*NonwhiteMale_{i}+\theta X_{i}+\epsilon_{i}  $  \space \space \space    (5)
\end{center}

\noindent where $Y_{i}$ is the contribution of participant $i$ in either the public goods provision or the minimum effort activity. $Moderate$-$D_{i}$ is a dummy variable indicating whether participant $i$ is in a moderately diverse team. $Most$-$D_{i}$ is a dummy variable indicating whether participant $i$ is in the most diverse team. $WhiteNon-male_{i}$ is an indicator variable for a white non-male identifying participant,  $NonwhiteNonmale_{i}$ is an indicator for a non-white non-male identifying participant and $NonwhiteMale_{i}$ is an indicator variable for a non-white male. $Moderate$-$D_{i}*WhiteNonmale_{i}$ is an interaction variable for a white non-male identifying individual in a moderately diverse team. $Moderate$-$D_{i}*NonwhiteNonmale_{i}$ is an interaction variable for a non-white non-male identifying individual in a moderately diverse team and $Moderate$-$D_{i}*NonwhiteMale_{i}$ is an interaction variable for a non-white man in a moderately diverse team. Similarly, $Most$-$D_{i}*WhiteNonmale_{i}$ is an interaction variable for a non-male identifying white individual in the most diverse team.  $Most$-$D_{i}*NonwhiteNonmale_{i}$ is an interaction variable for a non-male, non-white individual in the most diverse team and $Most$-$D_{i}*NonwhiteMale_{i}$ is an interaction variable for a non-white man in the most diverse team. $X_{i}$ is a set of individual characteristics such as age, major, income, individual interactions with the group, parents' socioeconomic background and other personal characteristics. $\epsilon_{i}$ is the residual term. As the default inference method, I specify upper and lower bounds as well as cluster standard errors at the session level.


\hspace  *{0mm} The coefficient, $\beta_1$ identifies the causal impact of a white man in a moderately diverse team in comparison to a white man in the least diverse team, $\beta_2$ identifies the causal impact of a white man in the most diverse team in comparison to a white man in the least diverse team, $\beta_3$ identifies the effect of a white non-male identifying individual in the least diverse team in comparison to a white man in the least diverse team, $\beta_4$ identifies the effect of a nonwhite non-male identifying individual in the least diverse team in comparison to a white man in the least diverse team and $\beta_5$ identifies the effect of a nonwhite man in the least diverse team in comparison to a white man in the least diverse team. The combined effects of $\beta_1$ , $\beta_3$ and $\beta_6$ identify the causal impact of a white non-male identifying individual in a moderately diverse team in comparison with a white man in the least diverse team, the combined effects of $\beta_1$ , $\beta_4$ and $\beta_7$ identify the causal impact of a nonwhite non-male identifying individual in a moderately diverse team in comparison with a white man in the least diverse team, the combined effects of $\beta_1$ , $\beta_5$ and $\beta_8$ identify the causal impact of a nonwhite man in a moderately diverse team in comparison with a white man in the least diverse team, the combined effects of $\beta_2$ , $\beta_3$ and $\beta_9$ identify the causal impact of a white non-male identifying individual in the most diverse team in comparison with a white man in the least diverse team, the combined effects of $\beta_2$, $\beta_4$ and $\beta_{10}$ identify the causal impact of a nonwhite non-male identifying individual in the most diverse team in comparison with a white man in the least diverse team. Finally, the combined effects of $\beta_2$ , $\beta_5$ and $\beta_{11}$ identify the causal impact of a nonwhite man in the most diverse team in comparison with a white man in the least diverse team. 

\subsubsection{Overall Incumbent and Newcomers}
Similar to the specifications above, I use the tobit regression model to estimate the causal impact of status on overall economic decision making across the 10 rounds of the study. An individual's status defines whether the person is an incumbent (part of the incumbent 3-person team) or a newcomer during the study. I employ Tobit regression models, as specified in equations 6a and 6b below, to investigate the influence of an individual's status and identity on their choices related to cooperation and coordination. Equations 6a and 6b adhere to the format used in equations 3 and 4, respectively, concerning gender and race identification. Equation 6a examines how behavior varies by gender across status with the omitted category being incumbent men. Equation 6b examines how behavior varies by race across status with the omitted group being white incumbents.

\begin{center}
 $ Y_{i} = \beta_1 Newcomer_{i} + \beta_2 Notmale_{i} + \beta_3 Newcomer_{i}*Notmale_{i} + \theta X_{i} + \epsilon_{i}  $  \space   (6a)
\end{center}

 \noindent where $Y_{i}$ is the contribution of participant $i$ in either the public goods provision or the minimum effort activity. $Newcomer_{i}$ is a dummy variable indicating whether participant $i$ is a newcomer during the 10 rounds of the study. $Notmale_{i}$ is an indicator variable for participant gender.$Newcomer_{i}*Notmale_{i}$ is an interaction variable of non-male identifying newcomer. $X_{i}$ is a set of individual characteristics such as age, income, individual interactions with the group, parents' socioeconomic background and other personal characteristics. $\epsilon_{i}$ is the residual term. As the default inference method, I specify upper and lower bounds as well as cluster standard errors at the session level.

\begin{center}
 $ Y_{i} = \beta_1 Newcomer_{i} + \beta_2 Notwhite_{i} + \beta_3 Newcomer_{i}*Notwhite_{i} + \theta X_{i} + \epsilon_{i}  $  \space  (6b)
\end{center}

 \noindent where $Y_{i}$ is the contribution of participant $i$ in either the public goods provision or the minimum effort activity. $Newcomer_{i}$ is a dummy variable indicating whether participant $i$ is a newcomer during the 10 rounds of the study.  $Notwhite_{i}$ is an indicator variable for participant race. $Newcomer_{i}*Notwhite_{i}$ is an interaction variable for a non-white individual who is a newcomer. $X_{i}$ is a set of individual characteristics such as age, major, income, individual interactions with the group, parents' socioeconomic background and other personal characteristics. $\epsilon_{i}$ is the residual term. As the default inference method, I specify upper and lower bounds as well as cluster standard errors at the session level.

\subsubsection{Incumbent and Newcomers After Changes in Group Composition}

Next, I examine how the cooperation and coordination choices of an individual differ based on the individual's status and the period of the study.  Actions in Part I are examined as before changes to the group composition, while actions in Part II of the study are considered to be after changes in group composition. The contributions of an individual are analyzed using the tobit regression specified in equation 7 below, with the omitted group being the incumbents before changes in group composition.

\begin{center}
 $ Y_{i} = \beta_1 Newcomerbefore_{i} + \beta_2 Newcomerafter_{i} + \beta_3 Incumbentafter_{i} + \theta X_{i} + \epsilon_{i}  $  \space \space \space  \space \space \space  \space \space \space \space  \space \space \space  \space \space \space   (7)
\end{center}

\noindent where $Y_{i}$ is the contribution of participant $i$ in either the public goods provision or the minimum effort activity. $Newcomerbefore_{i}$ is a dummy variable indicating whether participant $i$ is a newcomer during the first 5 rounds of the study. $Newcomerafter_{i}$ is a dummy variable indicating whether participant $i$ is a newcomer after changes in group composition at the end of round 5. $Incumbentafter_{i}$ is a dummy variable indicating an incumbent's economic decisions after changes in the team composition post round 5. $X_{i}$ is a set of individual characteristics such as age, major, income, individual interactions with the group, parents' socioeconomic background and other personal characteristics. $\epsilon_{i}$ is the residual term. As the default inference method, I specify upper and lower bounds as well as cluster standard errors at the session level.

\subsubsection{Newcomers in a New Team}
In addition to the above, I consider how newcomers cooperate and coordinate in different teams of varying diversity. Teams are classified based on both gender and racial diversity in the team, unlike the previous specifications. Teams that are gender and racially diverse are considered the most diverse. Teams that are only gender diverse or only racially diverse are considered moderately diverse. Teams that are homogeneous on gender and race are considered the least diverse. I utilize equation 8a to examine the causal impact of the existing team racial and gender diversity on economic decisions of the newcomer with the omitted group being newcomers in the least diverse team. 

\begin{center}
 $ Y_{i} = \beta_1 Moderately$-$diverse_{i} + \beta_2 Most$-$diverse_{i} + \theta X_{i} + \epsilon_{i}  $  \space \space \space    (8a)
\end{center}

\noindent where $Y_{i}$ is the contribution of newcomer $i$ in either the public goods provision or the minimum effort activity. $Moderately$-$diverse_{i}$ is a dummy variable indicating whether participant $i$, the newcomer is in a moderately diverse team. $Most$-$diverse_{i}$ is a dummy variable indicating whether a newcomer is in the most diverse team and $X_{i}$ is a set of individual characteristics such as age, major, income, individual interactions with the group, parents' socioeconomic background and other personal characteristics. $\epsilon_{i}$ is the residual term. As the default inference method in the tobit regression, I specify upper and lower bounds as well as cluster standard errors at the session level.

Next, I look at how the existing gender diversity affects economic decisions of a newcomer by gender of the newcomer. Incumbent teams are classified based on the levels of gender diversity of the existing team following the format of equation 3 above. The homogeneous gender teams have the same gender. A team is considered gender diverse if the team has a heterogeneous gender composition. The decisions of a newcomer are analyzed using the tobit model specified in equation 8b below with the omitted group being a man in a homogeneous gender team.  


\begin{center}
$Y_{i}=\beta_1Gender$-$diverse_{i}+\beta_2Notmale_{i}+\beta_3Gender$-$diverse_{i}*Notmale_{i}+\theta X_{i} + \epsilon_{i} $ (8b)
\end{center}

\noindent where $Y_{i}$ is the contribution of participant $i$ in either the public goods provision or the minimum effort activity. $Gender$-$diverse_{i}$ is a dummy variable indicating whether participant $i$,  the newcomer is joining a gender diverse team. $Notmale_{i}$ is an indicator for participant gender. $Gender$-$diverse_{i}*Notmale_{i}$ is an interaction of non-male identifying individual in a gender diverse team. $X_{i}$ is a set of individual characteristics such as age, major, income, individual interactions with the group, parents' socioeconomic background and other personal characteristics. $\epsilon_{i}$ is the residual term. As the default inference method in the tobit regression, I specify upper and lower bounds as well as cluster standard errors at the session level.


\hspace  *{0mm} Finally, I consider how the interaction of the racial diversity of the team and racial identity of the individual affects the cooperation and coordination choices of the newcomer in the team. I follow the definitions established in equation 4 above where an individual's racial identity is categorized into white versus non-whites (minorities). I then utilize equation 8c below to examine how cooperation and coordination choices of an individual is impacted by the interaction of the team racial diversity and individual racial identity. The omitted group in equation 8c below is a white person in a racially homogeneous team. 

\begin{center}
 $ Y_{i} = \beta_1 Race$-$diverse_{i} + \beta_2 Notwhite_{i} + \beta_3 Race$-$diverse_{i}*Notwhite_{i} + \theta X_{i} + \epsilon_{i}  $  \space   (8c)
\end{center}

\noindent where $Y_{i}$ is the contribution of participant $i$ in either the public goods provision or the minimum effort activity. $Race$-$diverse_{i}$ is a dummy variable indicating whether participant $i$ is in a racially diverse team. $Notwhite_{i}$ is an indicator variable for participant race. $Race$-$diverse_{i}*Notwhite_{i}$ is an interaction variable for a non-white individual in a racially diverse team. $X_{i}$ is a set of individual characteristics such as age, major, income, individual interactions with the group, parents' socioeconomic background and other personal characteristics. $\epsilon_{i}$ is the residual term. As the default inference method, I specify upper and lower bounds as well as cluster standard errors at the session level.


\subsubsection{Incumbents after a Newcomer Joins}

Finally, I look at how decisions of incumbent members of the team are affected by the racial identity and gender identity of a newcomer after changes in the group composition, after round 5. Incumbent members of the team are classified based on whether they share gender with the newcomer or share race as a racial minority (non-white) or white. This leads to incumbents that share both identities with the newcomer, incumbents that share gender identity but not racial identity, incumbents that share racial identity but not gender and incumbents that share neither racial identity or gender identity with the newcomer. The decisions of incumbent members of the team are analyzed using the tobit model specified in equation 9 below with the omitted group being incumbents that do not share race and gender with the newcomer.  

\begin{center}
 $ Y_{i} = \beta_1 Congruentgender_{i} +\beta_2 Congruentrace_{i} + \beta_3 Congruentboth_{i} + \theta X_{i} + \epsilon_{i}  $  \space \space \space    (9)
\end{center}

\noindent where $Y_{i}$ is the contribution of participant $i$ in either the public goods provision or the minimum effort activity.  $Congruentgender_{i}$ is a dummy variable indicating whether incumbent participant $i$ shares gender but not race with the newcomer. $Congruentrace_{i}$ is a dummy variable indicating whether incumbent participant $i$ shares race but not gender with the newcomer and $Congruentboth_{i}$  is a dummy variable indicating whether incumbent participant $i$ shares gender and race with the newcomer. $X_{i}$ is a set of individual characteristics such as age, income, individual interactions with the group, parents' socioeconomic background and other personal characteristics. $\epsilon_{i}$ is the residual term. As the default inference method, I specify upper and lower bounds as well as cluster standard errors at the session level.

\subsection{Cooperation in the Public Goods Provision}
\noindent I start the analyses by examining the average cooperation in the first part of the study among incumbent teams to understand whether the diversity of a team affects individual performance in teams of varying degrees of diversity. It is important to note "incumbents" in this study are participants in the 3-person teams prior to changes in team composition. As is the norm in the public goods provision, I consider the contributions a player makes toward the public account($C_i$ in equation 1 above) to represent that individual's level of cooperation within the team. Individual participant payoffs are based on equation 1 outlined above. I use the tobit regression models presented in the \hyperref[subsec:Specification]{empirical section} of the paper for my analyses. I also cluster the errors by session and present the results of the marginal effects, as is standard for causal inference.  

\noindent\textbf{\textit{Gender Diversity}} 

\noindent First, I consider how the gender diversity of the incumbent teams affect the cooperation across genders. The results of the  regression model are presented in table 3 below. The results presented in table 3 demonstrate the impact of incumbent team diversity on behavior of different incumbents in their cooperation decisions. The first column of table 3 illustrates the foundational model, estimated using equation 3 as previously specified above. The model considers a man in homogeneous gender team as the omitted group in the base model in column 1. The findings reveal that men in gender homogeneous teams tend to exhibit higher levels of cooperation compared to the non-male identifying individuals in homogeneous gender teams. As compared to men in homogeneous gender teams, non-male identifying individuals contribute 15.4\% ($P-value<0.05$) less of their endowment toward the group. Additionally, men in the gender diverse team do not statistically significantly cooperate at a level different from men in the homogeneous gender teams ($P-value=0.37$). However, non-male identifying individuals in diverse gender groups cooperate at a lower level than  men in the homogeneous gender group and men in the diverse gender group.  Specifically, non-male identifying individuals contribute 40.3\% ($P-value<0.01$) less than men in the homogeneous gender group and 45.1\% ($P-value<0.01$) less than men in the gender diverse team. The inclusion of controls for age, level of education and political affiliation controls in column 2 do not affect the direction and significance of the coefficients of cooperation. Further controls of minority status (such as gender minority status and racial minority status) and the level of interactions of the individual in the team do not affect the significance level of the non-male identifying individuals. Additionally, positive communication, as assessed by the video interactions in the team, affects cooperation ($P-value<0.1$). The findings suggest that men cooperate at similar levels in gender diverse and homogeneous gender teams. Non-male identifying individuals are most affected by the type of team they are in. Non-male identifying individuals cooperate less in the homogeneous gender teams. They are even worse cooperators in a gender diverse team. This suggests that in assigning non-male identifying individuals, it is more important to consider environments where they can be more cooperative. It is also worth noting that positive chat with the team is positively related to cooperation rates and is statistically significant.  

\begin{table}[H]
 \captionsetup{justification=raggedright,singlelinecheck=false}
\caption{Incumbent Team Diversity and Cooperation by Gender}
        \begin{table}[htbp]
    \begin{tabular}{c c c c}
    \toprule
    \textbf{Variables} & \textbf{(1)} & \textbf{(2)} & \textbf{(3)}         \\ 
\midrule
Diverse$-$Gender            &     4.77    &    5.69                               &  3.74  \\
                            &     (5.32)  &    (5.22)                             &  (5.72) \\

Notmale                  &      -15.37\sym{**} &    -15.23\sym{**}             &  -10.79  \\
                            &      (6.11)         &    (6.16)                     &  (7.06) \\
Diverse*Notmale         &      -29.66\sym{***}   &  -30.10\sym{***}            &  -22.77\sym{**}\\
                            &      (8.29)         &    (8.31)                     &  (9.75) \\


Positive Chat                        &                     &                       &  12.85\sym{*}  \\
                                 &                     &                           &  (7.40)  \\
Negative  Chat                       &                     &                       &  -0.84  \\
                                 &                     &                           &  (4.13)  \\
Engagement Chat                      &                     &                       &  -7.45  \\
                                 &                     &                           &  (5.34)  \\
\midrule
Other Controls                   &    No               &    Yes                    &    Yes        \\
Status Controls                    &    No               &    No                   &    Yes        \\
Chat                             &    No               &    No                     &    Yes        \\
\midrule
Number of Participants           &    333               &    333                   &    324        \\
\midrule
Observations                     &       1665          &       1665                &  1620    \\
\bottomrule
\end{tabular}
\begin{footnotesize}
\newline
*P$<$0.1, **P$<$0.05, ***P$<$0.01
\newline
Note: Robust standard errors clustered at the group level. Marginal effects of Tobit Model reported.
\newline
The dependent variable is the contributions toward the group in public goods provision in Part I. \\
\end{footnotesize}
\end{table}


\end{table}

\noindent \textbf{\textit{Racial Diversity}} 

\noindent Next, I explore the impact of the incumbent team's racial diversity on the cooperation rates of white and non-white individuals. The results derived from the model specified in equation 4 above in the \hyperref[subsec:Specification]{empirical specification} section are presented in table 4 below. Column 1 shows the base model of cooperation of white and non-white individuals in the teams of varying racial diversity. The omitted group of individuals is a white individual in a homogeneous all-white team. The results show that, whites in the racially diverse teams cooperate more than whites in the homogeneous racial team ($P-value<0.1$). However, the cooperation of non-white individuals in the racially diverse teams is the opposite. They cooperate at lower levels equivalent to 46.9\% (9.9-25-31.9) of their endowment less as compared to whites in a homogeneous racial team ($P-value<0.01$). Importantly, the inclusion of additional controls for age, level of education and political affiliation in column 2 do not affect the significance and direction of the coefficients. However, it is important to highlight that positive interactions within the group has a statistically significant effect on cooperation ($P-value<0.01$). This further enforces the findings in the previous paragraph. 

% In fact, a one standard deviation increase in positive communication is associated with more than 100\% increase in cooperation. Addition of controls for risk seeking behavior and interactions with the team does not diminish the statistical significance of the effects.
% \begin{center}
\begin{table}[H]
\captionsetup{justification=raggedright,singlelinecheck=false}
\caption{Incumbent Team Diversity and Cooperation by Race}
    
        \begin{table}[htbp]
\begin{left}
    
    \begin{tabular}{c c c c}
    \toprule
    \textbf{Variables} & \textbf{(1)} & \textbf{(2)} & \textbf{(3)}      \\ 
\midrule
Diverse$-$Race        &     9.89\sym{*}    &    8.89\sym{*}                   &  7.56 \\
                            &     (5.29)          &     (5.27)                &  (5.66) \\
Notwhite                  &      -24.99\sym{***}  &    -23.44\sym{***}      &  -24.74\sym{***}     \\
                            &      (5.46)         &    (6.67)          &  (6.77) \\
Diverse$-$Race*Notwhite    &      -31.78\sym{***}   &  -31.77\sym{***}      &  -32.28\sym{***}\\
                            &      (6.19)         &    (7.01)              &  (7.65) \\

Positive Chat                        &                     &                       &  13.31\sym{***}  \\
                                 &                     &                           &  (7.59)  \\
Negative  Chat                       &                     &                       &  -0.37  \\
                                 &                     &                           &  (4.06)  \\
Engagement Chat                      &                     &                       &  -8.36  \\
                                 &                     &                           &  (5.35)  \\
\midrule
Other Controls                   &    No               &    Yes                    &    Yes        \\
Status Controls                    &    No               &    No                   &    Yes        \\
Chat                             &    No               &    No                     &    Yes        \\
\midrule
Number of Participants           &    333               &    333                   &    324        \\
\midrule
Observations                     &       1665          &       1665                &  1620    \\
\bottomrule

\end{tabular}
\begin{footnotesize}
\newline
*P$<$0.1, **P$<$0.05, ***P$<$0.01
\newline
Note: Robust standard errors clustered at the group level. Marginal effects of Tobit Model reported.
\newline
The dependent variable is the contributions toward the group in public goods provision in Part I.\end{footnotesize}
\end{left}

\end{table}
    
\end{table}
% \end{center}

\noindent\textbf{\textit{Gender and Racial Diversity}} 

\noindent Finally, I proceed to assess how the incumbent team gender and racial diversity influences individuals of various racial and gender backgrounds. To achieve this, I employ equation 5 specified above in the \hyperref[subsec:Specification]{empirical specification} section. The outcomes of this analysis are presented in Table 5 below. Column 1 shows the base model without additional controls. As mentioned above, teams are classified based on the levels of racial and gender diversity. Teams that have diverse gender or racial composition are considered moderately diverse. Teams that have diverse gender and diverse racial compositions are considered most diverse. Finally, individuals in homogeneous racial and homogeneous gender teams are considered to be in the least diverse team. The omitted group of individuals in the analysis presented in table 5 are white men in the least diverse team. The findings show that white men in the more diverse teams cooperate more than white men in the least diverse team. They contribute on average 12.6\% and 10.5\% of their endowment more in the moderately diverse team and the most diverse teams respectively ($P-value<0.05$). Furthermore, non-white individuals in the least diverse teams cooperate less than white men in the least diverse team. The contributions of non-male identifying white individuals in the least diverse team is 14.8\% of their endowment less ($P-value<0.05$). In the moderately diverse teams, white individuals who are not male contribute 28.2\% of their endowment less toward the group ($P-value<0.01$). Non-white males contribute 47.6\% of their endowment less on average than white men in the least diverse team ($P-value<0.01$) and non-white individuals who are not male also contribute 80.7\% of their total endowment less ($P-value<0.01$). In the most diverse teams, non-male identifying whites contribute on average 45.3\% less of their endowment on average ($P-value<0.01$) than whites in the least diverse teams, non-white males contribute 66.4\% of their endowment less ($P-value<0.01$) and non-white individuals who are not male contribute about 86.8\% less on average as compared to white men in the least diverse team ($P-value<0.01$). The results underscore that the level of cooperation exhibited by individuals varies according to the diversity of the team they belong to. Not-white individuals are most affected by racial diversity of the team they are a part of while non-male identifying individuals are most affected by the gender diversity of the team they join. When we consider the dual racial and gender identities of an individual, white men tend to cooperate in more diverse teams. This observation leads to my initial finding: 

\textbf{Result 1a: Cooperation is lower among minorities and non-male identifying individuals but white men tend to cooperate more in diverse teams. }

% \begin{landscape}
\begin{table}[H]
 \captionsetup{justification=raggedright,singlelinecheck=false}
\caption{Incumbent Team Diversity and Cooperation by Gender and Race} \label{table:3c}
    \begin{center}
        \begin{table}[htbp]
    \begin{tabular}{c c c c}
    \toprule
    \textbf{Variables} & \textbf{(1)} & \textbf{(2)} & \textbf{(3)}       \\ 
\midrule
Moderately$-$Diverse        &     12.58\sym{**}    &    12.46\sym{**}     &  11.86\sym{**}  \\
                            &     (6.69)          &     (5.69)             &  (6.16) \\
Most$-$Diverse              &      10.53\sym{**}             &     9.77                & 7.31   \\
                            &      (6.36)         &     (6.63)           &  (7.57)  \\
White$-$Notmale          &      -14.77\sym{**}        &    -14.74\sym{**}    &  -10.86  \\
                            &      (6.86)         &    (6.89)             & (7.26)  \\
NonwhiteMale         &       -20.85\sym{**}   &  -19.44\sym{**}      &  -20.13\sym{**}\\
                            &      (7.67)         &    (8.83)             &  (9.32) \\
NonwhiteNotmale    &  -39.13\sym{***}          &  37.58\sym{***}               &  -37.32\sym{***} \\
                            &      (6.47)         &    (7.52)              &   (9.13) \\
Moderate*WhiteNotmale    &   -26.00 \sym{***} & -25.01\sym{**}       &  -20.23 \sym{***} \\
                                 &  (11.24)        &   (7.15)             &  (10.34)  \\
Moderate*NonwhiteMale      &   -39.37\sym{***} &  -39.93\sym{***}     &  -37.63\sym{***}  \\
                                 &  (12.68)        &     (12.56)             &  (13.32)   \\
Moderate*NonwhiteNotmale  &  -54.12\sym{***} &     -56.96\sym{***}     &  -52.93\sym{***} \\
                                 &  (10.96)        &     (11.15)             &  (12.83)   \\
Most*WhiteNotmale          &  -41.07\sym{***} &     -41.10\sym{***}     &  -34.55\sym{***}  \\
                                 &  (15.35)        &     (15.46)             &  (16.55)  \\
Most*NonwhiteMale         &   -56.11\sym{***} &     -56.04\sym{***}     &   -60.82\sym{***}  \\
                                 &  (17.61)        &     (17.82)             &  (19.21) \\
Most*NonwhiteNotmale    &   -58.15\sym{***} &   -60.51\sym{***}     &  -54.07\sym{***}  \\
                                 &  (13.09)        &     (12.64)             &  (16.00)   \\
Positive Chat                        &                     &                       &  13.66\sym{*}  \\
                                 &                     &                           &  (7.44)  \\
Negative  Chat                       &                     &                       &  0.13  \\
                                 &                     &                           &  (3.98)  \\
Engagement Chat                      &                     &                       &  -7.73  \\
                                 &                     &                           &  (5.18)  \\
\midrule
Other Controls                   &    No               &    Yes                    &    Yes        \\
Status Controls                    &    No               &    No                   &    Yes        \\
Chat                             &    No               &    No                     &    Yes        \\
\midrule
Number of Participants           &    333               &    333                   &    324        \\
\midrule
Observations                     &       1665          &       1665                &  1620    \\
\bottomrule

\end{tabular}
\begin{footnotesize}
\newline
*P$<$0.1, **P$<$0.05, ***P$<$0.01
\newline
Note: Robust standard errors clustered at the group level. Marginal effects of Tobit Model reported. 
\newline
The dependent variable is the contributions toward the group in public goods provision in Part I.
\end{footnotesize}
\end{table}
    \end{center}
\end{table}
% \end{landscape}

\hspace  *{0mm} Moving forward, I delve into the cooperation of incumbents and newcomers in the public goods provision. First, I plot of cooperation rates by levels of diversity in figure 2 below. Dimensions of diversity in the team is based on the classification from the previous paragraph. I combine the racial and gender diversity of the teams as described above into least diverse, moderately diverse and most diverse. Notably, incumbents consistently exhibit higher levels of cooperation across teams of varying degrees of diversity. This is corroborated by the results of the tobit regression analysis presented in tables 6 and 7 below. It is evident that newcomers generally cooperate less than incumbents over the 10 rounds. In table 6, I analyze cooperation rates by considering the gender of individuals using equation 6a specified above in the \hyperref[subsec:Specification]{empirical specification} section. The results in the base model show that male newcomers contribute on average 16.8\% ($P-value<0.01$) of their endowment less toward the group than incumbent males. Furthermore, non-male identifying incumbents are less cooperative and contribute 8.3\% of their endowment less than incumbent males ($P-value<0.1$). The results are robust to the inclusion of additional controls in columns 2 and 3.

In table 7, Column 1, I present the cooperation rates of  newcomers and incumbents by racial identity using equation 6b specified above in the \hyperref[subsec:Specification]{empirical specification} section. It is evident that White newcomers are less cooperative than white incumbents. However, non-white individuals are generally less cooperative as compared to incumbents in the team. Non-white incumbents contribute 17.7\% less of their endowment toward the group as compared to incumbent whites ($P-value<0.01$). Similarly, non-white newcomers contribute 9.7\% less on average as compared to white incumbents but the effect is statistically insignificant in the base model. The coefficients are lower and statistically significant at the 10\% significance level once additional controls are included in columns 2 and 3 of table 7. Furthermore, the coefficients for a newcomer and not-white individuals are robust to the inclusions of controls in columns 2 and 3.  Note that this analysis is examining actions toward the group during the 10 rounds of the public goods provision activity, five of which, the newcomers are not part of the group. \footnote{I later examine whether newcomers' contributions change after joining a team.} Newcomers cooperate less than incumbents. However, gender identity of the newcomer does not affect cooperation of newcomers but the race of an individual play a role in cooperation in the team. These insights lead to my second result.

\textbf{Result 2a:Incumbents cooperate more than newcomers during the duration of the study. }

\begin{figure}[H]
 \captionsetup{justification=raggedright,singlelinecheck=false}
\caption{Incumbent Versus Newcomer Overall Cooperation}
\includegraphics[scale=0.2]{Figures/Overall_ppg_new_inc.png} 
\end{figure}

\begin{table}[H]
 \captionsetup{justification=raggedright,singlelinecheck=false}
\caption{Incumbent Versus Newcomer Overall Cooperation  by Gender} \label{tab:table4}
    \begin{center}
        \begin{table}[htbp]
    \begin{tabular}{c c c c}
    \toprule
    \textbf{Variables} & \textbf{(1)} & \textbf{(2)} & \textbf{(3)}      \\ 
\midrule
Newcomer                         &     -16.81\sym{***} &     -15.63\sym{***}  &  -14.69\sym{***}   \\
                                 &     (4.51)          &     (4.74)          &  (5.10)             \\
Notmale                         &        -8.33\sym{*} &     -9.04\sym{*}  &  -7.56     \\
                                 &     (5.22)          &     (5.22)          &  (5.55)             \\
Newcomer*Notmale                  &     5.56         &  3.22               &  3.75    \\
                                 &     (6.51)          &   (6.46)          &  (6.39)             \\


\midrule
Other Controls                   &    No               &    Yes              &    Yes            \\
Status Controls                    &    No               &    No               &    Yes           \\
Chat                             &    No               &    No               &    No             \\
\midrule
Number of Participants           &    444               &    444              &    444               \\
\midrule
Observations                     &       4440          &       4440          &  4440            \\
\bottomrule

\end{tabular}
\begin{footnotesize}
\newline
*P$<$0.1, **P$<$0.05, ***P$<$0.01
\newline
Note: Robust standard errors clustered at the group level. Marginal effects of Tobit Model reported.
\newline
The dependent variable is the contributions toward the group in public goods provision in both Parts.
\end{footnotesize}
\end{table}
    \end{center}
\end{table}


\begin{table}[H]
 \captionsetup{justification=raggedright,singlelinecheck=false}
\caption{Incumbent Versus Newcomer Overall Cooperation by Race} \label{tab:table4}
    \begin{center}
        \begin{table}[htbp]
    \begin{tabular}{c c c c}
    \toprule
    \textbf{Variables} & \textbf{(1)} & \textbf{(2)} & \textbf{(3)}       \\ 
\midrule
Newcomer                         &     -14.72\sym{***} &     -13.59\sym{***}  &  -10.92\sym{*}   \\
                                 &     (4.58)          &     (4.75)          &  (5.13)             \\
Notwhite                         &  -17.70\sym{***} &     -19.07\sym{***}  &  -19.56\sym{***}   \\
                                 &     (4.78)          &     (5.68)          &  (6.11)             \\
Newcomer*Notwhite              &     -9.65              &     -11.35\sym{*}  &  -11.68\sym{*}   \\
                                 &     (6.45)          &     (7.09)          &  (7.09)             \\


\midrule
Other Controls                   &    No               &    Yes              &    Yes            \\
Status Controls                    &    No               &    No               &    Yes           \\
Chat                             &    No               &    No               &    No             \\
\midrule
Number of Participants           &    444               &    444              &    444               \\
\midrule
Observations                     &       4440          &       4440          &  4440            \\
\bottomrule

\end{tabular}
\begin{footnotesize}
\newline
*P$<$0.1, **P$<$0.05, ***P$<$0.01
\newline
Note: Robust standard errors clustered at the group level. Marginal effects of Tobit Model reported. 
\newline
The dependent variable is the contributions toward the group in public goods provision in both Parts. 
\end{footnotesize}
\end{table}
    \end{center}
\end{table}


\hspace  *{0mm} I now shift focus to the cooperation of individuals, both prior to the inclusion of newcomers and after their integration. I established in result 2a that newcomers cooperate less than incumbents over the span of 10 rounds. In table 8, column 1, I compare cooperation of incumbents and newcomers before and after group composition changes using the model specified in equation 7 above in the \hyperref[subsec:Specification]{empirical specification} section. I find that the newcomers primarily cooperate less in Part I when they participate in the public goods provision without a team-building activity ($P-value<0.01$). However, newcomers substantially increase their cooperation rates later in the study ($P-value<0.01$). Furthermore, incumbents statistically significantly increase their cooperation overall after the changes in group composition but the increase is lower than that of the newcomers ($P-value<0.01$). This increase in newcomer cooperation compensates for the low cooperation prior to their integration, leading to parity in cooperation between the newcomers and the base treatment group---the incumbents---before team composition changes ($P-value=0.91$). The results are robust to the inclusion of additional controls in columns 2 and 3. Newcomers cooperate more in teams and incumbents increase cooperation after a newcomer joins. These changes lead to my third primary result: 

\textbf{Result 3a: Incumbents and newcomers increase their levels of cooperation after changes in the team composition but the increase in cooperation is higher among newcomers.}

\begin{table}[H]
 \captionsetup{justification=raggedright,singlelinecheck=false}
\caption{Incumbent and Newcomer Cooperation Before and After Group Composition Changes} \label{tab:table5}
    \begin{center}
        \begin{table}[htbp]
    \begin{tabular}{c c c c}
    \toprule
    \textbf{Variables} & \textbf{(1)} & \textbf{(2)} & \textbf{(3)}       \\ 
\midrule
Newcomer(Before=1)               &     -26.61\sym{***}    &    -25.12\sym{***}  &  -22.46\sym{***}   \\
                                 &     (4.56)             &     (4.87)          &  (5.44)         \\
\addlinespace
Newcomer(After=1)                &     -0.64           &     0.80         &  3.27          \\
                                 &     (5.69)          &     (5.80)        &  (6.22)         \\
\addlinespace
Incumbent(After=1)               &     5.90\sym{**}    &     5.87\sym{**}  &  5.74\sym{**}     \\
                                 &     (2.91)          &     (2.89)        &  (2.88)         \\
\midrule
Other Controls                   &    No               &    Yes              &    Yes             \\
Status Controls                    &    No               &    No               &    Yes             \\
Chat                             &    No               &    No               &    No               \\
\midrule
Number of Participants           &    444               &    444              &    444               \\
\midrule
Observations                     &       4440          &       4440          &  4440          \\
\bottomrule

\end{tabular}
\begin{footnotesize}
\newline
*P$<$0.1, **P$<$0.05, ***P$<$0.01
\newline
Note: Robust standard errors clustered at the group level. Marginal effects of Tobit Model reported. 
\newline
The dependent variable is the contributions toward the group in public goods provision in both Parts.\end{footnotesize}
\end{table}
    \end{center}
\end{table}




\hspace  *{0mm} Moving forward, my analysis delves into the distinctive behaviors displayed by different newcomers following their assimilation into the team using the model specified in equation 8a above in the \hyperref[subsec:Specification]{empirical specification} section. First, I consider combined effects of racial and gender diversity on newcomer cooperation. As previously stated, I classify newcomers into 3 groups along the lines of the racial and gender diversity of the team they join. The base treatment is newcomers that join the least diverse team, teams that are homogeneous on both gender and racial compositions. Newcomers to moderately diverse teams join teams of heterogeneous gender composition or heterogeneous racial composition. Finally, newcomers that join heterogeneous racial and heterogeneous gender teams are classified as joining the most diverse teams. The results of the analysis are shown in table 9. The analysis shows that newcomers that join moderately diverse teams contribute on average 17.7\% more of their endowment towards the group account as compared to newcomers that join the least diverse incumbent teams ($P-value<0.1$). Additionally, newcomers that join the most diverse teams do not cooperate at a level different from newcomers that join the least diverse teams ($P-value=0.60$). Further analysis of the results as shown in the appendix reveals that the higher cooperation in moderately diverse teams is driven by white newcomers. White newcomers to the least diverse teams significantly reduce their cooperation when their gender is in-congruent with the rest of the team. 

 Further exploration of the impact of gender diversity on newcomer cooperation by newcomer gender is presented in table 10 using equation 8b specified above in the \hyperref[subsec:Specification]{empirical specification} section. The results do not show any statistically significant effect of gender diversity on newcomer behavior by newcomer gender. In addition to this, I consider newcomers by race in table 11 using the tobit model specified in equation 8c above in the empirical specification section. The results show that non-white newcomers are most affected by the diversity of the team they join. Non-white newcomers cooperate less than whites when they join a homogeneous racial team ($P-value<0.1$) but cooperate even less in the diverse racial teams as compared to whites in the racially homogeneous teams ($P-value<0.05$). This shows that the racial diversity of the team does not affect cooperation of the white newcomers ($P-value=0.45$) overall except for cases where a white newcomer joins a homogeneous gender team, in-congruent with their gender. This is particularly important given that whites are the majority in the labor force of the United States\citep{b21}. It is noteworthy that the demographic distribution of our participants mean racially homogeneous groups are mostly groups of homogeneous all-white teams. Interestingly, non-white newcomers who join racially diverse teams cooperate at level that is 28.9\% of their endowment lower than white newcomers in homogeneous racial teams ($P-value<0.01$). The lower cooperation of non-white newcomers is robust to additional controls and the chat controls of participants in columns 2 and 3 respectively in table 11. The results presented here and further analysis \footnote{Further analysis of behavior of newcomers is presented in Appendix A.} lead to my fourth finding.   

\textbf{Result 4a: Newcomers are sensitive to the existing team diversity but the nature of cooperation is dependent on racial identity of the newcomer.}


\begin{table}[H]
 \captionsetup{justification=raggedright,singlelinecheck=false}
\caption{Incumbent Team Diversity and Newcomer Cooperation } \label{tab:table6}
    \begin{center}
        \begin{table}[htbp]
    \begin{tabular}{c c c c}
    \toprule
    \textbf{Variables} & \textbf{(1)} & \textbf{(2)} & \textbf{(3)}     \\ 
\midrule
Moderate$-$Diverse  &  17.71\sym{*} &  11.71            &  12.97     \\
                    &  (9.42)       &  (8.95)           &  (8.94)    \\
\addlinespace
Most$-$Diverse      &  5.86       &   3.05                &  3.95     \\
                    &  (11.21)      & (10.80)             &  (10.77)     \ \\

\addlinespace
Positive                 &           &                     &  -2.56     \\
                         &           &                     &  (7.81)   \\
\addlinespace
Negative                 &           &                      &  -1.19  \\
                          &          &                      &  (4.68)    \\
\addlinespace
Engagement               &          &                        & 5.29    \\
                         &          &                          &  (4.55)   \\

\midrule
Other Controls   &   No &  Yes &    Yes    \\
Status Controls &   No  &    No    &    Yes    \\
Chat          &    No    &    No   &    Yes          \\
\midrule
Number of Participants & 111   &    111 &    106    \\
\midrule
Observations          &   555   &  555  &  530         \\
\bottomrule

\end{tabular}
\begin{footnotesize}
\newline
*P$<$0.1, **P$<$0.05, ***P$<$0.01
\newline
Note: Robust standard errors clustered at the group level. Marginal effects of Tobit Model reported. 
\newline
The dependent variable is the contributions toward the group in public goods provision in Part II.
\end{footnotesize}
\end{table}

    \end{center}
\end{table}

\begin{table}[H]
\caption{Incumbent Team Diversity and Newcomer Cooperation by Gender } \label{tab:table6}
    \begin{center}
        \begin{table}[htbp]
    \begin{tabular}{c c c c}
    \toprule
    \textbf{Variables} & \textbf{(1)} & \textbf{(2)} & \textbf{(3)}       \\ 
\midrule
Diverse$-$Gender            &     0.55    &    0.34                               &  1.68  \\
                            &     (8.80)  &    (8.45)                             &  (8.63) \\

Notmale                  &      -2.39  &    -6.32                            &  -6.19  \\
                            &      (8.41)  &    (8.06)                        &  (8.63) \\
Diverse*Notmale         &      -11.37   &  -14.08                            &  -16.90 \\
                            &      (14.58)  &    (13.66)                     &  (15.59) \\


Positive Chat                        &                     &                       &  -0.32  \\
                                 &                     &                           &  (7.78)  \\
Negative  Chat                       &                     &                       &  -2.46  \\
                                 &                     &                           &  (4.81)  \\
Engagement Chat                      &                     &                       &  6.68  \\
                                 &                     &                           &  (4.33)  \\
\midrule
Other Controls   &   No &  Yes &    Yes    \\
Status Controls &   No  &    No    &    Yes    \\
Chat          &    No    &    No   &    Yes          \\
\midrule
Number of Participants & 111   &    111 &    106    \\
\midrule
Observations          &   555   &  555  &  530         \\
\bottomrule

\end{tabular}
\begin{footnotesize}
\newline
*P$<$0.1, **P$<$0.05, ***P$<$0.01
\newline
Note: Robust standard errors clustered at the group level. Marginal effects of Tobit Model reported.
\newline
The dependent variable is the contributions toward the group in public goods provision in Part II.\end{footnotesize}
\end{table}

    \end{center}
\end{table}


\begin{table}[H]
 \captionsetup{justification=raggedright,singlelinecheck=false}
\caption{Incumbent Team Diversity and Newcomer Cooperation by Race } \label{tab:table6}
    \begin{center}
        \begin{table}[htbp]
    \begin{tabular}{c c c c}
    \toprule
    \textbf{Variables} & \textbf{(1)} & \textbf{(2)} & \textbf{(3)}      \\ 
\midrule
Diverse$-$Race            &     5.03     &    0.12                               &  -0.12  \\
                            &   (8.71)   &    (8.56)                             &  (8.34) \\

Notwhite                  &      -15.63\sym{*}  &    -20.61\sym{**}                 &  -20.27\sym{**}   \\
                            &      (8.38)          &    (9.35)                        &  (9.43) \\
Diverse*Notwhite         &      -18.07\sym{*}   &  -23.64\sym{**}                 &  -22.79\sym{*}  \\
                            &      (10.07)         &    (11.21)                     &  (12.09) \\


Positive Chat                        &                     &                       &  -4.24  \\
                                 &                     &                           &  (7.93)  \\
Negative  Chat                       &                     &                       &  -2.53  \\
                                 &                     &                           &  (4.54)  \\
Engagement Chat                      &                     &                       &  6.43  \\
                                 &                     &                           &  (4.34)  \\
\midrule
Other Controls   &   No &  Yes &    Yes    \\
Status Controls &   No  &    No    &    Yes    \\
Chat          &    No    &    No   &    Yes          \\
\midrule
Number of Participants & 111   &    111 &    106    \\
\midrule
Observations          &   555   &  555  &  530         \\
\bottomrule
\end{tabular}

\begin{footnotesize}
\newline
*P$<$0.1, **P$<$0.05, ***P$<$0.01
\newline
Note: Robust standard errors clustered at the group level. Marginal effects of Tobit Model reported.
\newline
The dependent variable is the contributions toward the group in public goods provision in Part II. \end{footnotesize}
\end{table}

    \end{center}
\end{table}





\hspace  *{0mm} Finally, I look at how incumbents’ contributions are impacted conditional on the identity of the incumbent and the newcomer. First, participants are classified by whether they are whites or minorities. Consequently, incumbents are grouped by whether they share identities with the newcomers. Incumbents of the same gender as the newcomers are considered as sharing gender identity. Non-white Minorities are considered to share the same racial identity as other minorities and whites are considered as sharing the same racial identity. This results in 4 groups of incumbents – those that share race and gender with the newcomer, those that share race but not gender with the newcomer, incumbents that gender but not race with the newcomer and incumbents that do not share race and gender with the newcomer. The tobit regression model specified in equation 9 above in the \hyperref[subsec:Specification]{empirical specification} section is used to estimate cooperation among incumbents in the public goods provision. Table 12 presents the results of the regression estimates. Column 1 presents the base model without controls and considers incumbents that do not share race and gender with the newcomer as the omitted category. It is evident that incumbents do not vary their cooperation conditional on the identities of the newcomer. The evidence shows that while incumbent individuals are on average more cooperative when they share the same race with a newcomer but not gender as compared to incumbents that do not share both race and gender, the coefficient is not statistically significant ($P-value=0.72$).

On the contrary, the average cooperation in the public goods provision activity is negative among incumbents that share the same gender as the newcomer but not the same race. However, the effect is also not statistically significant as compared to the omitted group ($P-value=0.57$). Finally, there is no statistically significant difference in the cooperation of incumbents that do not share the same race and gender and incumbents that do share both identities with the newcomer ($P-value=0.99$). These findings lead me to my fifth primary result:

\textbf{Result 5a: The identity of the newcomer does not impact the cooperation of incumbent members of the team.}
\begin{table}[H]
 \captionsetup{justification=raggedright,singlelinecheck=false}
\caption{Incumbent Cooperation and Newcomer Identity } \label{tab:table7}
    \begin{center}
        \begin{table}[htbp]
    \begin{tabular}{c c c c c}
    \toprule
    \textbf{Variables} & \textbf{(1)} & \textbf{(2)} & \textbf{(3)}      & \textbf{(4)}     \\ 
\midrule
Congruentgender           &     -4.20              &     -3.61          &  -2.88            &  -3.24 \\
                                            &     (7.41)             &     (7.57)         &  (7.55)           &  (8.33) \\
\addlinespace
Congruentrace           &      2.68              &     4.17           &   3.30            &  3.35   \\
                                            &      (6.44)            &     (6.76)         &   (6.80)          &  (7.27) \\
\addlinespace
Congruentboth                 &      -0.06              &   -0.22           &  0.47             &  0.17  \\
                                            &      (6.92)            &    (7.07)          &  (7.03)           &  (7.24) \\
\addlinespace
Positive                         &                     &                     &                    &  1.02  \\
                                 &                     &                     &                    &  (4.67)  \\
\addlinespace
Negative                         &                     &                     &                    &  -5.41  \\
                                 &                     &                     &                    &  (3.53)  \\
\addlinespace
Engagement                       &                     &                     &                    &  9.68\sym{**}  \\
                                 &                     &                     &                    &  (3.70)  \\

\midrule
Other Controls                   &    No                &    Yes              &    Yes          &    Yes        \\
Status Controls                    &    No                &    No               &    Yes          &    Yes        \\
Chat                             &    No                &    No               &    No           &    Yes        \\
\midrule
Number of Participants           &    333               &    333              &    333          &    318       \\
\midrule
Observations                     &       1665           &       1665          &  1665           &  1590    \\
\bottomrule


\end{tabular}

\begin{footnotesize}
\newline
*P$<$0.1, **P$<$0.05, ***P$<$0.01
\newline
Note: Robust standard errors clustered at the group level. Marginal effects of Tobit Model reported. 
\newline
The dependent variable is the contributions toward the group in public goods provision in Part II. \end{footnotesize}
\end{table}
    \end{center}
\end{table}

\subsection{Coordination in the Minimum Effort Activity}

\noindent To explore coordination choices in the study, I analyze the participants' hours contributed to the group activity in the minimum effort game. At the individual level, participants are tasked with selecting their preferred coordination choice in a group activity. The choices range from 10 hours to 70 hours. There are multiple equilibria in this activity. Team members coordinating at a higher level is pareto optimal and is beneficial to the team. The choices are analyzed as coordination behavior in the team. In each round, participants must determine the number of hours they are willing to dedicate to the group activity. Importantly, participants make their decisions with the knowledge that their individual payoffs are contingent upon the choices made by other team members. The determination of individual payoffs is calculated using equation 2 specified above in the \hyperref[subsec:Design]{experimental design} section.

\noindent\textbf{\textit{Gender Diversity}} 

\noindent Similar to the analysis of the cooperation of individuals in the teams, I initiate my analysis of team coordination by first examining incumbent teams. Incumbent teams (three-person teams) are classified by levels of diversity. Teams are classified based on gender diversity, racial diversity or both. As stated previously, gender diverse teams are non-homogeneous gender teams that have at least two individuals with different gender identities. I begin the analysis by looking at the impact of the level of gender diversity of the incumbent team on behavior across gender using the tobit regression model specified in equation 3. The results of the analyses are presented in table 13 below. Column 1 shows the base model and the omitted group is a man in a homogeneous gender team. The findings show that coordination choices of men do not vary by the level of gender diversity in the team ($P-value=0.90$). The situation is different for non-male identifying individuals. They coordinate at choices that are on average 12.2\% of the total hours possible lower than men in the homogeneous gender team ($P-value<0.05$). Furthermore, non-male identifying individuals in the diverse gender teams coordinate at choices that are 30.5\% lower than men in the homogeneous gender teams ($P-value<0.01$) and even lower compared to men in the gender diverse teams ($P-value<0.01$). The results are robust to the addition of controls in column 2. Additional controls for interactions among participants show a decrease in the coefficients but the findings are still robust. It is also evident that positive interactions with the team increase coordination choices ($P-value<0.1$). 

\begin{table}[H]
 \captionsetup{justification=raggedright,singlelinecheck=false}
\caption{Incumbent Team Diversity and Coordination by Gender } \label{tab:table8}
    \begin{center}
        \begin{table}[htbp]
    \begin{tabular}{c c c c}
    \toprule
    \textbf{Variables} & \textbf{(1)} & \textbf{(2)} & \textbf{(3)}         \\ 
\midrule
Diverse$-$Gender            &     0.42    &    0.35                               &  -0.25  \\
                            &     (3.39)  &    (3.32)                             &  (3.68) \\

Notmale                  &      -8.55\sym{**} &    -8.86\sym{**}             &  -6.87\sym{*}  \\
                            &      (3.69)         &    (3.67)                     &  (3.77) \\
Diverse*Notmale         &      -13.20\sym{***}   &  -13.93\sym{***}            &  -12.00\sym{**}\\
                            &      (4.90)         &    (4.90)                     &  (5.10) \\


Positive Chat                        &                     &                       &  7.19\sym{*}  \\
                                 &                     &                           &  (4.08)  \\
Negative  Chat                       &                     &                       &  -0.82  \\
                                 &                     &                           &  (2.05)  \\
Engagement Chat                      &                     &                       &  -3.48  \\
                                 &                     &                           &  (2.69)  \\
\midrule
Other Controls                   &    No               &    Yes                    &    Yes        \\
Status Controls                    &    No               &    No                   &    Yes        \\
Chat                             &    No               &    No                     &    Yes        \\
\midrule
Number of Participants           &    333               &    333                   &    324        \\
\midrule
Observations                     &       1665          &       1665                &  1620    \\
\bottomrule
\end{tabular}
\begin{footnotesize}
\newline
*P$<$0.1, **P$<$0.05, ***P$<$0.01
\newline
Note: Robust standard errors clustered at the group level. Marginal effects of Tobit Model reported.
\newline
The dependent variable is the hours toward the group activity in the minimum effort activity in Part I. 
\end{footnotesize}
\end{table}
    \end{center}
\end{table}

\noindent\textbf{\textit{Racial Diversity}} 

\noindent Now, I consider how the level of racial diversity of the incumbent team impact behavior across race using equation 4 above in the \hyperref[subsec:Specification]{empirical specification} section. Individuals are grouped as white or non-white (minorities) based on their self-identified race. The results of the analysis are presented in table 14 below. Column 1 shows the base model and the omitted group is a white individual in the homogeneous racial team. The findings show that coordination choices of whites in the racially diverse team are higher than whites in the racially homogeneous team ($P-value<0.05$). The addition of controls in column 2 do not affect the significance of the result. Furthermore, non-white individuals coordinate at lower choices in  the homogeneous racial teams than whites in the homogeneous racial teams ($P-value<0.01$). Similarly, non-white individuals in racially diverse teams coordinate at a level that is 41.4\% of the total hours possible lower than whites in the homogeneous racial teams ($P-value<0.01$). Further controls for age, major, education level and political affiliation do not affect the statistical significance of the results. Further controls for interactions among teams members show that positive chat is important for higher coordination choices ($P-value<0.1$). However, the difference in coordination between whites in the diverse racial groups and whites in the homogeneous racial groups becomes statistically insignificant ($P-value=0.26$). 

\begin{table}[H]
 \captionsetup{justification=raggedright,singlelinecheck=false}
\caption{Incumbent Team Diversity and Coordination by Race } \label{tab:table8}
    \begin{center}
        \begin{table}[htbp]
    \begin{tabular}{c c c c}
    \toprule
    \textbf{Variables} & \textbf{(1)} & \textbf{(2)} & \textbf{(3)}       \\ 
\midrule
Diverse$-$Race        &     6.44\sym{**}                   &    5.62\sym{*}                   &  4.13 \\
                      &     (3.13)                 &     (3.01)                &  (3.65) \\
Notwhite           &      -9.80\sym{***}     &    -8.13\sym{***}      &  -8.52\sym{***}     \\
                      &      (2.77)              &    (3.06)                   &  (3.45) \\
Diverse$-$Race*Notwhite  &      -12.73\sym{***}   &  -12.36\sym{***}      &  -12.90\sym{***}\\
                            &      (3.20)               &    (3.47)              &  (3.92) \\

Positive Chat                        &                     &                       &  7.42\sym{*}  \\
                                 &                     &                           &  (4.33)  \\
Negative  Chat                       &                     &                       &  -0.64  \\
                                 &                     &                           &  (2.14)  \\
Engagement Chat                      &                     &                       &  -3.69  \\
                                 &                     &                           &  (2.74)  \\
\midrule
Other Controls                   &    No               &    Yes                    &    Yes        \\
Status Controls                    &    No               &    No                   &    Yes        \\
Chat                             &    No               &    No                     &    Yes        \\
\midrule
Number of Participants           &    333               &    333                   &    324        \\
\midrule
Observations                     &       1665          &       1665                &  1620    \\
\bottomrule

\end{tabular}
\begin{footnotesize}
\newline
*P$<$0.1, **P$<$0.05, ***P$<$0.01
\newline
Note: Robust standard errors clustered at the group level. Marginal effects of Tobit Model reported. 
\newline
The dependent variable is the hours toward the group activity in the minimum effort activity in Part I.\end{footnotesize}
\end{table}
    \end{center}
\end{table}

 \noindent\textbf{\textit{Gender and Racial Diversity}}

\noindent Finally, I consider the impact of the level of racial and gender diversity of the incumbent team on behavior across race and gender identities of the individual. Teams are classified based on whether there is moderate diversity - at least one dimension of diversity on gender or race, least diverse - no heterogeneity in gender and race and most diverse - there is heterogeneity in race and gender. Individuals are also grouped into four types - white males, non-male identifying whites, non-white males and individuals who are not white and non-male identifying. The regression model specified in equation 5 from above in the \hyperref[subsec:Specification]{empirical specification} section is estimated to analyze the coordination choices of participants. The results of the analysis are presented in table 15 below. Column 1 shows the base model. The omitted group is a white man in the least diverse incumbent team. There is no statistically significant difference between the coordination choices of white men in the least diverse team and white men in the more diverse teams ($P-value=0.70$ for moderately diverse and $P-value=0.32$ for the most diverse). However, other individuals in the least diverse team do coordinate at choices different from those of white men. Non-male identifying whites coordinate at 10.9\% of the total hours possible lower than white men in the least teams ($P-value<0.1$), while non-white males and non-white individuals who are not men coordinate at choices that is 13.7\% ($P-value<0.1$) and 20.0\%($P-value<0.01$) respectively lower than white men in the least diverse teams. Additionally, non-male identifying whites in the most diverse teams coordinate at choices that are 22.8\% of the total hours possible lower than white men in the least diverse team ($P-value<0.1$). The effect is robust to the addition of controls in columns 2 and 3 in table 15. However, positive chat among incumbents increases average coordination choices. These findings collectively lead to my next result:

\textbf{Result 1b: Individuals who are racial minorities coordinate at lower levels in the more diverse teams than whites in the least diverse teams.}

% \begin{landscape}
\begin{table}[H]
 \captionsetup{justification=raggedright,singlelinecheck=false}
\caption{Incumbent Team Diversity and Coordination by Individual Identity } \label{tab:table8}
    \begin{center}
        \begin{table}[htbp]
    \begin{tabular}{c c c c}
    \toprule
    \textbf{Variables} & \textbf{(1)} & \textbf{(2)} & \textbf{(3)}       \\ 
\midrule
Moderately$-$Diverse        &     1.42            &    0.80                 &  -2.28     \\
                            &     (3.63)          &     (3.71)             &  (4.23) \\
Most$-$Diverse              &     4.26             &     3.95                & 2.37   \\
                            &    (4.25)         &     (4.38)           &  (4.74)  \\
WhiteNotmale          &    -7.65\sym{*}        &    -7.75\sym{*}    &  -5.35  \\
                            &      (4.15)         &    (4.21)             & (4.18)  \\
NonwhiteMale         &       -9.59\sym{*}   &  -7.05                &  -6.79   \\
                            &      (5.55)         &    (5.77)             &  (6.32) \\
NonwhiteNotmale    &  -14.05\sym{***}    &  -13.75\sym{***}      &  -13.29\sym{***} \\
                            &      (3.97)         &    (4.43)              &   (4.54) \\
Moderate*WhiteNotmale    &   -5.88          & -5.94               &  -3.86  \\
                                 &  (7.40)        &   (7.38)             &  (7.18)  \\
Moderate*NonwhiteMale      &   -2.00       &  -4.01                &  -3.81  \\
                                 &  (10.20)       &     (9.92)             &  (10.77)   \\
Moderate*NonwhiteNotmale  &  -19.30\sym{***} & -18.77\sym{***}     &  -18.93\sym{**} \\
                                 &  (6.98)        &     (6.94)             &  (7.31)   \\
Most*WhiteNotmale          &  -12.57\sym{*} &     -13.96\sym{***}     &  -13.13\sym{*}   \\
                                 &  (7.86)        &     (7.96)             &  (8.22)  \\
Most*NonwhiteMale         &   -9.40        &     -12.96\sym{***}     &   -13.79  \\
                                 &  (10.47)        &     (10.30)             &  (12.35) \\
Most*NonwhiteNotmale    &   -28.31\sym{***} &   -28.05\sym{***}     &  -32.38\sym{***}  \\
                                 &  (7.75)        &     (7.90)             &  (8.86)   \\
Positive Chat                        &                     &                       &  7.71\sym{*}  \\
                                 &                     &                           &  (4.20)  \\
Negative  Chat                       &                     &                       &  -1.42  \\
                                 &                     &                           &  (2.25)  \\
Engagement Chat                      &                     &                       &  -3.97  \\
                                 &                     &                           &  (2.64)  \\
\midrule
Other Controls                   &    No               &    Yes                    &    Yes        \\
Status Controls                    &    No               &    No                   &    Yes        \\
Chat                             &    No               &    No                     &    Yes        \\
\midrule
Number of Participants           &    333               &    333                   &    324        \\
\midrule
Observations                     &       1665          &       1665                &  1620    \\
\bottomrule

\end{tabular}
\begin{footnotesize}
\newline
*P$<$0.1, **P$<$0.05, ***P$<$0.01
\newline
Note: Robust standard errors clustered at the group level. Marginal effects of Tobit Model reported. 
\newline
The dependent variable is the hours toward the group activity in the minimum effort activity in Part I.\end{footnotesize}
\end{table}
    \end{center}
\end{table}
% \end{landscape}

\hspace  *{0mm} Next, I compare the overall coordination choices between incumbents and newcomers in the entire study. As previously, I plot coordination choices by the levels of racial and ethnic diversity in figure 3 below. The least diverse teams are homogeneous racial and gender teams. Most diverse teams are teams that have heterogeneous race and gender and moderately diverse teams have heterogeneous racial or heterogeneous gender compositions. Notably, newcomers demonstrate a significantly lower level of coordination compared to incumbents during the study in all the teams. The results of the estimation of the tobit model specified in equation 6a in table 16 further emphasizes this. The base model shows that male newcomers coordinate at choices that is 15.9\% of the total hours possible lower than incumbent men - the omitted group ($P-value<0.01$). However, there is no statistically significant difference in coordination choices of men and non-male identifying individuals in the homogeneous gender team ($P-value=0.23$). Furthermore, the lower coordination choices of newcomers are robust to the inclusion of additional controls in columns 2 and 3 of table 16 below.  

Using the specification stated in equation 6b in the specification section above, I present the results of coordination choices by racial identity of the individual and racial composition of the group in table 17. The base model is a white incumbent. The results in column 1 show that not only do newcomers coordinate at lower choices but also choices vary by race. Non-white individuals in the incumbent teams coordinate at choices that is 8.3\% less than white incumbents ($P-value<0.01$). Furthermore, white newcomers coordinate at choices that is 15.3\% of the total possible hours lower than incumbent whites. However, non-white newcomers do not coordinate at a level different from white newcomers ($P-value=0.45$). The results of these findings lead to my next result: 

\textbf{Result 2b: Incumbents make higher coordination choices than newcomers but choices vary by race.}

\begin{figure}[H]
\captionsetup{justification=raggedright,singlelinecheck=false}
\caption{Incumbent Versus Newcomer Overall Coordination Choices}
\includegraphics[scale=0.2]{Figures/Overall_me_new_inc.png} 
\end{figure}

\begin{table}[H]
 \captionsetup{justification=raggedright,singlelinecheck=false}
\caption{Incumbent Versus Newcomer Overall Coordination by Gender} \label{tab:table9}
    \begin{center}
        \begin{table}[htbp]
    \begin{tabular}{c c c c}
    \toprule
    \textbf{Variables} & \textbf{(1)} & \textbf{(2)} & \textbf{(3)}       \\ 
\midrule
Newcomer                         &     -11.10\sym{***} &     -10.69\sym{***}  &  -3.26   \\
                                 &     (2.45)          &     (2.44)          &  (3.06)             \\
Notmale                         &    -3.34            &     -3.88   &  -6.50     \\
                                 &     (2.75)          &     (2.73)          &  (3.13)             \\
Newcomer*Notmale               &     5.71            &  4.34               &  3.80    \\
                                 &     (3.76)          &   (3.84)          &  (5.30)             \\


\midrule
Other Controls                   &    No               &    Yes              &    Yes            \\
Status Controls                    &    No               &    No               &    Yes           \\
Chat                             &    No               &    No               &    No             \\
\midrule
Number of Participants           &    444               &    444              &    444               \\
\midrule
Observations                     &       4440          &       4440          &  4440            \\
\bottomrule

\end{tabular}
\begin{footnotesize}
\newline
*P$<$0.1, **P$<$0.05, ***P$<$0.01
\newline
Note: Robust standard errors clustered at the group level. Marginal effects of Tobit Model reported. 
\newline
The dependent variable is the hours toward the group activity in the minimum effort activity in both Parts. \end{footnotesize}
\end{table}

    \end{center}
\end{table}


\begin{table}[H]
 \captionsetup{justification=raggedright,singlelinecheck=false}
\caption{Incumbent Versus Newcomer Overall Coordination by Race} \label{tab:table9}
    \begin{center}
        \begin{table}[htbp]
    \begin{tabular}{c c c c}
    \toprule
    \textbf{Variables} & \textbf{(1)} & \textbf{(2)} & \textbf{(3)}     \\ 
\midrule
Newcomer                         &     -10.74\sym{***} &     -10.50\sym{***}  &  -1.27       \\
                                 &     (2.54)          &     (2.49)          &  (3.19)             \\
Notwhite                         &  -5.84\sym{***} &     -4.56\sym{*}  &  -6.60\sym{**}   \\
                                 &     (2.16)          &     (2.37)          &  (2.77)             \\
Newcomer*Notwhite              &     -2.79              &     -1.42   &  -4.68   \\
                                 &     (3.71)          &     (3.90)          &  (4.95)             \\


\midrule
Other Controls                   &    No               &    Yes              &    Yes            \\
Status Controls                    &    No               &    No               &    Yes           \\
Chat                             &    No               &    No               &    No             \\
\midrule
Number of Participants           &    444               &    444              &    444               \\
\midrule
Observations                     &       4440          &       4440          &  4440            \\
\bottomrule

\end{tabular}
\begin{footnotesize}
\newline
*P$<$0.1, **P$<$0.05, ***P$<$0.01
\newline
Note: Robust standard errors clustered at the group level. Marginal effects of Tobit Model reported. 
\newline
The dependent variable is the hours toward the group activity in the minimum effort activity in both Parts. 
\end{footnotesize}
\end{table}

    \end{center}
\end{table}


\hspace  *{0mm} Building on the aforementioned findings, I  delve deeper into the coordination choices of incumbents and newcomers. Table 18 compares coordination before and after the newcomer joins the team based on the model specified in equation 7 above in the \hyperref[subsec:Specification]{empirical specification} section. The model considers incumbents in rounds 1 - 5 as the omitted group in the analyses. Interestingly, the coordination choices after group composition changes are lower for newcomers as compared to incumbents before changes in group composition ($P-value<0.01$). Similarly, incumbents decrease their coordination choices after group composition changes ($P-value<0.05$). However, similar to cooperation choices before joining the team, newcomers coordinate on lower levels as compared to the incumbents in rounds 1 to 5 ($P-value<0.01$). This pattern persists when controls for age, political affiliation and educational background are included in column 2. This consequential finding leads to my next result:

\textbf{Result 3b: Incumbents’ and newcomers’ coordination choices change after newcomers join the team.}

\begin{table}[H]
 \captionsetup{justification=raggedright,singlelinecheck=false}
\caption{Incumbent and Newcomer Coordination Before and After Group Composition Changes} \label{tab:table10}
    \begin{center}
        \begin{table}[htbp]
    \begin{tabular}{c c c c}
    \toprule
    \textbf{Variables} & \textbf{(1)} & \textbf{(2)} & \textbf{(3)}      \\ 
\midrule
Newcomer(Before=1)               &     -17.91\sym{***}    &    -17.33\sym{***}  &  -16.94\sym{***}   \\
                                 &     (2.43)             &     (2.38)          &  (2.79)         \\
\addlinespace
Newcomer(After=1)                &     -8.79\sym{***}     &     -8.18\sym{***}   &  -7.73          \\
                                 &     (2.88)             &     (2.86)           &  (3.26)         \\
\addlinespace
Incumbent(After=1)               &     -4.26\sym{**}      &     -4.27\sym{**}  &  -4.30\sym{**}     \\
                                 &     (1.70)             &     (1.70)         &  (1.70)         \\
\midrule
Other Controls                   &    No                  &    Yes              &    Yes             \\
Status Controls                    &    No                  &    No               &    Yes             \\
Chat                             &    No                  &    No               &    No               \\
\midrule
Number of Participants           &    444               &    444              &    444               \\
\midrule
Observations                     &       4440          &       4440          &  4440          \\
\bottomrule

\end{tabular}
\begin{footnotesize}
\newline
*P$<$0.1, **P$<$0.05, ***P$<$0.01
\newline
Note: Robust standard errors clustered at the group level. Marginal effects of Tobit Model reported. 
\newline
The dependent variable is the hours toward the group activity in the minimum effort activity in both Parts. 
\end{footnotesize}
\end{table}
    \end{center}
\end{table}

\hspace  *{0mm} My focus shifts to understanding the impact of the incumbent team diversity on the newcomer coordination choices after newcomers join the team. Similar to the analysis of cooperation in the public goods provision, incumbent teams are classified by the level of diversity existing in the team before changes to the group composition. I first consider how the level of gender and racial diversity of the incumbent teams affect newcomer coordination choices. As before, the most diverse teams are the ones that have heterogeneous gender and racial compositions. Moderately diverse teams have diverse racial composition or diverse gender composition. The findings presented in table 19 reflects the regression model presented in equation 8a above in the \hyperref[subsec:Specification]{empirical specification} section with the base group being newcomers that join teams with homogeneous racial and homogeneous gender compositions - the least diverse. The results shown in column 1 show that newcomers that join moderately diverse incumbent teams do not coordinate at levels different from newcomers in the least diverse teams ($P-value=0.89$). Similarly, the coordination choices of newcomers in the most diverse team is on average positive but is not statistically significant ($P-value=0.53$). Controlling for other covariates in columns 2 and 3 does not affect the magnitude and direction of the effects. 

Next, I consider how the gender diversity of a team impacts newcomer coordination choices by newcomer gender identity. The findings of the base model of the estimation of equation 8b is presented in table 20. The model considers male newcomers to the homogeneous gender team as the omitted group. The results show that men in diverse gender teams coordinate at higher choices but the effect is not statistically significant ($P-value=0.14$). Similarly, the coefficients for non-male identifying individuals in the homogeneous gender groups and non-male identifying individuals in the gender diverse teams are also not statistically significant despite non-male identifying individuals in both teams coordinating at levels that are higher than men in the homogeneous gender teams. 

Finally, I consider the effect of the team racial diversity on newcomer coordination by newcomer race in table 21. I utilize equation 8c specified above in the \hyperref[subsec:Specification]{empirical specification} section to examine how the level of racial diversity in the incumbent team affects newcomer coordination choices by newcomer race. Intriguingly, among white newcomers, there is no statistically significant difference between coordination choices in homogeneous racial groups and diverse racial groups ($P-value=0.58$). Furthermore, non-white newcomers in the homogeneous racial teams tend to coordinate at lower choices despite the coefficients being statistically insignificant ($P-value=0.41$). Addition of controls in columns 2 and 3 does not affect the direction and statistical significance of the coefficients. These findings collectively lead to the following result:

\textbf{Result 4b: The diversity of the existing team does not have a discernible impact on the coordination choices of newcomers after they join the team.}

\begin{table}[H]
 \captionsetup{justification=raggedright,singlelinecheck=false}
\caption{Incumbent Team Diversity and Newcomer Coordination } \label{tab:table11}
    \begin{center}
        \begin{table}[htbp]
    \begin{tabular}{c c c c}
    \toprule
    \textbf{Variables} & \textbf{(1)} & \textbf{(2)} & \textbf{(3)}    \\ 
\midrule
Moderate$-$Diverse  &  0.72         &  0.82            &  0.02     \\
                    &  (5.32)       &  (5.33)           &  (5.27)    \\
\addlinespace
Most$-$Diverse      &  3.93       &   4.94                &  4.90     \\
                    &  (6.30)      & (6.18)             &  (6.20)     \ \\

\addlinespace
Positive                 &           &                     &  6.13     \\
                         &           &                     &  (4.36)   \\
\addlinespace
Negative                 &           &                      &  -2.63  \\
                          &          &                      &  (2.38)    \\
\addlinespace
Engagement               &          &                        & -0.70    \\
                         &          &                          &  (3.52)   \\

\midrule
Other Controls   &   No &  Yes &    Yes    \\
Status Controls &   No  &    No    &    Yes    \\
Chat          &    No    &    No   &    Yes          \\
\midrule
Number of Participants & 111   &    111 &    106    \\
\midrule
Observations          &   555   &  555  &  530         \\
\bottomrule

\end{tabular}

\begin{footnotesize}
\newline
*P$<$0.1, **P$<$0.05, ***P$<$0.01
\newline
Note: Robust standard errors clustered at the group level. Marginal effects of Tobit Model reported. 
\newline
The dependent variable is the hours toward the group activity in the minimum effort activity in Part II. \end{footnotesize}
\end{table}

    \end{center}
\end{table}

\begin{table}[H]
 \captionsetup{justification=raggedright,singlelinecheck=false}
\caption{Incumbent Team Diversity and Newcomer Coordination by Gender } \label{tab:table11}
    \begin{center}
        \begin{table}[htbp]
    \begin{tabular}{c c c c}
    \toprule
    \textbf{Variables} & \textbf{(1)} & \textbf{(2)} & \textbf{(3)}       \\ 
\midrule
Diverse$-$Gender            &     7.12    &    6.98                               &  6.54  \\
                            &     (4.80)  &    (4.71)                             &  (4.74) \\

Notmale                  &      1.95  &    -0.34                            &  -8.08  \\
                            &      (4.63)  &    (4.85)                        &  (4.93) \\
Diverse*Notmale         &      4.47      &  5.61                            &  -4.70 \\
                            &      (7.76)  &    (7.63)                     &  (7.48) \\


Positive Chat                        &                     &                       &  7.49  \\
                                 &                     &                           &  (4.14)  \\
Negative  Chat                       &                     &                       &  -3.54  \\
                                 &                     &                           &  (2.46)  \\
Engagement Chat                      &                     &                       &  -1.09  \\
                                 &                     &                           &  (3.46)  \\
\midrule
Other Controls   &   No &  Yes &    Yes    \\
Status Controls &   No  &    No    &    Yes    \\
Chat          &    No    &    No   &    Yes          \\
\midrule
Number of Participants & 111   &    111 &    106    \\
\midrule
Observations          &   555   &  555  &  530         \\
\bottomrule

\end{tabular}
\begin{footnotesize}
\newline
*P$<$0.1, **P$<$0.05, ***P$<$0.01
\newline
Note: Robust standard errors clustered at the group level. Marginal effects of Tobit Model reported. 
\newline
The dependent variable is the hours toward the group activity in the minimum effort activity in Part II. 
\end{footnotesize}
\end{table}

    \end{center}
\end{table}

\begin{table}[H]
 \captionsetup{justification=raggedright,singlelinecheck=false}
\caption{Incumbent Team Diversity and Newcomer Coordination by Race } \label{tab:table11}
    \begin{center}
        \begin{table}[htbp]
    \begin{tabular}{c c c c}
    \toprule
    \textbf{Variables} & \textbf{(1)} & \textbf{(2)} & \textbf{(3)}      \\ 
\midrule
Diverse$-$Race              &     -2.62          &    -1.97                               &  -0.49  \\
                            &   (4.76)           &    (4.72)                             &  (4.67) \\

Notwhite                 &      -3.79       &    -4.09                       &  -4.78   \\
                            &      (4.56)          &    (5.07)                        &  (4.91) \\
Diverse*Notwhite         &      -6.73          &  -6.81                       &  -5.31  \\
                            &      (5.75)         &  (6.28)                     &  (6.43) \\


Positive Chat                        &                     &                       &  5.91  \\
                                 &                     &                           &  (4.46)  \\
Negative  Chat                       &                     &                       &  -2.92  \\
                                 &                     &                           &  (2.39)  \\
Engagement Chat                      &                     &                       &  -0.66  \\
                                 &                     &                           &  (3.48)  \\
\midrule
Other Controls   &   No &  Yes &    Yes    \\
Status Controls &   No  &    No    &    Yes    \\
Chat          &    No    &    No   &    Yes          \\
\midrule
Number of Participants & 111   &    111 &    106    \\
\midrule
Observations          &   555   &  555  &  530         \\
\bottomrule

\end{tabular}

\begin{footnotesize}
\newline
*P$<$0.1, **P$<$0.05, ***P$<$0.01
\newline
Note: Robust standard errors clustered at the group level. Marginal effects of Tobit Model reported.
\newline
The dependent variable is the hours toward the group activity in the minimum effort activity in Part II. 
\end{footnotesize}
\end{table}
    \end{center}
\end{table}



\hspace  *{0mm} Finally, I look at how the identity of the newcomers affects incumbents’ coordination choices. I employ the procedure previously used in the analysis of the cooperation of participants. Shared identity is defined based on congruence between the identity of an incumbent and the newcomer. Racial identity is categorized into white and non-whites (minorities). Incumbents are then classified based on similarity to the newcomer regarding racial identity and gender identity. In table 22, I report the results of the tobit regression of the coordination choices of the incumbents. The base treatment group consists of incumbents that do not share both racial and gender identities with the newcomer. The results are based on the estimation of equation 9 in the \hyperref[subsec:Specification]{empirical specification} section. The findings in column 1 show the base model. It is evident that there is no statistically significant difference in the incumbents’ coordination choices by identities of the newcomer. There is no difference in coordination between incumbents that share gender identity but not race with the newcomer and the incumbents that do not share identities with the newcomer ($P-value=0.74$). Similarly, incumbents that share race and gender identities with the newcomer do not coordinate differently from incumbents that do not share both identities ($P-value=0.75$). Furthermore, the difference between the coordination choices of incumbents that do not share both racial and gender identities with the newcomer and the incumbents that share only racial identity with the newcomer is not statistically significant ($P-value=0.84$). It is evident that regardless of the identity of the newcomer, the average coordination choices of the existing team members are not impacted. These findings lead to my final result: 

\textbf{Result 5b: The incumbent members of the team do not vary their coordination choices in the minimum effort activity regardless of the newcomer's identity.}

\begin{table}[H]
 \captionsetup{justification=raggedright,singlelinecheck=false}
\caption{ Incumbent Coordination and Newcomer Identity } \label{tab:table12}
    \begin{center}
        \begin{table}[htbp]
    \begin{tabular}{c c c c c}
    \toprule
    \textbf{Variables} & \textbf{(1)} & \textbf{(2)} & \textbf{(3)}      & \textbf{(4)}     \\ 
\midrule
Congruentgender            &     -0.15              &     -0.71          &  -0.35            &  -2.02 \\
                                            &     (3.80)             &     (3.61)         &  (3.55)           &  (3.62) \\
\addlinespace
CongruentRace           &      2.06              &     2.05           &   1.99            &  0.52   \\
                                            &      (3.29)            &     (3.38)         &   (3.46)          &  (3.57) \\
\addlinespace
Congruentboth                &      1.21              &     -0.03           &  0.25                & -1.45  \\
                                            &      (3.71)             &    (3.54)          &  (3.70)            &  (3.65) \\
\addlinespace
Positive                         &                     &                     &              & -0.86  \\
                                 &                     &                     &              &  (2.14)  \\
\addlinespace
Negative                         &                     &                     &            &  -3.86\sym{**}  \\
                                 &           &                     &                    &  (1.83)  \\
\addlinespace
Engagement                       &                     &     &                    &  2.38\sym{*}  \\
                                 &                     &   &                    &  (1.45)  \\


\midrule
Other Controls                   &    No                &    Yes              &    Yes          &    Yes        \\
Status Controls                    &    No                &    No               &    Yes          &    Yes        \\
Chat                             &    No                &    No               &    No           &    Yes        \\
\midrule
Number of Participants           &    333               &    333              &    333          &    318       \\
\midrule
Observations                     &       1665           &       1665          &  1665           &  1590    \\
\bottomrule
 
\end{tabular}

\begin{footnotesize}
\newline
*P$<$0.1, **P$<$0.05, ***P$<$0.01
\newline
Note: Robust standard errors clustered at the group level. Marginal effects of Tobit Model reported. 
\newline
The dependent variable is the hours toward the group activity in the minimum effort activity in Part II.\end{footnotesize}
\end{table}
    \end{center}
\end{table}

\section{Conclusion} \label{sec:Conclusion}

The changing demographic composition of the US labor force signals an increasing diversity within teams, promising novel experiences for individuals across various team settings. This paper significantly contributes to the literature by investigating the impact of race and gender identity on team cooperation and coordination in dynamic settings, particularly when introducing newcomers to incumbent teams to examine economic decision making.  This study adopts an experimental economics approach to uncover compelling results with significant implications for team decisions in cooperation and coordination decisions. 

Newcomers are exogenously assigned to teams with varying degrees of diversity, emphasizing the salience of race and gender identities in the experimental environment. The analysis centers on incumbent teams, distinguishing between gender and racial diversity. Notably, in gender-homogeneous teams, men exhibit higher cooperation than non-male identifying individuals, who contribute less. Racially diverse teams witness increased cooperation from white individuals but decreased contributions from non-white individuals. When considering the gender and racial identities of individuals, white men in more diverse teams cooperate more, while non-male identifying and non-white individuals in less diverse teams contribute less.

A key finding is that incumbents consistently demonstrate higher cooperation than newcomers, with newcomers exhibiting lower cooperation rates equivalent to 15\% less than incumbents in the initial phase of the study. However, cooperation levels between incumbents and newcomers converge after team composition changes, achieving parity. The study identifies that newcomers' cooperation varies based on team diversity, with increased cooperation observed in moderately diverse teams. Importantly, the identity of newcomers does not significantly impact the cooperation choices of incumbent team members.

In the realm of coordination, men in gender-diverse teams show no significant difference compared to those in homogeneous gender teams, while non-male identifying individuals coordinate less. In racially diverse teams, whites exhibit higher coordination choices than non-white individuals. The combined analysis of race and gender diversity reveals that non-white, non-male identifying individuals coordinate at significantly lower levels in more diverse teams. However, no significant variation is found in incumbent coordination choices based on shared racial or gender identity with the newcomer.  

In conclusion, this study sheds light on the complexity of cooperation and coordination choices in diverse teams, with implications for both incumbents and newcomers. These results, while complementing other findings in economic literature, underscore the importance of fostering team identity, especially in diverse teams. Through the application of social dilemma activities, valuable insights are provided into the dynamics of team cooperation and coordination within distinct contexts, emphasizing the impact of dynamic team composition, a frequent occurrence in various organizations.

Importantly, the study recognizes the limitations of its findings, applicable within an environment where specific demographics are a majority, and diverse individuals are expected to work in teams. Furthermore, the findings raise an intriguing, unresolved question, prompting a forthcoming study. Given the discovered influence of team composition on individual behaviors, it becomes imperative to consider environmental factors. The next phase of exploration will delve into how the environment shapes the cooperative and coordination choices of individuals, particularly examining the influence of physical interactions among team members. These future investigations aim to enhance our understanding of the nuanced dynamics of team decision-making in diverse settings.

\bibliography{mybiblio.bib}

\section{Online Appendix}
 
\subsection{Appendix A : Additional Analysis of Newcomer } \label{sec:App}
I provide further analysis of what is driving the behavior of newcomers once they join a team. As shown earlier in the analysis, newcomers to moderately diverse teams are more cooperative than newcomers in the least diverse team. Here, I consider a different specification to understand whether newcomers' behavior is driven by having individuals in the existing team that look like them. I devise a classification based on whether there exists similar individuals in the team like them in the new team. In this specification, newcomers that share gender with at least 1 member of the incumbent team are considered to share gender with the team. A similar approach is taken for race.

First, I utilize equation 10 below to examine how sharing gender with at least 1 incumbent member of the team impacts actions of a newcomer. 

\begin{center}
 $Y_{i}=\beta_1Gen$-$Match_{i}+\beta_2Notmale_{i}+\beta_3Gender$-$Match_{i}*Notmale_{i}+\theta X_{i}+\epsilon_{i}$\space (10)
\end{center}


\noindent where $Y_{i}$ is the contributions of newcomer $i$ in either the public goods provision or the minimum effort activity. $Gen$-$Match_{i}$ is a dummy variable indicating whether newcomer $i$, the newcomer is in a group where they share gender with at least 1 incumbent member of the team. $Notmale_{i}$ is an indicator for newcomer gender. $Gen$-$Match_{i}*Notmale_{i}$ is an interaction of non-male identifying newcomer who has a gender match in the team. $X_{i}$ is a set of individual characteristics such as age, major, income, individual interactions with the group, parents' socioeconomic background and other personal characteristics. $\epsilon_{i}$ is the residual term. 

The results are presented in table 23 in the appendix. The first 3 columns show the results of newcomers' actions in the cooperation activity while the last 3 columns show the results of newcomers' actions in the coordination activity. Similar to earlier analysis, the first column presents the base model with male newcomers without gender match in the team being the omitted group.  The results in column 1 show that having homogeneity in terms of gender with at least 1 member of the incumbent team does not affect the cooperation of the newcomer overall. In fact, newcomers who join teams where an existing member shares gender with them on average cooperate less despite the effect being statistically insignificant among men ($P-value=0.41$). Additionally, the effect does not vary by gender of the newcomer ($P-value=0.92$). This result emphasizes further that the overall diversity of the team matters more than specific gender matches between team members. Additional controls in column 2 and column 3 do not affect the direction and the significance of the effects. The results of the analysis of the base model of coordination choices presented in column 4 shows coordination choices of newcomers are not affected by having an incumbent member of the team the newcomer shares gender identity with among men ($P-value=0.95$) in the base model. However, additional controls in the third column of coordination choices show that at the 10\% significance, positive conversation with the incumbent team increases coordination choices of the newcomer. Furthermore, non-male individuals coordinate at lower levels than men in teams where they do not share gender with any incumbent ($P-value<0.1$). Non-male identifying individuals who join teams where they share gender with at least 1 member of the team also coordinate at choices lower than men in teams where they do not have a gender match ($P-value<0.1$).

\begin{table}[H]
 \captionsetup{justification=raggedright,singlelinecheck=false}
\caption{Newcomer Decisions: Gender Match} \label{tab:tablea1}
    \begin{center}
        \begin{table}[htbp]
    \begin{tabular}{c c c c c c c}
    \toprule
          & \multicolumn{3}{c}{Cooperation}   &\multicolumn{3}{c}{Coordination}     \\
\midrule
Gender$-$Match      &  -7.31     &  -9.97   &  -8.34     &    -0.28    &  -2.16         &  0.58   \\
                    &  (8.88)    &  (8.75)  &  (9.17)   &     (4.72)   &  (5.01)        &  (5.18)\\
\addlinespace
Notmale        &  -0.81      &   -4.82   &  -5.05    &   2.02   &   0.35             &  -8.50\sym{*}    \\
                  &  (8.48)     &    (8.18) &  (8.87)   &   (4.71) &   (4.96)           &  (5.07) \ \\
\addlinespace
Gender$-$Match*Notmale &  -5.32 & -10.12   & -12.31    &   2.30   &   0.84           &  -9.73\sym{*}    \\
                  &  (10.42)     &    (10.11) &  (11.00)   &  (5.93) & (6.03)           &  (6.19) \ \\
                        
\addlinespace
Positive  Chat    &           &           &  -1.22     &             &               &  7.91\sym{*} \\
                  &           &           &  (7.88)    &             &                &  (4.23)   \\
\addlinespace
Negative   Chat  &            &           &  -2.89    &               &                &  -3.80   \\
                 &            &           &  (4.86)   &               &                 &  (2.57)  \\
\addlinespace
Engagement  Chat   &         &            &   5.98     &              &                &  -0.92 \\
                   &         &            &  (4.33)   &               &                &  (3.31) \\

\midrule
Other Controls      &   No   &  Yes     &    Yes    &    No    &    Yes  &    Yes \\
Status Controls     &   No   &    No    &    Yes    &    No    &   No    &    Yes \\
Chat                &    No  &    No    &    Yes     &    No    &    No   &    Yes             \\
\midrule
Number of Participants & 111   &    111 &    106  &   62   &    62 &    58     \\
\midrule
Observations          &   555   &  555  &  530   &  310     &  310   &  290             \\
\bottomrule

\end{tabular}

\begin{footnotesize}
\newline
*P$<$0.1, **P$<$0.05, ***P$<$0.01
\newline
Note: Robust standard errors clustered at the group level. Marginal effects of Tobit Model reported. 
\newline
Columns 1 - 3: The dependent variable is the contributions toward the group in the public goods provision in Part II. 
\newline
Columns 4 - 6: The dependent variable is the hours toward the group activity in the minimum effort activity in Part II.
\end{footnotesize}
\end{table}
    \end{center}
\end{table}

Next, I consider the impact of having an individual that shares a newcomer's race on the newcomer's economic decisions using equation 11 below: 

\begin{center}
 $Y_{i} = \beta_1 Race$-$Match_{i} + \beta_2 Notwhite_{i} + \beta_3 Race$-$Match_{i}*Notwhite_{i} + \theta X_{i} + \epsilon_{i}$  \space (11)
\end{center}

\noindent where $Y_{i}$ is the contributions of participant $i$ in either the public goods provision or the minimum effort activity. $Race$-$Match_{i}$ is a dummy variable indicating whether participant $i$, the newcomer is in a group where they share race with at least 1 incumbent member of the team. $Notwhite_{i}$ is an indicator for participant race. $Race$-$Match_{i}*Notwhite_{i}$ is an interaction of non-white newcomer who joins a team where he/she shares race with a member of the team. $X_{i}$ is a set of individual characteristics such as age, major, income, individual interactions with the group, parents' socioeconomic background and other personal characteristics. $\epsilon_{i}$ is the residual term. 

The results of the effect of sharing race with at least 1 member of the incumbent team on newcomer economic decision$-$making are presented in table 24. The first three columns show cooperation choices and the last 3 columns show the results of newcomers in the coordination activity. Overall, cooperation is lower among non-white newcomers who join teams where there is no incumbent they share the same race with as compared to omitted group of white newcomers in teams where they do not share race with any member of the team ($P-value=0.29$).  However, the effect is statistically insignificant. The coefficient of newcomers in teams where they share race with at least an incumbent member is even lower though statistically insignificant ($P-value=0.17$). The coefficient attenuates and is statistically significant at the 5\% significance level once controls for major, age, income and minority status are included ($P-value=0.04$). The effect stays statistically significant and consistent when additional controls of risk aversion and interactions with the incumbent members of the team are controlled for ($P-value<0.05$). Interestingly, coordination choices are lower when they share race with incumbents in the base model in column 4 of table 24. Newcomers who share race with at least 1 member of the incumbent team they join actually coordinate at lower choices than those who join teams where they do not share race with anyone . Among white newcomers, the effect is 13.6\% of the maximum possible less than whites in teams where they do not share race with with anyone ($P-value<0.05$) while among non-whites, coordination is 48.4\% of the maximum possible less than whites in teams where they do not share race with with anyone ($P-value<0.1$). The coefficients are robust to the inclusion of additional controls in column 5.  


\begin{table}[H]
 \captionsetup{justification=raggedright,singlelinecheck=false}
\caption{Newcomer Decisions: Race Match} \label{tab:tablea1}
    \begin{center}
        \begin{table}[htbp]
    \begin{tabular}{c c c c c c c}
    \toprule
          & \multicolumn{3}{c}{Cooperation}   &\multicolumn{3}{c}{Coordination}     \\
\midrule
Race$-$Match      &  6.64                   &   8.80             &  4.95     &  -9.50\sym{**}  &  -12.35\sym{**}   &  -3.94   \\
                  &  (12.37)                &  (13.11)           &  (14.13)   &  (3.94)        &  (6.35)           &  (7.48) \\
\addlinespace
Notwhite        &  -12.23                &   -16.96         &  -18.07  &  -11.36\sym{**}  &   -12.80\sym{**}   &  -8.42     \\
                  &   (11.46)                &    (11.82)         &  (12.04)   &  (4.57)        &    (5.57)     &  (6.11) \\
\addlinespace
Race$-$Match*Notwhite & -19.39   & -29.19\sym{**}  & -26.98\sym{**} & -13.03\sym{*}  & -15.15\sym{**} & -17.06 \sym{**}     \\
                        & (14.38)            &  (14.00)      &  (13.58)   & (6.66)       &    (6.97)        &  (6.51) \\
                        
\addlinespace
Positive  Chat    &                        &           &  -3.01     &             &               &  6.63  \\
                  &                        &           &  (7.85)    &             &                &  (4.33)   \\
\addlinespace
Negative   Chat  &                        &           &  -2.42    &               &                &   -2.79    \\
                 &                        &           &  (4.66)   &               &                 &  (2.42)  \\
\addlinespace
Engagement  Chat   &                      &            &   6.29     &              &                &  -0.53 \\
                   &                     &             &  (4.30)   &               &                &  (3.28) \\

\midrule
Other Controls      &   No   &  Yes     &    Yes    &    No    &    Yes  &    Yes \\
Status Controls     &   No   &    No    &    Yes    &    No    &   No    &    Yes \\
Chat                &    No  &    No    &    Yes     &    No    &    No   &    Yes             \\
\midrule
Number of Participants & 111   &    111 &    106  &   62   &    62 &    58     \\
\midrule
Observations          &   555   &  555  &  530   &  310     &  310   &  290             \\
\bottomrule

\end{tabular}

\begin{footnotesize}
\newline
*P$<$0.1, **P$<$0.05, ***P$<$0.01
\newline
Note: Robust standard errors clustered at the group level. Marginal effects of Tobit Model reported.
\newline
Columns 1 - 3: The dependent variable is the contributions toward the group in the public goods provision in Part II. 
\newline
Columns 4 - 6: The dependent variable is the hours toward the group activity in the minimum effort activity in Part II. 
\end{footnotesize}
\end{table}
    \end{center}
\end{table}

Combining the findings from tables 23 and 24, it is evident that the effect driving the behavior of newcomers in moderately diverse teams is not based on matching identities with an individual. In fact, individuals sharing the same race cooperate less when they join homogeneous teams that they do not share gender with them. This is particularly true for white newcomers as shown in figure 11 of the appendix. White men and women cooperate at a statistically significantly lower level when they join all-white women groups and all-white men groups respectively. Their cooperation choices is however highest when they join moderately diverse teams where there is heterogeneous gender or heterogeneous racial compositions. In terms of coordination choices, individuals coordinate less when they share race with members of the incumbent, again emphasizing the fact the effect driven by indentity matches. 


\subsection{Appendix B : Tables} \label{sec:App}



% \begin{table}[H]
% \caption{Summary Statistics [Cooperation] - \texttt{Mean and Standard Deviation} } \label{tab:table3}
%     \begin{center}
%         \begin{table}
    \begin{tabular}{ccccccc}
    \toprule
        & Incumbents & Newcomer & Newcomer(1-5) & Newcomer(6-10) &  Incumbent(1-5) & Incumbent(6-10) \\
        \midrule
        All-Men      & 60.59 & 49.25  & 39.76    & 58.73   &    57.71 & 63.46\\
                    & (39.67) & (31.49) & (27.96)  & (32.08)  &  (37.60) & (41.50)\\
        \addlinespace
        All-Women &   55.84 &   47.52 & 42.50    &  52.55  &    54.92 &   56.76\\
                        & (35.24) & (33.51) & (32.52) & (33.81) &  (34.87) & (35.61)\\
        \addlinespace
        Mixed Gender     & 60.39 &  47.73 & 40.55 & 54.91  &    61.42 &    59.35 \\
                        & (37.08) &   (36.46) & (35.19)  & (36.36) &  (35.87) &  (38.25)\\
        \addlinespace
        Combined &    58.53 &  47.92 & 41.21   &  54.62  &    58.06&    59.02 \\
                        & (36.88) & (34.37) & (32.85) & (34.58) &  (35.88) &  (37.86) \\
    \bottomrule
    \end{tabular}
    \caption{Summary Statistics [Cooperation] - \texttt{Mean and Standard Deviation} }
    \label{tab:my_label}
\end{table}

%     \end{center}
% \end{table}

%  

\begin{table}[H]
 \captionsetup{justification=raggedright,singlelinecheck=false}
\caption{Categories of Chat Content } \label{tab:table4}
        \begin{table}[h]
\begin{tabular}{|l|p{8cm}|}
\hline
\textbf{Category} & \textbf{Description} \\
\hline
Frustration & Displayed frustration during the puzzle-solving process (1 = yes, 0 = no) \\
\hline
Confusion & Expressions of confusion related to the puzzle-solving process (1 = yes, 0 = no) \\
\hline
Talk in agreement & Conversation related to the puzzle-solving process - agreement with another participant (1 = yes, 0 = no) \\
\hline
Talk in disagreement & Conversation related to the puzzle-solving process - disagreement with another participant (1 = yes, 0 = no) \\
\hline
Confident & Was confident in their abilities to solve the puzzle (1 = not confident at all...5 = very confident) \\
\hline
Excitement & Expressions of excitement or satisfaction related to the puzzle-solving process (1 = yes, 0 = no) \\
\hline
Assertive & Was assertive in their communication with other participant(s) (1 = not assertive at all...5 = very assertive) \\
\hline
Comfortable & How did the participant's language manifest or what nonverbal cues did the participant exhibit? (1 = not showing this at all...5 = shows very clear signs of this) - Comfortability \\
\hline
\end{tabular}
\label{tab:categories}
\end{table}
\end{table}

\begin{table}[H]
 \captionsetup{justification=raggedright,singlelinecheck=false}
\caption{Summary Statistics  } \label{tab:table4}
    \begin{center}
        \begin{tabular}{l*{5}{c}}
\toprule
                & Incumbent & Newcomer & Newcomer(M) & Newcomer(W) &    Combined \\
\midrule
All-Men      & 54.45  & 44.2  &    41.45&    47.56&    51.89\\
                &(22.11) &(20.85)&  (22.09)&  (18.81)&  (22.24)\\
\addlinespace
All-Women&   53.25 &    45.96 &    45.32&    46.54&    51.43\\
                & (18.17) & (20.03)&  (20.05)&  (20.02)&  (18.91)\\
\addlinespace
Mixed Gender     &    53.49&      49.22&    45.36&    52.91&    52.42\\
                &   (17.95)&   (20.22)&  (22.19)&  (17.41)&  (20.06)\\
\addlinespace
Combined &    53.56&     46.96&    44.56&    49.32&    51.91\\
                &  (17.97)&  (20.33)&  (21.36)&  (19.00)&  (20.01)\\

\bottomrule
\end{tabular}

    \end{center}
\end{table}

 

% \begin{table}[H]
% \caption{Summary Statistics [Coordination] - \texttt{Mean and Standard Deviation} } \label{tab:table5}
%     \begin{center}
%         
\begin{tabular}{l*{6}{c}}
\toprule
                & Incumbent & Newcomer & Newcomer(1-5) & Newcomer(6-10) &  Incumbent(1-5) & Incumbent(6-10)\\
\midrule
All-Men      & 54.45  & 44.2  & 43.3   &   45.1  & 57.73 & 51.6 \\
                &(22.11) &(20.85)& (20.94) & (20.82)  &(19.82)&(23.87)\\
\addlinespace
All-Women&   53.25 &    45.96 &  43.39  &  48.52  &  54.26&   52.25  \\
                & (18.17) & (20.03)&  (19.84)&  (19.92)&  (17.05)& (19.19)\\
\addlinespace
Mixed Gender     &    53.49&      49.22  & 46.27  & 52.18  &    56.12&    50.86 \\
                &   (17.95)&   (20.22)&  (20.58)&  (19.46) &   (17.95)&   (21.37) \\
\addlinespace
Combined &    53.56&     46.96 &   44.54  & 49.39  &  55.56 &    51.57\\
                &  (17.97)&  (20.33)&  (20.36)&  (20.04)&  (17.97)&  (20.99)\\

\bottomrule
\end{tabular}

%     \end{center}
% \end{table}

% \vfill



\subsection{Appendix C : Figures}

\begin{figure}[H]
 \captionsetup{justification=raggedright,singlelinecheck=false}
\caption{Part I Puzzle Pieces}
\includegraphics[scale=0.6]{Figures/S1.png} 
\end{figure}

 

\begin{figure}[H]
 \captionsetup{justification=raggedright,singlelinecheck=false}
\caption{Part I Puzzle Solution}
\includegraphics[scale=0.6]{Figures/Completed1.png} 
\end{figure}

\begin{figure}[H]
 \captionsetup{justification=raggedright,singlelinecheck=false}
\caption{Part II Puzzle Pieces}
\includegraphics[scale=0.6]{Figures/S2.png} 
\end{figure}


\begin{figure}[H]
 \captionsetup{justification=raggedright,singlelinecheck=false}
\caption{Part II Puzzle Solution}
\includegraphics[scale=0.6]{Figures/Completed2.png} 
\end{figure}

\begin{figure}[H]
 \captionsetup{justification=raggedright,singlelinecheck=false}
\caption{Environment of Lab}
\includegraphics[scale=0.5]{Figures/E.png} 
\end{figure}

\begin{figure}[H]
 \captionsetup{justification=raggedright,singlelinecheck=false}
\caption{Incumbent Team Diversity and Newcomer Cooperation }
\includegraphics[scale=0.2]{Figures/Newcomer_types_in_diversity_4_ppg.png} 
\end{figure}



\begin{figure}[H]
 \captionsetup{justification=raggedright,singlelinecheck=false}
\caption{Incumbent Team Diversity and Newcomer Coordination }
\includegraphics[scale=0.2]{Figures/Newcomer_types_in_diversity_4_me.png} 
\end{figure}

\begin{figure}[H]
 \captionsetup{justification=raggedright,singlelinecheck=false}
\caption{Incumbent Team Diversity and Newcomer Cooperation }
\includegraphics[scale=0.2]{Figures/Newcomer_types_in_diversity_ppg.png} 
\end{figure}


\begin{figure}[H]
 \captionsetup{justification=raggedright,singlelinecheck=false}
\caption{Incumbent Team Diversity and Newcomer Coordination }
\includegraphics[scale=0.2]{Figures/Newcomer_types_in_diversity_me.png} 
\end{figure}


\subsection{Appendix D : Instructions}
\subsubsection{General Instructions}
Welcome! Thank you for coming to this experiment on group behavior. Please power off your cell phone. This study has some stages that allow communication and other stages that don’t. So, we ask there be no talking among the participants, unless you are allowed to.  Violations will disqualify you for this study. If you have a question, please raise your hand. An experimenter will come to help you.
This experiment has different parts. You will be given instructions at the beginning of each part. The parts that involve group members’ interactions will be video recorded. The parts with decision making on the computers will not be video recorded. 
At the end of the experiment, you will be paid individually and privately in cash based on an exchange rate: 
100 Experimental Currency Units (ECUs) = \$1. 
No one else but the experimenters will know a participant’s decisions and earnings. So, you are under no obligations to share this information with other participants. 
There are 4 participants in this session. Each of you has received a sticker with your ID (A to D) on the back. Please don’t reveal it to anyone else except the experimenters. Now an experimenter will come. Please show your ID privately to the experimenter and he will lead each of you to your assigned lab. 

\subsubsection{Part I Instructions (Incumbents)}
\subsubsection{\textbf{Stage 1}}
\newline
Part I of this study has two stages. In the first stage, you and two other participants in your group will play a puzzle game. There are 3 envelopes on the table, one for every group member. Each envelope contains FOUR pieces of cardboard. Your task is to use these four pieces to form a Triangle like the one on the table, with one right angle of 90° and two 45° angles. To complete the task, each group member will need to complete their Triangle. 
All the Triangles were cut in the same way, and the pieces were shuffled so that the four pieces in your envelope cannot form a Triangle. You need to find the right piece or pieces from your group members. Group members are encouraged to share ideas and talk to each other with some rules to follow: 
\begin{itemize}
    \item Group members may give pieces to other group members but may not take pieces from other group members. 
    \item You must give the piece or pieces directly to one another instead of throwing the piece or pieces in the middle for others to take. 
    \item When making your Triangle, the pieces cannot overlap each other.
\end{itemize}
Each participant will receive payoffs based on the number of the pieces all the group members correctly place at a rate of 10 ECUs per correctly placed piece. For example, correctly placing all four pieces by each group member will earn everyone 120 ECUs (12*10 ECUs). 
You have 10 minutes to work on this task. When you finish, please raise your hand, the experimenter will check. You will find out about your payoffs at the end of the experiment. Just a gentle reminder. This puzzle game will be video recorded. 
\subsubsection{\textbf{Stage 2}}
Please find the computer that matches your ID. We’re now starting Stage 2 of part I. This stage of decision making on computers will not be video recorded. Please do not talk to each other in this stage. Raise your hand if you have any questions. 
\newline
\textbf{\textit{Instructions on Computer Screen}}
\newline
\textit{Screen 1}
\newline
In Stage 2, you and the 2 other group members will play 2 decision-making games on the computers. You will receive instructions at the beginning of each game. Your payoffs from each game will depend on your decisions and the other two group members’ decisions in that game. You will be paid based on one randomly chosen game. Since each game has the equal probability of being chosen, it is important for you to make decisions in each game as if it is the one that will be chosen to compute earnings. You will not see your payoffs in each game or which game is chosen for earnings until the end of this stage after all the games are played.

Now please click ‘Next’ to go to the first game. 
\newline
\textit{Game 1}
\newline
Game 1 Instructions
\newline
Game 1 consists of 5 rounds. In each round, you and the other 2 group members will receive an initial endowment of 100 ECUs and then decide how much of the initial endowment to keep and how much to allocate towards a group account.

For each ECU allocated to the group account, every group member will earn 0.5 ECUs. That is, every ECU you allocate to the group account will generate 0.5 ECUs for you and 0.5 ECUs for everyone else in your group, therefore this leads to a total of 0.5*3 = 1.5 ECUs for the group per ECU allocated. Similarly, if someone else allocates 1 ECU to the group account, you and everyone else in your group will earn 0.5 ECUs per person.

Therefore, your round payoffs are calculated as follows:
\begin{itemize}
    \item Your Round Payoffs = ECUs you keep for yourself + 0.5*(total amount allocated of all members in your group).
\end{itemize}

For example, if you allocate 50 ECUs to the group account and keep 50 ECUs for yourself. If the total amount allocated to the group account is 150 ECUs. Then, your round payoff will be the ECUs you kept for yourself + 0.5 times the total amount allocated to the group account making 125 ECUs (i.e. 50 ECUs + 75 ECUs).

Now please click ‘Next’, and you will be redirected to a page with questions to check your comprehension of the instructions.
\newline
\textit{Game 2}
\newline
Game 2 Instructions
\newline
Game 2 has 5 rounds too. Consider yourself and the other 2 participants work in a group at a firm. You can think of a round of this game as being a workweek.

In each week, you and your group members can each choose to spend up to 70 hours on an activity at work. The payoff that each group member receives in a round depends on the number of hours she/he chooses to spend on the activity and the number of hours the two other group members choose to spend. The formula below determines your round payoff, which is the minimum hours of all the group members minus 0.75 times your hours plus 85 ECUs. The payoff table is provided for you below, so you do not need to memorize this formula. This payoff table will be available at any point where you need to make a decision. Your round payoffs are calculated as follows:
\begin{itemize}
    \item Your Round Payoff (ECUs) = Minimum Hours of Your Group - 0.75*Your Hours + 85 ECUs (Please note that ECUs are rounded up to the nearest integer)
\end{itemize}


For example, if you spend 70 hours on the activity, and the other two group members spend 70 hours and 60 hours respectively: then your round payoff will be 60 – 0.75*70 +85 ECUs making 93 ECUs (highlighted in blue in the table below). Your round payoff will be determined by tracing row 70 for "your hours" and column 60 for "group minimum hours" in the payoff table. The payoff of the group member who spent 70 will be identical. However, the payoff of the group member who spent 60 will be 60 - 0.75*60 + 85 ECUs making 100 ECUs (highlighted in green in the table below).

Now please click ‘Next’, and you will be redirected to a page with questions to check your comprehension of the instructions.

\subsubsection{Part I Instructions (Newcomer)}
\subsubsection{\textbf{Stage 1}}
\newline
Part I of this study has two stages. In the first stage, you will wait for 10 minutes.
\subsubsection{\textbf{Stage 2}}
Please find the computer that matches your ID. We’re now starting Stage 2. This stage of decision making on computers will not be video recorded. Please raise your hand if you have any questions. 
\newline
\textbf{\textit{Instructions on Computer Screen}}
\newline
\textit{Screen 1}
\newline
In Stage 2, you and two computer robots will play 2 decision-making games on the computer. You will receive instructions at the beginning of each game. Your payoffs from each game will depend on your decisions and the computer randomly generated decisions of the robots. You will be paid based on one randomly chosen game. Since each game has the equal probability of being chosen, it is important for you to make decisions in each game as if it is the one that will be chosen to compute earnings. You will not see your payoffs in each game or which game is chosen for earnings until the end of this stage after all the games are played.

Now please click ‘Next’ to go to the first game.
\newline
\textit{Game 1}
\newline
Game 1 Instructions
\newline
Game 1 consists of 5 rounds. In each round, you and 2 computer robots will receive an initial endowment of 100 ECUs and then decide how much of the initial endowment to keep and how much to allocate towards a group account.

For each ECU allocated to the group account, every group member will earn 0.5 ECUs. That is, every ECU you allocate to the group account will generate 0.5 ECUs for you and 0.5 ECUs each for the two computer robots in your group, therefore this leads to a total of 0.5*3 = 1.5 ECUs for the group per ECU allocated. Similarly, if a computer robot allocates 1 ECU to the group account, you and the other computer robot in your group will earn 0.5 ECUs each. Please note that the decisions of the two robots are randomly generated by the computer. Therefore, your round payoffs are calculated as follows:
\begin{itemize}
    \item Your Round Payoffs = ECUs you keep for yourself + 0.5*(total amount allocated of all members in your group).
\end{itemize}

For example, if you allocate 50 ECUs to the group account and keep 50 ECUs for yourself. If the total amount allocated to the group account is 150 ECUs. Then, your round payoff will be the ECUs you kept for yourself + 0.5 times the total amount allocated to the group account making 125 ECUs (i.e., 50 ECUs + 75 ECUs).

Now please click ‘Next’, and you will be redirected to a page with questions to check your comprehension of the instructions.
\textit{Game 2}
Game 2 Instructions
Game 2 has 5 rounds too. Consider yourself and the other 2 computer robots work in a group at a firm. You can think of a round of this game as being a workweek.

In each week, you and your group members can each choose to spend up to 70 hours on an activity at work. The payoff that you receive in a round depends on the number of hours you choose to spend on the activity and the number of hours chosen by the two computer robots in your group. Please note that the decisions of the two robots are randomly generated by the computer. The formula below determines your round payoff, which is the minimum hours of all the group members minus 0.75 times your hours plus 85 ECUs. The payoff table is provided for you below, so you do not need to memorize this formula. This payoff table will be available at any point where you need to make a decision. Your round payoffs are calculated as follows:
\begin{itemize}
    \item Your Round Payoff (ECUs) = Minimum Hours of Your Group - 0.75*Your Hours + 85 ECUs (Please note that ECUs are rounded up to the nearest integer)
\end{itemize}

For example, if you spend 70 hours on the activity, and the other two computer robots spend 70 hours and 60 hours respectively: then your round payoff will be 60 – 0.75*70 +85 ECUs making 93 ECUs (highlighted in blue in the table below). Your round payoff will be determined by tracing row 70 for "your hours" and column 60 for "group minimum hours" in the payoff table. The payoff of the group member who spent 70 will be identical. However, the payoff of the group member who spent 60 will be 60 - 0.75*60 + 85 ECUs making 100 ECUs (highlighted in green in the table below).

Now please click ‘Next’, and you will be redirected to a page with questions to check your comprehension of the instructions.

\subsubsection{Part II Instructions (Incumbents and Newcomers)}
\newline
\subsubsection{\textbf{Stage 1}}
\newline
We are starting Part II of this study. A new participant has joined the group so there are 4 group members now including the 3 old members and one new member. The old 3-person group played a triangle-puzzle game and the 2 decision-making games among themselves. The new member played the same 2 decision-making games with two computer robots. 
Similar to Part I, Part II has two stages. In the first stage, everyone in the 4-person group will be given new envelopes and play a different puzzle game.   
There are 4 envelopes on the table, one for every group member. Each envelope contains SIX pieces of cardboard. Your task is to use these six pieces to form a Triangle like the one on the table, with one right angle of 90° and two 45° angles. To complete the task, each group member will need to complete their Triangle. 
Similar to the previous triangle puzzle, all the Triangles were cut in the same way, and the pieces were shuffled so that the six pieces in your envelope cannot form a Triangle. You need to find the right piece or pieces from your group members. Group members are encouraged to share ideas and talk to each other with the same rules as previously: 
\begin{itemize}
    \item Group members may give pieces to other group members but may not take pieces from other group members.
    \item You must give the piece or pieces directly to one another instead of throwing the piece or pieces in the middle for others to take. 
    \item When making your Triangle, the pieces cannot overlap each other.
\end{itemize}
Each participant will receive payoffs based on the number of the pieces all the group members correctly place at a rate of 10 ECUs per correctly placed piece. For example, correctly placing all six pieces by each group member will earn everyone 240 ECUs (24*10 ECUs). 
You have 10 minutes to work on this task. When you finish, please raise your hand, the experimenter will check. You will find out about your payoffs at the end of the experiment. Note this puzzle game will be video recorded.
\newline
\subsubsection{\textbf{Stage 2}}
\newline
Please find the computer that matches your ID. We’re now starting Stage 2 of Part II. This stage of decision making on computers will not be video recorded.  Please do not talk to each other in this stage. Raise your hand if you have any questions. 
\newline
\textbf{\textit{Instructions on Computer Screen}}
\newline
\textit{Screen 1}
\newline
In Stage 2, you and the 3 other group members will play 3 decision-making games on the computers. Your payoff from game 3 will be added to the payoff from the randomly chosen game from this part. Otherwise, the rules are the same as before. Since a new member joined, let’s recap these rules. You will receive instructions at the beginning of each game. Your payoffs from each game will depend on your decisions and the other 3 group members’ decisions in that game. You will be paid based on one randomly chosen game from games 1 and 2, and game 3. Since each of the first two games has the equal probability of being chosen, it is important for you to make decisions in each game as if it is the one that will be chosen to compute earnings. You will not see your payoffs in each game or which game is chosen for earnings until the end of this stage after all the games are played.

Now please click ‘Next’ to go to the first game.
\newline
\textit{Game 1}
\newline
Game 1 Instructions
\newline
Game 1 consists of 5 rounds. In each round, you and the other 3 group members will receive an initial endowment of 100 ECUs and then decide how much of the initial endowment to keep and how much to allocate towards a group account.

For each ECU allocated to the group account, every group member will earn 0.438 ECUs. That is, every ECU you allocate to the group account will generate 0.438 ECUs for you and 0.438 ECUs for everyone else in your group, therefore this leads to a total of 0.438*4 = 1.752 ECUs for the group per ECU allocated. Similarly, if someone else allocates 1 ECU to the group account, you and everyone else in your group will earn 0.438 ECUs per person.

Therefore, your round payoffs are calculated as follows:
\begin{itemize}
    \item Your Round Payoffs = ECUs you keep for yourself + 0.438*(total amount allocated of all members in your group).
\end{itemize}

For example, if you allocate 50 ECUs to the group account and keep 50 ECUs for yourself. If the total amount allocated to the group account is 200 ECUs. Then, your round payoff will be the ECUs you kept for yourself + 0.438 times the total amount allocated to the group account making 137.6 ECUs (i.e. 50 ECUs + 87.6 ECUs).

Now please click ‘Next’, and you will be redirected to a page with questions to check your comprehension of the instructions.
\newline
\textit{Game 2}
\newline
Game 2 Instructions
\newline
Game 2 has 5 rounds too. Consider yourself and the other 3 participants work in a group at a firm. You can think of a round of this game as being a workweek.

In each week, you and your group members can each choose to spend up to 70 hours on an activity at work. The payoff that each group member receives in a round depends on the number of hours she/he chooses to spend on the activity and the number of hours the 3 other group members choose to spend. The formula below determines your round payoff, which is the minimum hours of all the group members minus 0.75 times your hours plus 85 ECUs. The payoff table is provided for you below, so you do not need to memorize this formula. This payoff table will be available at any point where you need to make a decision. Your round payoffs are calculated as follows:
\begin{itemize}
    \item Your Round Payoff (ECUs) = Minimum Hours of Your Group - 0.75*Your Hours + 85 ECUs (Please note that ECUs are rounded up to the nearest integer)
\end{itemize}


For example, if you spend 70 hours on the activity, and the other 3 group members spend 70 hours, 10 hours and 60 hours respectively: then your round payoff will be 10 – 0.75*70 +85 ECUs making 43 ECUs (highlighted in blue in the table below). Your round payoff will be determined by tracing row 70 for "your hours" and column 10 for "group minimum hours" in the payoff table. The payoff of the group member who spent 70 will be identical. However, the payoff of the group member who spent 60 will be 10 - 0.75*60 + 85 ECUs making 50 ECUs (highlighted in green in the table below). Finally, the payoff of the group member who spent 10 will be 10 - 0.75*10 + 85 ECUs making 88 ECUs (highlighted in pink in the table below).

Now please click ‘Next’, and you will be redirected to a page with questions to check your comprehension of the instructions.

 
% \end{document}
% \clearpage

% \part{Group bias, anticipation and performance}
% % \documentclass[12pt]{article}
% % \documentclass[12pt]{report}
% \usepackage[a4paper, total={6in, 8in}]{geometry}
% \large
% \usepackage{booktabs}
% \usepackage{setspace}
% \usepackage[hidelinks]{hyperref}
% \usepackage{graphicx}
% \usepackage{float}
% \usepackage{xcolor}
% \usepackage{lscape}
% \usepackage{setspace}
% \usepackage{authblk}


% % BIBLIOGRAPHY %%%%%%%%%%%%%%
% \usepackage[natbibapa]{apacite} % to enable '\citet' and '\citep' macros
% \bibliographystyle{aer}
% % %%%%%%%%%%%%%%%%%%%%%%%%%%%%

% \title{
% {Group bias, anticipation and performance} }\\

% \author[1]{George Agyeah }
% \author[2]{Yufei Ren}
% \affil[1]{Department of Economics, University of Arkansas}
% \affil[2]{Department of Economics, University of Minnesota-Duluth \thanks{Financial support from the \href{https://bbrl.uark.edu/}{Walton College Behavioral Business Research Lab} and Department of Economics of the University of Minnesota-Duluth are duly acknowledged.}}

% % \author{ George Agyeah } 
% \date{\today}
% % \date{November 26, 2023} 

% \begin{document}

% \maketitle 
% \begin{center}
%     \href{https://wordpress.com/block-editor/page/ag-yeah.com/1254}{ \textcolor{black}{Click here for the latest version}} 
% \end{center}
% \begin{abstract}
% \small \noindent 
% Companies utilize various methods to assess employee performance, which directly impact decisions on promotions, salaries, and recognition. Despite this, the understanding of how these evaluations and worker relationships influence employee behavior remains limited. This study addresses this gap through a two-institution lab experiment. The experiment explores how the nature of performance evaluations and group affiliation affect individuals' perception of their evaluations. By leveraging existing group settings at two universities, this study investigates how group identity impacts the perception of discrimination among 240 participants. The findings indicate a heightened perception of bias when group identity is salient. Further analysis shows that positive bias, rather than negative bias is a significant driver of actions among participants.  \\

% \noindent\tiny\textbf{Keywords: perceived discrimination, performance evaluations, experiments }\\
% \noindent\textbf{JEL Codes: C92, D83, D91} \\
% \end{abstract}
% \setcounter{page}{0}
% \thispagestyle{empty}
% \pagebreak \newpage
\pagestyle{plain} 

\doublespacing

\section{Introduction} \label{sec:introduction}
Job evaluations are crucial for workplace decisions. Many organizations use a mix of objective and subjective reviews to assess employee performance. A key discussion point is how subjectivity in these reviews impacts the actions of supervisors and employees. Economic research has identified issues linked to subjective evaluations. One major concern is discrimination documented in studies \citep{goldin2000,bertrand2004}. A comprehensive review by \cite{bertrand2016} explores evidence of discrimination in this area. Similarly, favoritism can also arise from evaluation methods \citep{prendergast1996}. A question that remains is whether employees are aware of these biases, and to what extent employees adjust their behavior accordingly to reflect their anticipation of this bias. Research shows that discrimination, whether subtle or overt, can significantly impact individuals \citep{jones2016}. 

This study explores how group affiliation influences perceived discrimination among participants in economic decision making. Participants from two U.S. universities are randomly paired to complete and/or evaluate tasks. Participants' beliefs about evaluations in scenarios with varying degrees of potential subjectivity in evaluations are elicited. The findings suggest that group affiliation noticeably affects anticipation of bias and decisions. Interestingly, the results indicate a stronger focus on gaining positive bias from in-group membership compared to concerns about negative bias from out-group members.

This study replicates a common workplace situation. Imagine an individual working in an environment where group membership of employees and supervisors are public information. Group membership can be based on demographics or any other social identifier. In this scenario, performance evaluations that a supervisor conducts can be received differently based on an employee's group membership. An employee's anticipation of bias can therefore influence their behavior. Understanding how this perception affects decision-making is crucial especially among racial and gender minorities who often find themselves as outsiders in the work place. This experiment introduces a unique method to measure perceived bias that allows separation of the reason for anticipation of bias. 

\hspace *{0mm} This study makes several contributions to the literature. Firstly, it broadens our understanding of perceived bias by investigating the impact of natural group membership on decision-making. Secondly, the paper examines how performance measures influence effort levels. This allows for a more accurate assessment of how widely used evaluation mechanisms affect both employees and supervisors. The rest of the paper is organized as follows. Section \hyperref[sec:literature]{2} presents the literature review. Section \hyperref[sec:Design]{3} details the experimental design. Section \hyperref[sec:Hypotheses]{4} introduces hypotheses. Section \hyperref[sec:Analysis]{5} discusses the empirical analyses and results in the study. Section \hyperref[sec:Conclusion]{6} concludes.


\section{Literature Review} \label{sec:literature}
\hspace *{0mm} The American workforce is rapidly becoming more diverse. This includes along factors like location, gender, and race \citep{b21}. Having a diverse workforce that functions smoothly is essential for economic success \citep{eg05}. An extant literature shows that membership of a group can influence various economic outcomes for individuals \citep{charness2020social}. Indeed, an individual's perception of themselves can be affected by their environment and their group membership. For instance, gender stereotypes can influence how women perceive their own abilities in a group setting \citep{bordalo2019beliefs}. Perception of bias can also differ by race. \cite{ruebeck2023} finds that belonging to a racial minority can heighten the perception of bias. \cite{charness2020} explore this further by investigating the role of gender in anticipating discrimination. Their findings show that men and women make different choices about revealing their gender identity based on the task at hand. Similar results are found by \cite{aksoy2023}. They show that individuals mask signals about their affinity with the LGBTQ+ community in response to anticipated discrimination in prosocial behavior. The findings show varied behavior by gender identity in situations where discrimination is a possibility.

While a crucial factor in anticipating discrimination is the type of evaluation system used, several studies demonstrate that perception of bias can impact behavior. For example, \cite{angeli2023} show that anticipating bias can affect interview performance. Similarly, \cite{bedard2019} shows that perception about ability negatively impacts competition in classroom settings with tournament-style evaluations. Further evidence by \cite{angelovski2016} suggests that managers are more likely to exhibit escalation bias in situations where subjective evaluations are possible. A recent study reveals that employers often favor their own group members for promotions, impacting the effort levels of both promoted and non-promoted individuals \citep{vdurinik2023}. 

This paper aims to investigate the connection between group membership and perceived bias. Firstly, we will examine if individuals in an environment where group identity is salient have a higher anticipation of bias. Secondly, the study examines whether workers and supervisors differ in their perception of bias. Finally, the study explores the factors that influence individuals' anticipation of bias including membership of a group that has historically experienced discrimination.


\section{Experimental Design} \label{sec:Design}

The experimental design consists of two between-subjects treatments named, group identified (henceforth, Salient Identity) and control. Two hundred and forty participants are recruited from two universities: The University of Arkansas and the University of Minnesota, Duluth. Participants' natural affiliation with their respective universities serves as the basis for grouping them. Furthermore, we utilize group identity priming to make group membership salient in Salient Identity. 

Participants are recruited from classes in both institutions. The study is conducted synchronously across the two institutions. Initially, participants are unaware they are part of a larger study spanning both universities. Each session involves eight individuals from both universities (four from each institution). Participants in the "group identified" treatment, Salient Identity are primed about their university identity after initial instructions of the structure of the experiment are read live over zoom for both sets of participants. Once initial general instructions are read, participants proceed to complete tasks in 2 segments. Segment 1 consists of 4 parts. Participants perform 2 practice rounds after initial introduction to the tasks before proceeding to paid parts. Each part starts with instructions for the part before participants attempt the tasks. In part 1, all participants complete tasks that are evaluated by the preprogrammed software - the objective evaluator. In parts 2 and 3, the tasks completed by the workers' are evaluated by either someone from their own university (in-group) or someone from the other university (out-group). In part 4, the beliefs of the participants are elicited using the Vickrey auction procedure \citep{karni2009} before task completion. There is a slight variation in part 4 for supervisors. Supervisors in part 4 complete 2 separate evaluations per round - one from an in-group partner and the other from an out-group partner.

In segment 1, participants work in pairs with either another participant or a computer. Pairing is randomly assigned to prevent reassignment with past partners and to ensure each pair encounters both in-group and out-group partners at various  parts of the study. The tasks participants are engaged in is a modified version of a task requiring participants to adjust sliders and/or evaluate the accuracy of aligned sliders \citep{gill2012}. Participants that are assigned slider completion tasks have a 75-second time frame to correctly align 30 bars of sliders. These participants are henceforth referred to as workers. Additionally, workers are given the opportunity to choose an alternative option that suits their preferences. This fosters incentive compatibility, a concept discussed in \cite{dutcher2015}. Supervisors, evaluators of completed tasks, can be a software program or a participant (in-group or out-group member). Supervisors evaluate the sliders completed by the workers. The decision of what sliders are correctly aligned is based on the evaluation of the supervisor only. 

In each session, participants are either assigned Salient Group or the Control. To make group identity salient, participants are asked to fill out one of two questionaires at the beginning of the study. The questions are subtle enough to ensure the questions make participants’ university identity salient without having participants aware that they are being primed. Participants in this treatment are asked to answer questions on why they chose this university and the role the university plays in their lives. In addition to priming, participants in Salient Identity treatment are able to see the university affiliation of their paired partner (refer to appendix for images of university affiliation). The control treatment follows the same overall structure as Salient Identity, with two crucial differences - university affiliation is not mentioned at any point during the experiment and participants are unaware of the affiliation of paired partners. Instead of the university-related identity priming questions during the introductory phase, participants in the control treatment answer questions about their cell phone usage. A summary of the segment 1 of the experiment is presented in figure 1 below. 

\begin{figure}[H]
 \captionsetup{justification=raggedright,singlelinecheck=false}
\caption{Flowchart of Segment 1 of Experimental Procedure}
\includegraphics[scale=0.6]{images/Flowchart.png} 
\end{figure}
 

 In segment 2, self-identified individual characteristics are take. Risk preferences of participants are then elicited using an incentivized lottery task \citep{eckel2008}. Within each part, workers are paid based on the sum of earnings from the number of sliders correctly aligned and the earnings from time spent on the alternative activity. Supervisors are paid a flat rate for each part of the study. Finally, it is common knowledge in both treatments that earnings of workers are based on a randomly chosen part in segment 1 and all of the payoffs from the risk elicitation portion in segment 2.   

\section{Hypotheses} \label{sec:Hypotheses}
In many organizations, performance evaluation is a common phenomenon. Many situations exist where individuals are evaluated by measures that have varying degrees of subjectivity. It is also common for individuals to feel as outsiders within the groups they work. The proliferation of subjective measures of performance and the existence of several groups of individuals in many work environments present a situation where perceived bias could affect how individuals behave at the workplace. Economic research sheds light on how group identity can influence individual decision-making. Highlighting group identity has been shown to impact various behaviors, including performance \citep{aronson1998} and perception \citep{bargh1982}. Building on this foundation, the following first hypothesis is proposed: 
 
 \textit{Hypothesis 1:  Individuals in the primed treatment have a heightened anticipation of bias in economic decision-making scenarios. }

 Existing literature explores potential incentives that can skew evaluations \citep{carpenter2010}. Our unique design allows us to investigate bias about both positive and negative bias. We define positive bias as favorable treatment, while negative bias is unfavorable treatment. Individuals who anticipate positive bias may be willing to pay to avoid objective evaluations. Conversely, a willingness to pay to avoid an evaluation by an out-group suggests the expectation of negative bias. This leads to our second hypothesis:

  \textit{Hypothesis 2: Both positive and negative bias play a significant role in participants' decision-making.}

The burgeoning research on anticipated bias has shown that individuals that have historically suffered discrimination are more likely to anticipate discrimination \citep{charness2020,aksoy2023}. Consistent with this literature, we propose our final hypothesis: 

    \textit{Hypothesis 3: Gender and racial minorities that have historically experienced discrimination will anticipate more bias regardless of work roles.}

\section{Analysis} \label{sec:Analysis}
The analysis for this section of the paper is organized into four main sections for clarity. The initial segment provides an overview of the study participants in the summary statistics section. Details of the demographic distributions of the participant are also presented in this segment. The actions and decision-making of participants that are randomly assigned workers are presented in the second section. Following this, the focus shifts to the results of the actions of participants randomly assigned supervisors during the study. Finally, the analysis delves into whether actions and decisions in the study vary by demographic characteristics.

\subsection{Summary Statistics}
A total of 240 participants are recruited from the University of Arkansas, primarily sourced through the Walton College Behavioral Business Research Laboratory Sona System, and the University of Minnesota - Duluth campus. The gender distribution of the participants is presented in table 1 below. Among the participants, 135 individuals, accounting for 56.25\% of the sample, self-identified as men, while 104 participants identify as women. Additionally, one participant did not identify with either gender identity named.

\begin{table}[H]
 \captionsetup{justification=raggedright,singlelinecheck=false}
\caption{Gender Distribution of Participants } \label{tab:table1}
        {
	\def\sym#1{\ifmmode^{#1}\else\(^{#1}\)\fi}
	\begin{tabular}{l*{1}{ccccccc}}
		\toprule
		& Info & RCL & No Info  & Feedback(t=1,2) & Feedback(t=9,10) & Description   \\
		\midrule
		20         &      120 &         59 &      120   &      8  &   16  &  60 \\
		\midrule
		30         &      240 &        118 &      240   &      21  &   22 &  120 \\
		\midrule
		50         &      480 &        236 &      480   &      55  &   51 &  240 \\
		\midrule
		70         &      240 &        118 &      240   &      24  &   21 &  120 \\
		\midrule
		80         &      120 &         59 &      120   &      12  &   10 &  60 \\
		\bottomrule
	\end{tabular}
}
\end{table}

The self-identified racial distribution of the participants is presented in table 2 below. A notable majority, comprising 74.17\% of the 240 participants, identify as white. Five percent identify as black/African American, 6.25\% identify as Hispanic, 7.92\% as Asian, 1.25\% as Native American, 1.25\% as Middle Eastern, and 3.75\% identify as being from the Indian Subcontinent. Within the participant pool, 160 individuals are assigned to the Salient Identity treatment and 80 participants are assigned to the control treatment.
\begin{table}[H]
 \captionsetup{justification=raggedright,singlelinecheck=false}
\caption{Racial Distribution of Participants} \label{tab:table2}
        \begin{table}[htbp]\centering
    \begin{tabular}{c c c c c c}
    \toprule
    \multicolumn{6}{c }{\textbf{Summary of Main Variables}}  \\
    \midrule
    Variables            &    Obs      &  Mean   &   Standard Deviation &    Min     &   Max     \\
    \midrule
    Investor Interests   &    13,044      &  2.87   &    2.55 &    1     &   34     \\
    Mean Amount Raised   &    7,940       &  1,890,525   &   2,239,282 &    1,000     &   74,000,000     \\
    Number of Females    &    13,045      &  0.29   &  0.56 &    0     &   4     \\
    Company Age          &    13,045      &  6.73   &  2.45 &    2     &   11     \\
    Number of Founders   &    13,045     &  2.04   &  1.00  &    1     &   10     \\

    \bottomrule
    \end{tabular}
 \end{table}

\end{table}

% \vfill




\subsection{Workers}
First, the study looks at the time spent on the tasks across the different treatments by workers. Each participant can choose to spend time completing tasks or switch to an alternative activity where their payment is not based on effort but the time spent on the activity. The bar graph of the seconds left when workers switch to the alternative activity are presented in figure 2 below. The results presented in the figure only consider parts of the study where workers are assigned human evaluators they are unable to change. The treated (1) refers to the Salient Group treatment and treated (0) refers to the Control.  The results show that while treated workers spend an extra 1.9 seconds on average on completing tasks, the difference is not statistically significant. The results of this part of the paper show that workers do not vary their effort based on the salience of group membership. 

\begin{figure}[H]
 \captionsetup{justification=raggedright,singlelinecheck=false}
\caption{Time Spent on Leisure by Treatment}
\includegraphics[scale=0.2]{images/Fig 1.png} 
\end{figure}

The analysis then investigates potential differences in leisure time based on the supervisor's relationship with the worker (in-group vs. out-group). The results are presented in figure 3 below. The two bars on the left represent the control treatment and the two bars on the right represent the Salient Group treatment. Within each treatment, 1 represents in-group match while 0 represent out-group match. It is evident that participants in the control group spend less time working when compared to the participants in the treatment group despite the difference being statistically insignificant. Additionally, there are no statistically significant differences by nature of pairing, in-group or out-group. Individuals in the treated group do not appear to vary effort based on who their supervisor is. 

\begin{figure}[H]
 \captionsetup{justification=raggedright,singlelinecheck=false}
\caption{Time Spent on Leisure by Treatment}
\includegraphics[scale=0.2]{images/Fig 2.png} 
\end{figure}

The paper now examines how worker behavior regarding leisure varies across all parts of the study involving human evaluators. In particular, the section seeks to examine whether there are changes in effort levels in response to the options available in different parts. The results of the analysis are presented in figure 4 below. The three bars on the left represent the Control treatment and the three bars on the right represent the Salient Group treatment. Part 1 is the first time a worker is assigned an individual evaluator, part 2 is the second time a worker is assigned a human evaluator and part 3 is the third time a worker is assigned a human evaluator. Workers in both treatments increase their leisure between parts 1 and 2 even though the difference is not statistically significant.  

\begin{figure}[H]
 \captionsetup{justification=raggedright,singlelinecheck=false}
\caption{Leisure by Parts}
\includegraphics[scale=0.2]{images/Fig 1c Leisure by Parts.png} 
\end{figure}
\subsubsection{Anticipation of Bias among Workers}

This section explores whether workers' perceptions of bias differ based on the salience of group identity. The proportion of participants willing to pay for a preferred evaluator are analyzed. In a scenario where having a preferred evaluator offers no perceived advantage, workers would not be willing to pay anything. In contrast, a willingness to pay signifies a value placed on having their preferred evaluator assess their task in this context. This anticipated value is termed as perceived bias.

Figure 5 illustrates the percentage of workers willing to pay for an evaluator from their own institution (in-group) when randomly assigned an objective computer program evaluator (control group). The treated (1) refers to the Salient Group treatment and treated (0) refers to the Control. The figure reveals that 42.5\% of the control group is willing to pay for an in-group supervisor's evaluation. However, this number increases to 65.0\% for the treated group. This statistically significant difference (P-value = 0.019) suggests an anticipation of positive bias. Furthermore, among the treated participants willing to pay, the average price is 12.32, representing 12. 32\% of the earnings. Workers are willing to pay to switch to an in-group evaluator, presumably to gain an advantage from being evaluated by an in-group supervisor.

\begin{figure}[H]
 \captionsetup{justification=raggedright,singlelinecheck=false}
\caption{Anticipation of Positive Bias}
\includegraphics[scale=0.2]{images/Fig 3 WTP for Ing vs Obj.png} 
\end{figure}

Perceived bias regarding evaluations can manifest in two ways. The first involves an expectation of positive bias, leading participants to pay for a preferred evaluator in hopes of gaining an advantage. Conversely, individuals might perceive negative bias from an evaluator and choose to pay to avoid them altogether. To explore the possibility of negative bias, this section examines the proportion of participants willing to pay for an objective supervisor when randomly assigned an out-group supervisor (someone not from their institution). The treated (1) refers to the Salient Group treatment and treated (0) refers to the Control.Figure 6 depicts the percentage of participants choosing an objective evaluator over a randomly assigned out-group evaluator. The treated (1) refers to the Salient Group treatment and treated (0) refers to the Control. There is no statistically significant difference between the treated and control groups (P-value = 0.889). This suggests that the perception of negative bias is not a key driver of decisions. However, treated workers demonstrate a nuanced understanding of bias. While willing to pay for a positive in-group evaluation (as shown earlier), they are also willing to pay to avoid out-group evaluators  when they can have in-group evaluator(P-value = 0.048). This statistically significant finding suggests that treated workers strategically used the option to gain potential bias from in-group evaluators. The average amount workers are willing to pay to avoid an out-group member is 9.19 representing 9.19\% of their payoff. This signals a higher role of positive bias in decisions instead of negative bias.


\begin{figure}[H]
 \captionsetup{justification=raggedright,singlelinecheck=false}
\caption{Beliefs about Out-group Evaluators}
\includegraphics[scale=0.2]{images/Fig 4 a WTP for obj vs out.png} 
\end{figure}

\subsubsection{Regression Analysis}

This section delves into the impact of the treatment on workers' expectations of bias using a probit regression model. This model estimates the influence of the treatment on the predicted likelihood of a participant anticipating bias. The results of this analysis are presented in Table 3. Table 3 is organized into four main columns. Columns 1 and 2 focus on the anticipation of positive bias, while columns 3 and 4 explore the anticipation of negative bias.


\begin{table}[H]
 \captionsetup{justification=raggedright,singlelinecheck=false}
    \caption{Beliefs Among Workers by Treated }
         \begin{table}[htbp]
    \begin{tabular}{c c c c c c c }
    \toprule
    &\multicolumn{2}{c}{Positive}         &\multicolumn{2}{c}{Negative}      \\

    \textbf{Variables} & \textbf{(1)} & \textbf{(2)}  & \textbf{(1)} & \textbf{(2)}              \\ 

    \textbf & \textbf & \textbf & \textbf{ Base } & \textbf   \\ 

    \midrule
     Treated           &  0.225***   & 0.235***    & 0.013     & 0.009  \\
                       & (0.084)     & (0.097)     & (0.081)    & (0.091)   \\
                       &             &             &            &                  \\
\midrule
    
\textbf{Controls} & \textbf{ NO } & \textbf{YES}  & \textbf{ NO } & \textbf{YES}  \\ 
    \midrule
     N                  &   120          &      120  &     120  &     120        \\          
    \bottomrule
    \addlinespace[1ex]
    \multicolumn{3}{l}{\textsuperscript{***}$p<0.01$, 
      \textsuperscript{**}$p<0.05$, 
      \textsuperscript{*}$p<0.1$}
    \end{tabular}
    \newline
    Note: Clustered standard errors by sessions
\end{table}

\end{table}

The results in table 3 above reveal that group membership significantly impacts the anticipation of positive bias. Workers assigned to the treatment group are 22.5\% more likely to anticipate positive bias compared to the control group (P-value $<$ 0.01). This effect remains strong even after accounting for factors like risk tolerance, age, gender, income, employment status, and religion. In fact, the influence of the treatment on anticipation of positive bias increases slightly to 23.5\% (P-value $<$ 0.01). These robust findings suggest that in subjective environments, workers expect favorable treatment from those within their group.

Moving on to the anticipation of negative bias (columns 3 and 4 of Table 3), the results show no statistically significant differences between the treated and control groups (p-value $=$ 0.877). This suggests that anticipation of negative bias is not a significant driver of workers decisions. Rational participants who expect to be treated negatively will pay an amount equivalent to the value of having an objective evaluator to avoid an out-group evaluator. Even after including control variables, the results remain consistent (P-value $=$ 0.919). 

\subsection{Supervisors}
This section examines supervisor behavior across the treatment and control groups. It is important to note that supervisors receive a fixed payment per evaluation round regardless of worker group affiliation. First, the section examines whether there is a difference in the performance of supervisors. To do this, the accuracy of the evaluations of the supervisors across treatment is examined. Figure 7 illustrates the average accuracy of supervisors in both treatments. The two bars on the left represent the Control treatment and the two bars on the right represent the Salient Group treatment. Within each treatment, 1 represents in-group match while 0 represent out-group match. The data do not reveal significant differences in the supervisor's accuracy between the treated and control sessions (P-value = 0.35). In other words, supervisor performance are unaffected by whether they are  assigned to the treatment or control group. Furthermore, within the treatments, accuracy does not vary by the group affiliation of the worker. 

\begin{figure}[H]
 \captionsetup{justification=raggedright,singlelinecheck=false}
\caption{Accuracy of Supervisor}
\includegraphics[scale=0.2]{images/Fig 5 Accuracy of supervisors.png} 
\end{figure}


This section analyzes whether supervisor performance varies throughout the evaluation process. The error rates across the three parts where the supervisors are human are presented in figure 8 below. The three bars on the left represent the Control treatment and the three bars on the right represent the Salient Group treatment. The data reveals that error rates remain consistent between parts 1 and 2. However, the error rate increases significantly in part 3, where supervisors are tasked with evaluating two workers simultaneously - one from their in-group and one from an out-group.

\begin{figure}[H]
 \captionsetup{justification=raggedright,singlelinecheck=false}
\caption{Accuracy of Supervisors by Parts}
\includegraphics[scale=0.2]{images/Fig 5 b Accuracy of supervisors by Parts.png} 
\end{figure}

\subsubsection{Perceived Discrimination}

This section considers the perception of bias among supervisors. First, a bar graph is used to investigate whether supervisors perception of positive bias varies by treatment group. The treated (1) refers to the Salient Group treatment and treated (0) refers to the Control. The results shown in figure 9 below show that there is no statistically significant difference in anticipation of positive bias among supervisors (P-value=0.70). 

\begin{figure}[H]
 \captionsetup{justification=raggedright,singlelinecheck=false}
\caption{Beliefs about Out-group Members}
\includegraphics[scale=0.2]{images/Fig 6 Supervisors willingness to avoid Outgroup.png} 
\end{figure}
Next, this section examines whether the anticipation of bias varies by treatment group and group affiliation among supervisors. Again, the treated (1) refers to the Salient Group treatment and treated (0) refers to the Control. The results are presented in figure 10 below. It is evident that there is no difference in anticipation of positive bias among treated supervisors by group affiliation (P-value = 0.89).

\begin{figure}[H]
 \captionsetup{justification=raggedright,singlelinecheck=false}
\caption{Beliefs of Supervisors }
\includegraphics[scale=0.2]{images/Fig 6 B Supervisors willingness to avoid Objective.png} 
\end{figure}

\subsubsection{Regression Analysis}

Following the analysis, a probit regression model is used to assess the likelihood that supervisors anticipate bias based on the intervention they receive. The results of this analysis are presented in Table 4 below. 

\begin{table}[H]
 \captionsetup{justification=raggedright,singlelinecheck=false}
    \caption{Beliefs Among Supervisors by Treated }
        \begin{table}[htbp]\centering
    \caption{Matrix of Correlations}
    \begin{tabular}{c c c c c}
    \toprule
    \multicolumn{4}{c}{\textbf{Correlations}}  \\
    \midrule
    Variables      &      (1)      &     (2)      &      (3)      &     (4)\\
    \midrule
    Female Founded * Maternity Score     &    1.000   &               &        &       \\

                &               &                     &        &       \\
    Female Founded * Percentage Female    & 0.763    &     1.000      &           &       \\
                &               &                   &          &       \\
    Amount Raised               &     0.024    &       -0.046        &     1.000     &       \\
                &               &                     &            &       \\
    Number of Venture Firm Interests       &     0.043   &      0.016       &      0.220     &    1.000 \\
    \bottomrule
    \end{tabular}
\end{table}
\end{table}

The results in table 4 are presented in a similar way to the results on worker anticipation of bias above. Columns 1 and 2 present the results on the probability of anticipating positive bias while columns 3 and 4 present the probability of a supervisor anticipating negative bias. It is important to note that, supervisors have been supervising up to the point the decision is made but they do not know what role they will play in the next part when they make this decision. 

The results in the base model shows that treated supervisors do not differ in their likelihood to anticipate positive bias (P-value = 0.90). The effect is consistent if additional controls are added in column 2 (P-value = 0.96). It is apparent that unlike workers, supervisors do not anticipate positive bias despite the subjectivity of the evaluation mechanism. 

The results of the probability of supervisors to anticipate negative bias are presented in columns 3 and 4 of table 4. In the base model, there is no evidence that group membership affects anticipation of bias among supervisors (P-value = 0.144). Inclusion of additional controls for risk preferences, age, gender, income, employment status and religious affiliation do not affect the significance level despite the consistency of the direction of the effect. The overall average expectation of negative among supervisors in the treated group is still not statistically significant (P-value=0.146).


\subsection{Beliefs by Demographic Qualities}
\textbf{Bias and Gender}
\newline
This section explores how gender influences worker and supervisor perceptions of bias. There is no significant difference in who anticipates positive bias based on gender when looking at all participants, regardless of job role (P-value = 0.394). However, a breakdown by job role reveals a trend. Women workers in the treated (P-value=0.082) and control (P-value=0.031) groups are more likely to anticipate positive bias compared to men. Interestingly, there is a general difference across genders where women are more likely to expect negative bias (P-value=0.013). This pattern does not hold true when looking specifically at treated (P-value=0.131) or control worker groups (P-value=0.267).

\textbf{Bias and Race}
\newline
Next, the analysis look at the impact of race in the anticipation of bias. Participants are categorized into whites and non-whites to ensure there is enough power to test the proportions of participants that anticipate bias in the evaluations. There are racial differences in how participants anticipate positive bias, with non-white participants expecting more bias than white participants regardless of job role (P-value=0.015). Furthermore, race does not significantly impact how treated or control worker groups anticipate negative bias. There is no statistically significant difference in anticipation of negative bias by race among treated workers (P-value=0.562) and control workers (P-value=0.440). 


\subsubsection{Regression Analysis}

Finally, to gain a broader understanding of how race and gender interact with bias perception, the data from all participants are analyzed using a probit model. This model analyzes how a participant's self-identified race and gender influence their anticipation of discrimination. Participants are categorized based on their race (white or non-white) and gender (male or non-male). The results of this combined analysis are presented in Table 5 below.

\begin{table}[H]
  \captionsetup{justification=raggedright,singlelinecheck=false}
   \caption{Beliefs by Demographic Qualities }
         \begin{table}[htbp]
    \begin{tabular}{c c c c c}
    \toprule
    \textbf{Variables} & \textbf{(1)} & \textbf{(2)} & \textbf{(3)}         \\ 
    \textbf & \textbf{VC Interests} & \textbf{VC Interests} & \textbf{VC Interests}   \\ 

    \midrule
    WomenLed         &    0.304*** &      0.338***  &        0.292***   \\
                        &    (0.113)  &      (0.080)   &         (0.107)  \\
                        &             &                &                  \\

                        &             &                &             \\
                        &             &                &                    \\
    Firm Age            &             &      -0.048*** &    0.070***        \\
                        &             &      (0.015)   &    (0.023)       \\
                        &             &                &                 \\
                        &             &                &                \\

Industry and Firm Controls   &   No        &   Yes       &          Yes    \\
    Funding Controls    &   No             &   No        &        Yes        \\

    \midrule
     N                  &   13044          &      13044  &     13044      \\          
    \bottomrule
  
    \addlinespace[1ex]
    \multicolumn{3}{l}{\textsuperscript{***}$P<0.01$, 
      \textsuperscript{**}$P<0.05$, 
      \textsuperscript{*}$P<0.1$}

\end{tabular}
\newline
Note: The table reports marginal effects of Tobit Model.

\end{table}

\end{table}

Similar to the previous analysis, Table 5 above presents the findings in separate columns. Columns 1 and 2 focus on positive bias anticipation, while columns 3 and 4 address negative bias anticipation. The results reveal that white participants are generally less likely to anticipate positive bias compared to non-white participants. For instance, the coefficient for white women (-0.174) indicates a 17.4\% decrease in their probability of anticipating positive bias compared to non-white women (P-value $<$ 0.01). Similarly, white men are significantly less likely to anticipate positive bias than non-white women  (P-value $<$ 0.01). This effect remains robust even after accounting for additional factors in column 2 (P-value = 0.023). 

Interestingly, the analysis shows a gender difference in negative bias anticipation (P-value= 0.038). Women, regardless of race, are more likely to anticipate negative bias compared to men. However, race does not significantly influence negative bias anticipation for non-white participants (P-value= 0.946) compared to white men. 

In summary, these findings suggest that women and racial minorities are more likely to anticipate bias, with women being more concerned about negative bias and non-white participants having a higher expectation of positive bias.


\section{Conclusion} \label{sec:Conclusion}

This study delves into the intricate interplay between group membership, perceived bias, and decision-making processes within economic contexts. By examining the behavior of participants from two U.S. universities in a controlled experimental setting, the study uncovers significant insights on how group affiliation influences beliefs and actions regarding performance evaluations. There is significant evidence that workers anticipate bias when there is subjectivity in evaluations. The nature of the anticipation of bias follows the chivalry and solidarity arguments advanced by \cite{eckel2001chivalry}. Workers, unlike supervisors anticipate solidarity from in-group members and value this effect to be about 12.32\% of their payoff. Furthermore, there is evidence demographic qualities are a big driver in the formulation of beliefs about bias. The findings, as demonstrated in the study and the complementary research referenced, underscore the nuanced dynamics at play, highlighting a stronger inclination towards seeking favorable outcomes within in-group settings compared to concerns about potential bias from out-group members.

Notably, participants' willingness to pay to avoid certain evaluators based on group affiliation indicates the real-world implications of perceived biases in evaluation processes. This study extends prior literature by providing empirical evidence on how individuals navigate subjective evaluation systems, illuminating the complexities inherent in workplace dynamics. Moreover, the investigation sheds light on the influence of demographic factors such as race and gender on perceived bias and behavioral responses. The analysis reveals disparities in anticipation of bias based on race and gender. These findings underscore the need for organizations to address systemic biases and foster inclusive environments. The observed differences in anticipation of bias among workers and supervisors, as highlighted in the study show potential areas for intervention and policy reform to mitigate discriminatory practices.

Overall, this study, when viewed alongside the complementary literature, contributes to a deeper understanding of the multifaceted factors shaping decision-making in diverse settings. By uncovering patterns of behavior and perception related to group membership and evaluation processes, this study provides valuable insights for organizations aiming to promote fairness and equity in their workforce. However, acknowledging the limitations of this study, further research is warranted to explore the broader implications of these findings and to identify effective strategies for reducing anticipation of bias and mitigating bias in the workplace.

\bibliography{mybiblio.bib}


\section{Online Appendix}




\subsection{Appendix D : Instructions}
\subsubsection{Part 0}
\textbf{Treated}
Please answer the following survey questions. Your answers will not affect your earnings during this experiment and will be used for this study only. Individual data will not be exposed.
\newline
1.What is your college mascot (Razorback Hog vs Bulldog)? 
\newline
2. Grade/Year:
\newline
(a) Freshmen 
\newline
(b) Sophomore
\newline
(c) Junior 
\newline
(d) Senior
\newline
(e) > 4 years
\newline
(f) Graduate student
\newline
3. What is your university mascot? 
\newline
4. What is the university of most of your friends?
\newline
5. Please list three things that you like the most on your university campus.
\newline
6. What made you decide to attend your university? 
\newline
7. Please rate on a 5-point scale from “strongly disagree” to “strongly agree” with the following statements.
\newline
a). Being a part of my university is an important part of who I am.
\newline
b). Being a part of my university is an important part of the image that I project. 
\newline
c). Being a part of my university is a source of pride for me. 
\newline
University of Arkansas
\newline
1. What is your college mascot (Razorback Hog vs Bulldog)? 
\newline
2. Grade/Year:
\newline
(a) Freshmen 
\newline
(b) Sophomore
\newline
(c) Junior 
\newline
(d) Senior
\newline
(e) > 4 years
\newline
(f) Graduate student
\newline
3. What is your university mascot?
\newline
4. What is the university of most of your friends?
\newline
5. Please list three things that you like the most on your university campus.
\newline
6. What made you decide to attend your university? 
\newline
7. Please rate on a 5-point scale from “strongly disagree” to “strongly agree” with the following statements.
\newline
a). Being a part of my university is an important part of who I am.
\newline
b). Being a part of my university is an important part of the image that I project. 
\newline
c). Being a part of my university is a source of pride for me. 
\newline
\textbf{Control}
\newline
Please answer the following survey questions. Your answers will not affect your earnings during this experiment and will be used for this study only. Individual data will not be exposed.
\newline
1. What is your phone service provider?  AT\&T    Verizon  T-Mobile Other
\newline
2. What is the phone service provider of most of your friends on campus?
\newline
3. Please list three characteristics that make your phone service provider different from the other phone service providers.
\newline
4. Please list three characteristics that make your phone service provider similar to the other phone service providers.
\newline
5. Please rate on a 5-point scale from “strongly disagree” to “strongly agree” with the following statements.
\newline
a). Being part of my phone service provider is an important part of who I am.
\newline
b). Being part of my phone service provider is an important part of the image that I project. 
\newline
c). Being part of my phone service provider is a source of pride for me.
\newline

\subsubsection{General Introduction} 
Welcome!  This is an experiment in decision-making.  During this experiment, you will participate in a series of tasks.  The amount of money you make will depend partly on your actions in these tasks, partly on chance and partly on other participants' actions.  Please turn off mobile phones and any other electronic devices.  They must remain turned off for the duration of this experiment.
\newline

There will be 5 (A- E) separate parts of today’s experiment involving completely separate and unrelated decision tasks.  Parts A-D contain 3 rounds each and part E is played for a round. You will go through each part separately, meaning that after we have gone through instructions for each part, you will make decisions in this part. Your total earnings will be the sum of your earnings from one randomly chosen part from parts A to D, part E and a \$5 participation fee.  Your earnings are given in experimental currency units (ECUs).  At the end of the experiment, you will be paid in private and IN CASH based on the following exchange rate:

150 Experimental Currency Units = \$1. 

You must not communicate with each other.  If you have questions, please raise your hand and an experimenter will come to help you. Before we proceed, please make sure you have a piece of paper and a pen to write with. 

\subsubsection{Instructions on Tasks}
\newline
\textbf{Slider Completion Task}
\newline
In each paying round, you will undertake a task that lasts 75 seconds.  The task will consist of a screen with 30 bars.  We will call these bars “sliders” as on each bar there is a marker you can slide along the bar with your mouse. To move the marker, you can click on the marker and drag it along. Each marker is initially positioned at a point that is different from the center.  Your task will be to set the markers on as many sliders as you choose, to a position in the center. You will earn a payoff of 32 ECUs for each slider correctly aligned. A correctly aligned slider is one evaluated to be in the center of the bar, i.e., between 48 and 52. [Indicate 48 to 52 point]

Each paying round of the slider completion task begins with 30 sliders arranged in 1 column. We will call this screen 1. On this screen, you will be able to align sliders for as much of a 75-second round as you wish. In each of the 3 paying rounds of each part, you will be assigned a role – role A and role B. Role A has to complete the slider tasks. The payoff of role A in each round is determined by the number of correct sliders aligned in screen 1, at a rate of 32 ECUs per correctly placed slider. Remember, a correctly placed slider is one evaluated to be placed in the center of the bar. Your payoff from screen 1 is calculated as sliders correctly aligned multiplied by 32 ECUs. 

Additionally, you will also have the opportunity to engage in some alternate activity available on screen 2 if you choose to do so by pressing the “Go to screen 2” button at the end of screen 1. Your potential payoff from screen 2 is based on the total time spent on screen 2. This amount drops the longer you spend on screen 1.  We have provided a simulator below to help you keep track of how much you could earn should you decide to go to screen 2. 

[To be shown on screen 2: On this screen, you can do a word jumble. For the word jumble, you will see a matrix of letters. Inside that matrix are hidden words going across, down, backwards or diagonal. When you see a word, you can type it into the text box below the matrix and click on the “send” button. A word must be a minimum of three letters to be valid. Remember, your payoff from this screen is not affected by the activity you do here but by the time spend here.]

In sum, you can spend all of your time on screen 1 and none on screen 2, all of your time on screen 2 and none on screen 1 or a mix of time spent on both. Your total payoff for a round is the sum of your earnings from screen 1 and screen 2. At the end of the screen, you can also click “See Potential Screen 2 Earnings” to use the simulator. 


[To be shown on simulator area: In order for you to more easily see how these costs work, you can click the button "See Potential Screen 2 Earnings" where you can move the handle to see how much you will earn if you switch to screen 2 at different times.]

\textbf{Evaluation Task}
\newline
In each paying round, you will undertake an identical task that lasts 120 seconds. You will evaluate tasks completed by another participant. Your task is to evaluate whether the purple marker is aligned in the center of the bar, i.e., between 48 and 52. To evaluate the bars, you must review the position of each of the markers on the sliders. Once you evaluate a bar, unclick on the checkbox located on the left side of the bar to evaluate the bar as being incorrectly aligned. 

The decision of whether a slider is correct is determined by your evaluation of the position of the marker. Please note that each bar has a different starting point and a corresponding different center.  In each of the 3 paying rounds of each part, you will be assigned a role – role A and role B. Role B has to evaluate the sliders.  In each round, the round payoff of role B is a flat rate of 450 ECUs. Once you are done evaluating all the bars, click on the “next” at the button of the page to proceed to the next page. 

Practice Round [Instructions Before they begin practice round Sliders]
We will now begin with a practice round. There are a few additions to this screen over the version for the paying rounds. First, you will note the button “Go to Screen 2”. During a paying round, you will be able to click on this button to switch to screen 2 but once you choose to go to screen 2, you will not be able to come back to screen 1 in that round. For this practice screen, you will be able to switch back and forth freely. At the end of the screen, you can also click “See Potential Screen 2 Earnings” to see your potential screen 2 earnings. Remember, during a paid round, you can switch to screen 2 when you decide you no longer want to align sliders. To see the activities available on screen 2, click “Go to screen 2”.

While on this screen, you can do a word jumble. For the word jumble, you will see a matrix of letters. Inside that matrix are hidden words going across, down, backwards or diagonal. When you see a word, you can type it into the text box at the lower right and click on the submit button. You can complete the word jumble to the best of your abilities until the time allocated is finished. These activities will be available for you to engage in as you wish. Remember, there are no earnings for completing these activities because compensation for screen 2 is calculated by the time spent on screen 2. Once you switch to this page, you will stay on it until the round ends. However, in this practice round, you can click on “Go back screen 1” to return to screen 1.
Practice Round [Instructions Before they begin practice round – Slider Evaluation]
Now, you will evaluate tasks completed by another participant. Your task is to evaluate whether the purple marker is aligned in the center of the bar, i.e., between 48 and 52. To evaluate the bars, you must review the position of each of the markers on the sliders. Once you evaluate a bar, unclick on the checkbox located on the left side of the bar to evaluate the bar as being incorrectly aligned.
Once you are done evaluating all the bars, click on the “next” at the button of the page to proceed to the next page.
\newline
\textbf{Paying Rounds: General Reminder}
\newline
The practice rounds are finished. We will now move on to the paying rounds. In each part, you will be assigned a role as: role A (Slider task completion) or role B (Slider task evaluation). Your role will be fixed throughout each part. During each part, a role A will be randomly paired with a role B. You will never be paired with the same person you are paired with in any part for the duration of this experiment. That is, you will never be paired with the same person twice during this experiment.

In each of the 3 paying rounds of each part, role A has to complete the slider tasks. Role B evaluates the slider tasks. The task screen for each player will show the time remaining, his/her university mascot and his/her pairing’s university mascot, if applicable. Please note, Role B doesn’t undertake the tasks but evaluates the tasks after their paired role A finishes.

In each round, the payoff of role A is determined by the number of correct sliders aligned in screen 1, at a rate of 32 ECUs per correctly placed slider, and the earnings from screen 2. Specifically, Role A’s payoff = sliders correctly aligned*20 ECUS + the amount earned from time spent on Screen 2. For example, if role A (slider completion) correctly aligned 15 sliders as evaluated by the evaluator and spent no time on screen 2, then the player’s payoffs are as follows:
Role A’s payoffs: 15*32 ECUs + 0 = 480 ECUs


In each round, the payoff of role B is a flat fee. Role B’s Round payoff = a flat rate of 450 ECUs. For example, if a role B evaluates the 30 sliders completed by another participant, then the player’s payoff are as follows: 
Role B’s payoffs: 450 ECUs
Are there any questions?

\subsubsection{Group Pairing}
\newline
Who’s Paired with Whom?

We will now go over the pairing rules. 

For the paying rounds, each participant will be paired with different persons or a computer, one in each part.  The pairings will be changed after each part, and no one will be paired with the same person twice.  Note, you will always be paired with someone of a different role such that each role A will always be paired with a role B. Additionally, your pair can have your university mascot or the other university mascot, if applicable. For the purposes of this experiment, a university mascot is the mascot of the university a participant currently attends.  

\newline

The pairings are done in such a way to guarantee the following: 
\newline
 
  (1). who you will be paired with does not depend on your previous actions;
  \newline

  (2). the actions you take in one part cannot affect, either directly or indirectly, the actions of the people you will be paired with in later parts;
  \newline

  (3). the actions of the person you are paired with in any given part cannot be affected by your actions in earlier parts; 
  \newline

Additionally, you will be shown information about your match on the task screens. 
A task completion role (role A) will be represented by an emoticon holding a hammer as follows:  
A task evaluation role (role B) will be represented by an emoticon holding a pen as follows:  

Furthermore, the mascots of both pairings will be represented by their mascots and shown on the task screens. 

Hence, a role A from Minnesota (University) will be represented as:  

Similarly, a role B from Arkansas (University) will be represented as:  
Finally, a computer accomplishing the task will be represented as:  

Are there any questions?  We will start the first paying round after everyone answers the following quiz question. 

\textbf{Comprehension Questions}
\newline
Quiz 1: For the 3 paying rounds, everyone will be randomly paired with 3 different persons from one’s own group.  No one’s previous decisions will have any impact on whom he/she will be paired with in later rounds. (True/False)
\newline
Quiz 2: Each part begins with a role A and a role B as a group pairing. 
(True/False)
\newline
Quiz 3: This image (see image of computer evaluator below in figures):    represents:
\newline
Quiz 4: This image (see image 2 of Arkansas Worker below in figures):    represents:



\subsubsection{Vickrey Belief Elicitation}
\newline
In this part, there are “robots” available to help you change your evaluator should you choose to. We have 100 different robots; each has a different rate. Each Robot is equally likely to be chosen. Each Robot has a rate corresponding to an integer between 1 and 100, inclusive. That is, there is a Robot that charges 1\% of the payoff, a Robot that charges 2\% of the payoff, a Robot that charges 3\% of the payoff, ... , all the way up to a Robot that charges 100\% of the payoff. A Robot that charges 100\% of the payoff will charge 100\% of the payoff of screen 1 after the change in evaluator. Similarly, a robot that charges 1\% of the payoff will charge 1\% of the payoff of screen 1 after the change in evaluator. 
Once you decide, the change is implemented by a randomly picked Robot if your willingness to pay is greater than or equal to the charges of the randomly picked Robot. Additionally, if you are willing to switch to more than one evaluator, one of your preferred evaluators is randomly chosen for the change by the software. 


In summary, if you want to be evaluated by your preferred evaluator and your willingness to pay is greater than or equal to the rate charged by the randomly chosen Robot then, you will be evaluated by your preferred evaluator.  So, the score from screen 1 will be the score as if you were evaluated by your preferred evaluator.
 Your payoff will be as follows: (100\% – \%Charge of the Robot) * Your Payoff as evaluated by your preferred evaluator + Payoff from time spent on screen 2. 
 
 
For example, if you chose 1\% ( or 100\%) as your willingness to pay to switch from your randomly assigned evaluator to a different evaluator, and the Robot randomly selected for that change charges 1\% (or 100\%), this change will be implemented. Assuming your payoff you’re your preferred evaluator is 500 ECUs. 
Your payoff will be: (100-1/100) * Your Payoff as Evaluated by your preferred evaluator + Earnings from time spent on screen 2.
Your payoff will be: (99/100) * 500 + Earnings from time spent on screen 2. 
Your payoff will be: 495 + Earnings from time spent on screen 2. 

 
For more details, please click on the drop-down button.
\newline
[To be shown under Details Button: Recall, you have been randomly assigned an evaluator from your home mascot, the other mascot or the computer in this part of the experiment. You will be asked to decide whether you will be willing to switch your randomly paired evaluator to a preferred evaluator. If you choose to switch, you will be asked to enter the percentage of your payoff you are willing to pay to switch to a preferred evaluator.  If this part is chosen for payment and your willingness to pay is greater than or equal to the charges of the randomly chosen Robot, we apply these three steps to adjust your score: 
Step 1: There are three scores in this part – computer evaluator, an evaluator with your mascot and an evaluator with the other mascot. If this part is chosen for payment, then you are randomly assigned one of the three evaluators. 
Step 2: If your willingness to pay to switch from your randomly assigned evaluator to your preferred evaluator is greater than or equal to the rate charged by the randomly chosen robot to make the change, then the change is implemented, and your screen 1 payoff reflects evaluation done by your preferred evaluator minus the charges of the Robot that executed the change. 
Step 3: If you chose to switch to more than one alternative preferred evaluators, one of your preferred choices will be randomly chosen for the switch if conditions in step 1  and step 2 are met for both alternatives. 
 
[For example, if your randomly assigned evaluator is the computer and you prefer to pay 50\% to have your preferred evaluator; if the Robot randomly chosen to implement the change charges less than or equal to 50\%, then your payoff will be calculated as follows: (100\% – \% Charges of the robot) * Your Payoff as evaluated by your preferred evaluator + Payoff from time spent on screen 2]
\newline
\textbf{Comprehension}
\newline
Quiz 1: If you chose to pay 1\% of your payoff to switch to your preferred evaluator, and the Robot randomly selected for that change charges 20\%. Will the change be implemented? 
\newline
Quiz 2: If you chose to pay 50\% of your payoff to switch to your preferred evaluator, and the Robot randomly selected for that change charges 20\%. How much will you be charged if the change is implemented? 
\newline
Quiz 3: What is the range (minimum and maximum) of charges of the Robots? 

\subsubsection{Figures}
\newline
\begin{figure}[H]
 \captionsetup{justification=raggedright,singlelinecheck=false}
\caption{Image of Computer Evaluator }
\includegraphics[scale=0.2]{figure2/C.jpg} 
\end{figure}

\begin{figure}[H]
 \captionsetup{justification=raggedright,singlelinecheck=false}
\caption{Image of Arkansas Worker }
\includegraphics[scale=0.2]{figure2/WA.jpg} 
\end{figure}

\begin{figure}[H]
 \captionsetup{justification=raggedright,singlelinecheck=false}
\caption{Screenshot of 30 bars in screen 1  }
\includegraphics[scale=0.2]{figure2/Screen1.png} 
\end{figure}

\begin{figure}[H]
 \captionsetup{justification=raggedright,singlelinecheck=false}
\caption{Screenshot of Alternative Task in screen 2  }
\includegraphics[scale=0.2]{figure2/Screen2.png} 
\end{figure}
% \end{document} 
% \clearpage

% \part{Do investors show team composition preferences in funding Startup teams? An analysis of Venture Funding}
% % \documentclass[12pt]{article}
% % \documentclass[12pt]{report}
% \usepackage[a4paper, total={6in, 8in}]{geometry}
% \large
% \usepackage{booktabs}
% \usepackage{setspace}
% \usepackage[hidelinks]{hyperref}
% \usepackage{graphicx}
% \usepackage{float}
% \usepackage{xcolor}
% \usepackage{lscape}
% % \usepackage[english]{babel}
% % %Includes "References" in the table of contents
% \usepackage[nottoc]{tocbibind}
% % \bibliographystyle{apa}

% \usepackage[natbibapa]{apacite}  % to enable '\citet' and '\citep' macros
% \bibliographystyle{aer}

% BIBLIOGRAPHY %%%%%%%%%%%%%%
% \usepackage[natbibapa]{apacite}  % to enable '\citet' and '\citep' macros
% \bibliographystyle{apacite}
% %%%%%%%%%%%%%%%%%%%%%%%%%%%%

% \title{
% {Do investors show team composition preferences in funding Startup teams? An analysis of Venture Funding}\\
% {\large University of Arkansas}\\
% % {\includegraphics{university.jpg}}
% }
% \author{George Agyeah\thanks{University of Arkansas}}


% \date{\today
% \begin{document}

% \maketitle

% \begin{abstract}
% \noindent This paper examines the relationship between gender composition of early-stage startups and investor interest, as well as funding success. While women-led startups attract greater interest from venture capitalists, the impact on funding remains unclear.  The tobit regression model suggests a decrease in mean funding for women-led startups, while the instrumental variable (IV) regression reveals a different picture. Further analysis reveals that women tend to gravitate towards underfunded industries. Additionally, female-led startups exhibit lower funding performance compared to their peers within those industries, and they are less likely to enter sectors with more balanced funding, such as technology and finance.  \\
% \vspace{0in}\\
% \noindent\textbf{Keywords: Corporate Finance, Gender, Startups}\\
% \vspace{0in}\\
% \noindent\textbf{JEL Codes:G3, J16, J82 } \\

% \end{abstract}
% \setcounter{page}{0}
% \thispagestyle{empty}
% \title{
% {Do investors show team composition preferences in funding Startup teams? An analysis of Venture Funding} }\\

% \maketitle 

\pagestyle{plain} 


\section{Introduction} \label{sec:introduction}



Despite significant progress towards gender parity in many professions, women remain underrepresented in Science, Technology, Engineering, and Mathematics (STEM) fields. This disparity extends to entrepreneurship, where women founders are a distinct minority, even within industries dominated by female workers. The data paints a clear picture: in 2019, women made up over half (51.8\%) of the workforce in management, professional, and related occupations. Women have also surpassed men in certain sectors, such as education and health (74.8\% women), financial services (52.6\% women), and leisure and hospitality (51.2\% women). \footnote[1]{https://www.bls.gov/opub/reports/womens-databook/2020/home.htm} However, existing literature suggests that women entrepreneurs make up only 10-15\% of all entrepreneurs. Understanding the root causes of this phenomenon is crucial in developing strategies to bridge the gender gap.

The influence of startups extends far beyond individual lives. Household names like Uber, Facebook, Google, and Tesla all relied on VC funding during their early stages. Given the significance of both startups and the VC firms that support them, understanding how team composition affects a company's success becomes crucial. This study examines how the gender makeup of a startup team impacts their ability to attract investor interest and secure funding during the critical seed stage. Seed funding is vital. It provides the initial resources to develop the product and propel the company forward. Venture capital (VC) plays a central role at this stage, with over 60\% of US companies achieving an initial public offering (IPO) having received VC financing \citep{kaplan2010}. VC-backed companies demonstrably drive innovation, contributing to 44\% of R\&D activities among US public companies \cite{gornall2020}.

This study focuses on a specific question:   Does the gender makeup of a founding team influence their success in attracting venture capital interests and funds?  In simpler terms, do teams with different gender configurations experience variations in their fundraising interests and outcomes?

\section{Literature Review} \label{sec:literature}
\hspace *{0mm} Investigations on the gender gap in entrepreneurship have gained prominence in the past few years. Studies by \cite{gompers2017diversity} reveal a concerning trend: women comprise less than 10\% of the entrepreneurial and venture capital workforce, highlighting a substantial gender disparity. Interestingly, \cite{gompers2017diversity} also find a positive correlation between gender diversity in investor firms and improved deal flow and fund performance. \cite{koning2019} explore how the gender composition of entrepreneurial teams influences the nature of their innovations. Their findings suggest that teams with female inventors are more likely to focus on patents related to women's health, hinting at a potential link between inventor demographics and the direction of innovation efforts. Similarly, \cite{einio2019} show that innovators tend to create products that resonate more with customers who share similar characteristics, including gender with the innovators.

Despite the potential for innovation, securing funding can be an unequal process for diverse startups. Research by \cite{ewens2020} exposes a concerning bias in investor behavior. The results of their study reveal that male investors express less interest in women entrepreneurs. This bias can create a significant hurdle for women seeking funding for their ventures. A compelling study by \cite{calder2021} examines the impact of the high-profile gender discrimination case of Ellen Pao v. Kleiner Perkins on the industry. Their findings suggest that the case led to increased hiring of women partners in relevant venture firms. This shift in hiring practices ultimately resulted in more funding being directed towards Women$-$led startups. This highlights a potential ripple effect from legal action promoting diversity. 

Building on this research, this study aims to contribute to a deeper understanding of the gender gap in entrepreneurship. Through an empirical investigation, The paper analyzes how the gender makeup of entrepreneurial teams impacts their ability to secure funding and attract interests from venture capital firms. By addressing these aspects, this paper hopes to illuminate the challenges women face and identify opportunities to foster a more equitable environment for innovation and success.

\section{Data} \label{sec:data}
\subsection{Data Construction}
\hspace *{0mm} My primary data source for this study is proprietary data from Crunchbase, a comprehensive platform often referred to as the "LinkedIn of company data".  Originally part of TechCrunch, Crunchbase became a separate entity focused on empowering investors with data-driven decision making. While the data is self-reported by startups, Crunchbase employs verification methods using two sources: investor firms and the startups themselves.  Startups have a vested interest in accurate reporting as it signals confidence to potential investors. Investors benefit by publicizing their investments, enhancing their portfolio companies' appeal and potential returns.

The Crunchbase database encompasses information on companies, key personnel, and investment activities. To facilitate this analysis, I integrate various datasets within the Crunchbase platform. Table 1 below summarizes these key datasets, which cover startups, founders, investors, venture funding organizations, and fundraising outcomes. The data spans from the 1980s to 2021, but data comprehensiveness weakens for earlier years. Therefore, the analysis focuses on the 2010-2020 timeframe. Each dataset in Table 1 includes relevant variables. For example, the "people" dataset provides names, genders, locations, affiliations, and social media links for founders and investors. Similarly, the "investments" dataset tracks investment details like date, investor(s), recipient organization, and the specific funding event.

\begin{table}[H]
 \captionsetup{justification=raggedright,singlelinecheck=false}
    \caption{Summary Statistics: Companies}
        {
	\def\sym#1{\ifmmode^{#1}\else\(^{#1}\)\fi}
	\begin{tabular}{l*{1}{ccccccc}}
		\toprule
		& Info & RCL & No Info  & Feedback(t=1,2) & Feedback(t=9,10) & Description   \\
		\midrule
		20         &      120 &         59 &      120   &      8  &   16  &  60 \\
		\midrule
		30         &      240 &        118 &      240   &      21  &   22 &  120 \\
		\midrule
		50         &      480 &        236 &      480   &      55  &   51 &  240 \\
		\midrule
		70         &      240 &        118 &      240   &      24  &   21 &  120 \\
		\midrule
		80         &      120 &         59 &      120   &      12  &   10 &  60 \\
		\bottomrule
	\end{tabular}
}
\end{table}


\hspace *{0mm} The analysis focuses on the seed funding stage, which is the critical period where startups seek initial external funding to propel their growth.  Venture capital firms play a central role in seed funding.  The broader startup ecosystem progresses through four stages: pre-seed, seed, post-seed series, and finally, the exit stage. Startups may revisit these stages based on funding needs. Given the data's nature, I consider two key metrics: the average amount of capital raised per startup and the average number of venture capital firm interests a startup garners. Table 2 summarizes these variables. The average seed funding amount in the data set is approximately \$1.9 million, which is slightly lower than the broader ecosystem's average of \$2.2 million \footnote[2]{https://www.fundz.net/what-is-series-a-funding-series-b-funding-and-more}. Similarly, the average number of interested venture capital firms per startup is 2.87, with a range of 1 to 34 firms expressing interest per startup.

Importantly, women founders are underrepresented in the data set, comprising only about 11\%.  Regarding other startup characteristics, the average company age is 6.73 years (range: 2-11 years). The number of founders per startup also varies, with a minimum of 1 and a maximum of 10, and an average founding team size of 2.04 people.  This data will be crucial for analyzing how gender makeup of founding teams impacts their ability to secure funding from VCs.

\begin{table}[H]
 \captionsetup{justification=raggedright,singlelinecheck=false}
    \caption{Summary Statistics: Companies}
        \begin{table}[htbp]\centering
    \begin{tabular}{c c c c c c}
    \toprule
    \multicolumn{6}{c }{\textbf{Summary of Main Variables}}  \\
    \midrule
    Variables            &    Obs      &  Mean   &   Standard Deviation &    Min     &   Max     \\
    \midrule
    Investor Interests   &    13,044      &  2.87   &    2.55 &    1     &   34     \\
    Mean Amount Raised   &    7,940       &  1,890,525   &   2,239,282 &    1,000     &   74,000,000     \\
    Number of Females    &    13,045      &  0.29   &  0.56 &    0     &   4     \\
    Company Age          &    13,045      &  6.73   &  2.45 &    2     &   11     \\
    Number of Founders   &    13,045     &  2.04   &  1.00  &    1     &   10     \\

    \bottomrule
    \end{tabular}
 \end{table}

\end{table}

% \vfill

\subsection{Summary Statistics}

\hspace  *{0mm} To comprehensively understand the startups under consideration, various data points for each founding team are collected. This data encompasses the team's gender composition, educational background, size, company age, industry, funding outcomes, number of venture capitalist investors, location, funding year, and current status. Table 3 summarizes the key observations within the final dataset. The dataset includes 13,045 startups. Notably, 3,074 (or 23.56\%) of these startups have at least one woman founder. I refer to these startups as Women$-$Led. All-men teams make up the remaining firms. 

\begin{table}[H]
 \captionsetup{justification=raggedright,singlelinecheck=false}
    \caption{Summary Statistics: Companies}
         \begin{table}[htbp]
    \begin{tabular}{c c c c c c c }
    \toprule
    &\multicolumn{2}{c}{Positive}         &\multicolumn{2}{c}{Negative}      \\

    \textbf{Variables} & \textbf{(1)} & \textbf{(2)}  & \textbf{(1)} & \textbf{(2)}              \\ 

    \textbf & \textbf & \textbf & \textbf{ Base } & \textbf   \\ 

    \midrule
     Treated           &  0.225***   & 0.235***    & 0.013     & 0.009  \\
                       & (0.084)     & (0.097)     & (0.081)    & (0.091)   \\
                       &             &             &            &                  \\
\midrule
    
\textbf{Controls} & \textbf{ NO } & \textbf{YES}  & \textbf{ NO } & \textbf{YES}  \\ 
    \midrule
     N                  &   120          &      120  &     120  &     120        \\          
    \bottomrule
    \addlinespace[1ex]
    \multicolumn{3}{l}{\textsuperscript{***}$p<0.01$, 
      \textsuperscript{**}$p<0.05$, 
      \textsuperscript{*}$p<0.1$}
    \end{tabular}
    \newline
    Note: Clustered standard errors by sessions
\end{table}

\end{table}
 
\hspace *{0mm} Next, the distribution of startups across industries is examined. \footnote[3]{I use refinitiv industry classification, the details of which can be found at the website: $www.refinitiv.com/content/dam/marketing/en_us/documents/quick-reference-guides/trbc-business-classification-quick-guide.pdf$}. Figure 1 presents a time series visualization of the number of startups growth by industry between 2010 and 2020 in two panels. The left side of the figure depicts the all men$-$led startups, while the right side shows Women$-$led startups. Overall, the technology industry dominates the landscape of seed-seeking startups, constituting roughly 37\% of the data set. Consumer goods follows distantly at 19.7\%, with industries like industrials, education, government institutions, finance, and healthcare trailing behind. Notably, both all men$-$led and Women$-$led startups exhibit an upward trend in numbers since 2010. However, a significant decrease is observed in late 2019, potentially due to the global COVID-19 pandemic. Figure 2 delves deeper into the specific proportions of all-men to Women$-$led startup teams across industries.

\begin{figure}
 \captionsetup{justification=raggedright,singlelinecheck=false}
\caption{Time Series of Startups by Industries }
\includegraphics[scale=0.4]{figures_3/Figure1.png} 
\end{figure}


\hspace *{0mm} Figure 2 shows a breakdown of all-men versus Women$-$led startups across different industries. All men$-$led startups dominate the technology and financial sectors with a majority 84.3$\%$ and 80.7$\%$ of startups respectively being all men$-$led. Women$-$led startups account for 42.1$\%$ of the utilities sector and 29.2$\%$ of the institutions based startups. Women-led startups do not form a majority in any of the industries. This is not surprising considering the evidence that has been shown in other literature. On the extreme end left is the basic materials industry which is 100\% all men$-$led. This is partly explained by the number of companies in the basic materials industry being only 5, representing 0.04\% of the total. 

\begin{figure}
 \captionsetup{justification=raggedright,singlelinecheck=false}
\caption{Time Series of Startups by Industries Grouped According to Founding Team Gender Composition }
\includegraphics[scale=0.4]{figures_3/Participation.png} 
\end{figure}

\section{Results} \label{sec:Empirical Strategy}

\subsection{Empirical Strategy}

This section outlines how the paper analyzes the impact of gender composition on two key startup outcomes during the seed funding stage: the number of interested venture capital firms and the average amount of capital raised. The Crunchbase data only includes companies with at least one investor. Hence, an ordinary least squares regression model would not be suitable. This is because the number of interested firms has a lower bound of 1.  To account for this "censored" data, I employ a Tobit model. This technique allows for a more accurate assessment of how team gender composition influences the number of venture capital firms interested in a startup. Additionally, analyzing the total amount raised can be challenging because the number of funding rounds varies across startups. Therefore, I focus on the average amount raised during the seed stage.  Furthermore, to address potential skewness in the data distribution of funding amounts, a common practice in economics, the natural logarithm (log) of the mean amount raised is used in the regression analysis.

The analysis proceeds in two main parts. First, this section of the paper examines how the gender composition of the founding team influences investor interest and overall funding outcomes for startups.The analyses explore the specific impact of having women founders on startup success in securing funding and attracting venture firm interests. Second, I utilize instrumental variable regression to verify the robustness of the findings. This technique helps isolate the causal effect of having a woman on a startup team on the two key outcomes: the number of interested venture capital firms and the amount of seed funding raised.

\subsection{Tobit Regression}
\hspace *{0mm} To understand the relationship between the gender composition of a startup team on its success, I estimate the following tobit regression model using equation 1 below: 

\begin{center}
$Y_{i}=\alpha +\beta_1WomenLed_i+\beta_2X_{i} + \beta_{3}\chi_{k} + \beta_4\omega_{j}  + \epsilon_{i} \quad $(1)$ $
\end{center}
        
\noindent where $Y_{i}$ is either the number of venture firms that made investments or the amount of money invested in startup i during the seed stage of fundraising;  $WomenLed_i$ is dummy variable that is 1 if the startup team has a woman in the founding team, $X_{i}$ is the startup (firm) related characteristics, $\chi_{k}$ is the funding related characteristics, $\omega_{j} $ is the state related characteristics and $\epsilon_{i}$ is the error term. 

The first outcome variable I investigate is whether being a Women$-$Led startup affects the venture firm interests garnered.As early stated, in the context of venture capital financing, a startup typically receives its first outside investment from a venture capital firm, and this investment marks the beginning of the startup's engagement with outside investors. For this reason, the minimum number of investor firm interests is censored on the lower end at 1 in the tobit regression to  estimate the effect of the presence of a woman on the startup team on the number of venture firm interests attracted. The results of the tobit model are presented in table 4 below. 


\begin{table}[H]
 \captionsetup{justification=raggedright,singlelinecheck=false}
    \caption{Tobit: Number of Venture Firm Investments}
         \begin{table}[htbp]
    \begin{tabular}{c c c c c}
    \toprule
    \textbf{Variables} & \textbf{(1)} & \textbf{(2)} & \textbf{(3)}         \\ 
    \textbf & \textbf{VC Interests} & \textbf{VC Interests} & \textbf{VC Interests}   \\ 

    \midrule
    WomenLed         &    0.304*** &      0.338***  &        0.292***   \\
                        &    (0.113)  &      (0.080)   &         (0.107)  \\
                        &             &                &                  \\

                        &             &                &             \\
                        &             &                &                    \\
    Firm Age            &             &      -0.048*** &    0.070***        \\
                        &             &      (0.015)   &    (0.023)       \\
                        &             &                &                 \\
                        &             &                &                \\

Industry and Firm Controls   &   No        &   Yes       &          Yes    \\
    Funding Controls    &   No             &   No        &        Yes        \\

    \midrule
     N                  &   13044          &      13044  &     13044      \\          
    \bottomrule
  
    \addlinespace[1ex]
    \multicolumn{3}{l}{\textsuperscript{***}$P<0.01$, 
      \textsuperscript{**}$P<0.05$, 
      \textsuperscript{*}$P<0.1$}

\end{tabular}
\newline
Note: The table reports marginal effects of Tobit Model.

\end{table}

\end{table}

The tobit model shows that the composition of a startup team has statistically significant impact on the number of VC interests the startup attracts. The results from column 1 of table 4 show that Women$-$Led teams on average attract 0.304 more venture firm interests (P-value$<$0.001). The results in column 2 of table 4 show that the effect of a startup team having a woman co-founder is robust to the inclusion of industry and firm controls. In fact, the magnitude of the effect of being a Women$-$Led is higher at 0.338 increase in venture interest on average for Women$-$Led startups (P-value $<$ 0.001). The effect of a startup being Women$-$Led stays consistent and statistically significant (P-value $<$ 0.001) in column 3 which includes funding controls as well as industry and firm controls. Overall, the impact of being a Women$-$Led startup is robust to additional controls. The most conservative effect size of 0.292 represents 11.45\% of the average number of venture firm interests. The estimate for the effect of firm age on venture firm interest is inconsistent despite being statistically significant (P-value $<$0.01). In column 2, firm age is negatively correlated with venture firm interests while in column 3, the age of a firm is positively correlated with venture firm interests. 

Next, I look at the impact of the gender composition on funding realizations during the seed stage of investments. The base specification regresses the log of the mean amount raised by a startup on a categorical variable of the startup being Women$-$Led as expressed in equation 1 above. Similar to the analysis on the number of venture firm interests attracted, the tobit regression model is utilized and the lower bound of the data is censored (now \$1000) to account for the dynamic of the data set. The results of the tobit model are presented in table 5 below. The base model presented in column 1 of table 5 shows that the presence of a woman in a startup team is negatively related to the amount the team raises (P-value$<$0.001). Women$-$Led startups raise on average 16.64\% less money than homogeneous men startup teams. The magnitude of the effect reduces as additional controls are added in column 2 and column 3 of table 5 but the direction and robustness of the effect remains consistent. In column 2 of table 5, the addition of controls for the startup and industry characteristics reduces the magnitude of the effect to 13.32\% less funds raised on average by Women$-$Led startups as compared to all$-$men startup teams (P-value$<$0.001). Finally, in column 3 of table 5, the results of the estimate for the impact of gender composition on fund raising shows that Women$-$Led startups raise 14.27\% less money as compared to homogeneous men teams (P-value$<$0.001). This result reveals an interesting mechanism at play. It is apparent that, the increase in venture firm interests that Women$-$Led startups gain does not translate into an increase in amount raised by the Women$-$Led startups. In essence, while the presence of a woman on a team is helpful in attracting venture firm interests, the intensive margin of how much is raised is negatively affected. Furthermore, it appears older firms on average raise less money (P-value$<$0.001).

\begin{table}[H]
 \captionsetup{justification=raggedright,singlelinecheck=false}
    \caption{Tobit: Funds Realized}
         \begin{table}[htbp]
    \begin{tabular}{c c c c c}
    \toprule
    \textbf{Variables} & \textbf{(1)} & \textbf{(2)} & \textbf{(3)}               \\ 
    \textbf & \textbf{Log Amount} & \textbf{Log Amount} & \textbf{Log Amount}  \\ 

    \midrule
     Women$-$Led        & -0.182***   &      -0.143*** &       -0.154***   \\
                        &  (0.042)    &      (0.034)   &        (0.034)  \\
                        &             &                &                \\

                        &             &                &               \\
                        &             &                &         \\
    Firm Age            &             &      -0.149*** &     -0.054***    \\
                        &             &      (0.006)   &      (0.010)       \\
                        &             &                &                  \\
    
Industry and Firm Controls   &   No        &   Yes       &        Yes    \\
    Funding Controls    &   No             &   No        &         Yes        \\

    \midrule
     N                  &   7940          &      7940  &     7940        \\          
    \bottomrule
    \addlinespace[1ex]
    \multicolumn{3}{l}{\textsuperscript{***}$p<0.01$, 
      \textsuperscript{**}$p<0.05$, 
      \textsuperscript{*}$p<0.1$}
    \end{tabular}
    Note: The table reports marginal effects of Tobit Model.
\end{table}

\end{table}

% \vfill

\subsection{IV Regression}
The analysis above could be biased due to omitted variables that might influence both the percentage of women founders and the funding outcomes (number of investors and amount raised).  One concern is that successful startups might strategically add women founders to their teams after achieving initial traction. To address this and strengthen the findings, I employ Instrumental Variable (IV) regression. An effective instrument in this context would be a variable correlated with the presence of women founders but not directly linked to the funding outcomes. For this purpose, I utilize a startup's state maternity leave benefits as an instrumental variable. The rationale is that states with generous maternity leave policies are likely to have a higher percentage of women founders, as these policies allow women to return to work after childbirth.

This instrumental variable approach is supported by prior research. \cite{dustmann2012} demonstrate a strong correlation between expanded maternity benefits and a mother's return to work. Similarly, \cite{gottlieb2022} found that improved maternity benefits in Canada led to an increase in women starting businesses.  These studies support the notion that maternity leave policies can influence women's career choices, including entrepreneurship. To examine the potential link between state maternity leave policies and startup success, maternity benefit score developed by the National Partnership for Women and Families (NPWF) is used \footnote[4]{Information about them is accessible at: https://www.nationalpartnership.org/about-us/}. This non-governmental organization (NGO) advocates for policies that improve work-life balance for families, with a particular focus on working women. Every two years since 2012, the NPWF releases a report that grades each state's laws regarding paid leave, family leave, and maternity leave. Their goal is to encourage states to enact legislation that supports working mothers and their career aspirations. For the analysis, the average maternity benefit score for each state across the years 2012, 2014, 2016, and 2018 is calculated. This average score provides a more comprehensive picture of a state's maternity leave policies compared to a single year's data. Given my focus on the 2010-2020 time frame, this average score is representative of the maternity leave landscape during that period.

The data reveals a significant disparity in maternity leave policies across states.  States like Alabama, Idaho, Michigan, Mississippi, South Dakota, and Wyoming consistently receive the lowest grades (0 points) throughout the measured period. Conversely, California, Washington D.C., Connecticut, and New Jersey consistently rank at the top for their robust maternity leave policies. While I expect a state's maternity leave score to correlate with the percentage of women employed within that state and the proportion of women entrepreneurs, it is unlikely the maternity policy will directly influence the number of venture capital interests a startup gets or funding secured by the startups except through the presence of a woman on the team. The primary influence of maternity leave policies is therefore likely to be felt indirectly, through its impact on the presence of women on a startup team. In the first stage specification of the instrumental variable regression, the mean maternity leave score is considered as an instrument for the presence of women in the startup team. The first stage is estimated using equation 2 below: 
\begin{center}
$WomenLed_{i}=\alpha +\beta_1Maternity Score_i+\beta_2X_{i} +\beta_{3}\chi_{k} + \beta_4\omega_{j}  + \epsilon_{i} \quad $(2)$ $
\end{center}

where $WomenLed_{i}$ is a dummy variable that takes the value 1 if the startup has a woman on the team during the seed stage of fundraising;  $Maternity Score_i$ is the maternity leave score of the state associated with the startup, $X_{i}$ is the startup (firm) related characteristics, $\chi_{k}$ is the funding related characteristics, $\omega_{j}$ is the state related characteristics and $\epsilon_{i}$ is the error term. 
        
\hspace *{0mm} The results of the first-stage regression is presented in table 6 below. The table shows that the maternity leave score of the startup impacts the probability of the startup being Women$-$Led (P-value $<$ 0.001). A standard deviation increase in the mean maternity leave score increases the probability of the startup having a woman by 19.53\% . This effect is robust to firm and industry controls (P-value$<$0.001) and increases to 20.93\%. Furthermore, the inclusion of industry and firm controls as well as funding controls also strengthens the magnitude and the statistical significance of the relationship to 21.6\% increase per one standard deviation increase in the maternity leave score of the startup (P-value$<$0.001). It is also worth noting that the likelihood ratio test is used as a test statistic for assessing the significance of the model. The maternity leave score of the state of the startup is a significant predictor of the presence of a woman on the startup team at the 10\% significance level. 

\begin{table}[H]
 \captionsetup{justification=raggedright,singlelinecheck=false}
    \caption{First Stage Regression}
        % First Stage for Both
 \begin{table}[htbp]
 \centering
    \begin{tabular}{c c c c c}
    \toprule
    \textbf{Variables} & \textbf{(1)} & \textbf{(2)} & \textbf{(3)}                 \\ 
    \textbf & \textbf{WomenLed  } & \textbf{WomenLed } & \textbf{WomenLed }   \\ 

    \midrule
    Maternity Score      &    0.0042*** &      0.0045***  &      0.0047***   \\
                         &    (0.0008) &     (0.0008)   &   (0.0008)  \\
                         &             &                &                \\

                        &             &                &          \\
                        &             &                &           \\
    Firm Age            &             &   -0.037***   &     0.012        \\
                        &             &      (0.0095)  &   (0.015)       \\
                        &             &                &                \\
                        &             &                &               \\
Constant                &  -1.385***  &   -13.25       &    -13.310              \\
                        &   (0.053)   &  (-143.800)   &     (143.300)           \\
                        &             &                &              \\

Industry and Firm Controls   &   No        &   Yes       &       Yes    \\
    Funding Controls    &   No             &   No        &      Yes        \\

    \midrule
     N                  &   13044          &      13044  &      13044      \\          
    \bottomrule
    \addlinespace[1ex]
    \multicolumn{3}{l}{\textsuperscript{***}$P<0.01$, 
      \textsuperscript{**}$P<0.05$, 
      \textsuperscript{*}$P<0.1$}
    \end{tabular}
\end{table}

\end{table}

Following my confirmation of the validity of the IV variable, I run the reduced form regressions of the instrumental variables using equation 3 below: 

\begin{center}
$Y_{i}=\alpha +\beta_1Maternity Score_i+\beta_2X_{i} + \beta_{3}\chi_{k} + \beta_4\omega_{j}  + \epsilon_{i} \quad $(3)$ $
\end{center}

\noindent where $Y_{i}$ is either the number of venture firms that made investments or the amount of money invested in startup i during the seed stage of fundraising;  $Maternity Score_i$ is the maternity leave score of the state associated with the startup, $X_{i}$ is the startup (firm) related characteristics, $\chi_{k}$ is the funding related characteristics, $\omega_{j} $ is the state related characteristics and $\epsilon_{i}$ is the error term. 

Table 7 below presents the effect of maternity leave score associated with the startup on venture firm interests attracted. Similar to the other results discussed above, column 1 presents the base model, column 2 presents the results of the base model with additional controls for industry and firm characteristics of the startup. The results of the base model shows that, a one standard deviation increase in the maternity leave score of startup is associated with 0.51 increase in venture firm interests (P-value$<$0.001). The addition of firm and industry controls in column 2 of table 7 diminishes the size of the effect to 0.23 increase in the number of venture firm interests attracted but the effect is still statistically significant (P-value$<$0.001). The inclusion of additional controls in column 3 slightly affects the magnitude of the effect but not the significance. A one standard deviation increase in the maternity leave score associated with the startup increases the number of venture firms attracted to 0.33 (P-value$<$0.001). It is noteworthy that the direction of the effect of the maternity leave score on the number of venture firm interests attracted is consistent. Additionally, the effect of the age of a firm on the number of venture interests attracted is statistically significant (P-value$<$0.001) but inconsistent. 

\begin{table}[H]
 \captionsetup{justification=raggedright,singlelinecheck=false}
    \caption{Reduced Form Regression}
        % Reduced Form VC Interests
 \begin{table}[htbp]
    \begin{tabular}{c c c c}
    \toprule
    \textbf{Variables} & \textbf{(1)} & \textbf{(2)} & \textbf{(3)}               \\ 
    \textbf & \textbf{VC Interests} & \textbf{VC Interests} & \textbf{VC Interests}  \\ 

    \midrule
    Maternity Score     &    0.011*** &      0.005***    &    0.0065***   \\
                        &    (0.0014)  &      (0.0013)   &   (0.0012)  \\
                        &             &                  &               \\

                        &             &                  &               \\
                        &             &                  &          \\
    Firm Age            &             &      -0.048***   &    0.080***        \\
                        &             &      (0.015)     &   (0.023)       \\
                        &             &                  &               \\
                        &             &                  &            \\

Industry and Firm Controls   &   No        &   Yes       &        Yes    \\
    Funding Controls    &   No             &   No        &       Yes        \\

    \midrule
     N                  &   13044          &      13044  &      13044      \\          
    \bottomrule
    \addlinespace[1ex]
    \multicolumn{3}{l}{\textsuperscript{***}$P<0.01$, 
      \textsuperscript{**}$P<0.05$, 
      \textsuperscript{*}$P<0.1$}
    \end{tabular}
    \newline
    Note: The table reports marginal effects of Tobit Model.
\end{table}

\end{table}

Next, the results of the reduced form regression of the effect of a startup being Women−led on the mean amount raised by the startup are presented in table 8 below, replacing the independent variable with the maternity leave score associated with the startup. Again, I present the tobit model results because firms enter my dataset once they receive an investment. In the base model, a standard deviation increase in the maternity leave score is associated with an increase in the mean amount raised by 49.19\% (P-value$<$0.001). This effect diminishes as additional controls are added. In column 2, the effect size diminishes to 24.44\% increase in amount raised per a standard deviation increase in the maternity leave score of the state of the startup (P-value$<$0.001). The effect of a one standard deviation increase maternity leave score accounts for a 27.37\% increase in mean amount raised by a startup once controls for industry, firm and funding are accounted for (P-value$<$0.001). On the contrary, older firms raise significantly less at the time of their seed stage (P-value$<$0.001). The effect represented 13.58\% (column 2) and 4.50\% (column 3) less for an additional year spent in the seed stage. 

\begin{table}[H]
\captionsetup{justification=raggedright,singlelinecheck=false}
 \caption{Reduced Form Regression}
        % Reduced form log amount
 \begin{table}[htbp]
    \begin{tabular}{c c c c }
    \toprule
    \textbf{Variables} & \textbf{(1)} & \textbf{(2)} & \textbf{(3)}             \\ 
    \textbf & \textbf{Log Amount} & \textbf{Log Amount} & \textbf{Log Amount}   \\ 

    \midrule
    Maternity Score     &   0.0086*** &      0.0047*** &     0.0052***   \\
                        &  (0.0006)  &      (0.0003)   &       (0.0005)  \\
                        &             &                &                \\

                        &             &                &               \\
                        &             &                &                \\
    Firm Age            &             &      -0.146*** &      -0.046***    \\
                        &             &      (0.006)   &      (0.010)       \\
                        &             &                &                    \\
   Industry and Firm Controls   &   No        &   Yes       &        Yes    \\
    Funding Controls    &   No             &   No        &     Yes        \\

    \midrule
     N                  &   7940          &      7940  &      7940      \\          
    \bottomrule
    \addlinespace[1ex]
    \multicolumn{3}{l}{\textsuperscript{***}$P<0.01$, 
      \textsuperscript{**}$P<0.05$, 
      \textsuperscript{*}$P<0.1$}
    \end{tabular}
    \newline
    Note: The table reports marginal effects of Tobit Model.
\end{table}

\end{table}

Finally, the estimates of the structural model are presented. The structural model of the instrumental variable specification utilizes the estimates of the dependent variable from the first stage specified in equation 2 above. Classification of the firms is based on a threshold of the predicted value being above the third quartile. Using the prediction, the structural model of the effect of a startup being Women−Led on its outcome is estimated using equation 4 below:

\begin{center}
$Y_{i}=\alpha +\beta_1\widehat{Women-Led}_i+\beta_2X_{i} + \beta_{3}\chi_{k} + \beta_4\omega_{j}  + \epsilon_{i} \quad (4) $
\end{center}

\noindent where $Y_{i}$ is either the number of venture firms that made investments or the amount of money invested in startup i during the seed stage of fundraising;  $\widehat{Women-Led_i}$ is the estimated classification of the startup as Women$-$Led, $X_{i}$ is the startup (firm) related characteristics, $\chi_{k}$ is the funding related characteristics, $\omega_{j}$ is the state related characteristics and $\epsilon_{i}$ is the error term. 

First, the structural specification of the model is used to examine the impact of a startup being Women$-$Led on the number of venture firm interests attracted. The results of the estimation are presented in table 9 below. On average, Women$-$Led startups attract 0.683 more venture firm interests as compared to the homogeneous men teams (P-value $<$ 0.001). The addition of controls for industry and firm characteristics of the startup does not affect the direction and the significance of the effect. However, the magnitude of the effect diminishes to 0.186 increase in the number of venture firm interests attracted on average for Women-Led startups (P-value $<$ 0.1). In column 3 of table 9, the effect is robust to the inclusion of all the controls. Women$-$Led startups on average attract 0.193 more venture firm interests. This result collaborates the finding from the tobit model estimate presented above in table 5. The consistency of this finding points to a situation where firms recognise the dearth of representation and support women-led teams. Additionally, the effect of firm age on funds raised is inconsistent. In column 2, older firms appear to attract less interest (P-value $<$ 0.001) but the effect is the opposite in column 3 (P-value $<$ 0.001).

\begin{table}[H]
 \captionsetup{justification=raggedright,singlelinecheck=false}
    \caption{Structural Form Regression}
        % Structural VC Interests
 \begin{table}[htbp]
    \begin{tabular}{c c c c}
    \toprule
    \textbf{Variables} & \textbf{(1)} & \textbf{(2)} & \textbf{(3)}               \\ 
    \textbf & \textbf{VC Interests} & \textbf{VC Interests} & \textbf{VC Interests}   \\ 

    \midrule
    \widehat{Women$-$Led} &   0.683***  &    0.186*    &      0.193*   \\
                         &    (0.098)   &   (0.112)    &      (0.108)      \\
                         &             &               &                  \\

                        &             &                &                   \\
                        &             &                &                    \\
    Firm Age            &             &      -0.046*** &   0.068***        \\
                        &             &      (0.015)   &  (0.023)       \\
                        &             &                &               \\
                        &             &                &         \\

Industry and Firm Controls   &   No        &   Yes       &       Yes    \\
    Funding Controls    &   No             &   No        &       Yes        \\

    \midrule
     N                  &   13044          &      13044  &       13044      \\          
    \bottomrule
    \addlinespace[1ex]
    \multicolumn{3}{l}{\textsuperscript{***}$p<0.01$, 
      \textsuperscript{**}$p<0.05$, 
      \textsuperscript{*}$p<0.1$}
    \end{tabular}
    Note: The table reports marginal effects of Tobit Model.

\end{table}

\end{table}
% \vfill 

Similarly, the structural specification of the model is used to examine the impact of the startup being Women$-$Led on the amount raised by the startup. The results are presented in table 10 below. In the base model, the results show that in fact Women$-$Led startups out raise their all$-$men counterparts. The coefficient of 0.51 represents 66.53\% more money raised by Women$-$Led startups as compared to the all$-$men startups. The magnitude of the effect decreases to 14.34\% more once controls for the industry and firm of the startup are added (P-value $<$ 0.001). The addition of further controls in column 3 of table 10 impacts the magnitude of the coefficient but not the statistical significance ( P-value $<$ 0.001). After accounting for the other characteristics of a startup, Women-Led startups raise on average 19.69\% more on average than all-men homogeneous startup teams. On the contrary, the effect of the age of the firm on the amount realized in fundraising is negative. If only the industry and firm controls are added, the effect of age is 13.50\% less for an additional year the firm spends in this stage (P-value $<$ 0.001). The direction of the coefficient is robust to the inclusion of all controls. However, the effect of a firm spending an additional year in the seed stage is slightly lower as older firms raised 5.35\% less per additional year spent in the seed stage (P-value $<$ 0.001). 

\begin{table}[H]
 \captionsetup{justification=raggedright,singlelinecheck=false}
    \caption{Structural Form Regression}
        % Structural Log Amount
 \begin{table}[htbp]
    \begin{tabular}{c c c c }
    \toprule
    \textbf{Variables} & \textbf{(1)} & \textbf{(2)} & \textbf{(3)}                \\ 
    \textbf & \textbf{Log Amount} & \textbf{Log Amount} & \textbf{Log Amount}   \\ 

    \midrule
    \widehat{Women$-$Led} &   0.510***  &      0.134***   &    0.179***   \\
                        &  (0.046)    &      (0.047)    &     (0.047)  \\
                        &             &                 &              \\

                        &             &                &              \\
                        &             &                &          \\
    Firm Age            &             &      -0.145***  &    -0.055***    \\
                        &             &      (0.006)   &     (0.010)       \\
                        &             &                &                   \\
   

Industry and Firm Controls   &   No        &   Yes       &      Yes    \\
    Funding Controls    &   No             &   No        &      Yes        \\

    \midrule
     N                  &   7940          &      7940  &     7940      \\          
    \bottomrule
    \addlinespace[1ex]
    \multicolumn{3}{l}{\textsuperscript{***}$p<0.01$, 
      \textsuperscript{**}$p<0.05$, 
      \textsuperscript{*}$p<0.1$}
    \end{tabular}
    Note: The table reports marginal effects of Tobit Model.
\end{table}

\end{table}


\section{Discussion}

This study delves into the intricate relationship between gender composition within startup teams and their fundraising success during the crucial seed funding stage. The core objective was to illuminate the influence of the presence of women in these teams on attracting venture capital interest and the mean amount of funding ultimately secured. The findings offer valuable insights into the dynamics of gender diversity within entrepreneurial ventures and its ramifications for accessing critical financial resources.

One of the most significant revelations is that startups with women founders attract venture capital interest at a demonstrably higher rate compared to their all-men counterparts. This suggests a potential shift in venture capital firms' preferences, possibly reflecting a growing recognition of the value proposition that gender diversity brings to innovation and decision-making processes. This aligns with existing research highlighting the positive impact of gender diversity on team performance and the quality of decisions made.

However, a concerning disparity emerges – despite attracting greater investor interest, the tobit regression model reveals that Women-Led startups tend to raise less capital on average compared to all-men teams. This discrepancy exposes a significant gap in funding outcomes. While women-led ventures may pique investor curiosity and initial interest, they still face formidable challenges in securing the necessary funding to thrive. This disparity may be attributed to a confluence of factors, including unconscious bias influencing investment decisions, limitations in access to networks and crucial resources, and systemic barriers that women entrepreneurs often encounter.

To strengthen the findings and address potential endogeneity concerns, the study utilizes instrumental variable regression. By employing state-level maternity benefits as an instrument for the presence of women in startups, the results partly corroborate the initial findings. This analysis confirms that gender diversity within startup teams positively influences the level of venture capital interest garnered. However, Women-Led startups raised more money than their all-men counterparts in the IV regression specification.

In conclusion, these findings underscore the criticality of promoting gender diversity and fostering an inclusive environment within the entrepreneurial ecosystem. Addressing persistent disparities in funding outcomes based on gender necessitates a collaborative effort from policymakers, investors, and all stakeholders within the industry. Implementing targeted diversity initiatives, providing dedicated mentorship and support programs for women entrepreneurs, and fostering inclusive investment practices are some potential strategies to level the playing field and create a more equitable environment for startup funding.

This study significantly advances our understanding of how gender composition affects a startup's fundraising success during the seed stage. While there appears to be progress in terms of venture firms expressing interest in Women-Led startups. Highlighting the value of gender diversity and implementing inclusive practices are crucial steps that stakeholders can take to create opportunities for women entrepreneurs to flourish and contribute meaningfully to both innovation and economic growth. Further research and policy efforts are essential to promote gender equality and empower inclusive entrepreneurship.



\section{Conclusion} \label{sec:conclusion}

In conclusion, this paper delves into the critical issue of gender diversity within the labor force, particularly within the entrepreneurial landscape. The findings underscore the complex interplay between team gender composition, investor interests, and fundraising outcomes during the seed stage of investment. While previous research has highlighted the persistent gender gap in entrepreneurship, this study offers nuanced insights into the mechanisms underlying these disparities.

The analysis reveals compelling evidence that the presence of women founders impacts venture firm interest, albeit with inconsistent implications for fundraising success. The results from the tobit regression model shows Women$-$led startups tend to raise lower amounts of funds compared to their all-men counterparts. However, the instrumental variable regression model suggests increasing startup gender diversity can enhance fundraising amounts for Women$-$led startups.


For policymakers and stakeholders, these findings underscore the need to foster an environment conducive to gender-inclusive entrepreneurship. Encouraging Women$-$led startups to innovate in industries with high funding potential could offer a strategic approach to mitigating disparities in fundraising outcomes. By addressing structural barriers and promoting diversity in entrepreneurship, we can create a more equitable landscape that fosters innovation and economic growth.



\bibliography{mybiblio.bib}

\section{Appendix} \label{sec:Appendix}


\subsection{Gender Discrimination}
To test whether the results is driven by gender stereotype-based discrimination, I define a dummy variable for a founding team being in a women dominant industry within the data. A startup is said to be in a women$-$dominant if the majority of the startups in the industry are women$-$led. The logic of this is that if there is no evidence of discrimination, there should not be a difference between the outcome of being a Women$-$led startup in a women dominated industry or a male dominant industry. I then include this dummy variable as well as the interaction of being in a congruent industry and the team being Women$-$led. The results of this specification are presented in tables 11 and 12 below. The results in table 11 show the effect on venture firm interests attracted. The results show that there is no difference in venture funds attracted by startups that are in a congruent industry.  

\begin{table}[H]
 \captionsetup{justification=raggedright,singlelinecheck=false}
    \caption{Gender Discrimination and Venture Interests Attracted}
         \begin{table}[htbp]
    \begin{tabular}{c c c c c}
    \toprule
    \textbf{Variables} & \textbf{(1)} & \textbf{(2)} & \textbf{(3)}         \\ 
    \textbf & \textbf{VC Interests} & \textbf{VC Interests} & \textbf{VC Interests}   \\ 

    \midrule
    Women$-$Led         &   0.259*** &     0.271***    &        0.180*   \\
                        &   (0.105)  &      (0.010)    &        (0.096)  \\
                        &             &                &                  \\
    Women$-$Dominant    &    0.073    &      0.199     &        0.297  \\
                        &    (0.098)  &      (0.133)   &        (0.128)  \\
                        &             &                &                    \\
Women Dominant*Led      &    0.085     &    0.1475     &       0.245   \\
                        &   (0.179)   &     (0.169)    &      (0.163)  \\                    
                        &             &                &             \\
                        &             &                &                    \\
    Firm Age            &             &      -0.048*** &    0.068***        \\
                        &             &      (0.015)   &    (0.023)       \\
                        &             &                &                 \\
                        &             &                &                \\

Industry and Firm Controls   &   No        &   Yes       &          Yes    \\
    Funding Controls    &   No             &   No        &        Yes        \\

    \midrule
     N                  &   13044          &      13044  &     13044      \\          
    \bottomrule
  
    \addlinespace[1ex]
    \multicolumn{3}{l}{\textsuperscript{***}$p<0.01$, 
      \textsuperscript{**}$p<0.05$, 
      \textsuperscript{*}$p<0.1$}

\end{tabular}
Note: The table reports marginal effects of Tobit Model.

\end{table}

\end{table}


In table 12, the paper analyzes the possibility of gender stereotype-based discrimination in funds raised. The effect of being a women$-$led startup in an industry dominated by women is actually negative and the effect is not statistically significant (P-value > 0.1). This is suggestive evidence against a case of discrimination against Women$-$led startups. 

\begin{table}[H]
 \captionsetup{justification=raggedright,singlelinecheck=false}
    \caption{Gender Discrimination and Amount Raised}
         \begin{table}[htbp]
    \begin{tabular}{c c c c c}
    \toprule
    \textbf{Variables} & \textbf{(1)} & \textbf{(2)} & \textbf{(3)}               \\ 
    \textbf & \textbf{Log Amount} & \textbf{Log Amount} & \textbf{Log Amount}  \\ 

    \midrule
    Women$-$Led         &   -0.192*** &      -0.178***  &        -0.188***   \\
                        &    (0.052)  &      (0.042)   &         (0.042)  \\
                        &             &                &                  \\
    Women$-$Dominant    &    -0.126*** &      -0.017    &        -0.021  \\
                        &    (0.047)  &      (0.056)   &         (0.056)  \\
                        &             &                &                    \\
    Women Dominant*Led      &    0.080     &      0.099  &        0.097   \\
                        &   (0.089)   &      (0.071)   &      (0.071)  \\                    
                        &             &                &             \\
                        &             &                &                    \\
    Firm Age            &             &      -0.149*** &    -0.054***        \\
                        &             &      (0.006)   &    (0.010)       \\
                        &             &                &                 \\
                        &             &                &                \\
Industry and Firm Controls   &   No        &   Yes       &        Yes    \\
    Funding Controls    &   No             &   No        &         Yes        \\

    \midrule
     N                  &   7940          &      7940  &     7940        \\          
    \bottomrule
    \addlinespace[1ex]
    \multicolumn{3}{l}{\textsuperscript{***}$p<0.01$, 
      \textsuperscript{**}$p<0.05$, 
      \textsuperscript{*}$p<0.1$}
    \end{tabular}
    Note: The table reports marginal effects of Tobit Model.
\end{table}

\end{table}

% \vfill

\subsection{Sorting}
This section examines the possibility that our previous results are influenced by sorting bias. Figure 1 suggests that industries like utilities, healthcare, education, government, and industrials have a higher proportion of women-led startups. Conversely, finance, technology, real estate, and energy have lower proportions. Figure 3 explores the average amount raised by industry. We see that finance, technology, and real estate industries have the highest average funding, while healthcare, utilities, and education/government sectors have the lowest. Figure 4 provides a more granular breakdown. In industries with a higher concentration of women-led startups (utilities, education/government, industrials, and healthcare), they tend to under-raise compared to all-male teams. Interestingly, these are also the industries with the lowest average funding overall. However, a different picture emerges in high-funding industries like finance and technology. Here, while women-led startups represent a smaller proportion, they are able to raise capital on par with, or even outperform, their all-male counterparts.  

% \begin{figure}
% \caption{Time Series of Startups by Industries }
% \centering
% \includegraphics[scale=0.4]{figures_3/Figure1.png} 
% \end{figure}



% \begin{figure}
% \caption{Time Series of Startups by Industries Grouped According to Founding Team Gender Composition }
% \centering
% \includegraphics[scale=0.4]{figures_3/Participation.png} 
% \end{figure}

\begin{figure}
 \captionsetup{justification=raggedright,singlelinecheck=false}
\caption{Amount Raised by Industries}
\includegraphics[scale=0.4]{figures_3/Industry.png} 
\end{figure}


\begin{figure}
 \captionsetup{justification=raggedright,singlelinecheck=false}
\caption{Amount Raised by Industries Grouped by Gender of Founding Team }
\includegraphics[scale=0.4]{figures_3/Industry_gender.png} 
\end{figure}


% \begin{itemize}
%         \item Adams, R. B., \& Ferreira, D. (2009). Women in the boardroom and their impact on governance and performance. Journal of financial economics, 94(2), 291-309.
%     \item Alesina, A. F., Lotti, F., \& Mistrulli, P. E. (2013). Do women pay more for credit? Evidence from Italy. Journal of the European Economic Association, 11(suppl1), 45-66.
%     \item Ali, M., Kulik, C. T., \& Metz, I. (2011). The gender diversity–performance relationship in services and manufacturing organizations. The International Journal of Human Resource Management, 22(07), 1464-1485.
%     \item Alsan, M., Garrick, O., \& Graziani, G. (2019). Does diversity matter for health? Experimental evidence from Oakland. American Economic Review, 109(12), 4071-4111.
%     \item Beckman, C. M., Burton, M. D., \& O'Reilly, C. (2007). Early teams: The impact of team demography on VC financing and going public. Journal of business venturing, 22(2), 147-173.
%     \item Bernstein, S., Korteweg, A., \& Laws, K. (2017). Attracting early‐stage investors: Evidence from a randomized field experiment. The Journal of Finance, 72(2), 509-538.
%     \item Bertrand, Marianne, and Esther Duflo. 2016. “Field Experiments on Discrimination.” National Bureau of Economic Research (NBER) Working Paper 22014.
%     \item Brooks, A. W., Huang, L., Kearney, S. W., \& Murray, F. E. (2014). Investors prefer entrepreneurial ventures pitched by attractive men. Proceedings of the National Academy of Sciences, 111(12), 4427-4431.
% \item Calder-Wang, S., Gompers, P., \& Sweeney, P. (2021). Venture Capital’s “Me Too” Moment (No. w28679). National Bureau of Economic Research.
%     \item Dessein, W., \& Santos, T. (2006). Adaptive organizations. Journal of Political Economy, 114(5), 956-995.
%     \item ell, S. T., Villado, A. J., Lukasik, M. A., Belau, L., \& Briggs, A. L. (2011). Getting specific about demographic diversity variable and team performance relationships: A meta-analysis. Journal of management, 37(3), 709-743.
%     \item Ewens, M., \& Townsend, R. R. (2020). Are early stage investors biased against women?. Journal of Financial Economics, 135(3), 653-677.
%     \item Gompers, P. A., \& Wang, S. Q. (2017). And the children shall lead: Gender diversity and performance in venture capital (No. w23454). National Bureau of Economic Research.
%     \item Gompers, P. A., \& Wang, S. Q. (2017). Diversity in innovation (No. w23082). National Bureau of Economic Research.
%     \item Gompers, P., V. Mukharlyamov, and Y. Xuan, “The Cost of Friendship,” Journal of Financial Economics, 119 (2016), 626–644
%     \item Gompers, P., V. Mukharlyamov, E. Weisburst, and Y. Xuan, “Gender Effects in Venture Capital,” forthcoming in Journal of Financial and Quantitative Analysis (2020).
%     \item Gornall, W., \& Strebulaev, I. A. (2020). Gender, race, and entrepreneurship: A randomized field experiment on venture capitalists and angels. Available at SSRN 3301982.
%     \item  Guzman, J., \& Kacperczyk, A. O. (2019). Gender gap in entrepreneurship. Research Policy, 48(7), 1666-1680.
%     \item Hebert, C. (2020, March). Gender stereotypes and entrepreneur financing. In 10th Miami Behavioral Finance Conference.
%     \item Hellmann, T., Mostipan, I., \& Vulkan, N. (2019). Be careful what you ask for: Fundraising strategies in equity crowdfunding (No. w26275). National Bureau of Economic Research.
%     \item Hu, A., \& Ma, S. (2020). Human interactions and financial investment: A video-based approach. Available at SSRN.
%     \item Joshi, A., \& Roh, H. (2007). Context matters: a multilevel framework for work team diversity research. In J. Martocchio (Ed.), Research in Personnel and Human Resource Management, Vol. 26. (pp. 148). Greenwich, CT: JAI Press.
%     \item Kim, D., \& Starks, L. T. (2016). Gender diversity on corporate boards: Do women contribute unique skills?. American Economic Review, 106(5), 267$-$71.
%     \item Koning, R., Samila, S., \& Ferguson, J. P. (2019). Female inventors and inventions. Available at SSRN 3401889.
%     \item Lyons, E. (2017). Team production in international labor markets: Experimental evidence from the field. American Economic Journal: Applied Economics, 9(3), 70-104. 
%     \item Mannix, E., \& Neale, M. A. (2005). What differences make a difference? Psychological Science in the Public Interest, 6, 3155.
%     \item Mathieu, J. E., Tannenbaum, S. I., Donsbach, J. S., \& Alliger, G. M. (2014). A review and integration of team composition models: Moving toward a dynamic and temporal framework. Journal of Management, 40(1), 130-160.
%     \item Rasul, I., \& Rogger, D. (2018). Management of bureaucrats and public service delivery: Evidence from the nigerian civil service. The Economic Journal, 128(608), 413-446.
%     \item Robb, A. M., \& Robinson, D. T. (2014). The capital structure decisions of new firms. The Review of Financial Studies, 27(1), 153-179.
%     \item Ruigrok, W., Peck, S., \& Tacheva, S. (2007). Nationality and gender diversity on Swiss corporate boards. Corporate governance: an international review, 15(4), 546-557.
%     \item Shore, L. M., Chung-Herrera, B. G., Dean, M. A., Ehrhart, K. H., Jung, D. I., Randel, A. E., \& Singh, G. (2009). Diversity in organizations: Where are we now and where are we going?. Human resource management review, 19(2), 117-133.
%     \item The Refinitiv business classifications. Refinitiv Business Classification . (n.d.). Retrieved July 11, 2021, from $https://www.refinitiv.ru/content/dam/marketing/en_us/documents/methodology/trbc-business-classifcation-methodology.pdf$. 
%     \item Webber, S. S., \& Donahue, L. M. (2001). Impact of highly and less job-related diversity on work group cohesion and performance a meta-analysis. Journal of Management, 27, 141162.
%     \item Wegge, J., Roth, C., Neubach, B., Schmidt, K. H., \& Kanfer, R. (2008). Age and gender diversity as determinants of performance and health in a public organization: the role of task complexity and group size. Journal of Applied Psychology, 93(6), 1301.
%     \item	Harrison, D. A., \& Klein, K. J. (2007). What's the difference? Diversity constructs as separation, variety, or disparity in organizations. Academy of management review, 32(4), 1199-1228.
%     \item	Hoogendoorn, S., Oosterbeek, H., \& Van Praag, M. (2013). The impact of gender diversity on the performance of business teams: Evidence from a field experiment. Management Science, 59(7), 1514-1528. 


    
% \end{itemize}



% \end{document} 
% \clearpage

% % \documentclass[12pt]{article}
% % \documentclass[12pt]{report}
% \usepackage[a4paper, total={6in, 8in}]{geometry}
% \large
% \usepackage{booktabs}
% \usepackage{setspace}
% \usepackage{hyperref}
% \usepackage{graphicx}
% \usepackage{float}
% \usepackage{xcolor}
% \usepackage{lscape}

% \begin{document}
\doublespacing

\begin{center}
\section*{Conclusion}\label{sec:Conclusion_all}
\end{center}

The culmination of this dissertation underscores the intricate interplay between diversity, team composition, perceived discrimination in workplace settings and economic decision making. Through a comprehensive exploration across three chapters, several key insights have emerged, offering valuable implications for organizations aiming to foster inclusivity and enhance performance.

In the first chapter, the research highlights the significant impact of team diversity on economic decision-making. Findings reveal that team diversity influences individual behavior, particularly among newcomers, underscoring the importance of understanding the effects of team composition on performance outcomes. By acknowledging the complexities of team dynamics, organizations can leverage diversity as a strategic asset to drive innovation and success.

In the second chapter, the research further explores the nuanced dynamics of group membership on perceived discrimination, and decision-making processes within economic contexts. By examining behaviors and perceptions in controlled experimental settings, the study provides empirical evidence on how individuals navigate subjective evaluation systems. These insights offer valuable implications for organizations seeking to create more inclusive and equitable work environments, highlighting the importance of addressing bias and promoting diversity to enhance organizational performance.

The final chapter investigates the influence of gender composition within startup teams on venture firm interests and funding outcomes during the seed funding stage. It reveals that women-led startups attract greater interest from venture capital firms, yet tend to raise lower amounts of funds compared to male-dominated startups, highlighting a significant gap in funding outcomes. Despite efforts to address potential discriminatory practices, such as instrumental variable regressions and analyses of industry contexts, disparities persist, indicating the need for comprehensive strategies to foster gender-inclusive entrepreneurship environments. The findings underscore the complexity of gender dynamics in the entrepreneurial landscape and emphasize the importance of addressing structural barriers to promote gender equality and innovation.

In conclusion, the findings from this dissertation contribute to a deeper understanding of the complexities inherent in team dynamics, job evaluations, and perceptions of discrimination. By integrating these insights into organizational practices and policies, companies can cultivate environments that foster collaboration, innovation, and mutual respect, ultimately driving success in an increasingly diverse and dynamic workplace landscape.


\end{document}


% \clearpage

% \end{document}