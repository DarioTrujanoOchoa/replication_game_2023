% \documentclass[12pt]{article}
% % \documentclass[12pt]{report}
% \usepackage[a4paper, total={6in, 8in}]{geometry}
% \large
% \usepackage{booktabs}
% \usepackage{setspace}
% \usepackage{hyperref}
% \usepackage{graphicx}
% \usepackage{float}
% \usepackage{xcolor}
% \usepackage{lscape}

% \begin{document}
\doublespacing

\begin{center}
\section*{Conclusion}\label{sec:Conclusion_all}
\end{center}

The culmination of this dissertation underscores the intricate interplay between diversity, team composition, perceived discrimination in workplace settings and economic decision making. Through a comprehensive exploration across three chapters, several key insights have emerged, offering valuable implications for organizations aiming to foster inclusivity and enhance performance.

In the first chapter, the research highlights the significant impact of team diversity on economic decision-making. Findings reveal that team diversity influences individual behavior, particularly among newcomers, underscoring the importance of understanding the effects of team composition on performance outcomes. By acknowledging the complexities of team dynamics, organizations can leverage diversity as a strategic asset to drive innovation and success.

In the second chapter, the research further explores the nuanced dynamics of group membership on perceived discrimination, and decision-making processes within economic contexts. By examining behaviors and perceptions in controlled experimental settings, the study provides empirical evidence on how individuals navigate subjective evaluation systems. These insights offer valuable implications for organizations seeking to create more inclusive and equitable work environments, highlighting the importance of addressing bias and promoting diversity to enhance organizational performance.

The final chapter investigates the influence of gender composition within startup teams on venture firm interests and funding outcomes during the seed funding stage. It reveals that women-led startups attract greater interest from venture capital firms, yet tend to raise lower amounts of funds compared to male-dominated startups, highlighting a significant gap in funding outcomes. Despite efforts to address potential discriminatory practices, such as instrumental variable regressions and analyses of industry contexts, disparities persist, indicating the need for comprehensive strategies to foster gender-inclusive entrepreneurship environments. The findings underscore the complexity of gender dynamics in the entrepreneurial landscape and emphasize the importance of addressing structural barriers to promote gender equality and innovation.

In conclusion, the findings from this dissertation contribute to a deeper understanding of the complexities inherent in team dynamics, job evaluations, and perceptions of discrimination. By integrating these insights into organizational practices and policies, companies can cultivate environments that foster collaboration, innovation, and mutual respect, ultimately driving success in an increasingly diverse and dynamic workplace landscape.


\end{document}

