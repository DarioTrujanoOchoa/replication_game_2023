% \documentclass[12pt]{article}
% % \documentclass[12pt]{report}
% \usepackage[a4paper, total={6in, 8in}]{geometry}
% \large
% \usepackage{booktabs}
% \usepackage{setspace}
% \usepackage{hyperref}
% \usepackage{graphicx}
% \usepackage{float}
% \usepackage{xcolor}
% \usepackage{lscape}

% \begin{document}
\thispagestyle{empty} % no page number for title page
\doublespacing

\begin{center}
\section*{Introduction}\label{sec:intro}
\end{center}

Fairness and equity are paramount in economic decision-making, impacting both individual opportunities and overall economic well-being.  Understanding the intricacies of team composition, perceptions of discrimination and the impact of team composition on fundraising success is essential for fostering inclusive and effective work environments. This dissertation delves into these areas, aiming to uncover insights that can inform organizational practices and policies.

\textbf{Chapter 1: Team Diversity and Economic Decision-Making}

Teams have become integral to various environments, offering diverse perspectives and skills that enhance decision-making processes. However, challenges such as coordination issues and delays in decision-making can arise, particularly in diverse teams. This chapter investigates the impact of team diversity on economic decision-making through a series of experiments. By exploring how team diversity influences individual behavior and performance, the research aims to provide valuable insights into optimizing team composition for efficient outcomes.

\textbf{Chapter 2: Group bias, anticipation and performance }

In the second chapter, the research delves into the role of subjectivity in job evaluations and its implications for supervisors and employees. Combining insights from economic literature and experimental studies conducted with participants from two U.S. universities, the study elucidates concerns such as discrimination and favoritism in evaluation processes. It investigates how group affiliation influences both the anticipation of bias and individuals' behaviors regarding performance evaluations. The results underscore significant disparities in the perception of discrimination based on race and gender, shedding light on the necessity for organizations to address systemic biases and cultivate inclusive environments. Furthermore, the research unveils a stronger emphasis on achieving positive outcomes from in-group membership compared to concerns about potential bias from out-group evaluators.

\textbf{Chapter 3: Analysis of Venture funding at the Seed Stage}

The third chapter explores the persistent gender gap in entrepreneurship, particularly within Science, Technology, Engineering, and Mathematics (STEM) fields. The study leverages data on startup teams to investigate how gender composition influences their ability to attract investment, especially during the crucial seed funding stage. The findings reveal a concerning disparity: while startups with women founders attract higher venture capital interest rates, they tend to struggle where it matters most.Women$-$Led startups are concentrated in areas where the area amount is often low and they less represented in high evaluation sectors of science and technology. The analysis suggests potential industry-based sorting bias, with women-led ventures performing on par or better in high-funding industries.

Collectively, these chapters contribute to a deeper understanding of team dynamics,perceptions of discrimination in the workplace and startup teams. By uncovering insights and implications for organizational practices, policies, and interventions, this dissertation aims to facilitate the creation of more inclusive and effective work environments. 



% \end{document}