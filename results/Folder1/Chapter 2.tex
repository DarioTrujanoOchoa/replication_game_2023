% \documentclass[12pt]{article}
% % \documentclass[12pt]{report}
% \usepackage[a4paper, total={6in, 8in}]{geometry}
% \large
% \usepackage{booktabs}
% \usepackage{setspace}
% \usepackage[hidelinks]{hyperref}
% \usepackage{graphicx}
% \usepackage{float}
% \usepackage{xcolor}
% \usepackage{lscape}
% \usepackage{setspace}
% \usepackage{authblk}


% % BIBLIOGRAPHY %%%%%%%%%%%%%%
% \usepackage[natbibapa]{apacite} % to enable '\citet' and '\citep' macros
% \bibliographystyle{aer}
% % %%%%%%%%%%%%%%%%%%%%%%%%%%%%

% \title{
% {Group bias, anticipation and performance} }\\

% \author[1]{George Agyeah }
% \author[2]{Yufei Ren}
% \affil[1]{Department of Economics, University of Arkansas}
% \affil[2]{Department of Economics, University of Minnesota-Duluth \thanks{Financial support from the \href{https://bbrl.uark.edu/}{Walton College Behavioral Business Research Lab} and Department of Economics of the University of Minnesota-Duluth are duly acknowledged.}}

% % \author{ George Agyeah } 
% \date{\today}
% % \date{November 26, 2023} 

% \begin{document}

% \maketitle 
% \begin{center}
%     \href{https://wordpress.com/block-editor/page/ag-yeah.com/1254}{ \textcolor{black}{Click here for the latest version}} 
% \end{center}
% \begin{abstract}
% \small \noindent 
% Companies utilize various methods to assess employee performance, which directly impact decisions on promotions, salaries, and recognition. Despite this, the understanding of how these evaluations and worker relationships influence employee behavior remains limited. This study addresses this gap through a two-institution lab experiment. The experiment explores how the nature of performance evaluations and group affiliation affect individuals' perception of their evaluations. By leveraging existing group settings at two universities, this study investigates how group identity impacts the perception of discrimination among 240 participants. The findings indicate a heightened perception of bias when group identity is salient. Further analysis shows that positive bias, rather than negative bias is a significant driver of actions among participants.  \\

% \noindent\tiny\textbf{Keywords: perceived discrimination, performance evaluations, experiments }\\
% \noindent\textbf{JEL Codes: C92, D83, D91} \\
% \end{abstract}
% \setcounter{page}{0}
% \thispagestyle{empty}
% \pagebreak \newpage
\pagestyle{plain} 

\doublespacing

\section{Introduction} \label{sec:introduction}
Job evaluations are crucial for workplace decisions. Many organizations use a mix of objective and subjective reviews to assess employee performance. A key discussion point is how subjectivity in these reviews impacts the actions of supervisors and employees. Economic research has identified issues linked to subjective evaluations. One major concern is discrimination documented in studies \citep{goldin2000,bertrand2004}. A comprehensive review by \cite{bertrand2016} explores evidence of discrimination in this area. Similarly, favoritism can also arise from evaluation methods \citep{prendergast1996}. A question that remains is whether employees are aware of these biases, and to what extent employees adjust their behavior accordingly to reflect their anticipation of this bias. Research shows that discrimination, whether subtle or overt, can significantly impact individuals \citep{jones2016}. 

This study explores how group affiliation influences perceived discrimination among participants in economic decision making. Participants from two U.S. universities are randomly paired to complete and/or evaluate tasks. Participants' beliefs about evaluations in scenarios with varying degrees of potential subjectivity in evaluations are elicited. The findings suggest that group affiliation noticeably affects anticipation of bias and decisions. Interestingly, the results indicate a stronger focus on gaining positive bias from in-group membership compared to concerns about negative bias from out-group members.

This study replicates a common workplace situation. Imagine an individual working in an environment where group membership of employees and supervisors are public information. Group membership can be based on demographics or any other social identifier. In this scenario, performance evaluations that a supervisor conducts can be received differently based on an employee's group membership. An employee's anticipation of bias can therefore influence their behavior. Understanding how this perception affects decision-making is crucial especially among racial and gender minorities who often find themselves as outsiders in the work place. This experiment introduces a unique method to measure perceived bias that allows separation of the reason for anticipation of bias. 

\hspace *{0mm} This study makes several contributions to the literature. Firstly, it broadens our understanding of perceived bias by investigating the impact of natural group membership on decision-making. Secondly, the paper examines how performance measures influence effort levels. This allows for a more accurate assessment of how widely used evaluation mechanisms affect both employees and supervisors. The rest of the paper is organized as follows. Section \hyperref[sec:literature]{2} presents the literature review. Section \hyperref[sec:Design]{3} details the experimental design. Section \hyperref[sec:Hypotheses]{4} introduces hypotheses. Section \hyperref[sec:Analysis]{5} discusses the empirical analyses and results in the study. Section \hyperref[sec:Conclusion]{6} concludes.


\section{Literature Review} \label{sec:literature}
\hspace *{0mm} The American workforce is rapidly becoming more diverse. This includes along factors like location, gender, and race \citep{b21}. Having a diverse workforce that functions smoothly is essential for economic success \citep{eg05}. An extant literature shows that membership of a group can influence various economic outcomes for individuals \citep{charness2020social}. Indeed, an individual's perception of themselves can be affected by their environment and their group membership. For instance, gender stereotypes can influence how women perceive their own abilities in a group setting \citep{bordalo2019beliefs}. Perception of bias can also differ by race. \cite{ruebeck2023} finds that belonging to a racial minority can heighten the perception of bias. \cite{charness2020} explore this further by investigating the role of gender in anticipating discrimination. Their findings show that men and women make different choices about revealing their gender identity based on the task at hand. Similar results are found by \cite{aksoy2023}. They show that individuals mask signals about their affinity with the LGBTQ+ community in response to anticipated discrimination in prosocial behavior. The findings show varied behavior by gender identity in situations where discrimination is a possibility.

While a crucial factor in anticipating discrimination is the type of evaluation system used, several studies demonstrate that perception of bias can impact behavior. For example, \cite{angeli2023} show that anticipating bias can affect interview performance. Similarly, \cite{bedard2019} shows that perception about ability negatively impacts competition in classroom settings with tournament-style evaluations. Further evidence by \cite{angelovski2016} suggests that managers are more likely to exhibit escalation bias in situations where subjective evaluations are possible. A recent study reveals that employers often favor their own group members for promotions, impacting the effort levels of both promoted and non-promoted individuals \citep{vdurinik2023}. 

This paper aims to investigate the connection between group membership and perceived bias. Firstly, we will examine if individuals in an environment where group identity is salient have a higher anticipation of bias. Secondly, the study examines whether workers and supervisors differ in their perception of bias. Finally, the study explores the factors that influence individuals' anticipation of bias including membership of a group that has historically experienced discrimination.


\section{Experimental Design} \label{sec:Design}

The experimental design consists of two between-subjects treatments named, group identified (henceforth, Salient Identity) and control. Two hundred and forty participants are recruited from two universities: The University of Arkansas and the University of Minnesota, Duluth. Participants' natural affiliation with their respective universities serves as the basis for grouping them. Furthermore, we utilize group identity priming to make group membership salient in Salient Identity. 

Participants are recruited from classes in both institutions. The study is conducted synchronously across the two institutions. Initially, participants are unaware they are part of a larger study spanning both universities. Each session involves eight individuals from both universities (four from each institution). Participants in the "group identified" treatment, Salient Identity are primed about their university identity after initial instructions of the structure of the experiment are read live over zoom for both sets of participants. Once initial general instructions are read, participants proceed to complete tasks in 2 segments. Segment 1 consists of 4 parts. Participants perform 2 practice rounds after initial introduction to the tasks before proceeding to paid parts. Each part starts with instructions for the part before participants attempt the tasks. In part 1, all participants complete tasks that are evaluated by the preprogrammed software - the objective evaluator. In parts 2 and 3, the tasks completed by the workers' are evaluated by either someone from their own university (in-group) or someone from the other university (out-group). In part 4, the beliefs of the participants are elicited using the Vickrey auction procedure \citep{karni2009} before task completion. There is a slight variation in part 4 for supervisors. Supervisors in part 4 complete 2 separate evaluations per round - one from an in-group partner and the other from an out-group partner.

In segment 1, participants work in pairs with either another participant or a computer. Pairing is randomly assigned to prevent reassignment with past partners and to ensure each pair encounters both in-group and out-group partners at various  parts of the study. The tasks participants are engaged in is a modified version of a task requiring participants to adjust sliders and/or evaluate the accuracy of aligned sliders \citep{gill2012}. Participants that are assigned slider completion tasks have a 75-second time frame to correctly align 30 bars of sliders. These participants are henceforth referred to as workers. Additionally, workers are given the opportunity to choose an alternative option that suits their preferences. This fosters incentive compatibility, a concept discussed in \cite{dutcher2015}. Supervisors, evaluators of completed tasks, can be a software program or a participant (in-group or out-group member). Supervisors evaluate the sliders completed by the workers. The decision of what sliders are correctly aligned is based on the evaluation of the supervisor only. 

In each session, participants are either assigned Salient Group or the Control. To make group identity salient, participants are asked to fill out one of two questionaires at the beginning of the study. The questions are subtle enough to ensure the questions make participants’ university identity salient without having participants aware that they are being primed. Participants in this treatment are asked to answer questions on why they chose this university and the role the university plays in their lives. In addition to priming, participants in Salient Identity treatment are able to see the university affiliation of their paired partner (refer to appendix for images of university affiliation). The control treatment follows the same overall structure as Salient Identity, with two crucial differences - university affiliation is not mentioned at any point during the experiment and participants are unaware of the affiliation of paired partners. Instead of the university-related identity priming questions during the introductory phase, participants in the control treatment answer questions about their cell phone usage. A summary of the segment 1 of the experiment is presented in figure 1 below. 

\begin{figure}[H]
 \captionsetup{justification=raggedright,singlelinecheck=false}
\caption{Flowchart of Segment 1 of Experimental Procedure}
\includegraphics[scale=0.6]{images/Flowchart.png} 
\end{figure}
 

 In segment 2, self-identified individual characteristics are take. Risk preferences of participants are then elicited using an incentivized lottery task \citep{eckel2008}. Within each part, workers are paid based on the sum of earnings from the number of sliders correctly aligned and the earnings from time spent on the alternative activity. Supervisors are paid a flat rate for each part of the study. Finally, it is common knowledge in both treatments that earnings of workers are based on a randomly chosen part in segment 1 and all of the payoffs from the risk elicitation portion in segment 2.   

\section{Hypotheses} \label{sec:Hypotheses}
In many organizations, performance evaluation is a common phenomenon. Many situations exist where individuals are evaluated by measures that have varying degrees of subjectivity. It is also common for individuals to feel as outsiders within the groups they work. The proliferation of subjective measures of performance and the existence of several groups of individuals in many work environments present a situation where perceived bias could affect how individuals behave at the workplace. Economic research sheds light on how group identity can influence individual decision-making. Highlighting group identity has been shown to impact various behaviors, including performance \citep{aronson1998} and perception \citep{bargh1982}. Building on this foundation, the following first hypothesis is proposed: 
 
 \textit{Hypothesis 1:  Individuals in the primed treatment have a heightened anticipation of bias in economic decision-making scenarios. }

 Existing literature explores potential incentives that can skew evaluations \citep{carpenter2010}. Our unique design allows us to investigate bias about both positive and negative bias. We define positive bias as favorable treatment, while negative bias is unfavorable treatment. Individuals who anticipate positive bias may be willing to pay to avoid objective evaluations. Conversely, a willingness to pay to avoid an evaluation by an out-group suggests the expectation of negative bias. This leads to our second hypothesis:

  \textit{Hypothesis 2: Both positive and negative bias play a significant role in participants' decision-making.}

The burgeoning research on anticipated bias has shown that individuals that have historically suffered discrimination are more likely to anticipate discrimination \citep{charness2020,aksoy2023}. Consistent with this literature, we propose our final hypothesis: 

    \textit{Hypothesis 3: Gender and racial minorities that have historically experienced discrimination will anticipate more bias regardless of work roles.}

\section{Analysis} \label{sec:Analysis}
The analysis for this section of the paper is organized into four main sections for clarity. The initial segment provides an overview of the study participants in the summary statistics section. Details of the demographic distributions of the participant are also presented in this segment. The actions and decision-making of participants that are randomly assigned workers are presented in the second section. Following this, the focus shifts to the results of the actions of participants randomly assigned supervisors during the study. Finally, the analysis delves into whether actions and decisions in the study vary by demographic characteristics.

\subsection{Summary Statistics}
A total of 240 participants are recruited from the University of Arkansas, primarily sourced through the Walton College Behavioral Business Research Laboratory Sona System, and the University of Minnesota - Duluth campus. The gender distribution of the participants is presented in table 1 below. Among the participants, 135 individuals, accounting for 56.25\% of the sample, self-identified as men, while 104 participants identify as women. Additionally, one participant did not identify with either gender identity named.

\begin{table}[H]
 \captionsetup{justification=raggedright,singlelinecheck=false}
\caption{Gender Distribution of Participants } \label{tab:table1}
        {
	\def\sym#1{\ifmmode^{#1}\else\(^{#1}\)\fi}
	\begin{tabular}{l*{1}{ccccccc}}
		\toprule
		& Info & RCL & No Info  & Feedback(t=1,2) & Feedback(t=9,10) & Description   \\
		\midrule
		20         &      120 &         59 &      120   &      8  &   16  &  60 \\
		\midrule
		30         &      240 &        118 &      240   &      21  &   22 &  120 \\
		\midrule
		50         &      480 &        236 &      480   &      55  &   51 &  240 \\
		\midrule
		70         &      240 &        118 &      240   &      24  &   21 &  120 \\
		\midrule
		80         &      120 &         59 &      120   &      12  &   10 &  60 \\
		\bottomrule
	\end{tabular}
}
\end{table}

The self-identified racial distribution of the participants is presented in table 2 below. A notable majority, comprising 74.17\% of the 240 participants, identify as white. Five percent identify as black/African American, 6.25\% identify as Hispanic, 7.92\% as Asian, 1.25\% as Native American, 1.25\% as Middle Eastern, and 3.75\% identify as being from the Indian Subcontinent. Within the participant pool, 160 individuals are assigned to the Salient Identity treatment and 80 participants are assigned to the control treatment.
\begin{table}[H]
 \captionsetup{justification=raggedright,singlelinecheck=false}
\caption{Racial Distribution of Participants} \label{tab:table2}
        \begin{table}[htbp]\centering
    \begin{tabular}{c c c c c c}
    \toprule
    \multicolumn{6}{c }{\textbf{Summary of Main Variables}}  \\
    \midrule
    Variables            &    Obs      &  Mean   &   Standard Deviation &    Min     &   Max     \\
    \midrule
    Investor Interests   &    13,044      &  2.87   &    2.55 &    1     &   34     \\
    Mean Amount Raised   &    7,940       &  1,890,525   &   2,239,282 &    1,000     &   74,000,000     \\
    Number of Females    &    13,045      &  0.29   &  0.56 &    0     &   4     \\
    Company Age          &    13,045      &  6.73   &  2.45 &    2     &   11     \\
    Number of Founders   &    13,045     &  2.04   &  1.00  &    1     &   10     \\

    \bottomrule
    \end{tabular}
 \end{table}

\end{table}

% \vfill




\subsection{Workers}
First, the study looks at the time spent on the tasks across the different treatments by workers. Each participant can choose to spend time completing tasks or switch to an alternative activity where their payment is not based on effort but the time spent on the activity. The bar graph of the seconds left when workers switch to the alternative activity are presented in figure 2 below. The results presented in the figure only consider parts of the study where workers are assigned human evaluators they are unable to change. The treated (1) refers to the Salient Group treatment and treated (0) refers to the Control.  The results show that while treated workers spend an extra 1.9 seconds on average on completing tasks, the difference is not statistically significant. The results of this part of the paper show that workers do not vary their effort based on the salience of group membership. 

\begin{figure}[H]
 \captionsetup{justification=raggedright,singlelinecheck=false}
\caption{Time Spent on Leisure by Treatment}
\includegraphics[scale=0.2]{images/Fig 1.png} 
\end{figure}

The analysis then investigates potential differences in leisure time based on the supervisor's relationship with the worker (in-group vs. out-group). The results are presented in figure 3 below. The two bars on the left represent the control treatment and the two bars on the right represent the Salient Group treatment. Within each treatment, 1 represents in-group match while 0 represent out-group match. It is evident that participants in the control group spend less time working when compared to the participants in the treatment group despite the difference being statistically insignificant. Additionally, there are no statistically significant differences by nature of pairing, in-group or out-group. Individuals in the treated group do not appear to vary effort based on who their supervisor is. 

\begin{figure}[H]
 \captionsetup{justification=raggedright,singlelinecheck=false}
\caption{Time Spent on Leisure by Treatment}
\includegraphics[scale=0.2]{images/Fig 2.png} 
\end{figure}

The paper now examines how worker behavior regarding leisure varies across all parts of the study involving human evaluators. In particular, the section seeks to examine whether there are changes in effort levels in response to the options available in different parts. The results of the analysis are presented in figure 4 below. The three bars on the left represent the Control treatment and the three bars on the right represent the Salient Group treatment. Part 1 is the first time a worker is assigned an individual evaluator, part 2 is the second time a worker is assigned a human evaluator and part 3 is the third time a worker is assigned a human evaluator. Workers in both treatments increase their leisure between parts 1 and 2 even though the difference is not statistically significant.  

\begin{figure}[H]
 \captionsetup{justification=raggedright,singlelinecheck=false}
\caption{Leisure by Parts}
\includegraphics[scale=0.2]{images/Fig 1c Leisure by Parts.png} 
\end{figure}
\subsubsection{Anticipation of Bias among Workers}

This section explores whether workers' perceptions of bias differ based on the salience of group identity. The proportion of participants willing to pay for a preferred evaluator are analyzed. In a scenario where having a preferred evaluator offers no perceived advantage, workers would not be willing to pay anything. In contrast, a willingness to pay signifies a value placed on having their preferred evaluator assess their task in this context. This anticipated value is termed as perceived bias.

Figure 5 illustrates the percentage of workers willing to pay for an evaluator from their own institution (in-group) when randomly assigned an objective computer program evaluator (control group). The treated (1) refers to the Salient Group treatment and treated (0) refers to the Control. The figure reveals that 42.5\% of the control group is willing to pay for an in-group supervisor's evaluation. However, this number increases to 65.0\% for the treated group. This statistically significant difference (P-value = 0.019) suggests an anticipation of positive bias. Furthermore, among the treated participants willing to pay, the average price is 12.32, representing 12. 32\% of the earnings. Workers are willing to pay to switch to an in-group evaluator, presumably to gain an advantage from being evaluated by an in-group supervisor.

\begin{figure}[H]
 \captionsetup{justification=raggedright,singlelinecheck=false}
\caption{Anticipation of Positive Bias}
\includegraphics[scale=0.2]{images/Fig 3 WTP for Ing vs Obj.png} 
\end{figure}

Perceived bias regarding evaluations can manifest in two ways. The first involves an expectation of positive bias, leading participants to pay for a preferred evaluator in hopes of gaining an advantage. Conversely, individuals might perceive negative bias from an evaluator and choose to pay to avoid them altogether. To explore the possibility of negative bias, this section examines the proportion of participants willing to pay for an objective supervisor when randomly assigned an out-group supervisor (someone not from their institution). The treated (1) refers to the Salient Group treatment and treated (0) refers to the Control.Figure 6 depicts the percentage of participants choosing an objective evaluator over a randomly assigned out-group evaluator. The treated (1) refers to the Salient Group treatment and treated (0) refers to the Control. There is no statistically significant difference between the treated and control groups (P-value = 0.889). This suggests that the perception of negative bias is not a key driver of decisions. However, treated workers demonstrate a nuanced understanding of bias. While willing to pay for a positive in-group evaluation (as shown earlier), they are also willing to pay to avoid out-group evaluators  when they can have in-group evaluator(P-value = 0.048). This statistically significant finding suggests that treated workers strategically used the option to gain potential bias from in-group evaluators. The average amount workers are willing to pay to avoid an out-group member is 9.19 representing 9.19\% of their payoff. This signals a higher role of positive bias in decisions instead of negative bias.


\begin{figure}[H]
 \captionsetup{justification=raggedright,singlelinecheck=false}
\caption{Beliefs about Out-group Evaluators}
\includegraphics[scale=0.2]{images/Fig 4 a WTP for obj vs out.png} 
\end{figure}

\subsubsection{Regression Analysis}

This section delves into the impact of the treatment on workers' expectations of bias using a probit regression model. This model estimates the influence of the treatment on the predicted likelihood of a participant anticipating bias. The results of this analysis are presented in Table 3. Table 3 is organized into four main columns. Columns 1 and 2 focus on the anticipation of positive bias, while columns 3 and 4 explore the anticipation of negative bias.


\begin{table}[H]
 \captionsetup{justification=raggedright,singlelinecheck=false}
    \caption{Beliefs Among Workers by Treated }
         \begin{table}[htbp]
    \begin{tabular}{c c c c c c c }
    \toprule
    &\multicolumn{2}{c}{Positive}         &\multicolumn{2}{c}{Negative}      \\

    \textbf{Variables} & \textbf{(1)} & \textbf{(2)}  & \textbf{(1)} & \textbf{(2)}              \\ 

    \textbf & \textbf & \textbf & \textbf{ Base } & \textbf   \\ 

    \midrule
     Treated           &  0.225***   & 0.235***    & 0.013     & 0.009  \\
                       & (0.084)     & (0.097)     & (0.081)    & (0.091)   \\
                       &             &             &            &                  \\
\midrule
    
\textbf{Controls} & \textbf{ NO } & \textbf{YES}  & \textbf{ NO } & \textbf{YES}  \\ 
    \midrule
     N                  &   120          &      120  &     120  &     120        \\          
    \bottomrule
    \addlinespace[1ex]
    \multicolumn{3}{l}{\textsuperscript{***}$p<0.01$, 
      \textsuperscript{**}$p<0.05$, 
      \textsuperscript{*}$p<0.1$}
    \end{tabular}
    \newline
    Note: Clustered standard errors by sessions
\end{table}

\end{table}

The results in table 3 above reveal that group membership significantly impacts the anticipation of positive bias. Workers assigned to the treatment group are 22.5\% more likely to anticipate positive bias compared to the control group (P-value $<$ 0.01). This effect remains strong even after accounting for factors like risk tolerance, age, gender, income, employment status, and religion. In fact, the influence of the treatment on anticipation of positive bias increases slightly to 23.5\% (P-value $<$ 0.01). These robust findings suggest that in subjective environments, workers expect favorable treatment from those within their group.

Moving on to the anticipation of negative bias (columns 3 and 4 of Table 3), the results show no statistically significant differences between the treated and control groups (p-value $=$ 0.877). This suggests that anticipation of negative bias is not a significant driver of workers decisions. Rational participants who expect to be treated negatively will pay an amount equivalent to the value of having an objective evaluator to avoid an out-group evaluator. Even after including control variables, the results remain consistent (P-value $=$ 0.919). 

\subsection{Supervisors}
This section examines supervisor behavior across the treatment and control groups. It is important to note that supervisors receive a fixed payment per evaluation round regardless of worker group affiliation. First, the section examines whether there is a difference in the performance of supervisors. To do this, the accuracy of the evaluations of the supervisors across treatment is examined. Figure 7 illustrates the average accuracy of supervisors in both treatments. The two bars on the left represent the Control treatment and the two bars on the right represent the Salient Group treatment. Within each treatment, 1 represents in-group match while 0 represent out-group match. The data do not reveal significant differences in the supervisor's accuracy between the treated and control sessions (P-value = 0.35). In other words, supervisor performance are unaffected by whether they are  assigned to the treatment or control group. Furthermore, within the treatments, accuracy does not vary by the group affiliation of the worker. 

\begin{figure}[H]
 \captionsetup{justification=raggedright,singlelinecheck=false}
\caption{Accuracy of Supervisor}
\includegraphics[scale=0.2]{images/Fig 5 Accuracy of supervisors.png} 
\end{figure}


This section analyzes whether supervisor performance varies throughout the evaluation process. The error rates across the three parts where the supervisors are human are presented in figure 8 below. The three bars on the left represent the Control treatment and the three bars on the right represent the Salient Group treatment. The data reveals that error rates remain consistent between parts 1 and 2. However, the error rate increases significantly in part 3, where supervisors are tasked with evaluating two workers simultaneously - one from their in-group and one from an out-group.

\begin{figure}[H]
 \captionsetup{justification=raggedright,singlelinecheck=false}
\caption{Accuracy of Supervisors by Parts}
\includegraphics[scale=0.2]{images/Fig 5 b Accuracy of supervisors by Parts.png} 
\end{figure}

\subsubsection{Perceived Discrimination}

This section considers the perception of bias among supervisors. First, a bar graph is used to investigate whether supervisors perception of positive bias varies by treatment group. The treated (1) refers to the Salient Group treatment and treated (0) refers to the Control. The results shown in figure 9 below show that there is no statistically significant difference in anticipation of positive bias among supervisors (P-value=0.70). 

\begin{figure}[H]
 \captionsetup{justification=raggedright,singlelinecheck=false}
\caption{Beliefs about Out-group Members}
\includegraphics[scale=0.2]{images/Fig 6 Supervisors willingness to avoid Outgroup.png} 
\end{figure}
Next, this section examines whether the anticipation of bias varies by treatment group and group affiliation among supervisors. Again, the treated (1) refers to the Salient Group treatment and treated (0) refers to the Control. The results are presented in figure 10 below. It is evident that there is no difference in anticipation of positive bias among treated supervisors by group affiliation (P-value = 0.89).

\begin{figure}[H]
 \captionsetup{justification=raggedright,singlelinecheck=false}
\caption{Beliefs of Supervisors }
\includegraphics[scale=0.2]{images/Fig 6 B Supervisors willingness to avoid Objective.png} 
\end{figure}

\subsubsection{Regression Analysis}

Following the analysis, a probit regression model is used to assess the likelihood that supervisors anticipate bias based on the intervention they receive. The results of this analysis are presented in Table 4 below. 

\begin{table}[H]
 \captionsetup{justification=raggedright,singlelinecheck=false}
    \caption{Beliefs Among Supervisors by Treated }
        \begin{table}[htbp]\centering
    \caption{Matrix of Correlations}
    \begin{tabular}{c c c c c}
    \toprule
    \multicolumn{4}{c}{\textbf{Correlations}}  \\
    \midrule
    Variables      &      (1)      &     (2)      &      (3)      &     (4)\\
    \midrule
    Female Founded * Maternity Score     &    1.000   &               &        &       \\

                &               &                     &        &       \\
    Female Founded * Percentage Female    & 0.763    &     1.000      &           &       \\
                &               &                   &          &       \\
    Amount Raised               &     0.024    &       -0.046        &     1.000     &       \\
                &               &                     &            &       \\
    Number of Venture Firm Interests       &     0.043   &      0.016       &      0.220     &    1.000 \\
    \bottomrule
    \end{tabular}
\end{table}
\end{table}

The results in table 4 are presented in a similar way to the results on worker anticipation of bias above. Columns 1 and 2 present the results on the probability of anticipating positive bias while columns 3 and 4 present the probability of a supervisor anticipating negative bias. It is important to note that, supervisors have been supervising up to the point the decision is made but they do not know what role they will play in the next part when they make this decision. 

The results in the base model shows that treated supervisors do not differ in their likelihood to anticipate positive bias (P-value = 0.90). The effect is consistent if additional controls are added in column 2 (P-value = 0.96). It is apparent that unlike workers, supervisors do not anticipate positive bias despite the subjectivity of the evaluation mechanism. 

The results of the probability of supervisors to anticipate negative bias are presented in columns 3 and 4 of table 4. In the base model, there is no evidence that group membership affects anticipation of bias among supervisors (P-value = 0.144). Inclusion of additional controls for risk preferences, age, gender, income, employment status and religious affiliation do not affect the significance level despite the consistency of the direction of the effect. The overall average expectation of negative among supervisors in the treated group is still not statistically significant (P-value=0.146).


\subsection{Beliefs by Demographic Qualities}
\textbf{Bias and Gender}
\newline
This section explores how gender influences worker and supervisor perceptions of bias. There is no significant difference in who anticipates positive bias based on gender when looking at all participants, regardless of job role (P-value = 0.394). However, a breakdown by job role reveals a trend. Women workers in the treated (P-value=0.082) and control (P-value=0.031) groups are more likely to anticipate positive bias compared to men. Interestingly, there is a general difference across genders where women are more likely to expect negative bias (P-value=0.013). This pattern does not hold true when looking specifically at treated (P-value=0.131) or control worker groups (P-value=0.267).

\textbf{Bias and Race}
\newline
Next, the analysis look at the impact of race in the anticipation of bias. Participants are categorized into whites and non-whites to ensure there is enough power to test the proportions of participants that anticipate bias in the evaluations. There are racial differences in how participants anticipate positive bias, with non-white participants expecting more bias than white participants regardless of job role (P-value=0.015). Furthermore, race does not significantly impact how treated or control worker groups anticipate negative bias. There is no statistically significant difference in anticipation of negative bias by race among treated workers (P-value=0.562) and control workers (P-value=0.440). 


\subsubsection{Regression Analysis}

Finally, to gain a broader understanding of how race and gender interact with bias perception, the data from all participants are analyzed using a probit model. This model analyzes how a participant's self-identified race and gender influence their anticipation of discrimination. Participants are categorized based on their race (white or non-white) and gender (male or non-male). The results of this combined analysis are presented in Table 5 below.

\begin{table}[H]
  \captionsetup{justification=raggedright,singlelinecheck=false}
   \caption{Beliefs by Demographic Qualities }
         \begin{table}[htbp]
    \begin{tabular}{c c c c c}
    \toprule
    \textbf{Variables} & \textbf{(1)} & \textbf{(2)} & \textbf{(3)}         \\ 
    \textbf & \textbf{VC Interests} & \textbf{VC Interests} & \textbf{VC Interests}   \\ 

    \midrule
    WomenLed         &    0.304*** &      0.338***  &        0.292***   \\
                        &    (0.113)  &      (0.080)   &         (0.107)  \\
                        &             &                &                  \\

                        &             &                &             \\
                        &             &                &                    \\
    Firm Age            &             &      -0.048*** &    0.070***        \\
                        &             &      (0.015)   &    (0.023)       \\
                        &             &                &                 \\
                        &             &                &                \\

Industry and Firm Controls   &   No        &   Yes       &          Yes    \\
    Funding Controls    &   No             &   No        &        Yes        \\

    \midrule
     N                  &   13044          &      13044  &     13044      \\          
    \bottomrule
  
    \addlinespace[1ex]
    \multicolumn{3}{l}{\textsuperscript{***}$P<0.01$, 
      \textsuperscript{**}$P<0.05$, 
      \textsuperscript{*}$P<0.1$}

\end{tabular}
\newline
Note: The table reports marginal effects of Tobit Model.

\end{table}

\end{table}

Similar to the previous analysis, Table 5 above presents the findings in separate columns. Columns 1 and 2 focus on positive bias anticipation, while columns 3 and 4 address negative bias anticipation. The results reveal that white participants are generally less likely to anticipate positive bias compared to non-white participants. For instance, the coefficient for white women (-0.174) indicates a 17.4\% decrease in their probability of anticipating positive bias compared to non-white women (P-value $<$ 0.01). Similarly, white men are significantly less likely to anticipate positive bias than non-white women  (P-value $<$ 0.01). This effect remains robust even after accounting for additional factors in column 2 (P-value = 0.023). 

Interestingly, the analysis shows a gender difference in negative bias anticipation (P-value= 0.038). Women, regardless of race, are more likely to anticipate negative bias compared to men. However, race does not significantly influence negative bias anticipation for non-white participants (P-value= 0.946) compared to white men. 

In summary, these findings suggest that women and racial minorities are more likely to anticipate bias, with women being more concerned about negative bias and non-white participants having a higher expectation of positive bias.


\section{Conclusion} \label{sec:Conclusion}

This study delves into the intricate interplay between group membership, perceived bias, and decision-making processes within economic contexts. By examining the behavior of participants from two U.S. universities in a controlled experimental setting, the study uncovers significant insights on how group affiliation influences beliefs and actions regarding performance evaluations. There is significant evidence that workers anticipate bias when there is subjectivity in evaluations. The nature of the anticipation of bias follows the chivalry and solidarity arguments advanced by \cite{eckel2001chivalry}. Workers, unlike supervisors anticipate solidarity from in-group members and value this effect to be about 12.32\% of their payoff. Furthermore, there is evidence demographic qualities are a big driver in the formulation of beliefs about bias. The findings, as demonstrated in the study and the complementary research referenced, underscore the nuanced dynamics at play, highlighting a stronger inclination towards seeking favorable outcomes within in-group settings compared to concerns about potential bias from out-group members.

Notably, participants' willingness to pay to avoid certain evaluators based on group affiliation indicates the real-world implications of perceived biases in evaluation processes. This study extends prior literature by providing empirical evidence on how individuals navigate subjective evaluation systems, illuminating the complexities inherent in workplace dynamics. Moreover, the investigation sheds light on the influence of demographic factors such as race and gender on perceived bias and behavioral responses. The analysis reveals disparities in anticipation of bias based on race and gender. These findings underscore the need for organizations to address systemic biases and foster inclusive environments. The observed differences in anticipation of bias among workers and supervisors, as highlighted in the study show potential areas for intervention and policy reform to mitigate discriminatory practices.

Overall, this study, when viewed alongside the complementary literature, contributes to a deeper understanding of the multifaceted factors shaping decision-making in diverse settings. By uncovering patterns of behavior and perception related to group membership and evaluation processes, this study provides valuable insights for organizations aiming to promote fairness and equity in their workforce. However, acknowledging the limitations of this study, further research is warranted to explore the broader implications of these findings and to identify effective strategies for reducing anticipation of bias and mitigating bias in the workplace.

\bibliography{mybiblio.bib}


\section{Online Appendix}




\subsection{Appendix D : Instructions}
\subsubsection{Part 0}
\textbf{Treated}
Please answer the following survey questions. Your answers will not affect your earnings during this experiment and will be used for this study only. Individual data will not be exposed.
\newline
1.What is your college mascot (Razorback Hog vs Bulldog)? 
\newline
2. Grade/Year:
\newline
(a) Freshmen 
\newline
(b) Sophomore
\newline
(c) Junior 
\newline
(d) Senior
\newline
(e) > 4 years
\newline
(f) Graduate student
\newline
3. What is your university mascot? 
\newline
4. What is the university of most of your friends?
\newline
5. Please list three things that you like the most on your university campus.
\newline
6. What made you decide to attend your university? 
\newline
7. Please rate on a 5-point scale from “strongly disagree” to “strongly agree” with the following statements.
\newline
a). Being a part of my university is an important part of who I am.
\newline
b). Being a part of my university is an important part of the image that I project. 
\newline
c). Being a part of my university is a source of pride for me. 
\newline
University of Arkansas
\newline
1. What is your college mascot (Razorback Hog vs Bulldog)? 
\newline
2. Grade/Year:
\newline
(a) Freshmen 
\newline
(b) Sophomore
\newline
(c) Junior 
\newline
(d) Senior
\newline
(e) > 4 years
\newline
(f) Graduate student
\newline
3. What is your university mascot?
\newline
4. What is the university of most of your friends?
\newline
5. Please list three things that you like the most on your university campus.
\newline
6. What made you decide to attend your university? 
\newline
7. Please rate on a 5-point scale from “strongly disagree” to “strongly agree” with the following statements.
\newline
a). Being a part of my university is an important part of who I am.
\newline
b). Being a part of my university is an important part of the image that I project. 
\newline
c). Being a part of my university is a source of pride for me. 
\newline
\textbf{Control}
\newline
Please answer the following survey questions. Your answers will not affect your earnings during this experiment and will be used for this study only. Individual data will not be exposed.
\newline
1. What is your phone service provider?  AT\&T    Verizon  T-Mobile Other
\newline
2. What is the phone service provider of most of your friends on campus?
\newline
3. Please list three characteristics that make your phone service provider different from the other phone service providers.
\newline
4. Please list three characteristics that make your phone service provider similar to the other phone service providers.
\newline
5. Please rate on a 5-point scale from “strongly disagree” to “strongly agree” with the following statements.
\newline
a). Being part of my phone service provider is an important part of who I am.
\newline
b). Being part of my phone service provider is an important part of the image that I project. 
\newline
c). Being part of my phone service provider is a source of pride for me.
\newline

\subsubsection{General Introduction} 
Welcome!  This is an experiment in decision-making.  During this experiment, you will participate in a series of tasks.  The amount of money you make will depend partly on your actions in these tasks, partly on chance and partly on other participants' actions.  Please turn off mobile phones and any other electronic devices.  They must remain turned off for the duration of this experiment.
\newline

There will be 5 (A- E) separate parts of today’s experiment involving completely separate and unrelated decision tasks.  Parts A-D contain 3 rounds each and part E is played for a round. You will go through each part separately, meaning that after we have gone through instructions for each part, you will make decisions in this part. Your total earnings will be the sum of your earnings from one randomly chosen part from parts A to D, part E and a \$5 participation fee.  Your earnings are given in experimental currency units (ECUs).  At the end of the experiment, you will be paid in private and IN CASH based on the following exchange rate:

150 Experimental Currency Units = \$1. 

You must not communicate with each other.  If you have questions, please raise your hand and an experimenter will come to help you. Before we proceed, please make sure you have a piece of paper and a pen to write with. 

\subsubsection{Instructions on Tasks}
\newline
\textbf{Slider Completion Task}
\newline
In each paying round, you will undertake a task that lasts 75 seconds.  The task will consist of a screen with 30 bars.  We will call these bars “sliders” as on each bar there is a marker you can slide along the bar with your mouse. To move the marker, you can click on the marker and drag it along. Each marker is initially positioned at a point that is different from the center.  Your task will be to set the markers on as many sliders as you choose, to a position in the center. You will earn a payoff of 32 ECUs for each slider correctly aligned. A correctly aligned slider is one evaluated to be in the center of the bar, i.e., between 48 and 52. [Indicate 48 to 52 point]

Each paying round of the slider completion task begins with 30 sliders arranged in 1 column. We will call this screen 1. On this screen, you will be able to align sliders for as much of a 75-second round as you wish. In each of the 3 paying rounds of each part, you will be assigned a role – role A and role B. Role A has to complete the slider tasks. The payoff of role A in each round is determined by the number of correct sliders aligned in screen 1, at a rate of 32 ECUs per correctly placed slider. Remember, a correctly placed slider is one evaluated to be placed in the center of the bar. Your payoff from screen 1 is calculated as sliders correctly aligned multiplied by 32 ECUs. 

Additionally, you will also have the opportunity to engage in some alternate activity available on screen 2 if you choose to do so by pressing the “Go to screen 2” button at the end of screen 1. Your potential payoff from screen 2 is based on the total time spent on screen 2. This amount drops the longer you spend on screen 1.  We have provided a simulator below to help you keep track of how much you could earn should you decide to go to screen 2. 

[To be shown on screen 2: On this screen, you can do a word jumble. For the word jumble, you will see a matrix of letters. Inside that matrix are hidden words going across, down, backwards or diagonal. When you see a word, you can type it into the text box below the matrix and click on the “send” button. A word must be a minimum of three letters to be valid. Remember, your payoff from this screen is not affected by the activity you do here but by the time spend here.]

In sum, you can spend all of your time on screen 1 and none on screen 2, all of your time on screen 2 and none on screen 1 or a mix of time spent on both. Your total payoff for a round is the sum of your earnings from screen 1 and screen 2. At the end of the screen, you can also click “See Potential Screen 2 Earnings” to use the simulator. 


[To be shown on simulator area: In order for you to more easily see how these costs work, you can click the button "See Potential Screen 2 Earnings" where you can move the handle to see how much you will earn if you switch to screen 2 at different times.]

\textbf{Evaluation Task}
\newline
In each paying round, you will undertake an identical task that lasts 120 seconds. You will evaluate tasks completed by another participant. Your task is to evaluate whether the purple marker is aligned in the center of the bar, i.e., between 48 and 52. To evaluate the bars, you must review the position of each of the markers on the sliders. Once you evaluate a bar, unclick on the checkbox located on the left side of the bar to evaluate the bar as being incorrectly aligned. 

The decision of whether a slider is correct is determined by your evaluation of the position of the marker. Please note that each bar has a different starting point and a corresponding different center.  In each of the 3 paying rounds of each part, you will be assigned a role – role A and role B. Role B has to evaluate the sliders.  In each round, the round payoff of role B is a flat rate of 450 ECUs. Once you are done evaluating all the bars, click on the “next” at the button of the page to proceed to the next page. 

Practice Round [Instructions Before they begin practice round Sliders]
We will now begin with a practice round. There are a few additions to this screen over the version for the paying rounds. First, you will note the button “Go to Screen 2”. During a paying round, you will be able to click on this button to switch to screen 2 but once you choose to go to screen 2, you will not be able to come back to screen 1 in that round. For this practice screen, you will be able to switch back and forth freely. At the end of the screen, you can also click “See Potential Screen 2 Earnings” to see your potential screen 2 earnings. Remember, during a paid round, you can switch to screen 2 when you decide you no longer want to align sliders. To see the activities available on screen 2, click “Go to screen 2”.

While on this screen, you can do a word jumble. For the word jumble, you will see a matrix of letters. Inside that matrix are hidden words going across, down, backwards or diagonal. When you see a word, you can type it into the text box at the lower right and click on the submit button. You can complete the word jumble to the best of your abilities until the time allocated is finished. These activities will be available for you to engage in as you wish. Remember, there are no earnings for completing these activities because compensation for screen 2 is calculated by the time spent on screen 2. Once you switch to this page, you will stay on it until the round ends. However, in this practice round, you can click on “Go back screen 1” to return to screen 1.
Practice Round [Instructions Before they begin practice round – Slider Evaluation]
Now, you will evaluate tasks completed by another participant. Your task is to evaluate whether the purple marker is aligned in the center of the bar, i.e., between 48 and 52. To evaluate the bars, you must review the position of each of the markers on the sliders. Once you evaluate a bar, unclick on the checkbox located on the left side of the bar to evaluate the bar as being incorrectly aligned.
Once you are done evaluating all the bars, click on the “next” at the button of the page to proceed to the next page.
\newline
\textbf{Paying Rounds: General Reminder}
\newline
The practice rounds are finished. We will now move on to the paying rounds. In each part, you will be assigned a role as: role A (Slider task completion) or role B (Slider task evaluation). Your role will be fixed throughout each part. During each part, a role A will be randomly paired with a role B. You will never be paired with the same person you are paired with in any part for the duration of this experiment. That is, you will never be paired with the same person twice during this experiment.

In each of the 3 paying rounds of each part, role A has to complete the slider tasks. Role B evaluates the slider tasks. The task screen for each player will show the time remaining, his/her university mascot and his/her pairing’s university mascot, if applicable. Please note, Role B doesn’t undertake the tasks but evaluates the tasks after their paired role A finishes.

In each round, the payoff of role A is determined by the number of correct sliders aligned in screen 1, at a rate of 32 ECUs per correctly placed slider, and the earnings from screen 2. Specifically, Role A’s payoff = sliders correctly aligned*20 ECUS + the amount earned from time spent on Screen 2. For example, if role A (slider completion) correctly aligned 15 sliders as evaluated by the evaluator and spent no time on screen 2, then the player’s payoffs are as follows:
Role A’s payoffs: 15*32 ECUs + 0 = 480 ECUs


In each round, the payoff of role B is a flat fee. Role B’s Round payoff = a flat rate of 450 ECUs. For example, if a role B evaluates the 30 sliders completed by another participant, then the player’s payoff are as follows: 
Role B’s payoffs: 450 ECUs
Are there any questions?

\subsubsection{Group Pairing}
\newline
Who’s Paired with Whom?

We will now go over the pairing rules. 

For the paying rounds, each participant will be paired with different persons or a computer, one in each part.  The pairings will be changed after each part, and no one will be paired with the same person twice.  Note, you will always be paired with someone of a different role such that each role A will always be paired with a role B. Additionally, your pair can have your university mascot or the other university mascot, if applicable. For the purposes of this experiment, a university mascot is the mascot of the university a participant currently attends.  

\newline

The pairings are done in such a way to guarantee the following: 
\newline
 
  (1). who you will be paired with does not depend on your previous actions;
  \newline

  (2). the actions you take in one part cannot affect, either directly or indirectly, the actions of the people you will be paired with in later parts;
  \newline

  (3). the actions of the person you are paired with in any given part cannot be affected by your actions in earlier parts; 
  \newline

Additionally, you will be shown information about your match on the task screens. 
A task completion role (role A) will be represented by an emoticon holding a hammer as follows:  
A task evaluation role (role B) will be represented by an emoticon holding a pen as follows:  

Furthermore, the mascots of both pairings will be represented by their mascots and shown on the task screens. 

Hence, a role A from Minnesota (University) will be represented as:  

Similarly, a role B from Arkansas (University) will be represented as:  
Finally, a computer accomplishing the task will be represented as:  

Are there any questions?  We will start the first paying round after everyone answers the following quiz question. 

\textbf{Comprehension Questions}
\newline
Quiz 1: For the 3 paying rounds, everyone will be randomly paired with 3 different persons from one’s own group.  No one’s previous decisions will have any impact on whom he/she will be paired with in later rounds. (True/False)
\newline
Quiz 2: Each part begins with a role A and a role B as a group pairing. 
(True/False)
\newline
Quiz 3: This image (see image of computer evaluator below in figures):    represents:
\newline
Quiz 4: This image (see image 2 of Arkansas Worker below in figures):    represents:



\subsubsection{Vickrey Belief Elicitation}
\newline
In this part, there are “robots” available to help you change your evaluator should you choose to. We have 100 different robots; each has a different rate. Each Robot is equally likely to be chosen. Each Robot has a rate corresponding to an integer between 1 and 100, inclusive. That is, there is a Robot that charges 1\% of the payoff, a Robot that charges 2\% of the payoff, a Robot that charges 3\% of the payoff, ... , all the way up to a Robot that charges 100\% of the payoff. A Robot that charges 100\% of the payoff will charge 100\% of the payoff of screen 1 after the change in evaluator. Similarly, a robot that charges 1\% of the payoff will charge 1\% of the payoff of screen 1 after the change in evaluator. 
Once you decide, the change is implemented by a randomly picked Robot if your willingness to pay is greater than or equal to the charges of the randomly picked Robot. Additionally, if you are willing to switch to more than one evaluator, one of your preferred evaluators is randomly chosen for the change by the software. 


In summary, if you want to be evaluated by your preferred evaluator and your willingness to pay is greater than or equal to the rate charged by the randomly chosen Robot then, you will be evaluated by your preferred evaluator.  So, the score from screen 1 will be the score as if you were evaluated by your preferred evaluator.
 Your payoff will be as follows: (100\% – \%Charge of the Robot) * Your Payoff as evaluated by your preferred evaluator + Payoff from time spent on screen 2. 
 
 
For example, if you chose 1\% ( or 100\%) as your willingness to pay to switch from your randomly assigned evaluator to a different evaluator, and the Robot randomly selected for that change charges 1\% (or 100\%), this change will be implemented. Assuming your payoff you’re your preferred evaluator is 500 ECUs. 
Your payoff will be: (100-1/100) * Your Payoff as Evaluated by your preferred evaluator + Earnings from time spent on screen 2.
Your payoff will be: (99/100) * 500 + Earnings from time spent on screen 2. 
Your payoff will be: 495 + Earnings from time spent on screen 2. 

 
For more details, please click on the drop-down button.
\newline
[To be shown under Details Button: Recall, you have been randomly assigned an evaluator from your home mascot, the other mascot or the computer in this part of the experiment. You will be asked to decide whether you will be willing to switch your randomly paired evaluator to a preferred evaluator. If you choose to switch, you will be asked to enter the percentage of your payoff you are willing to pay to switch to a preferred evaluator.  If this part is chosen for payment and your willingness to pay is greater than or equal to the charges of the randomly chosen Robot, we apply these three steps to adjust your score: 
Step 1: There are three scores in this part – computer evaluator, an evaluator with your mascot and an evaluator with the other mascot. If this part is chosen for payment, then you are randomly assigned one of the three evaluators. 
Step 2: If your willingness to pay to switch from your randomly assigned evaluator to your preferred evaluator is greater than or equal to the rate charged by the randomly chosen robot to make the change, then the change is implemented, and your screen 1 payoff reflects evaluation done by your preferred evaluator minus the charges of the Robot that executed the change. 
Step 3: If you chose to switch to more than one alternative preferred evaluators, one of your preferred choices will be randomly chosen for the switch if conditions in step 1  and step 2 are met for both alternatives. 
 
[For example, if your randomly assigned evaluator is the computer and you prefer to pay 50\% to have your preferred evaluator; if the Robot randomly chosen to implement the change charges less than or equal to 50\%, then your payoff will be calculated as follows: (100\% – \% Charges of the robot) * Your Payoff as evaluated by your preferred evaluator + Payoff from time spent on screen 2]
\newline
\textbf{Comprehension}
\newline
Quiz 1: If you chose to pay 1\% of your payoff to switch to your preferred evaluator, and the Robot randomly selected for that change charges 20\%. Will the change be implemented? 
\newline
Quiz 2: If you chose to pay 50\% of your payoff to switch to your preferred evaluator, and the Robot randomly selected for that change charges 20\%. How much will you be charged if the change is implemented? 
\newline
Quiz 3: What is the range (minimum and maximum) of charges of the Robots? 

\subsubsection{Figures}
\newline
\begin{figure}[H]
 \captionsetup{justification=raggedright,singlelinecheck=false}
\caption{Image of Computer Evaluator }
\includegraphics[scale=0.2]{figure2/C.jpg} 
\end{figure}

\begin{figure}[H]
 \captionsetup{justification=raggedright,singlelinecheck=false}
\caption{Image of Arkansas Worker }
\includegraphics[scale=0.2]{figure2/WA.jpg} 
\end{figure}

\begin{figure}[H]
 \captionsetup{justification=raggedright,singlelinecheck=false}
\caption{Screenshot of 30 bars in screen 1  }
\includegraphics[scale=0.2]{figure2/Screen1.png} 
\end{figure}

\begin{figure}[H]
 \captionsetup{justification=raggedright,singlelinecheck=false}
\caption{Screenshot of Alternative Task in screen 2  }
\includegraphics[scale=0.2]{figure2/Screen2.png} 
\end{figure}
% \end{document}