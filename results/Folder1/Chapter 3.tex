% \documentclass[12pt]{article}
% % \documentclass[12pt]{report}
% \usepackage[a4paper, total={6in, 8in}]{geometry}
% \large
% \usepackage{booktabs}
% \usepackage{setspace}
% \usepackage[hidelinks]{hyperref}
% \usepackage{graphicx}
% \usepackage{float}
% \usepackage{xcolor}
% \usepackage{lscape}
% % \usepackage[english]{babel}
% % %Includes "References" in the table of contents
% \usepackage[nottoc]{tocbibind}
% % \bibliographystyle{apa}

% \usepackage[natbibapa]{apacite}  % to enable '\citet' and '\citep' macros
% \bibliographystyle{aer}

% BIBLIOGRAPHY %%%%%%%%%%%%%%
% \usepackage[natbibapa]{apacite}  % to enable '\citet' and '\citep' macros
% \bibliographystyle{apacite}
% %%%%%%%%%%%%%%%%%%%%%%%%%%%%

% \title{
% {Do investors show team composition preferences in funding Startup teams? An analysis of Venture Funding}\\
% {\large University of Arkansas}\\
% % {\includegraphics{university.jpg}}
% }
% \author{George Agyeah\thanks{University of Arkansas}}


% \date{\today
% \begin{document}

% \maketitle

% \begin{abstract}
% \noindent This paper examines the relationship between gender composition of early-stage startups and investor interest, as well as funding success. While women-led startups attract greater interest from venture capitalists, the impact on funding remains unclear.  The tobit regression model suggests a decrease in mean funding for women-led startups, while the instrumental variable (IV) regression reveals a different picture. Further analysis reveals that women tend to gravitate towards underfunded industries. Additionally, female-led startups exhibit lower funding performance compared to their peers within those industries, and they are less likely to enter sectors with more balanced funding, such as technology and finance.  \\
% \vspace{0in}\\
% \noindent\textbf{Keywords: Corporate Finance, Gender, Startups}\\
% \vspace{0in}\\
% \noindent\textbf{JEL Codes:G3, J16, J82 } \\

% \end{abstract}
% \setcounter{page}{0}
% \thispagestyle{empty}
% \title{
% {Do investors show team composition preferences in funding Startup teams? An analysis of Venture Funding} }\\

% \maketitle 

\pagestyle{plain} 


\section{Introduction} \label{sec:introduction}



Despite significant progress towards gender parity in many professions, women remain underrepresented in Science, Technology, Engineering, and Mathematics (STEM) fields. This disparity extends to entrepreneurship, where women founders are a distinct minority, even within industries dominated by female workers. The data paints a clear picture: in 2019, women made up over half (51.8\%) of the workforce in management, professional, and related occupations. Women have also surpassed men in certain sectors, such as education and health (74.8\% women), financial services (52.6\% women), and leisure and hospitality (51.2\% women). \footnote[1]{https://www.bls.gov/opub/reports/womens-databook/2020/home.htm} However, existing literature suggests that women entrepreneurs make up only 10-15\% of all entrepreneurs. Understanding the root causes of this phenomenon is crucial in developing strategies to bridge the gender gap.

The influence of startups extends far beyond individual lives. Household names like Uber, Facebook, Google, and Tesla all relied on VC funding during their early stages. Given the significance of both startups and the VC firms that support them, understanding how team composition affects a company's success becomes crucial. This study examines how the gender makeup of a startup team impacts their ability to attract investor interest and secure funding during the critical seed stage. Seed funding is vital. It provides the initial resources to develop the product and propel the company forward. Venture capital (VC) plays a central role at this stage, with over 60\% of US companies achieving an initial public offering (IPO) having received VC financing \citep{kaplan2010}. VC-backed companies demonstrably drive innovation, contributing to 44\% of R\&D activities among US public companies \cite{gornall2020}.

This study focuses on a specific question:   Does the gender makeup of a founding team influence their success in attracting venture capital interests and funds?  In simpler terms, do teams with different gender configurations experience variations in their fundraising interests and outcomes?

\section{Literature Review} \label{sec:literature}
\hspace *{0mm} Investigations on the gender gap in entrepreneurship have gained prominence in the past few years. Studies by \cite{gompers2017diversity} reveal a concerning trend: women comprise less than 10\% of the entrepreneurial and venture capital workforce, highlighting a substantial gender disparity. Interestingly, \cite{gompers2017diversity} also find a positive correlation between gender diversity in investor firms and improved deal flow and fund performance. \cite{koning2019} explore how the gender composition of entrepreneurial teams influences the nature of their innovations. Their findings suggest that teams with female inventors are more likely to focus on patents related to women's health, hinting at a potential link between inventor demographics and the direction of innovation efforts. Similarly, \cite{einio2019} show that innovators tend to create products that resonate more with customers who share similar characteristics, including gender with the innovators.

Despite the potential for innovation, securing funding can be an unequal process for diverse startups. Research by \cite{ewens2020} exposes a concerning bias in investor behavior. The results of their study reveal that male investors express less interest in women entrepreneurs. This bias can create a significant hurdle for women seeking funding for their ventures. A compelling study by \cite{calder2021} examines the impact of the high-profile gender discrimination case of Ellen Pao v. Kleiner Perkins on the industry. Their findings suggest that the case led to increased hiring of women partners in relevant venture firms. This shift in hiring practices ultimately resulted in more funding being directed towards Women$-$led startups. This highlights a potential ripple effect from legal action promoting diversity. 

Building on this research, this study aims to contribute to a deeper understanding of the gender gap in entrepreneurship. Through an empirical investigation, The paper analyzes how the gender makeup of entrepreneurial teams impacts their ability to secure funding and attract interests from venture capital firms. By addressing these aspects, this paper hopes to illuminate the challenges women face and identify opportunities to foster a more equitable environment for innovation and success.

\section{Data} \label{sec:data}
\subsection{Data Construction}
\hspace *{0mm} My primary data source for this study is proprietary data from Crunchbase, a comprehensive platform often referred to as the "LinkedIn of company data".  Originally part of TechCrunch, Crunchbase became a separate entity focused on empowering investors with data-driven decision making. While the data is self-reported by startups, Crunchbase employs verification methods using two sources: investor firms and the startups themselves.  Startups have a vested interest in accurate reporting as it signals confidence to potential investors. Investors benefit by publicizing their investments, enhancing their portfolio companies' appeal and potential returns.

The Crunchbase database encompasses information on companies, key personnel, and investment activities. To facilitate this analysis, I integrate various datasets within the Crunchbase platform. Table 1 below summarizes these key datasets, which cover startups, founders, investors, venture funding organizations, and fundraising outcomes. The data spans from the 1980s to 2021, but data comprehensiveness weakens for earlier years. Therefore, the analysis focuses on the 2010-2020 timeframe. Each dataset in Table 1 includes relevant variables. For example, the "people" dataset provides names, genders, locations, affiliations, and social media links for founders and investors. Similarly, the "investments" dataset tracks investment details like date, investor(s), recipient organization, and the specific funding event.

\begin{table}[H]
 \captionsetup{justification=raggedright,singlelinecheck=false}
    \caption{Summary Statistics: Companies}
        {
	\def\sym#1{\ifmmode^{#1}\else\(^{#1}\)\fi}
	\begin{tabular}{l*{1}{ccccccc}}
		\toprule
		& Info & RCL & No Info  & Feedback(t=1,2) & Feedback(t=9,10) & Description   \\
		\midrule
		20         &      120 &         59 &      120   &      8  &   16  &  60 \\
		\midrule
		30         &      240 &        118 &      240   &      21  &   22 &  120 \\
		\midrule
		50         &      480 &        236 &      480   &      55  &   51 &  240 \\
		\midrule
		70         &      240 &        118 &      240   &      24  &   21 &  120 \\
		\midrule
		80         &      120 &         59 &      120   &      12  &   10 &  60 \\
		\bottomrule
	\end{tabular}
}
\end{table}


\hspace *{0mm} The analysis focuses on the seed funding stage, which is the critical period where startups seek initial external funding to propel their growth.  Venture capital firms play a central role in seed funding.  The broader startup ecosystem progresses through four stages: pre-seed, seed, post-seed series, and finally, the exit stage. Startups may revisit these stages based on funding needs. Given the data's nature, I consider two key metrics: the average amount of capital raised per startup and the average number of venture capital firm interests a startup garners. Table 2 summarizes these variables. The average seed funding amount in the data set is approximately \$1.9 million, which is slightly lower than the broader ecosystem's average of \$2.2 million \footnote[2]{https://www.fundz.net/what-is-series-a-funding-series-b-funding-and-more}. Similarly, the average number of interested venture capital firms per startup is 2.87, with a range of 1 to 34 firms expressing interest per startup.

Importantly, women founders are underrepresented in the data set, comprising only about 11\%.  Regarding other startup characteristics, the average company age is 6.73 years (range: 2-11 years). The number of founders per startup also varies, with a minimum of 1 and a maximum of 10, and an average founding team size of 2.04 people.  This data will be crucial for analyzing how gender makeup of founding teams impacts their ability to secure funding from VCs.

\begin{table}[H]
 \captionsetup{justification=raggedright,singlelinecheck=false}
    \caption{Summary Statistics: Companies}
        \begin{table}[htbp]\centering
    \begin{tabular}{c c c c c c}
    \toprule
    \multicolumn{6}{c }{\textbf{Summary of Main Variables}}  \\
    \midrule
    Variables            &    Obs      &  Mean   &   Standard Deviation &    Min     &   Max     \\
    \midrule
    Investor Interests   &    13,044      &  2.87   &    2.55 &    1     &   34     \\
    Mean Amount Raised   &    7,940       &  1,890,525   &   2,239,282 &    1,000     &   74,000,000     \\
    Number of Females    &    13,045      &  0.29   &  0.56 &    0     &   4     \\
    Company Age          &    13,045      &  6.73   &  2.45 &    2     &   11     \\
    Number of Founders   &    13,045     &  2.04   &  1.00  &    1     &   10     \\

    \bottomrule
    \end{tabular}
 \end{table}

\end{table}

% \vfill

\subsection{Summary Statistics}

\hspace  *{0mm} To comprehensively understand the startups under consideration, various data points for each founding team are collected. This data encompasses the team's gender composition, educational background, size, company age, industry, funding outcomes, number of venture capitalist investors, location, funding year, and current status. Table 3 summarizes the key observations within the final dataset. The dataset includes 13,045 startups. Notably, 3,074 (or 23.56\%) of these startups have at least one woman founder. I refer to these startups as Women$-$Led. All-men teams make up the remaining firms. 

\begin{table}[H]
 \captionsetup{justification=raggedright,singlelinecheck=false}
    \caption{Summary Statistics: Companies}
         \begin{table}[htbp]
    \begin{tabular}{c c c c c c c }
    \toprule
    &\multicolumn{2}{c}{Positive}         &\multicolumn{2}{c}{Negative}      \\

    \textbf{Variables} & \textbf{(1)} & \textbf{(2)}  & \textbf{(1)} & \textbf{(2)}              \\ 

    \textbf & \textbf & \textbf & \textbf{ Base } & \textbf   \\ 

    \midrule
     Treated           &  0.225***   & 0.235***    & 0.013     & 0.009  \\
                       & (0.084)     & (0.097)     & (0.081)    & (0.091)   \\
                       &             &             &            &                  \\
\midrule
    
\textbf{Controls} & \textbf{ NO } & \textbf{YES}  & \textbf{ NO } & \textbf{YES}  \\ 
    \midrule
     N                  &   120          &      120  &     120  &     120        \\          
    \bottomrule
    \addlinespace[1ex]
    \multicolumn{3}{l}{\textsuperscript{***}$p<0.01$, 
      \textsuperscript{**}$p<0.05$, 
      \textsuperscript{*}$p<0.1$}
    \end{tabular}
    \newline
    Note: Clustered standard errors by sessions
\end{table}

\end{table}
 
\hspace *{0mm} Next, the distribution of startups across industries is examined. \footnote[3]{I use refinitiv industry classification, the details of which can be found at the website: $www.refinitiv.com/content/dam/marketing/en_us/documents/quick-reference-guides/trbc-business-classification-quick-guide.pdf$}. Figure 1 presents a time series visualization of the number of startups growth by industry between 2010 and 2020 in two panels. The left side of the figure depicts the all men$-$led startups, while the right side shows Women$-$led startups. Overall, the technology industry dominates the landscape of seed-seeking startups, constituting roughly 37\% of the data set. Consumer goods follows distantly at 19.7\%, with industries like industrials, education, government institutions, finance, and healthcare trailing behind. Notably, both all men$-$led and Women$-$led startups exhibit an upward trend in numbers since 2010. However, a significant decrease is observed in late 2019, potentially due to the global COVID-19 pandemic. Figure 2 delves deeper into the specific proportions of all-men to Women$-$led startup teams across industries.

\begin{figure}
 \captionsetup{justification=raggedright,singlelinecheck=false}
\caption{Time Series of Startups by Industries }
\includegraphics[scale=0.4]{figures_3/Figure1.png} 
\end{figure}


\hspace *{0mm} Figure 2 shows a breakdown of all-men versus Women$-$led startups across different industries. All men$-$led startups dominate the technology and financial sectors with a majority 84.3$\%$ and 80.7$\%$ of startups respectively being all men$-$led. Women$-$led startups account for 42.1$\%$ of the utilities sector and 29.2$\%$ of the institutions based startups. Women-led startups do not form a majority in any of the industries. This is not surprising considering the evidence that has been shown in other literature. On the extreme end left is the basic materials industry which is 100\% all men$-$led. This is partly explained by the number of companies in the basic materials industry being only 5, representing 0.04\% of the total. 

\begin{figure}
 \captionsetup{justification=raggedright,singlelinecheck=false}
\caption{Time Series of Startups by Industries Grouped According to Founding Team Gender Composition }
\includegraphics[scale=0.4]{figures_3/Participation.png} 
\end{figure}

\section{Results} \label{sec:Empirical Strategy}

\subsection{Empirical Strategy}

This section outlines how the paper analyzes the impact of gender composition on two key startup outcomes during the seed funding stage: the number of interested venture capital firms and the average amount of capital raised. The Crunchbase data only includes companies with at least one investor. Hence, an ordinary least squares regression model would not be suitable. This is because the number of interested firms has a lower bound of 1.  To account for this "censored" data, I employ a Tobit model. This technique allows for a more accurate assessment of how team gender composition influences the number of venture capital firms interested in a startup. Additionally, analyzing the total amount raised can be challenging because the number of funding rounds varies across startups. Therefore, I focus on the average amount raised during the seed stage.  Furthermore, to address potential skewness in the data distribution of funding amounts, a common practice in economics, the natural logarithm (log) of the mean amount raised is used in the regression analysis.

The analysis proceeds in two main parts. First, this section of the paper examines how the gender composition of the founding team influences investor interest and overall funding outcomes for startups.The analyses explore the specific impact of having women founders on startup success in securing funding and attracting venture firm interests. Second, I utilize instrumental variable regression to verify the robustness of the findings. This technique helps isolate the causal effect of having a woman on a startup team on the two key outcomes: the number of interested venture capital firms and the amount of seed funding raised.

\subsection{Tobit Regression}
\hspace *{0mm} To understand the relationship between the gender composition of a startup team on its success, I estimate the following tobit regression model using equation 1 below: 

\begin{center}
$Y_{i}=\alpha +\beta_1WomenLed_i+\beta_2X_{i} + \beta_{3}\chi_{k} + \beta_4\omega_{j}  + \epsilon_{i} \quad $(1)$ $
\end{center}
        
\noindent where $Y_{i}$ is either the number of venture firms that made investments or the amount of money invested in startup i during the seed stage of fundraising;  $WomenLed_i$ is dummy variable that is 1 if the startup team has a woman in the founding team, $X_{i}$ is the startup (firm) related characteristics, $\chi_{k}$ is the funding related characteristics, $\omega_{j} $ is the state related characteristics and $\epsilon_{i}$ is the error term. 

The first outcome variable I investigate is whether being a Women$-$Led startup affects the venture firm interests garnered.As early stated, in the context of venture capital financing, a startup typically receives its first outside investment from a venture capital firm, and this investment marks the beginning of the startup's engagement with outside investors. For this reason, the minimum number of investor firm interests is censored on the lower end at 1 in the tobit regression to  estimate the effect of the presence of a woman on the startup team on the number of venture firm interests attracted. The results of the tobit model are presented in table 4 below. 


\begin{table}[H]
 \captionsetup{justification=raggedright,singlelinecheck=false}
    \caption{Tobit: Number of Venture Firm Investments}
         \begin{table}[htbp]
    \begin{tabular}{c c c c c}
    \toprule
    \textbf{Variables} & \textbf{(1)} & \textbf{(2)} & \textbf{(3)}         \\ 
    \textbf & \textbf{VC Interests} & \textbf{VC Interests} & \textbf{VC Interests}   \\ 

    \midrule
    WomenLed         &    0.304*** &      0.338***  &        0.292***   \\
                        &    (0.113)  &      (0.080)   &         (0.107)  \\
                        &             &                &                  \\

                        &             &                &             \\
                        &             &                &                    \\
    Firm Age            &             &      -0.048*** &    0.070***        \\
                        &             &      (0.015)   &    (0.023)       \\
                        &             &                &                 \\
                        &             &                &                \\

Industry and Firm Controls   &   No        &   Yes       &          Yes    \\
    Funding Controls    &   No             &   No        &        Yes        \\

    \midrule
     N                  &   13044          &      13044  &     13044      \\          
    \bottomrule
  
    \addlinespace[1ex]
    \multicolumn{3}{l}{\textsuperscript{***}$P<0.01$, 
      \textsuperscript{**}$P<0.05$, 
      \textsuperscript{*}$P<0.1$}

\end{tabular}
\newline
Note: The table reports marginal effects of Tobit Model.

\end{table}

\end{table}

The tobit model shows that the composition of a startup team has statistically significant impact on the number of VC interests the startup attracts. The results from column 1 of table 4 show that Women$-$Led teams on average attract 0.304 more venture firm interests (P-value$<$0.001). The results in column 2 of table 4 show that the effect of a startup team having a woman co-founder is robust to the inclusion of industry and firm controls. In fact, the magnitude of the effect of being a Women$-$Led is higher at 0.338 increase in venture interest on average for Women$-$Led startups (P-value $<$ 0.001). The effect of a startup being Women$-$Led stays consistent and statistically significant (P-value $<$ 0.001) in column 3 which includes funding controls as well as industry and firm controls. Overall, the impact of being a Women$-$Led startup is robust to additional controls. The most conservative effect size of 0.292 represents 11.45\% of the average number of venture firm interests. The estimate for the effect of firm age on venture firm interest is inconsistent despite being statistically significant (P-value $<$0.01). In column 2, firm age is negatively correlated with venture firm interests while in column 3, the age of a firm is positively correlated with venture firm interests. 

Next, I look at the impact of the gender composition on funding realizations during the seed stage of investments. The base specification regresses the log of the mean amount raised by a startup on a categorical variable of the startup being Women$-$Led as expressed in equation 1 above. Similar to the analysis on the number of venture firm interests attracted, the tobit regression model is utilized and the lower bound of the data is censored (now \$1000) to account for the dynamic of the data set. The results of the tobit model are presented in table 5 below. The base model presented in column 1 of table 5 shows that the presence of a woman in a startup team is negatively related to the amount the team raises (P-value$<$0.001). Women$-$Led startups raise on average 16.64\% less money than homogeneous men startup teams. The magnitude of the effect reduces as additional controls are added in column 2 and column 3 of table 5 but the direction and robustness of the effect remains consistent. In column 2 of table 5, the addition of controls for the startup and industry characteristics reduces the magnitude of the effect to 13.32\% less funds raised on average by Women$-$Led startups as compared to all$-$men startup teams (P-value$<$0.001). Finally, in column 3 of table 5, the results of the estimate for the impact of gender composition on fund raising shows that Women$-$Led startups raise 14.27\% less money as compared to homogeneous men teams (P-value$<$0.001). This result reveals an interesting mechanism at play. It is apparent that, the increase in venture firm interests that Women$-$Led startups gain does not translate into an increase in amount raised by the Women$-$Led startups. In essence, while the presence of a woman on a team is helpful in attracting venture firm interests, the intensive margin of how much is raised is negatively affected. Furthermore, it appears older firms on average raise less money (P-value$<$0.001).

\begin{table}[H]
 \captionsetup{justification=raggedright,singlelinecheck=false}
    \caption{Tobit: Funds Realized}
         \begin{table}[htbp]
    \begin{tabular}{c c c c c}
    \toprule
    \textbf{Variables} & \textbf{(1)} & \textbf{(2)} & \textbf{(3)}               \\ 
    \textbf & \textbf{Log Amount} & \textbf{Log Amount} & \textbf{Log Amount}  \\ 

    \midrule
     Women$-$Led        & -0.182***   &      -0.143*** &       -0.154***   \\
                        &  (0.042)    &      (0.034)   &        (0.034)  \\
                        &             &                &                \\

                        &             &                &               \\
                        &             &                &         \\
    Firm Age            &             &      -0.149*** &     -0.054***    \\
                        &             &      (0.006)   &      (0.010)       \\
                        &             &                &                  \\
    
Industry and Firm Controls   &   No        &   Yes       &        Yes    \\
    Funding Controls    &   No             &   No        &         Yes        \\

    \midrule
     N                  &   7940          &      7940  &     7940        \\          
    \bottomrule
    \addlinespace[1ex]
    \multicolumn{3}{l}{\textsuperscript{***}$p<0.01$, 
      \textsuperscript{**}$p<0.05$, 
      \textsuperscript{*}$p<0.1$}
    \end{tabular}
    Note: The table reports marginal effects of Tobit Model.
\end{table}

\end{table}

% \vfill

\subsection{IV Regression}
The analysis above could be biased due to omitted variables that might influence both the percentage of women founders and the funding outcomes (number of investors and amount raised).  One concern is that successful startups might strategically add women founders to their teams after achieving initial traction. To address this and strengthen the findings, I employ Instrumental Variable (IV) regression. An effective instrument in this context would be a variable correlated with the presence of women founders but not directly linked to the funding outcomes. For this purpose, I utilize a startup's state maternity leave benefits as an instrumental variable. The rationale is that states with generous maternity leave policies are likely to have a higher percentage of women founders, as these policies allow women to return to work after childbirth.

This instrumental variable approach is supported by prior research. \cite{dustmann2012} demonstrate a strong correlation between expanded maternity benefits and a mother's return to work. Similarly, \cite{gottlieb2022} found that improved maternity benefits in Canada led to an increase in women starting businesses.  These studies support the notion that maternity leave policies can influence women's career choices, including entrepreneurship. To examine the potential link between state maternity leave policies and startup success, maternity benefit score developed by the National Partnership for Women and Families (NPWF) is used \footnote[4]{Information about them is accessible at: https://www.nationalpartnership.org/about-us/}. This non-governmental organization (NGO) advocates for policies that improve work-life balance for families, with a particular focus on working women. Every two years since 2012, the NPWF releases a report that grades each state's laws regarding paid leave, family leave, and maternity leave. Their goal is to encourage states to enact legislation that supports working mothers and their career aspirations. For the analysis, the average maternity benefit score for each state across the years 2012, 2014, 2016, and 2018 is calculated. This average score provides a more comprehensive picture of a state's maternity leave policies compared to a single year's data. Given my focus on the 2010-2020 time frame, this average score is representative of the maternity leave landscape during that period.

The data reveals a significant disparity in maternity leave policies across states.  States like Alabama, Idaho, Michigan, Mississippi, South Dakota, and Wyoming consistently receive the lowest grades (0 points) throughout the measured period. Conversely, California, Washington D.C., Connecticut, and New Jersey consistently rank at the top for their robust maternity leave policies. While I expect a state's maternity leave score to correlate with the percentage of women employed within that state and the proportion of women entrepreneurs, it is unlikely the maternity policy will directly influence the number of venture capital interests a startup gets or funding secured by the startups except through the presence of a woman on the team. The primary influence of maternity leave policies is therefore likely to be felt indirectly, through its impact on the presence of women on a startup team. In the first stage specification of the instrumental variable regression, the mean maternity leave score is considered as an instrument for the presence of women in the startup team. The first stage is estimated using equation 2 below: 
\begin{center}
$WomenLed_{i}=\alpha +\beta_1Maternity Score_i+\beta_2X_{i} +\beta_{3}\chi_{k} + \beta_4\omega_{j}  + \epsilon_{i} \quad $(2)$ $
\end{center}

where $WomenLed_{i}$ is a dummy variable that takes the value 1 if the startup has a woman on the team during the seed stage of fundraising;  $Maternity Score_i$ is the maternity leave score of the state associated with the startup, $X_{i}$ is the startup (firm) related characteristics, $\chi_{k}$ is the funding related characteristics, $\omega_{j}$ is the state related characteristics and $\epsilon_{i}$ is the error term. 
        
\hspace *{0mm} The results of the first-stage regression is presented in table 6 below. The table shows that the maternity leave score of the startup impacts the probability of the startup being Women$-$Led (P-value $<$ 0.001). A standard deviation increase in the mean maternity leave score increases the probability of the startup having a woman by 19.53\% . This effect is robust to firm and industry controls (P-value$<$0.001) and increases to 20.93\%. Furthermore, the inclusion of industry and firm controls as well as funding controls also strengthens the magnitude and the statistical significance of the relationship to 21.6\% increase per one standard deviation increase in the maternity leave score of the startup (P-value$<$0.001). It is also worth noting that the likelihood ratio test is used as a test statistic for assessing the significance of the model. The maternity leave score of the state of the startup is a significant predictor of the presence of a woman on the startup team at the 10\% significance level. 

\begin{table}[H]
 \captionsetup{justification=raggedright,singlelinecheck=false}
    \caption{First Stage Regression}
        % First Stage for Both
 \begin{table}[htbp]
 \centering
    \begin{tabular}{c c c c c}
    \toprule
    \textbf{Variables} & \textbf{(1)} & \textbf{(2)} & \textbf{(3)}                 \\ 
    \textbf & \textbf{WomenLed  } & \textbf{WomenLed } & \textbf{WomenLed }   \\ 

    \midrule
    Maternity Score      &    0.0042*** &      0.0045***  &      0.0047***   \\
                         &    (0.0008) &     (0.0008)   &   (0.0008)  \\
                         &             &                &                \\

                        &             &                &          \\
                        &             &                &           \\
    Firm Age            &             &   -0.037***   &     0.012        \\
                        &             &      (0.0095)  &   (0.015)       \\
                        &             &                &                \\
                        &             &                &               \\
Constant                &  -1.385***  &   -13.25       &    -13.310              \\
                        &   (0.053)   &  (-143.800)   &     (143.300)           \\
                        &             &                &              \\

Industry and Firm Controls   &   No        &   Yes       &       Yes    \\
    Funding Controls    &   No             &   No        &      Yes        \\

    \midrule
     N                  &   13044          &      13044  &      13044      \\          
    \bottomrule
    \addlinespace[1ex]
    \multicolumn{3}{l}{\textsuperscript{***}$P<0.01$, 
      \textsuperscript{**}$P<0.05$, 
      \textsuperscript{*}$P<0.1$}
    \end{tabular}
\end{table}

\end{table}

Following my confirmation of the validity of the IV variable, I run the reduced form regressions of the instrumental variables using equation 3 below: 

\begin{center}
$Y_{i}=\alpha +\beta_1Maternity Score_i+\beta_2X_{i} + \beta_{3}\chi_{k} + \beta_4\omega_{j}  + \epsilon_{i} \quad $(3)$ $
\end{center}

\noindent where $Y_{i}$ is either the number of venture firms that made investments or the amount of money invested in startup i during the seed stage of fundraising;  $Maternity Score_i$ is the maternity leave score of the state associated with the startup, $X_{i}$ is the startup (firm) related characteristics, $\chi_{k}$ is the funding related characteristics, $\omega_{j} $ is the state related characteristics and $\epsilon_{i}$ is the error term. 

Table 7 below presents the effect of maternity leave score associated with the startup on venture firm interests attracted. Similar to the other results discussed above, column 1 presents the base model, column 2 presents the results of the base model with additional controls for industry and firm characteristics of the startup. The results of the base model shows that, a one standard deviation increase in the maternity leave score of startup is associated with 0.51 increase in venture firm interests (P-value$<$0.001). The addition of firm and industry controls in column 2 of table 7 diminishes the size of the effect to 0.23 increase in the number of venture firm interests attracted but the effect is still statistically significant (P-value$<$0.001). The inclusion of additional controls in column 3 slightly affects the magnitude of the effect but not the significance. A one standard deviation increase in the maternity leave score associated with the startup increases the number of venture firms attracted to 0.33 (P-value$<$0.001). It is noteworthy that the direction of the effect of the maternity leave score on the number of venture firm interests attracted is consistent. Additionally, the effect of the age of a firm on the number of venture interests attracted is statistically significant (P-value$<$0.001) but inconsistent. 

\begin{table}[H]
 \captionsetup{justification=raggedright,singlelinecheck=false}
    \caption{Reduced Form Regression}
        % Reduced Form VC Interests
 \begin{table}[htbp]
    \begin{tabular}{c c c c}
    \toprule
    \textbf{Variables} & \textbf{(1)} & \textbf{(2)} & \textbf{(3)}               \\ 
    \textbf & \textbf{VC Interests} & \textbf{VC Interests} & \textbf{VC Interests}  \\ 

    \midrule
    Maternity Score     &    0.011*** &      0.005***    &    0.0065***   \\
                        &    (0.0014)  &      (0.0013)   &   (0.0012)  \\
                        &             &                  &               \\

                        &             &                  &               \\
                        &             &                  &          \\
    Firm Age            &             &      -0.048***   &    0.080***        \\
                        &             &      (0.015)     &   (0.023)       \\
                        &             &                  &               \\
                        &             &                  &            \\

Industry and Firm Controls   &   No        &   Yes       &        Yes    \\
    Funding Controls    &   No             &   No        &       Yes        \\

    \midrule
     N                  &   13044          &      13044  &      13044      \\          
    \bottomrule
    \addlinespace[1ex]
    \multicolumn{3}{l}{\textsuperscript{***}$P<0.01$, 
      \textsuperscript{**}$P<0.05$, 
      \textsuperscript{*}$P<0.1$}
    \end{tabular}
    \newline
    Note: The table reports marginal effects of Tobit Model.
\end{table}

\end{table}

Next, the results of the reduced form regression of the effect of a startup being Women−led on the mean amount raised by the startup are presented in table 8 below, replacing the independent variable with the maternity leave score associated with the startup. Again, I present the tobit model results because firms enter my dataset once they receive an investment. In the base model, a standard deviation increase in the maternity leave score is associated with an increase in the mean amount raised by 49.19\% (P-value$<$0.001). This effect diminishes as additional controls are added. In column 2, the effect size diminishes to 24.44\% increase in amount raised per a standard deviation increase in the maternity leave score of the state of the startup (P-value$<$0.001). The effect of a one standard deviation increase maternity leave score accounts for a 27.37\% increase in mean amount raised by a startup once controls for industry, firm and funding are accounted for (P-value$<$0.001). On the contrary, older firms raise significantly less at the time of their seed stage (P-value$<$0.001). The effect represented 13.58\% (column 2) and 4.50\% (column 3) less for an additional year spent in the seed stage. 

\begin{table}[H]
\captionsetup{justification=raggedright,singlelinecheck=false}
 \caption{Reduced Form Regression}
        % Reduced form log amount
 \begin{table}[htbp]
    \begin{tabular}{c c c c }
    \toprule
    \textbf{Variables} & \textbf{(1)} & \textbf{(2)} & \textbf{(3)}             \\ 
    \textbf & \textbf{Log Amount} & \textbf{Log Amount} & \textbf{Log Amount}   \\ 

    \midrule
    Maternity Score     &   0.0086*** &      0.0047*** &     0.0052***   \\
                        &  (0.0006)  &      (0.0003)   &       (0.0005)  \\
                        &             &                &                \\

                        &             &                &               \\
                        &             &                &                \\
    Firm Age            &             &      -0.146*** &      -0.046***    \\
                        &             &      (0.006)   &      (0.010)       \\
                        &             &                &                    \\
   Industry and Firm Controls   &   No        &   Yes       &        Yes    \\
    Funding Controls    &   No             &   No        &     Yes        \\

    \midrule
     N                  &   7940          &      7940  &      7940      \\          
    \bottomrule
    \addlinespace[1ex]
    \multicolumn{3}{l}{\textsuperscript{***}$P<0.01$, 
      \textsuperscript{**}$P<0.05$, 
      \textsuperscript{*}$P<0.1$}
    \end{tabular}
    \newline
    Note: The table reports marginal effects of Tobit Model.
\end{table}

\end{table}

Finally, the estimates of the structural model are presented. The structural model of the instrumental variable specification utilizes the estimates of the dependent variable from the first stage specified in equation 2 above. Classification of the firms is based on a threshold of the predicted value being above the third quartile. Using the prediction, the structural model of the effect of a startup being Women−Led on its outcome is estimated using equation 4 below:

\begin{center}
$Y_{i}=\alpha +\beta_1\widehat{Women-Led}_i+\beta_2X_{i} + \beta_{3}\chi_{k} + \beta_4\omega_{j}  + \epsilon_{i} \quad (4) $
\end{center}

\noindent where $Y_{i}$ is either the number of venture firms that made investments or the amount of money invested in startup i during the seed stage of fundraising;  $\widehat{Women-Led_i}$ is the estimated classification of the startup as Women$-$Led, $X_{i}$ is the startup (firm) related characteristics, $\chi_{k}$ is the funding related characteristics, $\omega_{j}$ is the state related characteristics and $\epsilon_{i}$ is the error term. 

First, the structural specification of the model is used to examine the impact of a startup being Women$-$Led on the number of venture firm interests attracted. The results of the estimation are presented in table 9 below. On average, Women$-$Led startups attract 0.683 more venture firm interests as compared to the homogeneous men teams (P-value $<$ 0.001). The addition of controls for industry and firm characteristics of the startup does not affect the direction and the significance of the effect. However, the magnitude of the effect diminishes to 0.186 increase in the number of venture firm interests attracted on average for Women-Led startups (P-value $<$ 0.1). In column 3 of table 9, the effect is robust to the inclusion of all the controls. Women$-$Led startups on average attract 0.193 more venture firm interests. This result collaborates the finding from the tobit model estimate presented above in table 5. The consistency of this finding points to a situation where firms recognise the dearth of representation and support women-led teams. Additionally, the effect of firm age on funds raised is inconsistent. In column 2, older firms appear to attract less interest (P-value $<$ 0.001) but the effect is the opposite in column 3 (P-value $<$ 0.001).

\begin{table}[H]
 \captionsetup{justification=raggedright,singlelinecheck=false}
    \caption{Structural Form Regression}
        % Structural VC Interests
 \begin{table}[htbp]
    \begin{tabular}{c c c c}
    \toprule
    \textbf{Variables} & \textbf{(1)} & \textbf{(2)} & \textbf{(3)}               \\ 
    \textbf & \textbf{VC Interests} & \textbf{VC Interests} & \textbf{VC Interests}   \\ 

    \midrule
    \widehat{Women$-$Led} &   0.683***  &    0.186*    &      0.193*   \\
                         &    (0.098)   &   (0.112)    &      (0.108)      \\
                         &             &               &                  \\

                        &             &                &                   \\
                        &             &                &                    \\
    Firm Age            &             &      -0.046*** &   0.068***        \\
                        &             &      (0.015)   &  (0.023)       \\
                        &             &                &               \\
                        &             &                &         \\

Industry and Firm Controls   &   No        &   Yes       &       Yes    \\
    Funding Controls    &   No             &   No        &       Yes        \\

    \midrule
     N                  &   13044          &      13044  &       13044      \\          
    \bottomrule
    \addlinespace[1ex]
    \multicolumn{3}{l}{\textsuperscript{***}$p<0.01$, 
      \textsuperscript{**}$p<0.05$, 
      \textsuperscript{*}$p<0.1$}
    \end{tabular}
    Note: The table reports marginal effects of Tobit Model.

\end{table}

\end{table}
% \vfill 

Similarly, the structural specification of the model is used to examine the impact of the startup being Women$-$Led on the amount raised by the startup. The results are presented in table 10 below. In the base model, the results show that in fact Women$-$Led startups out raise their all$-$men counterparts. The coefficient of 0.51 represents 66.53\% more money raised by Women$-$Led startups as compared to the all$-$men startups. The magnitude of the effect decreases to 14.34\% more once controls for the industry and firm of the startup are added (P-value $<$ 0.001). The addition of further controls in column 3 of table 10 impacts the magnitude of the coefficient but not the statistical significance ( P-value $<$ 0.001). After accounting for the other characteristics of a startup, Women-Led startups raise on average 19.69\% more on average than all-men homogeneous startup teams. On the contrary, the effect of the age of the firm on the amount realized in fundraising is negative. If only the industry and firm controls are added, the effect of age is 13.50\% less for an additional year the firm spends in this stage (P-value $<$ 0.001). The direction of the coefficient is robust to the inclusion of all controls. However, the effect of a firm spending an additional year in the seed stage is slightly lower as older firms raised 5.35\% less per additional year spent in the seed stage (P-value $<$ 0.001). 

\begin{table}[H]
 \captionsetup{justification=raggedright,singlelinecheck=false}
    \caption{Structural Form Regression}
        % Structural Log Amount
 \begin{table}[htbp]
    \begin{tabular}{c c c c }
    \toprule
    \textbf{Variables} & \textbf{(1)} & \textbf{(2)} & \textbf{(3)}                \\ 
    \textbf & \textbf{Log Amount} & \textbf{Log Amount} & \textbf{Log Amount}   \\ 

    \midrule
    \widehat{Women$-$Led} &   0.510***  &      0.134***   &    0.179***   \\
                        &  (0.046)    &      (0.047)    &     (0.047)  \\
                        &             &                 &              \\

                        &             &                &              \\
                        &             &                &          \\
    Firm Age            &             &      -0.145***  &    -0.055***    \\
                        &             &      (0.006)   &     (0.010)       \\
                        &             &                &                   \\
   

Industry and Firm Controls   &   No        &   Yes       &      Yes    \\
    Funding Controls    &   No             &   No        &      Yes        \\

    \midrule
     N                  &   7940          &      7940  &     7940      \\          
    \bottomrule
    \addlinespace[1ex]
    \multicolumn{3}{l}{\textsuperscript{***}$p<0.01$, 
      \textsuperscript{**}$p<0.05$, 
      \textsuperscript{*}$p<0.1$}
    \end{tabular}
    Note: The table reports marginal effects of Tobit Model.
\end{table}

\end{table}


\section{Discussion}

This study delves into the intricate relationship between gender composition within startup teams and their fundraising success during the crucial seed funding stage. The core objective was to illuminate the influence of the presence of women in these teams on attracting venture capital interest and the mean amount of funding ultimately secured. The findings offer valuable insights into the dynamics of gender diversity within entrepreneurial ventures and its ramifications for accessing critical financial resources.

One of the most significant revelations is that startups with women founders attract venture capital interest at a demonstrably higher rate compared to their all-men counterparts. This suggests a potential shift in venture capital firms' preferences, possibly reflecting a growing recognition of the value proposition that gender diversity brings to innovation and decision-making processes. This aligns with existing research highlighting the positive impact of gender diversity on team performance and the quality of decisions made.

However, a concerning disparity emerges – despite attracting greater investor interest, the tobit regression model reveals that Women-Led startups tend to raise less capital on average compared to all-men teams. This discrepancy exposes a significant gap in funding outcomes. While women-led ventures may pique investor curiosity and initial interest, they still face formidable challenges in securing the necessary funding to thrive. This disparity may be attributed to a confluence of factors, including unconscious bias influencing investment decisions, limitations in access to networks and crucial resources, and systemic barriers that women entrepreneurs often encounter.

To strengthen the findings and address potential endogeneity concerns, the study utilizes instrumental variable regression. By employing state-level maternity benefits as an instrument for the presence of women in startups, the results partly corroborate the initial findings. This analysis confirms that gender diversity within startup teams positively influences the level of venture capital interest garnered. However, Women-Led startups raised more money than their all-men counterparts in the IV regression specification.

In conclusion, these findings underscore the criticality of promoting gender diversity and fostering an inclusive environment within the entrepreneurial ecosystem. Addressing persistent disparities in funding outcomes based on gender necessitates a collaborative effort from policymakers, investors, and all stakeholders within the industry. Implementing targeted diversity initiatives, providing dedicated mentorship and support programs for women entrepreneurs, and fostering inclusive investment practices are some potential strategies to level the playing field and create a more equitable environment for startup funding.

This study significantly advances our understanding of how gender composition affects a startup's fundraising success during the seed stage. While there appears to be progress in terms of venture firms expressing interest in Women-Led startups. Highlighting the value of gender diversity and implementing inclusive practices are crucial steps that stakeholders can take to create opportunities for women entrepreneurs to flourish and contribute meaningfully to both innovation and economic growth. Further research and policy efforts are essential to promote gender equality and empower inclusive entrepreneurship.



\section{Conclusion} \label{sec:conclusion}

In conclusion, this paper delves into the critical issue of gender diversity within the labor force, particularly within the entrepreneurial landscape. The findings underscore the complex interplay between team gender composition, investor interests, and fundraising outcomes during the seed stage of investment. While previous research has highlighted the persistent gender gap in entrepreneurship, this study offers nuanced insights into the mechanisms underlying these disparities.

The analysis reveals compelling evidence that the presence of women founders impacts venture firm interest, albeit with inconsistent implications for fundraising success. The results from the tobit regression model shows Women$-$led startups tend to raise lower amounts of funds compared to their all-men counterparts. However, the instrumental variable regression model suggests increasing startup gender diversity can enhance fundraising amounts for Women$-$led startups.


For policymakers and stakeholders, these findings underscore the need to foster an environment conducive to gender-inclusive entrepreneurship. Encouraging Women$-$led startups to innovate in industries with high funding potential could offer a strategic approach to mitigating disparities in fundraising outcomes. By addressing structural barriers and promoting diversity in entrepreneurship, we can create a more equitable landscape that fosters innovation and economic growth.



\bibliography{mybiblio.bib}

\section{Appendix} \label{sec:Appendix}


\subsection{Gender Discrimination}
To test whether the results is driven by gender stereotype-based discrimination, I define a dummy variable for a founding team being in a women dominant industry within the data. A startup is said to be in a women$-$dominant if the majority of the startups in the industry are women$-$led. The logic of this is that if there is no evidence of discrimination, there should not be a difference between the outcome of being a Women$-$led startup in a women dominated industry or a male dominant industry. I then include this dummy variable as well as the interaction of being in a congruent industry and the team being Women$-$led. The results of this specification are presented in tables 11 and 12 below. The results in table 11 show the effect on venture firm interests attracted. The results show that there is no difference in venture funds attracted by startups that are in a congruent industry.  

\begin{table}[H]
 \captionsetup{justification=raggedright,singlelinecheck=false}
    \caption{Gender Discrimination and Venture Interests Attracted}
         \begin{table}[htbp]
    \begin{tabular}{c c c c c}
    \toprule
    \textbf{Variables} & \textbf{(1)} & \textbf{(2)} & \textbf{(3)}         \\ 
    \textbf & \textbf{VC Interests} & \textbf{VC Interests} & \textbf{VC Interests}   \\ 

    \midrule
    Women$-$Led         &   0.259*** &     0.271***    &        0.180*   \\
                        &   (0.105)  &      (0.010)    &        (0.096)  \\
                        &             &                &                  \\
    Women$-$Dominant    &    0.073    &      0.199     &        0.297  \\
                        &    (0.098)  &      (0.133)   &        (0.128)  \\
                        &             &                &                    \\
Women Dominant*Led      &    0.085     &    0.1475     &       0.245   \\
                        &   (0.179)   &     (0.169)    &      (0.163)  \\                    
                        &             &                &             \\
                        &             &                &                    \\
    Firm Age            &             &      -0.048*** &    0.068***        \\
                        &             &      (0.015)   &    (0.023)       \\
                        &             &                &                 \\
                        &             &                &                \\

Industry and Firm Controls   &   No        &   Yes       &          Yes    \\
    Funding Controls    &   No             &   No        &        Yes        \\

    \midrule
     N                  &   13044          &      13044  &     13044      \\          
    \bottomrule
  
    \addlinespace[1ex]
    \multicolumn{3}{l}{\textsuperscript{***}$p<0.01$, 
      \textsuperscript{**}$p<0.05$, 
      \textsuperscript{*}$p<0.1$}

\end{tabular}
Note: The table reports marginal effects of Tobit Model.

\end{table}

\end{table}


In table 12, the paper analyzes the possibility of gender stereotype-based discrimination in funds raised. The effect of being a women$-$led startup in an industry dominated by women is actually negative and the effect is not statistically significant (P-value > 0.1). This is suggestive evidence against a case of discrimination against Women$-$led startups. 

\begin{table}[H]
 \captionsetup{justification=raggedright,singlelinecheck=false}
    \caption{Gender Discrimination and Amount Raised}
         \begin{table}[htbp]
    \begin{tabular}{c c c c c}
    \toprule
    \textbf{Variables} & \textbf{(1)} & \textbf{(2)} & \textbf{(3)}               \\ 
    \textbf & \textbf{Log Amount} & \textbf{Log Amount} & \textbf{Log Amount}  \\ 

    \midrule
    Women$-$Led         &   -0.192*** &      -0.178***  &        -0.188***   \\
                        &    (0.052)  &      (0.042)   &         (0.042)  \\
                        &             &                &                  \\
    Women$-$Dominant    &    -0.126*** &      -0.017    &        -0.021  \\
                        &    (0.047)  &      (0.056)   &         (0.056)  \\
                        &             &                &                    \\
    Women Dominant*Led      &    0.080     &      0.099  &        0.097   \\
                        &   (0.089)   &      (0.071)   &      (0.071)  \\                    
                        &             &                &             \\
                        &             &                &                    \\
    Firm Age            &             &      -0.149*** &    -0.054***        \\
                        &             &      (0.006)   &    (0.010)       \\
                        &             &                &                 \\
                        &             &                &                \\
Industry and Firm Controls   &   No        &   Yes       &        Yes    \\
    Funding Controls    &   No             &   No        &         Yes        \\

    \midrule
     N                  &   7940          &      7940  &     7940        \\          
    \bottomrule
    \addlinespace[1ex]
    \multicolumn{3}{l}{\textsuperscript{***}$p<0.01$, 
      \textsuperscript{**}$p<0.05$, 
      \textsuperscript{*}$p<0.1$}
    \end{tabular}
    Note: The table reports marginal effects of Tobit Model.
\end{table}

\end{table}

% \vfill

\subsection{Sorting}
This section examines the possibility that our previous results are influenced by sorting bias. Figure 1 suggests that industries like utilities, healthcare, education, government, and industrials have a higher proportion of women-led startups. Conversely, finance, technology, real estate, and energy have lower proportions. Figure 3 explores the average amount raised by industry. We see that finance, technology, and real estate industries have the highest average funding, while healthcare, utilities, and education/government sectors have the lowest. Figure 4 provides a more granular breakdown. In industries with a higher concentration of women-led startups (utilities, education/government, industrials, and healthcare), they tend to under-raise compared to all-male teams. Interestingly, these are also the industries with the lowest average funding overall. However, a different picture emerges in high-funding industries like finance and technology. Here, while women-led startups represent a smaller proportion, they are able to raise capital on par with, or even outperform, their all-male counterparts.  

% \begin{figure}
% \caption{Time Series of Startups by Industries }
% \centering
% \includegraphics[scale=0.4]{figures_3/Figure1.png} 
% \end{figure}



% \begin{figure}
% \caption{Time Series of Startups by Industries Grouped According to Founding Team Gender Composition }
% \centering
% \includegraphics[scale=0.4]{figures_3/Participation.png} 
% \end{figure}

\begin{figure}
 \captionsetup{justification=raggedright,singlelinecheck=false}
\caption{Amount Raised by Industries}
\includegraphics[scale=0.4]{figures_3/Industry.png} 
\end{figure}


\begin{figure}
 \captionsetup{justification=raggedright,singlelinecheck=false}
\caption{Amount Raised by Industries Grouped by Gender of Founding Team }
\includegraphics[scale=0.4]{figures_3/Industry_gender.png} 
\end{figure}


% \begin{itemize}
%         \item Adams, R. B., \& Ferreira, D. (2009). Women in the boardroom and their impact on governance and performance. Journal of financial economics, 94(2), 291-309.
%     \item Alesina, A. F., Lotti, F., \& Mistrulli, P. E. (2013). Do women pay more for credit? Evidence from Italy. Journal of the European Economic Association, 11(suppl1), 45-66.
%     \item Ali, M., Kulik, C. T., \& Metz, I. (2011). The gender diversity–performance relationship in services and manufacturing organizations. The International Journal of Human Resource Management, 22(07), 1464-1485.
%     \item Alsan, M., Garrick, O., \& Graziani, G. (2019). Does diversity matter for health? Experimental evidence from Oakland. American Economic Review, 109(12), 4071-4111.
%     \item Beckman, C. M., Burton, M. D., \& O'Reilly, C. (2007). Early teams: The impact of team demography on VC financing and going public. Journal of business venturing, 22(2), 147-173.
%     \item Bernstein, S., Korteweg, A., \& Laws, K. (2017). Attracting early‐stage investors: Evidence from a randomized field experiment. The Journal of Finance, 72(2), 509-538.
%     \item Bertrand, Marianne, and Esther Duflo. 2016. “Field Experiments on Discrimination.” National Bureau of Economic Research (NBER) Working Paper 22014.
%     \item Brooks, A. W., Huang, L., Kearney, S. W., \& Murray, F. E. (2014). Investors prefer entrepreneurial ventures pitched by attractive men. Proceedings of the National Academy of Sciences, 111(12), 4427-4431.
% \item Calder-Wang, S., Gompers, P., \& Sweeney, P. (2021). Venture Capital’s “Me Too” Moment (No. w28679). National Bureau of Economic Research.
%     \item Dessein, W., \& Santos, T. (2006). Adaptive organizations. Journal of Political Economy, 114(5), 956-995.
%     \item ell, S. T., Villado, A. J., Lukasik, M. A., Belau, L., \& Briggs, A. L. (2011). Getting specific about demographic diversity variable and team performance relationships: A meta-analysis. Journal of management, 37(3), 709-743.
%     \item Ewens, M., \& Townsend, R. R. (2020). Are early stage investors biased against women?. Journal of Financial Economics, 135(3), 653-677.
%     \item Gompers, P. A., \& Wang, S. Q. (2017). And the children shall lead: Gender diversity and performance in venture capital (No. w23454). National Bureau of Economic Research.
%     \item Gompers, P. A., \& Wang, S. Q. (2017). Diversity in innovation (No. w23082). National Bureau of Economic Research.
%     \item Gompers, P., V. Mukharlyamov, and Y. Xuan, “The Cost of Friendship,” Journal of Financial Economics, 119 (2016), 626–644
%     \item Gompers, P., V. Mukharlyamov, E. Weisburst, and Y. Xuan, “Gender Effects in Venture Capital,” forthcoming in Journal of Financial and Quantitative Analysis (2020).
%     \item Gornall, W., \& Strebulaev, I. A. (2020). Gender, race, and entrepreneurship: A randomized field experiment on venture capitalists and angels. Available at SSRN 3301982.
%     \item  Guzman, J., \& Kacperczyk, A. O. (2019). Gender gap in entrepreneurship. Research Policy, 48(7), 1666-1680.
%     \item Hebert, C. (2020, March). Gender stereotypes and entrepreneur financing. In 10th Miami Behavioral Finance Conference.
%     \item Hellmann, T., Mostipan, I., \& Vulkan, N. (2019). Be careful what you ask for: Fundraising strategies in equity crowdfunding (No. w26275). National Bureau of Economic Research.
%     \item Hu, A., \& Ma, S. (2020). Human interactions and financial investment: A video-based approach. Available at SSRN.
%     \item Joshi, A., \& Roh, H. (2007). Context matters: a multilevel framework for work team diversity research. In J. Martocchio (Ed.), Research in Personnel and Human Resource Management, Vol. 26. (pp. 148). Greenwich, CT: JAI Press.
%     \item Kim, D., \& Starks, L. T. (2016). Gender diversity on corporate boards: Do women contribute unique skills?. American Economic Review, 106(5), 267$-$71.
%     \item Koning, R., Samila, S., \& Ferguson, J. P. (2019). Female inventors and inventions. Available at SSRN 3401889.
%     \item Lyons, E. (2017). Team production in international labor markets: Experimental evidence from the field. American Economic Journal: Applied Economics, 9(3), 70-104. 
%     \item Mannix, E., \& Neale, M. A. (2005). What differences make a difference? Psychological Science in the Public Interest, 6, 3155.
%     \item Mathieu, J. E., Tannenbaum, S. I., Donsbach, J. S., \& Alliger, G. M. (2014). A review and integration of team composition models: Moving toward a dynamic and temporal framework. Journal of Management, 40(1), 130-160.
%     \item Rasul, I., \& Rogger, D. (2018). Management of bureaucrats and public service delivery: Evidence from the nigerian civil service. The Economic Journal, 128(608), 413-446.
%     \item Robb, A. M., \& Robinson, D. T. (2014). The capital structure decisions of new firms. The Review of Financial Studies, 27(1), 153-179.
%     \item Ruigrok, W., Peck, S., \& Tacheva, S. (2007). Nationality and gender diversity on Swiss corporate boards. Corporate governance: an international review, 15(4), 546-557.
%     \item Shore, L. M., Chung-Herrera, B. G., Dean, M. A., Ehrhart, K. H., Jung, D. I., Randel, A. E., \& Singh, G. (2009). Diversity in organizations: Where are we now and where are we going?. Human resource management review, 19(2), 117-133.
%     \item The Refinitiv business classifications. Refinitiv Business Classification . (n.d.). Retrieved July 11, 2021, from $https://www.refinitiv.ru/content/dam/marketing/en_us/documents/methodology/trbc-business-classifcation-methodology.pdf$. 
%     \item Webber, S. S., \& Donahue, L. M. (2001). Impact of highly and less job-related diversity on work group cohesion and performance a meta-analysis. Journal of Management, 27, 141162.
%     \item Wegge, J., Roth, C., Neubach, B., Schmidt, K. H., \& Kanfer, R. (2008). Age and gender diversity as determinants of performance and health in a public organization: the role of task complexity and group size. Journal of Applied Psychology, 93(6), 1301.
%     \item	Harrison, D. A., \& Klein, K. J. (2007). What's the difference? Diversity constructs as separation, variety, or disparity in organizations. Academy of management review, 32(4), 1199-1228.
%     \item	Hoogendoorn, S., Oosterbeek, H., \& Van Praag, M. (2013). The impact of gender diversity on the performance of business teams: Evidence from a field experiment. Management Science, 59(7), 1514-1528. 


    
% \end{itemize}



% \end{document}