
% \bibliographystyle{aer}


% % BIBLIOGRAPHY %%%%%%%%%%%%%%
% % \usepackage[natbibapa]{apacite}  % to enable '\citet' and '\citep' macros
% % \bibliographystyle{apacite}
% % %%%%%%%%%%%%%%%%%%%%%%%%%%%%

% \title{
% {Racial and Gender Diversity, Newcomers and Team Performance in a Dynamic Setting}\\
% {\large University of Arkansas}\\
% % {\includegraphics{university.jpg}}
% }
% \author{ George Agyeah\thanks{{gagyeah@walton.uark.edu} University of Arkansas 
% \newline
% Financial support from the National Science Foundation (SES\#2215203),\href{https://bbrl.uark.edu/}{Behavioral Business Research Lab} and the \href{https://walton.uark.edu/diversity/} {Dr. Barbara A. Lofton Office of Diversity \& Inclusion} are acknowledged. The experiment was pre-registered with the AEA’s RCT registry under number \href{https://www.socialscienceregistry.org/trials/9048}{9048}.  } } 


% \begin{document}
% \setlength{\topmargin}{1in} % Set top margin
% \setlength{\oddsidemargin}{1in}  % Set left margin (odd pages)
% \setlength{\evensidemargin}{1in}  % Set left margin (even pages)
% \setlength{\textwidth}{6.5in}    % Adjust text width (considering margins)
% \setlength{\textheight}{9in}     % Adjust text height (considering margins)

% \title{
% {Exploring Diversity, Discrimination, and Performance Dynamics} }\

% \maketitle 
% \begin{center}
%     \href{https://wordpress.com/block-editor/page/ag-yeah.com/1254}{ \textcolor{black}{Click here for the latest version}} 
% \end{center}
% \begin{abstract}
% \small \noindent Teams play a crucial role in shaping decision-making processes within organizations and institutions by capitalizing on a diverse range of perspectives and skills. However, teams frequently encounter challenges, such as coordination issues and suboptimal cooperation, which may be amplified by the varied backgrounds and identities of team members. This study employs a lab experiment to explore the impact of both racial and gender diversity on individual economic decision-making within teams operating in a dynamic context. Across 111 independent sessions, 444 participants make cooperation and coordination decisions in the public goods provision and minimum effort activities. Video and audio recordings of participant interactions are then coded to capture participant communication. Incumbents in a team cooperate and coordinate at levels that are about 14\% higher than newcomers. Newcomers significantly increase their cooperation and coordination choices after joining a team. Further evidence shows that the existing incumbent team diversity affects the cooperation choices of newcomers, but not incumbents. The findings underscore the significance of bolstering team identity in team design and highlight the crucial consideration of the substantial influence that the specific team a newcomer joins has on their decision-making in organizations. \\

% \noindent\tiny\textbf{Keywords: Diversity, gender, race, public goods, experiment, teams }\\
% \noindent\textbf{JEL Codes: M14, J15, J16, H41, C91} \\
% \end{abstract}
% \setcounter{page}{0}
% \thispagestyle{empty}
% \pagebreak \newpage

\pagestyle{plain} 

\doublespacing

\section{Introduction} \label{sec:introduction}
 

Teams have become integral to the functioning of diverse environments ranging from corporate board rooms to academic research teams \citep{bjl14,lml}. Numerous reasons contribute to the significance of teams. Firstly, teams bring together a diverse range of perspectives and skills, enabling enhanced decision-making by leveraging the collective knowledge and experiences of each member. Secondly, teams facilitate workload sharing, particularly for complex or time-consuming decisions. This collaborative effort ensures that tasks are effectively handled. Thirdly, teams can foster a supportive and encouraging environment, promoting stress reduction and bolstering team morale. However, teams can also encounter challenges, such as coordination issues and potential delays in decision-making processes. Optimal outcomes require team members to cooperate, a process that can sometimes slow down decision-making in various settings. This effect can be further exacerbated by social category diversity if differences are magnified \citep{lau}. Social category diversity encompasses differences in race, gender, age, or cultural background among team members. By acknowledging and addressing these challenges, organizations can harness the benefits of team-based decision-making while effectively managing the complexities arising from social category diversity among team members. 

 % \hspace *{0mm} Team diversity can shape team behavior and overall performance. According to social identity theory, individuals tend to more readily identify with others who share similar social categories like them \citep{t78}. This can pose challenges when individuals from different backgrounds are brought together in a team. On the contrary, research has demonstrated that teams composed of individuals with diverse backgrounds and experiences can yield remarkable improvements in various aspects of team performance \citep{labbzs14,pln08}. Consequently, recognizing and comprehending the impact of diversity within teams becomes imperative for achieving efficient outcomes. 
 
 The public goods provision and the minimum effort activities are utilized to investigate the impact of team diversity and changes in the team dynamics on the performance of teams. The public goods provision activity and the minimum effort activity are social dilemma games that are useful in measuring behavior where the interest of the individual conflicts with the needs of the group. The dilemma lies in the fact that the pareto optimal outcome for the group is achieved when members of the group give at the highest levels of both activities. The dominant strategy for an individual in the public goods provision is to contribute at the lowest level to the group account. Cooperating in a social dilemma is essential for the functioning of teams while failure to cooperate has been linked to inefficiencies in teams \citep{stallen2023}. Contributions to the group in the public goods provision is analyzed as the level of cooperation in the team. Choices of the level of effort in the minimum effort activity are analyzed as the coordination choices within the group.  There is no dominant strategy but higher coordination choices reflect higher efficiency within the group. 
 
 I conduct a series of experiments to explore the impact of the level of team diversity on economic decision-making in a dynamic context.  Participants of varying racial and gender identities are randomly assigned to teams of three people. A newcomer is added to the team in the middle of the experiment to examine the evolution of cooperation and coordination among the members. The results indicate that team diversity has a significant effect on individual behavior. Additionally, the existing composition impacts contributions of incumbents prior to group composition changes. Furthermore, newcomers are affected by the diversity of the team they join but not incumbents. 

\hspace *{0mm} This work makes several contributions to the literature. Firstly, it extends our understanding of teams by investigating the influence of newcomers' identities on team dynamics and the impact of team diversity on both incumbents and newcomers. This approach enables me to adopt a functional perspective that closely mirrors real-world organizational settings. Secondly, my findings provide compelling evidence of the dual effect of racial and gender diversity on team performance within diverse teams. My design effectively highlights the gender and racial identity of each individual allowing individuals to perceive an individuals racial and gender racial identities based on interactions within the team. This enables me to more accurately account for the dual effects of racial identity and gender identity. Lastly, while prior studies have found mixed results regarding the impact of diversity on performance, my study tackles endogeneity and selection concerns that commonly arise in observational studies, thus offering more robust evidence.

\hspace *{0mm} The rest of the paper is organized as follows. Section \hyperref[sec:literature]{2} presents the literature review. Section \hyperref[sec:Design]{3} details the experimental design. Section \hyperref[sec:Hypotheses]{4} introduces hypotheses. Section \hyperref[sec:Analysis]{5} discusses the empirical analyses and results in the study. Section \hyperref[sec:Conclusion]{6} concludes.
 

\section{Literature Review} \label{sec:literature}
 
While numerous forms of diversity play a crucial role in influencing economic behavior, the extensive literature on the impact of diversity on economic behavior is too vast to comprehensively cover in this context. Therefore, I narrow the focus to a specific subset of literature that delves into the impact of gender and racial identities within teams. I specifically delve into the literature addressing teams, focusing on the impact of team diversity on both cooperation in the public goods provision activity and coordination in the minimum effort activity.

\subsection{Teams}
Research exploring the relationship between diversity and team behavior highlights the advantages and challenges in getting diverse teams to function efficiently  \citep{po15}. On one hand, a diverse team can offer a wealth of perspectives and experiences, fostering a climate conducive to creative and innovative solutions \citep{rvhv13}. For instance, diverse teams may generate novel ideas that would not emerge within a homogeneous group as diverse individuals are more likely to have different backgrounds and experiences that can provide different perspectives for the team. This has been shown in the literature to affect decision-making. For instance, according to \cite{ks16}, a diverse team is better positioned to utilize the diversity of skills in the team to make better-informed decisions. On the other hand, a diverse team may encounter obstacles such as increased conflict and coordination difficulties. These challenges can arise due to variations in styles and thinking processes among individuals from diverse backgrounds, leading to potential misunderstandings.  

\hspace *{0mm} Organizations have the potential to reap the benefits of diverse teams by effectively managing these challenges \citep{j23}. Furthermore, the demographic trends evident in census data highlight the ongoing surge in diversity within organizational settings within the US labor force, particularly along the dimensions of race and gender. Between 2010 and 2020, the proportion of the US population that was white decreased by 8.6\%.  Conversely, the percentage of the population identifying as black increased slightly, the Hispanic or Latino population grew a notable 23\%, and the population of individuals identifying as belonging to other racial categories surged by 129\% \citep{b20}. These shifts in population distribution underscore the diversification occurring along racial lines within the US population. Coupled with changes in labor force participation across gender identities, this indicates a rapid transformation in the composition of the labor force. As such, organizations that can effectively harness the potential of diverse teams stand to gain a competitive advantage in the evolving workforce landscape. Even organizations that currently lack diversity will increasingly witness the introduction of individuals from diverse demographic backgrounds, as the nature of the labor force more closely mirrors the melting pot that is the United States. 

\hspace *{0mm}Irrespective of the current diversity of work teams within an organization, if managed effectively, diversity can bring together diverse skill sets that enhance performance in an organization despite the potential for interpersonal tension \citep{rvhv13}. Evidence from research suggests that the benefits of diversity are driven by diverse teams having a diversity in expertise. For example, a study by \cite{ks16} finds that diverse teams are more likely to have a diversity of expertise. On the contrary, issues of conflicts and poor coordination common in teams can be further exacerbated in diverse teams. The predictions of similarity-attraction theory suggest that people prefer similarity in their interactions \citep{b71,bw62}. Hence, demographically distinct newcomers may disrupt established processes and struggle to gain acceptance in the new team if they are not well integrated \citep{c05,rkev13}.

\hspace *{0mm} Recent studies have explored the interplay between demographic diversity and team behavior. Notable studies highlight the impact of group diversity on aspects of team behavior. For instance, \cite{bjd13} find that the presence of a man in a fund management team increases the probability of selecting a higher risk investment, even though all-male teams do not inherently exhibit the highest risk-seeking tendencies. Generally, one can reasonably expect that individuals with dissimilar demographic backgrounds often have different cultural perspectives, have different educational experiences, and approach problems in a different way. Consequently, increasing the level of demographic diversity in a team could impact the level of skill diversity. For example, a Walmart store that serves a diverse community will benefit from having a diverse staff. Specifically, a diverse staff is more likely to include a staff member who distinctly understands the needs of a customer in search of an afro-pick (or any other item that is mostly used by a segment of the population). This dynamic applies in many settings including corporate America, where a culturally inappropriate product could be costly, both in terms of reputation and lost income.  Increasing the level of diversity in a team serves to increase the skill diversity available for decision-making \citep{jnn99}. However, reaping the benefits of skill diversity is contingent upon the ability of the team to coordinate and cooperate on the diverse ideas members bring forth from their respective backgrounds.

\hspace *{0mm} The positive performance of diverse teams can be impeded if team diversity negatively affects the behavior of individual team members. Substantial evidence indicates that individual behavior is impacted by the diversity of the team. For example, \cite{ch19} find that women in mixed-gender groups are twice as likely as women in single-gender groups to suffer from the gender stereotype effect, resulting in hesitancy to assume leadership roles or contribute ideas in gender-incongruent tasks. These findings are also corroborated by the work of \cite{brs22}. They find that women are more inclined to lead teams with a majority of females compared to teams with a majority of males. Diversity within teams not only influences the behavior of team members but also has effects on how individuals are treated within the team dynamics by others. The work of \cite{cfs21} sheds light on the impact of gender stereotypes on the selection of individuals to answer questions on behalf of the team, showing a preference for gender-congruent individuals in topics traditionally associated with specific genders. Their findings demonstrate that individuals who belong to a gender minority within a team are less inclined to engage in self-promotion. Similarly, \cite{skp21} provide compelling evidence that token women, standalone women in otherwise all male teams, tend to be less influential and receive less credit for their contributions. Furthermore, \cite{s17}has documented the role of gender in credit attribution within a team, highlighting its impact within academic research. It is worth noting that beyond the economic literature, \cite{bcn21} uncover that interns who share demographic similarities with senior managers tend to have more positive experiences and are more likely to receive job offers. These studies collectively emphasize the intricate dynamics of diversity in teams and underscore the need to address the biases and challenges that can arise from such diversity, ultimately fostering more equitable and inclusive environments. 

\hspace *{0mm} Exploring diversity within team environments presents a complex challenge. Teams can encompass diversity across various dimensions, such as age, gender, race, ethnicity, educational background, and work experience. Research investigating the impact of diversity reveal that the multifaceted nature of diversity can yield diverse behavioral outcomes across different contexts \citep{clls14}. Consequently, the impact of team diversity on performance can manifest in unique ways. The intricacy of studying diversity in team environments therefore necessitates grappling with the complex interplay between individual identity and its implications for team identification and diversity as well as the different dimensions of diversity. Research has demonstrated that individual identity significantly influences decision-making processes \citep{ ak08}. Recent evidence sheds light on the role of identity in shaping trust dynamics \citep{cde22}, anticipating instances of discrimination \citep{ack23}, and influencing income redistribution \citep{fghz23}. Moreover, team membership has also been shown to impact behavior across various contexts, encompassing phenomena such as shirking and free riding \citep{eg05}, preferences for outcomes \citep{crrbcdfggllmrwy07}, considerations of charitable acts and punishment \citep{chen2009group} and judicial claims \citep{sz11}. These studies underscore the significant influence of individual and group identities on behavior within distinct environments. Recognizing the diverse facets of identity becomes crucial when examining diversity within team settings, as different facets can yield distinct behavioral implications.

\hspace *{0mm} A growing body of literature delves into the implications of the different dimensions of diversity and team performance in economic decision-making. Much of this literature focuses on the effect of gender diversity on behavior. For instance, \cite{aai12} study the effect of the dynamics of gender diversity within teams. They utilize administrative data from endogenously formed teams with fixed compositions in the L'Oreal e-Strat Challenge. The findings reveal that all-women teams perform comparatively worse than all-men and heterogeneous gender teams, exhibiting less aggressive pricing strategies and placing a higher emphasis on social sustainability initiatives rather than research and development. While the study provides valuable insights into decision-making among homogeneous gender teams, it is limited in disentangling the specific mechanisms driving these decisions. The paper points to potential factors such as skill differences, sorting behaviors, and heterogeneous team dynamics. In another vein, \cite{am18} explore the influence of educational diversity within teams. In a randomized control trial, they investigate the significance of newcomers' educational backgrounds on team receptivity. Interestingly, the results demonstrate that "old-timers," incumbents in this study, are less accepting of newcomers with different educational backgrounds, particularly when they perceive the new arrivals as a threat to collective representation. Conversely, it is important to note that negative social categorization may still occur for newcomers in teams without the same concerns.

\subsection{Diversity and Cooperation }
\hspace *{0mm} Extensive research has demonstrated the complexities that arise from race and gender differences within teams and how they affect interactions among team members \citep{ha12}. According to Social Identity Theory \citep{t78,tbt79}, teams composed of individuals with diverse backgrounds and values may encounter challenges in integrating their unique perspectives and collaborating effectively. Furthermore, prior research widely agrees that people tend to feel more at ease working with others or groups they identify with. In contrast, the economics literature has found mixed results when exploring the impact of team member heterogeneity on cooperation in experimental settings utilizing public goods games. For instance, \cite{nt94} find that all-female groups demonstrate higher levels of cooperation compared to mixed-gender and all-male groups. In contrast, \cite{pbm19} find that mixed-gender groups are the most effective in cooperative collective action. For a comprehensive review of the literature, refer to \cite{blmv11} and \cite{cg09}. A growing body of studies seeks to reconcile these discrepancies, often explaining them through factors such as framing \citep{blmv11, ejmm13} and differences in conditional cooperativeness \citep{fkmmw21}.

\hspace *{0mm} While a significant body of work has investigated the impact of gender group diversity on contributions and cooperation in the provision of public goods, it is unclear how the race of individuals might impact cooperation within teams. Even less evident is how the combined racial and gender identities of an individual impact cooperation in the provision of public goods. It is important to note that an individual’s race is a significant component of their identity, and evidence from prior studies indicates that identities and common goals in teams can be influenced by racial identity \citep{burns2015}. The gender identity of an individual has also been shown to impact shirking behavior among team members. It is therefore particularly crucial for us to understand how diversity along racial and gender identities can affect efficiency of team production. The need to enhance cooperation can also be influenced by changes in group diversity, where existing members may leave, and new members may join. However, the impact of group dynamics, such as changes in the diversity of work teams, on the decisions of both existing and new team members, as well as the overall team performance, remains understudied. As aptly stated by \cite{ght18}, “[c]hanging group compositions over time, however, may alter a group in three different ways ceteris paribus: First, it divides up a group according to the entry of its individual members. Second, it implies the new group members’ adaption to the group. And third, there is also old group members’ adjustment to the new situation.” To address these dynamic aspects of group composition, my experimental design allows me to analyze and measure the impact of team diversity on team performance by distinguishing between these different mechanisms---gender, race and the addition of new members using the public goods provision activity. 

\subsection{Diversity and Coordination }

\hspace{0mm} Effective teamwork relies on coordination among group members to achieve efficient outcomes. A team can leverage its members' knowledge to facilitate such coordination, particularly when the group is homogeneous and shares certain demographic similarities. Hence, teams lacking such commonalities may encounter challenges in establishing connections that enable effective coordination \citep{pk12}. 

\hspace {0mm} While research on diversity and coordination predominantly focuses on gender, findings from \cite{g} indicate that smaller groups tend to coordinate more effectively than larger ones. Notably, they did not find significant gender differences in coordination. These results align with the findings of \cite{dg05}, who observed minor disparities in initial stages but not in the final stages of the repeated coordination activity. On a contrasting note, \cite{h00} finds that participants exhibited more aggressive behavior toward female co-players in a battle of sexes study . Surprisingly, this heterogeneous attitude to different genders actually facilitated coordination and increased earnings in mixed-gender groups compared to homogeneous pairs. Many of the arguments discussed above regarding the impact of group dynamic changes on cooperation can also affect the coordination among teams when group composition changes. In summary, newcomers could disturb existing channels due to their dissimilarity or their presence, changing the balance of the group. How incumbents and newcomers adjust to the group membership changes could impact choices within the group. 

 
\section{Experimental Design} \label{sec:Design}

To investigate the impact of group diversity on group and newcomer performances, I introduce a newcomer to a three-person pre-existing group, referred to as the incumbents.  111 sessions of 4 participants—444 undergraduate students overall—participated in the study conducted at the Behavioral Business Research Lab (BBRL, https://bbrl.uark.edu/) of the University of Arkansas between Fall 2022 and Spring 2023. A session in this study refers to the four-person participants that make decisions in the different parts of the study. Participants are recruited by gender and exogenously assigned gender teams in the laboratory when they arrive. In situations where possible, individuals are randomized into mixed-race teams or allwhite teams. Tables 1 and 2 below show the distribution of the treatments.Table 1 presents summary of sessions by gender composition and newcomer gender. Table 2 shows that there are 80 participants exogeneously assigned to incumbent all-men teams, 184 participants are assigned to incumbent all-women teams and 180 participants are assigned to mixed-gender teams. 


\begin{table}[H]
 \captionsetup{justification=raggedright,singlelinecheck=false}
\caption{Summary of Sessions by Gender Composition} \label{tab:table1}
\centering
\begin{table}[htbp]
\begin{tabular}{c c c c}
\toprule
                   &\multicolumn{3}{c}{Newcomer Gender}  \\
                   &      Male  &   Female &    Total    \\
\midrule
All-men            &      44 &         36 &       80     \\
\midrule
All-women          &      88 &         96 &      184   \\
\midrule
Mixed gender       &      88 &         92 &      180  \\
\midrule
Total              &      220 &       224 &      444  \\
\bottomrule
\end{tabular}
\end{table}

\end{table}

 

\begin{table}[H]
 \captionsetup{justification=raggedright,singlelinecheck=false}
\caption{Summary of Sessions by Racial Composition} \label{tab:table2}
\centering
\begin{table}[htbp]
\begin{tabular}{c c c c}
\toprule
                   &\multicolumn{3}{c}{Newcomer Race}  \\
                   &      Non-White  &   White &    Total    \\
\midrule
All-white            &      60 &         88 &       148     \\
\midrule
Mixed-race          &      184 &         112 &      296   \\
\midrule
Total              &      244 &       200 &      444  \\
\bottomrule
\end{tabular}
\end{table}
\end{table}

\noindent \textbf{ \textit{Part I}} 
 \newline
Each session involves four participants. When the participants arrive at the laboratory, they are randomly assigned IDs A to D conditional on their gender. A summary of the experimental design is presented in figure 1 below.  Subjects A, B, and C are sent to Lab 1, and D (referred to as "the standalone" hereafter) is sent to Lab 2. In Lab 1, participants are seated around a table with three chairs in the center of the room.  These team members are thus able to observe the gender and race of the other participants, a feature often lacking in other studies.  Participants are notified that the first part of the experiment consists of two stages. In the first stage, the group plays a triangle puzzle game designed to enhance group identity \citep{eg05}. Each group member is given an envelope containing four cut pieces of cardboard. The group is then told to make triangles similar to the sample on the table. The four pieces in each envelope are not enough to make a triangle, and group members are encouraged to communicate and trade pieces to be able to form the triangles. Additionally, participants are informed that interactions during the first stage (the puzzle stage) are being video and audio recorded. Group members are paid 10 Experimental Currency Units (ECUs) for each piece correctly placed by any group member. All teams correctly solved the puzzle. Earnings in ECUs from this part and other parts of the study are exchanged for dollars at the end of the study . The average payment per participant is 20 US dollars. It is public information that participants are informed about their payoff at the end of Stage 2 of Part II. Participants are given 10 minutes to work on the puzzle task. Simultaneously, the standalone in Lab 2 is instructed to wait for further instructions in 10 minutes. 

\begin{figure}[H]
\captionsetup{justification=raggedright,singlelinecheck=false}

\caption{Flowchart of Experimental Procedure}
\includegraphics[scale=0.6]{Figures/Design.png} 
\end{figure}
 
 
 \hspace  *{0mm} In Stage 2 of Part I, A, B, and C play economic decision games on their designated computers in Lab 1.  Each computer is located at a different corner of the lab separated by tall dividers for participants’ privacy as shown in figure 8 in the appendix. A, B, and C play the games on their own but as a group, and D---the standalone---plays the games with two computer robots in Lab 2. The participants play two games: a public goods provision game that involves a voluntary contribution mechanism (VCM) and a minimum effort game. The participants are given instructions at the beginning of each game and informed that they will play five rounds of each game. The marginal per capita return (MPCR) in Part I of the public provision game is 0.5.  Payoff of individual participants in the public goods provision game -voluntary contribution mechanism (VCM) are calculated as follows: 
 
 $ \pi_{i} = 100 $–$ c_{i} + \frac{M}{k}\sum_{j=1}^{k}(c_j )  $ \space \space \space  \space \space \space  \space  (1)
\newline
\noindent\textit{where $\pi_{i}$ represents individual \textit{i}'s payoff, $c_{i}$ represents individual $i’s$ contributions to the group account, and $c_{j}$ represents the individual contributions of all players to the group account. k is the number of participants in the group (i.e., three for the three-person group and four for the four-person group). Finally, M represents the multiplier, the constant by which contributions to the group are multiplied by.}

Payoffs of individual participants in the minimum effort are calculated as follows:

$\pi_{i}  =  85+min(H_{j} )- \frac{3}{4} h_{i}  $   \space \space \space  \space \space \space  \space \space  (2)   
\newline
\noindent\textit{where $h_i$ is the number of hours contributed by individual i toward the group activity, and min($H_j$) is the minimum hours contributed among all the individuals within group j. }

The participants are paid based on their cumulative payoffs of 5 rounds of one randomly chosen game out of the two games in accordance with incentive compatibility prescribed in \cite{ach18}. Additionally, participants are not provided feedback on choices of other participants until the end of Part I. 

\noindent \textbf{ \textit{Part II}}
\newline
Following the completion of Part I, participant D (the standalone) is brought into Lab 1 to join the three-person group of A, B, and C, and Part II of the experiment begins. The physical entrance of the standalone makes the individual's gender and race observable to the other participants, and vice versa. A, B, and C are invited back to their seats at the round table where they played the puzzle game in Part I, and an additional chair is added to the table for D. Participants are informed that in Part I, A, B, and C played two decision-making games as a group, and D played the same two decision-making games with two computer robots. 
  
The experimenter then reads instructions for Stage 1 of Part II, which is another puzzle game. The participants are then each given a new envelope containing six pieces. The participants are told that their individual task is to make a triangle similar to the sample shown. As in Part I, the six pieces in each envelope cannot form a triangle, and group members are encouraged to communicate and trade pieces to form triangles. Similar to Part I, participants are told that their interaction in the puzzle-solving stage is being video and audio recorded. Each group member is required to make their own triangle, and participants are paid 10 ECUs per correctly placed piece by each group member.  Once again, participants are given 10 minutes to complete the task. 65\% of teams correctly finished the puzzle in the allotted time. This activity is designed to enhance the group identity of the newly formed four-person group. The puzzle is purposefully more challenging than the one in Stage 1 of Part I. Since participant D (the standalone) did not participate in the similar puzzle in Part I, the pre-existing group members, A, B, and C, may help D complete the task.

 \hspace  *{0mm} After finishing the puzzle game, the four participants are again invited to their designated computers at the four corners of the room to play the two economic decision games—public goods provision game -voluntary contribution mechanism (VCM) and minimum effort —as in Stage 2 of Part I. The marginal per capita return (MPCR) in Part II is adjusted to 0.438 to avoid amplifying cooperation behavior. In addition, I elicit their risk preferences using a lottery mechanism prescribed by \cite{eg02}. The participants complete a post-experimental survey before seeing their Part II earnings and total earnings. Their Part II earnings include their payoffs in the puzzle game, the 5-round cumulative payoffs of a randomly chosen computerized decision game, and payoffs of the lottery. The total earnings for the experiment are the sum of earnings in Parts I and II and \$7 show-up fee.

  \hspace  *{0mm} The study occurs in two parts each consisting of 2 stages as summarized in flow chart diagram shown in figure 1 above. It is worth noting that in each part, participants engage in a group building activity in stage 1. The activity involves attempting a puzzle of varying difficulty. Communication and interactions during stage 1 are recorded from two angles. These recordings are coded by research assistants and included as chat controls in the regression analysis of decision making. In Part I, stage 1, the participants start with an incomplete set of pieces and need to communicate and trade pieces to get all required pieces as show in figure 4 in the appendix. Each individual participant is required to complete the puzzle by putting the pieces together to form triangle as shown figure 5 in the appendix. In Part II, stage 1, participants solve a different, more difficult puzzle requiring each participant to trade pieces to obtain all the required pieces as shown in figure 6. Similar to Part I, each participant in Part II is expected to complete the puzzle as shown in figure 6 in the appendix. Once participants finish the puzzle or the allotted 10 minutes for the puzzle is exhausted, participants proceed to stage 2 in both parts where decisions are made on the computer. Figure 8 in the appendix shows the environment of the lab. In the first and third images, pilot participants are shown working on the group building activity in stage I. The image in the middle of of figure 8 shows participants seated at their individual computers in the second stage. 

\section{Hypotheses} \label{sec:Hypotheses}
In many organizations, it is beneficial for employees to cooperate and coordinate. Many situations arise in which cooperation or coordination does not benefit the individual but is beneficial to the group or organization. In this study, I analyze group decision-making using a public goods activity and a minimum effort activity. The public goods game models a group production environment in which the provision of a good requires the contributions of a proportion of the group \citep{c10,stoop2012lab,kagel2020handbook}. The minimum effort activity is an economics decision-making activity modelled to represent a team environment in which the production of a good depends on the effort of the weakest link \citep{vbb91}.

 \hspace  *{0mm} In my study, individual decisions are made simultaneously under uncertainty. Willingness to cooperate is impacted by the behavioral interactions that the individual has with the group, their perceptions of the other people in the group, and their understanding of what is culturally expected of them. Findings in the economics literature shed light on how team composition affects cooperation and coordination tendencies. For example, \cite{nt94} find that all-women groups are more cooperative than all-men and mixed-gender groups. In contrast, \cite{pbm19} find that mixed-gender groups cooperate most effectively. Expanding on this extant literature, I propose my first hypothesis:
 
 \textit{Hypothesis 1: The diversity of a group affects the cooperation and coordination decisions of individual group members in a public goods provision and a minimum effort activity.}

\hspace  *{0mm} Evidence in the group identity literature shows that individuals who are part of a group are more cooperative. Members of teams build group identity and contribute more to the group compared to standalone individuals. Hence, the overall contributions to the group activity are likely to differ depending on the status of the individual as a newcomer or an incumbent. Additional evidence in the laboratory establishes that replacing established team members with newcomers yields a reduction in overall team performance driven partly by a breakdown in trust \citep{m2013}. Hence, newcomers are expected to exhibit different cooperation and coordination tendencies as compared to incumbents. This concept gives rise to my second hypothesis: 

  \textit{Hypothesis 2: Incumbents are expected to cooperate and coordinate better overall than newcomers.}

\hspace  *{0mm} Behavioral economics research focusing on team dynamics reveals that the behavior of established team members undergoes shifts based on the characteristics of newly joined individuals. An insightful study illustrates that existing members exhibit reduced openness towards newcomers who possess distinctive qualities, perceiving them as lacking in cooperation and competence \citep{am18}. Further evidence show partners and strangers in a team react differently when working together in a team \citep{ght18}. Expanding upon this literature, I postulate the following hypothesis: 

    \textit{Hypothesis 3: The cooperation and coordination of incumbents and newcomers change after newcomers join the team.}

\hspace  *{0mm} A wealth of research in the field of behavioral economics underscores the impact of group identity on the collective behavior of teams. These group dynamics can elevate performance across various contexts, even within diverse teams \citep{eg05}. This phenomenon is echoed in the findings of \citep{crrbcdfggllmrwy07}, who find that individuals who align their identity with a group exhibit distinct behavioral patterns compared to those who perceive themselves as isolated individuals within the same group. An extensive body of work in behavioral economics also underscores the influence of individual identities within a group on behavioral patterns \citep{har2009,chen11}. However, it is reasonable to expect a different effect in an environment where diverse individuals are joining the labor force. Diverse teams are likely to better assimilate newcomers than homogeneous teams as individuals are able to connect with members of the existing team. The congruence of an individual’s identity along racial and gender dimensions with the existing team could have an impact on newcomer behavior. Based on insights from these interconnected lines of research, I formulate my next hypothesis:

    \textit{Hypothesis 4: The diversity of the incumbent teams could impact cooperation and coordination choices of newcomers to the team.}

\hspace  *{0mm} Just as incumbents can influence the contributions of newcomers within a group, the identity of a newcomer can influence how incumbents contribute to the group. When an incumbent identifies more closely with a newcomer than other incumbents in the team, it can affect their subsequent contributions to the group. Conversely, the existing group identity among the incumbents may moderate the influence of an individual's identity on their contributions to the group. This argument forms the basis for my final hypothesis, which posits that the congruence between the identity of incumbents and the identity of newcomers influences the contributions of incumbents to the group.

    \textit{Hypothesis 5: The contributions of incumbents are influenced by the congruence of their identity and the identity of the newcomer.}


\section{Analysis} \label{sec:Analysis}
The analysis is organized into five main sections for clarity. The initial segment provides an overview of the study participants in the summary statistics section. Details of coding for the interactions in the group building activities, as described in the experimental design section, are presented in the second section. Section three provides information on the empirical specifications. Following this, the focus shifts to the results of cooperation within the public goods provision activity in the fourth section. Finally, the analysis delves into coordination choices within the minimum effort activity in the fifth section.

My analysis of cooperation in the public goods provision and coordination in the minimum effort activity proceed as follows. I first explore cooperation and coordination among incumbents in the first part of the study. I then broaden the scope of the analysis to look at the overall cooperation and coordination of both newcomers and incumbents. The evolution of actions in the second part of the study is then examined.

\subsection{Summary Statistics}
A total of 444 participants are recruited from the University of Arkansas, primarily sourced through the Walton College Behavioral Business Research Laboratory Sona System, along with targeted recruitment posters placed strategically around the Fayetteville campus. Among the participants, 178 individuals, accounting for 40\% of the sample, self-identified as men, while 261 participants identify as women. Additionally, four participants specify another gender identity, and one person opts not to disclose their gender identity.

Regarding racial identification, a notable majority, comprising 68\% of the total 444 participants, identify as white. Five percent identify as black/African American, 12\% as Hispanic, 10\% as Asian, 1\% as Middle Eastern, and 4\% as belonging to some other ethnicity. Within the participant pool, 184 individuals are purposefully assigned to all-women teams, 80 to all-men teams, and 180 to mixed-gender groups through an exogenous assignment process based on their gender. Furthermore, 33\% of the sample is placed into all-white homogeneous racial groups, while the remaining 67\% are assigned to other racial groups.


\subsection{Communication}
To examine how communication during the group building activity affects actions in the  public goods provision and minimum effort activities, we hire three English-fluent research assistants to code the content of video interactions. The coders receive a description of the experiment and experimental tasks but are not informed about the purpose of the study. The coders are told to code the messages independently using their own best judgement based on the pre-defined coding criteria. Video and audio interactions are divided into 30-seconds of video conversations called conversation segments. 

A conversation segment can be classified into multiple categories including: (1) frustration, (2) confusion, (3) talk in agreement, (4) talk in disagreement, (5) confident, (6) assertive, (7) excitement or satisfaction (8) comfortable as detailed in Table 25 in the appendix. Coders are tasked with assessing the presence or absence of frustration, with 1 denoting its presence and 0 indicating its absence. Similarly, expressions of confusion are assessed with a score of 1 during the 30-second segment if a coder assesses expressions of confusion. Participants engagement in conversations are also assigned binary values of whether they are affirming agreement (1 = present, 0 = absent) or expressing disagreement (1 = present, 0 = absent) with others during the puzzle-solving process. Participants confidence during the puzzle solving process is rated on a scale from 1 (not confident at all) to 5 (very confident). Expressions of excitement or satisfaction related to the puzzle-solving process are noted with 1, while their assertiveness in communication with others is measured on a scale from 1 (not assertive at all) to 5 (very assertive). Additionally, participants' comfort is evaluated based on language or nonverbal cues, ranging from 1 (not showing at all) to 5 (shows very clear signs). Each research assistant codes the interactions by 30-second segments for stage 1 of Part I and Part II of the study. The assessments of coders are summarized using majority voting for the binary assessments. The assessments that are on a likert scale are averaged across the three coders. The segments are then averaged per individual participant. 

It is important to note that, in addition to the decisions made by participants in stage 2, coded interactions of research assistants are included as chat controls. Chat controls are categorized into three main blocks using factorial analysis. "Positive chat" refers to the level of positive interactions a participant has with the other participants in the session and it is based on scores on assessments for assertiveness, excitement or satisfaction and comfort within the team. "Negative chat" assigns a value to the level of negative interactions a participant has with other participants. Variables included in the negative chat analysis include assessments of frustration, confusion and "talk in disagreement". Finally, the level of engagement of an individual in the puzzle-solving phase is assigned an engagement metric based on their speech and overall engagement in the task. 


\subsection{Empirical Specification} \label{subsec:Specification}
\subsubsection{Incumbent Teams in Part I} 
I use the tobit regression model to estimate the causal impact of team diversity on individual cooperation and coordination choices. First, I consider economic decision-making during the first part of the study before the newcomer is added to the team. Incumbent teams (3-person teams) are classified by levels of diversity. I start by analyzing the impact of the interaction between the gender diversity within a team and the gender identity of an individual on their cooperation and coordination choices in the activities. Gender-diverse teams consist of members with more than one gender identity, while gender-homogeneous teams consist of members sharing the same gender. Actions are modeled using equation 3 outlined below. The excluded group is a man in a gender-homogeneous team. 
\begin{center}      
 $ Y_{i}= \beta_1 Gender$-$diverse_{i} + \beta_2Notmale_{i}+\beta_3 Gender$-$diverse_{i}*Notmale_{i} + \theta X_{i} + \epsilon_{i} $ (3)
\end{center}

\noindent where $Y_{i}$ is the contribution of participant $i$ in either the public goods provision or the minimum effort activity. $Gender$-$diverse_{i}$ is a dummy variable indicating whether participant $i$ is in a gender diverse team. $NotMale_{i}$ is an indicator for participant gender. $Gender$-$diverse_{i}*Notmale_{i}$ is an interaction of non-male identifying individual in a gender diverse team. $X_{i}$ is a set of individual characteristics such as age, major, income, individual interactions with the group, parents' socioeconomic background and other personal characteristics. $\epsilon_{i}$ is the residual term. As the default inference method in the tobit regression, I specify upper and lower bounds. I also cluster standard errors at the session level.

Team assignment is randomly assigned. The coefficient, $\beta_1$ identifies the causal impact of a man in a gender diverse team in comparison to a man in the homogeneous gender team, $\beta_2$ identifies the effect of non-male identifying individual in homogeneous gender teams. The combined effects of $\beta_1$ , $\beta_2$ and $\beta_3$ identify the causal impact of a non-male identifying individual in a gender diverse team in comparison with a man in the homogeneous gender team. Under the null hypothesis of no treatment effects, there is no difference in contributions of individuals in different teams of varying gender diversity. 

\hspace  *{0mm} Next, I consider how the interaction of the racial diversity of the team and the racial identity of the individual affects the cooperation and coordination choices of the individual in the team. An individual's racial identity is categorized into white versus non-whites (minorities). A racially diverse team has more than one race while members of a racially homogeneous team share the same race. I utilize equation 4 below to examine how cooperation and coordination choices of an individual is impacted by the interaction of the team level of racial diversity and individual racial identity. In the equation below, the omitted category is a white person in a racially homogeneous team.

\begin{center}
 $ Y_{i} = \beta_1 Race$-$diverse_{i} + \beta_2 Notwhite_{i} + \beta_3 Race$-$diverse_{i}*Notwhite_{i} + \theta X_{i} + \epsilon_{i}  $    (4)
\end{center}

\noindent where $Y_{i}$ is the contribution of participant $i$ in either the public goods provision or the minimum effort activity. $Race$-$diverse_{i}$ is a dummy variable indicating whether participant $i$ is in a racially diverse team. $Notwhite_{i}$ is an indicator variable for a non-white person. $Race$-$diverse_{i}*Notwhite_{i}$ is an interaction variable for a non-white individual in a racially diverse team. $X_{i}$ is a set of individual characteristics such as age, major, income, individual interactions with the group, parents' socioeconomic background and other personal characteristics. $\epsilon_{i}$ is the residual term. As the default inference method, I specify upper and lower bounds as well as cluster standard errors at the session level.

The coefficient, $\beta_1$ identifies the causal impact of whites in a racially diverse team in comparison with whites in the racially homogeneous team, $\beta_2$ identifies the effect of non-white individuals in the homogeneous racial teams. The combined effects of $\beta_1$ , $\beta_2$ and $\beta_3$ identify the causal impact of a non-white individual in a racially diverse team in comparison with a white individual in the homogeneous racial team. Again, under the null hypothesis of no treatment effects, there is no difference in contributions of individuals in different teams of varying racial diversity. 

\hspace  *{0mm} Finally, I consider how the interaction of the diversity (racial and gender) of the team and the joint racial and gender identity of the individual affects the cooperation and coordination choices of the individual in the team. The experimental design reveals the salience of both identities. Teams that are homogeneous on both gender and race are classified as the least diverse. Teams that have heterogeneity on either race or gender are classified as "$Moderate$-$D$" diverse teams. Finally, teams that have heterogeneity on both race and gender are classified as the "$Most$-$D$" diverse. In addition to the classification of team diversity, I classify individuals into four main types - an interaction of the racial and gender identities of the individual. In the tobit regression model presented in equation 5 below, the omitted category is a white man in the least diverse team.  

\begin{center}
 $Y_{i}=\beta_1Moderate$-$D_{i}+\beta_2Most$-$D_{i}+\beta_3WhiteNonmale_{i}+\beta_4NonwhiteNonmale_{i}+\beta_5 NonwhiteMale_{i}+\beta_6Moderate$-$D_{i}*WhiteNonmale_{i}+\beta_7Moderate$-$D_{i}*NonwhiteNonmale_{i}+ \beta_8Moderate$-$D_{i}*NonwhiteMale_{i}+\beta_9Most$-$D_{i}*WhiteNonmale_{i}+\beta_{10}Most$-$D_{i}*NonwhiteNonmale_{i}+\beta_{11}Most$-$D_{i}*NonwhiteMale_{i}+\theta X_{i}+\epsilon_{i}  $  \space \space \space    (5)
\end{center}

\noindent where $Y_{i}$ is the contribution of participant $i$ in either the public goods provision or the minimum effort activity. $Moderate$-$D_{i}$ is a dummy variable indicating whether participant $i$ is in a moderately diverse team. $Most$-$D_{i}$ is a dummy variable indicating whether participant $i$ is in the most diverse team. $WhiteNon-male_{i}$ is an indicator variable for a white non-male identifying participant,  $NonwhiteNonmale_{i}$ is an indicator for a non-white non-male identifying participant and $NonwhiteMale_{i}$ is an indicator variable for a non-white male. $Moderate$-$D_{i}*WhiteNonmale_{i}$ is an interaction variable for a white non-male identifying individual in a moderately diverse team. $Moderate$-$D_{i}*NonwhiteNonmale_{i}$ is an interaction variable for a non-white non-male identifying individual in a moderately diverse team and $Moderate$-$D_{i}*NonwhiteMale_{i}$ is an interaction variable for a non-white man in a moderately diverse team. Similarly, $Most$-$D_{i}*WhiteNonmale_{i}$ is an interaction variable for a non-male identifying white individual in the most diverse team.  $Most$-$D_{i}*NonwhiteNonmale_{i}$ is an interaction variable for a non-male, non-white individual in the most diverse team and $Most$-$D_{i}*NonwhiteMale_{i}$ is an interaction variable for a non-white man in the most diverse team. $X_{i}$ is a set of individual characteristics such as age, major, income, individual interactions with the group, parents' socioeconomic background and other personal characteristics. $\epsilon_{i}$ is the residual term. As the default inference method, I specify upper and lower bounds as well as cluster standard errors at the session level.


\hspace  *{0mm} The coefficient, $\beta_1$ identifies the causal impact of a white man in a moderately diverse team in comparison to a white man in the least diverse team, $\beta_2$ identifies the causal impact of a white man in the most diverse team in comparison to a white man in the least diverse team, $\beta_3$ identifies the effect of a white non-male identifying individual in the least diverse team in comparison to a white man in the least diverse team, $\beta_4$ identifies the effect of a nonwhite non-male identifying individual in the least diverse team in comparison to a white man in the least diverse team and $\beta_5$ identifies the effect of a nonwhite man in the least diverse team in comparison to a white man in the least diverse team. The combined effects of $\beta_1$ , $\beta_3$ and $\beta_6$ identify the causal impact of a white non-male identifying individual in a moderately diverse team in comparison with a white man in the least diverse team, the combined effects of $\beta_1$ , $\beta_4$ and $\beta_7$ identify the causal impact of a nonwhite non-male identifying individual in a moderately diverse team in comparison with a white man in the least diverse team, the combined effects of $\beta_1$ , $\beta_5$ and $\beta_8$ identify the causal impact of a nonwhite man in a moderately diverse team in comparison with a white man in the least diverse team, the combined effects of $\beta_2$ , $\beta_3$ and $\beta_9$ identify the causal impact of a white non-male identifying individual in the most diverse team in comparison with a white man in the least diverse team, the combined effects of $\beta_2$, $\beta_4$ and $\beta_{10}$ identify the causal impact of a nonwhite non-male identifying individual in the most diverse team in comparison with a white man in the least diverse team. Finally, the combined effects of $\beta_2$ , $\beta_5$ and $\beta_{11}$ identify the causal impact of a nonwhite man in the most diverse team in comparison with a white man in the least diverse team. 

\subsubsection{Overall Incumbent and Newcomers}
Similar to the specifications above, I use the tobit regression model to estimate the causal impact of status on overall economic decision making across the 10 rounds of the study. An individual's status defines whether the person is an incumbent (part of the incumbent 3-person team) or a newcomer during the study. I employ Tobit regression models, as specified in equations 6a and 6b below, to investigate the influence of an individual's status and identity on their choices related to cooperation and coordination. Equations 6a and 6b adhere to the format used in equations 3 and 4, respectively, concerning gender and race identification. Equation 6a examines how behavior varies by gender across status with the omitted category being incumbent men. Equation 6b examines how behavior varies by race across status with the omitted group being white incumbents.

\begin{center}
 $ Y_{i} = \beta_1 Newcomer_{i} + \beta_2 Notmale_{i} + \beta_3 Newcomer_{i}*Notmale_{i} + \theta X_{i} + \epsilon_{i}  $  \space   (6a)
\end{center}

 \noindent where $Y_{i}$ is the contribution of participant $i$ in either the public goods provision or the minimum effort activity. $Newcomer_{i}$ is a dummy variable indicating whether participant $i$ is a newcomer during the 10 rounds of the study. $Notmale_{i}$ is an indicator variable for participant gender.$Newcomer_{i}*Notmale_{i}$ is an interaction variable of non-male identifying newcomer. $X_{i}$ is a set of individual characteristics such as age, income, individual interactions with the group, parents' socioeconomic background and other personal characteristics. $\epsilon_{i}$ is the residual term. As the default inference method, I specify upper and lower bounds as well as cluster standard errors at the session level.

\begin{center}
 $ Y_{i} = \beta_1 Newcomer_{i} + \beta_2 Notwhite_{i} + \beta_3 Newcomer_{i}*Notwhite_{i} + \theta X_{i} + \epsilon_{i}  $  \space  (6b)
\end{center}

 \noindent where $Y_{i}$ is the contribution of participant $i$ in either the public goods provision or the minimum effort activity. $Newcomer_{i}$ is a dummy variable indicating whether participant $i$ is a newcomer during the 10 rounds of the study.  $Notwhite_{i}$ is an indicator variable for participant race. $Newcomer_{i}*Notwhite_{i}$ is an interaction variable for a non-white individual who is a newcomer. $X_{i}$ is a set of individual characteristics such as age, major, income, individual interactions with the group, parents' socioeconomic background and other personal characteristics. $\epsilon_{i}$ is the residual term. As the default inference method, I specify upper and lower bounds as well as cluster standard errors at the session level.

\subsubsection{Incumbent and Newcomers After Changes in Group Composition}

Next, I examine how the cooperation and coordination choices of an individual differ based on the individual's status and the period of the study.  Actions in Part I are examined as before changes to the group composition, while actions in Part II of the study are considered to be after changes in group composition. The contributions of an individual are analyzed using the tobit regression specified in equation 7 below, with the omitted group being the incumbents before changes in group composition.

\begin{center}
 $ Y_{i} = \beta_1 Newcomerbefore_{i} + \beta_2 Newcomerafter_{i} + \beta_3 Incumbentafter_{i} + \theta X_{i} + \epsilon_{i}  $  \space \space \space  \space \space \space  \space \space \space \space  \space \space \space  \space \space \space   (7)
\end{center}

\noindent where $Y_{i}$ is the contribution of participant $i$ in either the public goods provision or the minimum effort activity. $Newcomerbefore_{i}$ is a dummy variable indicating whether participant $i$ is a newcomer during the first 5 rounds of the study. $Newcomerafter_{i}$ is a dummy variable indicating whether participant $i$ is a newcomer after changes in group composition at the end of round 5. $Incumbentafter_{i}$ is a dummy variable indicating an incumbent's economic decisions after changes in the team composition post round 5. $X_{i}$ is a set of individual characteristics such as age, major, income, individual interactions with the group, parents' socioeconomic background and other personal characteristics. $\epsilon_{i}$ is the residual term. As the default inference method, I specify upper and lower bounds as well as cluster standard errors at the session level.

\subsubsection{Newcomers in a New Team}
In addition to the above, I consider how newcomers cooperate and coordinate in different teams of varying diversity. Teams are classified based on both gender and racial diversity in the team, unlike the previous specifications. Teams that are gender and racially diverse are considered the most diverse. Teams that are only gender diverse or only racially diverse are considered moderately diverse. Teams that are homogeneous on gender and race are considered the least diverse. I utilize equation 8a to examine the causal impact of the existing team racial and gender diversity on economic decisions of the newcomer with the omitted group being newcomers in the least diverse team. 

\begin{center}
 $ Y_{i} = \beta_1 Moderately$-$diverse_{i} + \beta_2 Most$-$diverse_{i} + \theta X_{i} + \epsilon_{i}  $  \space \space \space    (8a)
\end{center}

\noindent where $Y_{i}$ is the contribution of newcomer $i$ in either the public goods provision or the minimum effort activity. $Moderately$-$diverse_{i}$ is a dummy variable indicating whether participant $i$, the newcomer is in a moderately diverse team. $Most$-$diverse_{i}$ is a dummy variable indicating whether a newcomer is in the most diverse team and $X_{i}$ is a set of individual characteristics such as age, major, income, individual interactions with the group, parents' socioeconomic background and other personal characteristics. $\epsilon_{i}$ is the residual term. As the default inference method in the tobit regression, I specify upper and lower bounds as well as cluster standard errors at the session level.

Next, I look at how the existing gender diversity affects economic decisions of a newcomer by gender of the newcomer. Incumbent teams are classified based on the levels of gender diversity of the existing team following the format of equation 3 above. The homogeneous gender teams have the same gender. A team is considered gender diverse if the team has a heterogeneous gender composition. The decisions of a newcomer are analyzed using the tobit model specified in equation 8b below with the omitted group being a man in a homogeneous gender team.  


\begin{center}
$Y_{i}=\beta_1Gender$-$diverse_{i}+\beta_2Notmale_{i}+\beta_3Gender$-$diverse_{i}*Notmale_{i}+\theta X_{i} + \epsilon_{i} $ (8b)
\end{center}

\noindent where $Y_{i}$ is the contribution of participant $i$ in either the public goods provision or the minimum effort activity. $Gender$-$diverse_{i}$ is a dummy variable indicating whether participant $i$,  the newcomer is joining a gender diverse team. $Notmale_{i}$ is an indicator for participant gender. $Gender$-$diverse_{i}*Notmale_{i}$ is an interaction of non-male identifying individual in a gender diverse team. $X_{i}$ is a set of individual characteristics such as age, major, income, individual interactions with the group, parents' socioeconomic background and other personal characteristics. $\epsilon_{i}$ is the residual term. As the default inference method in the tobit regression, I specify upper and lower bounds as well as cluster standard errors at the session level.


\hspace  *{0mm} Finally, I consider how the interaction of the racial diversity of the team and racial identity of the individual affects the cooperation and coordination choices of the newcomer in the team. I follow the definitions established in equation 4 above where an individual's racial identity is categorized into white versus non-whites (minorities). I then utilize equation 8c below to examine how cooperation and coordination choices of an individual is impacted by the interaction of the team racial diversity and individual racial identity. The omitted group in equation 8c below is a white person in a racially homogeneous team. 

\begin{center}
 $ Y_{i} = \beta_1 Race$-$diverse_{i} + \beta_2 Notwhite_{i} + \beta_3 Race$-$diverse_{i}*Notwhite_{i} + \theta X_{i} + \epsilon_{i}  $  \space   (8c)
\end{center}

\noindent where $Y_{i}$ is the contribution of participant $i$ in either the public goods provision or the minimum effort activity. $Race$-$diverse_{i}$ is a dummy variable indicating whether participant $i$ is in a racially diverse team. $Notwhite_{i}$ is an indicator variable for participant race. $Race$-$diverse_{i}*Notwhite_{i}$ is an interaction variable for a non-white individual in a racially diverse team. $X_{i}$ is a set of individual characteristics such as age, major, income, individual interactions with the group, parents' socioeconomic background and other personal characteristics. $\epsilon_{i}$ is the residual term. As the default inference method, I specify upper and lower bounds as well as cluster standard errors at the session level.


\subsubsection{Incumbents after a Newcomer Joins}

Finally, I look at how decisions of incumbent members of the team are affected by the racial identity and gender identity of a newcomer after changes in the group composition, after round 5. Incumbent members of the team are classified based on whether they share gender with the newcomer or share race as a racial minority (non-white) or white. This leads to incumbents that share both identities with the newcomer, incumbents that share gender identity but not racial identity, incumbents that share racial identity but not gender and incumbents that share neither racial identity or gender identity with the newcomer. The decisions of incumbent members of the team are analyzed using the tobit model specified in equation 9 below with the omitted group being incumbents that do not share race and gender with the newcomer.  

\begin{center}
 $ Y_{i} = \beta_1 Congruentgender_{i} +\beta_2 Congruentrace_{i} + \beta_3 Congruentboth_{i} + \theta X_{i} + \epsilon_{i}  $  \space \space \space    (9)
\end{center}

\noindent where $Y_{i}$ is the contribution of participant $i$ in either the public goods provision or the minimum effort activity.  $Congruentgender_{i}$ is a dummy variable indicating whether incumbent participant $i$ shares gender but not race with the newcomer. $Congruentrace_{i}$ is a dummy variable indicating whether incumbent participant $i$ shares race but not gender with the newcomer and $Congruentboth_{i}$  is a dummy variable indicating whether incumbent participant $i$ shares gender and race with the newcomer. $X_{i}$ is a set of individual characteristics such as age, income, individual interactions with the group, parents' socioeconomic background and other personal characteristics. $\epsilon_{i}$ is the residual term. As the default inference method, I specify upper and lower bounds as well as cluster standard errors at the session level.

\subsection{Cooperation in the Public Goods Provision}
\noindent I start the analyses by examining the average cooperation in the first part of the study among incumbent teams to understand whether the diversity of a team affects individual performance in teams of varying degrees of diversity. It is important to note "incumbents" in this study are participants in the 3-person teams prior to changes in team composition. As is the norm in the public goods provision, I consider the contributions a player makes toward the public account($C_i$ in equation 1 above) to represent that individual's level of cooperation within the team. Individual participant payoffs are based on equation 1 outlined above. I use the tobit regression models presented in the \hyperref[subsec:Specification]{empirical section} of the paper for my analyses. I also cluster the errors by session and present the results of the marginal effects, as is standard for causal inference.  

\noindent\textbf{\textit{Gender Diversity}} 

\noindent First, I consider how the gender diversity of the incumbent teams affect the cooperation across genders. The results of the  regression model are presented in table 3 below. The results presented in table 3 demonstrate the impact of incumbent team diversity on behavior of different incumbents in their cooperation decisions. The first column of table 3 illustrates the foundational model, estimated using equation 3 as previously specified above. The model considers a man in homogeneous gender team as the omitted group in the base model in column 1. The findings reveal that men in gender homogeneous teams tend to exhibit higher levels of cooperation compared to the non-male identifying individuals in homogeneous gender teams. As compared to men in homogeneous gender teams, non-male identifying individuals contribute 15.4\% ($P-value<0.05$) less of their endowment toward the group. Additionally, men in the gender diverse team do not statistically significantly cooperate at a level different from men in the homogeneous gender teams ($P-value=0.37$). However, non-male identifying individuals in diverse gender groups cooperate at a lower level than  men in the homogeneous gender group and men in the diverse gender group.  Specifically, non-male identifying individuals contribute 40.3\% ($P-value<0.01$) less than men in the homogeneous gender group and 45.1\% ($P-value<0.01$) less than men in the gender diverse team. The inclusion of controls for age, level of education and political affiliation controls in column 2 do not affect the direction and significance of the coefficients of cooperation. Further controls of minority status (such as gender minority status and racial minority status) and the level of interactions of the individual in the team do not affect the significance level of the non-male identifying individuals. Additionally, positive communication, as assessed by the video interactions in the team, affects cooperation ($P-value<0.1$). The findings suggest that men cooperate at similar levels in gender diverse and homogeneous gender teams. Non-male identifying individuals are most affected by the type of team they are in. Non-male identifying individuals cooperate less in the homogeneous gender teams. They are even worse cooperators in a gender diverse team. This suggests that in assigning non-male identifying individuals, it is more important to consider environments where they can be more cooperative. It is also worth noting that positive chat with the team is positively related to cooperation rates and is statistically significant.  

\begin{table}[H]
 \captionsetup{justification=raggedright,singlelinecheck=false}
\caption{Incumbent Team Diversity and Cooperation by Gender}
        \begin{table}[htbp]
    \begin{tabular}{c c c c}
    \toprule
    \textbf{Variables} & \textbf{(1)} & \textbf{(2)} & \textbf{(3)}         \\ 
\midrule
Diverse$-$Gender            &     4.77    &    5.69                               &  3.74  \\
                            &     (5.32)  &    (5.22)                             &  (5.72) \\

Notmale                  &      -15.37\sym{**} &    -15.23\sym{**}             &  -10.79  \\
                            &      (6.11)         &    (6.16)                     &  (7.06) \\
Diverse*Notmale         &      -29.66\sym{***}   &  -30.10\sym{***}            &  -22.77\sym{**}\\
                            &      (8.29)         &    (8.31)                     &  (9.75) \\


Positive Chat                        &                     &                       &  12.85\sym{*}  \\
                                 &                     &                           &  (7.40)  \\
Negative  Chat                       &                     &                       &  -0.84  \\
                                 &                     &                           &  (4.13)  \\
Engagement Chat                      &                     &                       &  -7.45  \\
                                 &                     &                           &  (5.34)  \\
\midrule
Other Controls                   &    No               &    Yes                    &    Yes        \\
Status Controls                    &    No               &    No                   &    Yes        \\
Chat                             &    No               &    No                     &    Yes        \\
\midrule
Number of Participants           &    333               &    333                   &    324        \\
\midrule
Observations                     &       1665          &       1665                &  1620    \\
\bottomrule
\end{tabular}
\begin{footnotesize}
\newline
*P$<$0.1, **P$<$0.05, ***P$<$0.01
\newline
Note: Robust standard errors clustered at the group level. Marginal effects of Tobit Model reported.
\newline
The dependent variable is the contributions toward the group in public goods provision in Part I. \\
\end{footnotesize}
\end{table}


\end{table}

\noindent \textbf{\textit{Racial Diversity}} 

\noindent Next, I explore the impact of the incumbent team's racial diversity on the cooperation rates of white and non-white individuals. The results derived from the model specified in equation 4 above in the \hyperref[subsec:Specification]{empirical specification} section are presented in table 4 below. Column 1 shows the base model of cooperation of white and non-white individuals in the teams of varying racial diversity. The omitted group of individuals is a white individual in a homogeneous all-white team. The results show that, whites in the racially diverse teams cooperate more than whites in the homogeneous racial team ($P-value<0.1$). However, the cooperation of non-white individuals in the racially diverse teams is the opposite. They cooperate at lower levels equivalent to 46.9\% (9.9-25-31.9) of their endowment less as compared to whites in a homogeneous racial team ($P-value<0.01$). Importantly, the inclusion of additional controls for age, level of education and political affiliation in column 2 do not affect the significance and direction of the coefficients. However, it is important to highlight that positive interactions within the group has a statistically significant effect on cooperation ($P-value<0.01$). This further enforces the findings in the previous paragraph. 

% In fact, a one standard deviation increase in positive communication is associated with more than 100\% increase in cooperation. Addition of controls for risk seeking behavior and interactions with the team does not diminish the statistical significance of the effects.
% \begin{center}
\begin{table}[H]
\captionsetup{justification=raggedright,singlelinecheck=false}
\caption{Incumbent Team Diversity and Cooperation by Race}
    
        \begin{table}[htbp]
\begin{left}
    
    \begin{tabular}{c c c c}
    \toprule
    \textbf{Variables} & \textbf{(1)} & \textbf{(2)} & \textbf{(3)}      \\ 
\midrule
Diverse$-$Race        &     9.89\sym{*}    &    8.89\sym{*}                   &  7.56 \\
                            &     (5.29)          &     (5.27)                &  (5.66) \\
Notwhite                  &      -24.99\sym{***}  &    -23.44\sym{***}      &  -24.74\sym{***}     \\
                            &      (5.46)         &    (6.67)          &  (6.77) \\
Diverse$-$Race*Notwhite    &      -31.78\sym{***}   &  -31.77\sym{***}      &  -32.28\sym{***}\\
                            &      (6.19)         &    (7.01)              &  (7.65) \\

Positive Chat                        &                     &                       &  13.31\sym{***}  \\
                                 &                     &                           &  (7.59)  \\
Negative  Chat                       &                     &                       &  -0.37  \\
                                 &                     &                           &  (4.06)  \\
Engagement Chat                      &                     &                       &  -8.36  \\
                                 &                     &                           &  (5.35)  \\
\midrule
Other Controls                   &    No               &    Yes                    &    Yes        \\
Status Controls                    &    No               &    No                   &    Yes        \\
Chat                             &    No               &    No                     &    Yes        \\
\midrule
Number of Participants           &    333               &    333                   &    324        \\
\midrule
Observations                     &       1665          &       1665                &  1620    \\
\bottomrule

\end{tabular}
\begin{footnotesize}
\newline
*P$<$0.1, **P$<$0.05, ***P$<$0.01
\newline
Note: Robust standard errors clustered at the group level. Marginal effects of Tobit Model reported.
\newline
The dependent variable is the contributions toward the group in public goods provision in Part I.\end{footnotesize}
\end{left}

\end{table}
    
\end{table}
% \end{center}

\noindent\textbf{\textit{Gender and Racial Diversity}} 

\noindent Finally, I proceed to assess how the incumbent team gender and racial diversity influences individuals of various racial and gender backgrounds. To achieve this, I employ equation 5 specified above in the \hyperref[subsec:Specification]{empirical specification} section. The outcomes of this analysis are presented in Table 5 below. Column 1 shows the base model without additional controls. As mentioned above, teams are classified based on the levels of racial and gender diversity. Teams that have diverse gender or racial composition are considered moderately diverse. Teams that have diverse gender and diverse racial compositions are considered most diverse. Finally, individuals in homogeneous racial and homogeneous gender teams are considered to be in the least diverse team. The omitted group of individuals in the analysis presented in table 5 are white men in the least diverse team. The findings show that white men in the more diverse teams cooperate more than white men in the least diverse team. They contribute on average 12.6\% and 10.5\% of their endowment more in the moderately diverse team and the most diverse teams respectively ($P-value<0.05$). Furthermore, non-white individuals in the least diverse teams cooperate less than white men in the least diverse team. The contributions of non-male identifying white individuals in the least diverse team is 14.8\% of their endowment less ($P-value<0.05$). In the moderately diverse teams, white individuals who are not male contribute 28.2\% of their endowment less toward the group ($P-value<0.01$). Non-white males contribute 47.6\% of their endowment less on average than white men in the least diverse team ($P-value<0.01$) and non-white individuals who are not male also contribute 80.7\% of their total endowment less ($P-value<0.01$). In the most diverse teams, non-male identifying whites contribute on average 45.3\% less of their endowment on average ($P-value<0.01$) than whites in the least diverse teams, non-white males contribute 66.4\% of their endowment less ($P-value<0.01$) and non-white individuals who are not male contribute about 86.8\% less on average as compared to white men in the least diverse team ($P-value<0.01$). The results underscore that the level of cooperation exhibited by individuals varies according to the diversity of the team they belong to. Not-white individuals are most affected by racial diversity of the team they are a part of while non-male identifying individuals are most affected by the gender diversity of the team they join. When we consider the dual racial and gender identities of an individual, white men tend to cooperate in more diverse teams. This observation leads to my initial finding: 

\textbf{Result 1a: Cooperation is lower among minorities and non-male identifying individuals but white men tend to cooperate more in diverse teams. }

% \begin{landscape}
\begin{table}[H]
 \captionsetup{justification=raggedright,singlelinecheck=false}
\caption{Incumbent Team Diversity and Cooperation by Gender and Race} \label{table:3c}
    \begin{center}
        \begin{table}[htbp]
    \begin{tabular}{c c c c}
    \toprule
    \textbf{Variables} & \textbf{(1)} & \textbf{(2)} & \textbf{(3)}       \\ 
\midrule
Moderately$-$Diverse        &     12.58\sym{**}    &    12.46\sym{**}     &  11.86\sym{**}  \\
                            &     (6.69)          &     (5.69)             &  (6.16) \\
Most$-$Diverse              &      10.53\sym{**}             &     9.77                & 7.31   \\
                            &      (6.36)         &     (6.63)           &  (7.57)  \\
White$-$Notmale          &      -14.77\sym{**}        &    -14.74\sym{**}    &  -10.86  \\
                            &      (6.86)         &    (6.89)             & (7.26)  \\
NonwhiteMale         &       -20.85\sym{**}   &  -19.44\sym{**}      &  -20.13\sym{**}\\
                            &      (7.67)         &    (8.83)             &  (9.32) \\
NonwhiteNotmale    &  -39.13\sym{***}          &  37.58\sym{***}               &  -37.32\sym{***} \\
                            &      (6.47)         &    (7.52)              &   (9.13) \\
Moderate*WhiteNotmale    &   -26.00 \sym{***} & -25.01\sym{**}       &  -20.23 \sym{***} \\
                                 &  (11.24)        &   (7.15)             &  (10.34)  \\
Moderate*NonwhiteMale      &   -39.37\sym{***} &  -39.93\sym{***}     &  -37.63\sym{***}  \\
                                 &  (12.68)        &     (12.56)             &  (13.32)   \\
Moderate*NonwhiteNotmale  &  -54.12\sym{***} &     -56.96\sym{***}     &  -52.93\sym{***} \\
                                 &  (10.96)        &     (11.15)             &  (12.83)   \\
Most*WhiteNotmale          &  -41.07\sym{***} &     -41.10\sym{***}     &  -34.55\sym{***}  \\
                                 &  (15.35)        &     (15.46)             &  (16.55)  \\
Most*NonwhiteMale         &   -56.11\sym{***} &     -56.04\sym{***}     &   -60.82\sym{***}  \\
                                 &  (17.61)        &     (17.82)             &  (19.21) \\
Most*NonwhiteNotmale    &   -58.15\sym{***} &   -60.51\sym{***}     &  -54.07\sym{***}  \\
                                 &  (13.09)        &     (12.64)             &  (16.00)   \\
Positive Chat                        &                     &                       &  13.66\sym{*}  \\
                                 &                     &                           &  (7.44)  \\
Negative  Chat                       &                     &                       &  0.13  \\
                                 &                     &                           &  (3.98)  \\
Engagement Chat                      &                     &                       &  -7.73  \\
                                 &                     &                           &  (5.18)  \\
\midrule
Other Controls                   &    No               &    Yes                    &    Yes        \\
Status Controls                    &    No               &    No                   &    Yes        \\
Chat                             &    No               &    No                     &    Yes        \\
\midrule
Number of Participants           &    333               &    333                   &    324        \\
\midrule
Observations                     &       1665          &       1665                &  1620    \\
\bottomrule

\end{tabular}
\begin{footnotesize}
\newline
*P$<$0.1, **P$<$0.05, ***P$<$0.01
\newline
Note: Robust standard errors clustered at the group level. Marginal effects of Tobit Model reported. 
\newline
The dependent variable is the contributions toward the group in public goods provision in Part I.
\end{footnotesize}
\end{table}
    \end{center}
\end{table}
% \end{landscape}

\hspace  *{0mm} Moving forward, I delve into the cooperation of incumbents and newcomers in the public goods provision. First, I plot of cooperation rates by levels of diversity in figure 2 below. Dimensions of diversity in the team is based on the classification from the previous paragraph. I combine the racial and gender diversity of the teams as described above into least diverse, moderately diverse and most diverse. Notably, incumbents consistently exhibit higher levels of cooperation across teams of varying degrees of diversity. This is corroborated by the results of the tobit regression analysis presented in tables 6 and 7 below. It is evident that newcomers generally cooperate less than incumbents over the 10 rounds. In table 6, I analyze cooperation rates by considering the gender of individuals using equation 6a specified above in the \hyperref[subsec:Specification]{empirical specification} section. The results in the base model show that male newcomers contribute on average 16.8\% ($P-value<0.01$) of their endowment less toward the group than incumbent males. Furthermore, non-male identifying incumbents are less cooperative and contribute 8.3\% of their endowment less than incumbent males ($P-value<0.1$). The results are robust to the inclusion of additional controls in columns 2 and 3.

In table 7, Column 1, I present the cooperation rates of  newcomers and incumbents by racial identity using equation 6b specified above in the \hyperref[subsec:Specification]{empirical specification} section. It is evident that White newcomers are less cooperative than white incumbents. However, non-white individuals are generally less cooperative as compared to incumbents in the team. Non-white incumbents contribute 17.7\% less of their endowment toward the group as compared to incumbent whites ($P-value<0.01$). Similarly, non-white newcomers contribute 9.7\% less on average as compared to white incumbents but the effect is statistically insignificant in the base model. The coefficients are lower and statistically significant at the 10\% significance level once additional controls are included in columns 2 and 3 of table 7. Furthermore, the coefficients for a newcomer and not-white individuals are robust to the inclusions of controls in columns 2 and 3.  Note that this analysis is examining actions toward the group during the 10 rounds of the public goods provision activity, five of which, the newcomers are not part of the group. \footnote{I later examine whether newcomers' contributions change after joining a team.} Newcomers cooperate less than incumbents. However, gender identity of the newcomer does not affect cooperation of newcomers but the race of an individual play a role in cooperation in the team. These insights lead to my second result.

\textbf{Result 2a:Incumbents cooperate more than newcomers during the duration of the study. }

\begin{figure}[H]
 \captionsetup{justification=raggedright,singlelinecheck=false}
\caption{Incumbent Versus Newcomer Overall Cooperation}
\includegraphics[scale=0.2]{Figures/Overall_ppg_new_inc.png} 
\end{figure}

\begin{table}[H]
 \captionsetup{justification=raggedright,singlelinecheck=false}
\caption{Incumbent Versus Newcomer Overall Cooperation  by Gender} \label{tab:table4}
    \begin{center}
        \begin{table}[htbp]
    \begin{tabular}{c c c c}
    \toprule
    \textbf{Variables} & \textbf{(1)} & \textbf{(2)} & \textbf{(3)}      \\ 
\midrule
Newcomer                         &     -16.81\sym{***} &     -15.63\sym{***}  &  -14.69\sym{***}   \\
                                 &     (4.51)          &     (4.74)          &  (5.10)             \\
Notmale                         &        -8.33\sym{*} &     -9.04\sym{*}  &  -7.56     \\
                                 &     (5.22)          &     (5.22)          &  (5.55)             \\
Newcomer*Notmale                  &     5.56         &  3.22               &  3.75    \\
                                 &     (6.51)          &   (6.46)          &  (6.39)             \\


\midrule
Other Controls                   &    No               &    Yes              &    Yes            \\
Status Controls                    &    No               &    No               &    Yes           \\
Chat                             &    No               &    No               &    No             \\
\midrule
Number of Participants           &    444               &    444              &    444               \\
\midrule
Observations                     &       4440          &       4440          &  4440            \\
\bottomrule

\end{tabular}
\begin{footnotesize}
\newline
*P$<$0.1, **P$<$0.05, ***P$<$0.01
\newline
Note: Robust standard errors clustered at the group level. Marginal effects of Tobit Model reported.
\newline
The dependent variable is the contributions toward the group in public goods provision in both Parts.
\end{footnotesize}
\end{table}
    \end{center}
\end{table}


\begin{table}[H]
 \captionsetup{justification=raggedright,singlelinecheck=false}
\caption{Incumbent Versus Newcomer Overall Cooperation by Race} \label{tab:table4}
    \begin{center}
        \begin{table}[htbp]
    \begin{tabular}{c c c c}
    \toprule
    \textbf{Variables} & \textbf{(1)} & \textbf{(2)} & \textbf{(3)}       \\ 
\midrule
Newcomer                         &     -14.72\sym{***} &     -13.59\sym{***}  &  -10.92\sym{*}   \\
                                 &     (4.58)          &     (4.75)          &  (5.13)             \\
Notwhite                         &  -17.70\sym{***} &     -19.07\sym{***}  &  -19.56\sym{***}   \\
                                 &     (4.78)          &     (5.68)          &  (6.11)             \\
Newcomer*Notwhite              &     -9.65              &     -11.35\sym{*}  &  -11.68\sym{*}   \\
                                 &     (6.45)          &     (7.09)          &  (7.09)             \\


\midrule
Other Controls                   &    No               &    Yes              &    Yes            \\
Status Controls                    &    No               &    No               &    Yes           \\
Chat                             &    No               &    No               &    No             \\
\midrule
Number of Participants           &    444               &    444              &    444               \\
\midrule
Observations                     &       4440          &       4440          &  4440            \\
\bottomrule

\end{tabular}
\begin{footnotesize}
\newline
*P$<$0.1, **P$<$0.05, ***P$<$0.01
\newline
Note: Robust standard errors clustered at the group level. Marginal effects of Tobit Model reported. 
\newline
The dependent variable is the contributions toward the group in public goods provision in both Parts. 
\end{footnotesize}
\end{table}
    \end{center}
\end{table}


\hspace  *{0mm} I now shift focus to the cooperation of individuals, both prior to the inclusion of newcomers and after their integration. I established in result 2a that newcomers cooperate less than incumbents over the span of 10 rounds. In table 8, column 1, I compare cooperation of incumbents and newcomers before and after group composition changes using the model specified in equation 7 above in the \hyperref[subsec:Specification]{empirical specification} section. I find that the newcomers primarily cooperate less in Part I when they participate in the public goods provision without a team-building activity ($P-value<0.01$). However, newcomers substantially increase their cooperation rates later in the study ($P-value<0.01$). Furthermore, incumbents statistically significantly increase their cooperation overall after the changes in group composition but the increase is lower than that of the newcomers ($P-value<0.01$). This increase in newcomer cooperation compensates for the low cooperation prior to their integration, leading to parity in cooperation between the newcomers and the base treatment group---the incumbents---before team composition changes ($P-value=0.91$). The results are robust to the inclusion of additional controls in columns 2 and 3. Newcomers cooperate more in teams and incumbents increase cooperation after a newcomer joins. These changes lead to my third primary result: 

\textbf{Result 3a: Incumbents and newcomers increase their levels of cooperation after changes in the team composition but the increase in cooperation is higher among newcomers.}

\begin{table}[H]
 \captionsetup{justification=raggedright,singlelinecheck=false}
\caption{Incumbent and Newcomer Cooperation Before and After Group Composition Changes} \label{tab:table5}
    \begin{center}
        \begin{table}[htbp]
    \begin{tabular}{c c c c}
    \toprule
    \textbf{Variables} & \textbf{(1)} & \textbf{(2)} & \textbf{(3)}       \\ 
\midrule
Newcomer(Before=1)               &     -26.61\sym{***}    &    -25.12\sym{***}  &  -22.46\sym{***}   \\
                                 &     (4.56)             &     (4.87)          &  (5.44)         \\
\addlinespace
Newcomer(After=1)                &     -0.64           &     0.80         &  3.27          \\
                                 &     (5.69)          &     (5.80)        &  (6.22)         \\
\addlinespace
Incumbent(After=1)               &     5.90\sym{**}    &     5.87\sym{**}  &  5.74\sym{**}     \\
                                 &     (2.91)          &     (2.89)        &  (2.88)         \\
\midrule
Other Controls                   &    No               &    Yes              &    Yes             \\
Status Controls                    &    No               &    No               &    Yes             \\
Chat                             &    No               &    No               &    No               \\
\midrule
Number of Participants           &    444               &    444              &    444               \\
\midrule
Observations                     &       4440          &       4440          &  4440          \\
\bottomrule

\end{tabular}
\begin{footnotesize}
\newline
*P$<$0.1, **P$<$0.05, ***P$<$0.01
\newline
Note: Robust standard errors clustered at the group level. Marginal effects of Tobit Model reported. 
\newline
The dependent variable is the contributions toward the group in public goods provision in both Parts.\end{footnotesize}
\end{table}
    \end{center}
\end{table}




\hspace  *{0mm} Moving forward, my analysis delves into the distinctive behaviors displayed by different newcomers following their assimilation into the team using the model specified in equation 8a above in the \hyperref[subsec:Specification]{empirical specification} section. First, I consider combined effects of racial and gender diversity on newcomer cooperation. As previously stated, I classify newcomers into 3 groups along the lines of the racial and gender diversity of the team they join. The base treatment is newcomers that join the least diverse team, teams that are homogeneous on both gender and racial compositions. Newcomers to moderately diverse teams join teams of heterogeneous gender composition or heterogeneous racial composition. Finally, newcomers that join heterogeneous racial and heterogeneous gender teams are classified as joining the most diverse teams. The results of the analysis are shown in table 9. The analysis shows that newcomers that join moderately diverse teams contribute on average 17.7\% more of their endowment towards the group account as compared to newcomers that join the least diverse incumbent teams ($P-value<0.1$). Additionally, newcomers that join the most diverse teams do not cooperate at a level different from newcomers that join the least diverse teams ($P-value=0.60$). Further analysis of the results as shown in the appendix reveals that the higher cooperation in moderately diverse teams is driven by white newcomers. White newcomers to the least diverse teams significantly reduce their cooperation when their gender is in-congruent with the rest of the team. 

 Further exploration of the impact of gender diversity on newcomer cooperation by newcomer gender is presented in table 10 using equation 8b specified above in the \hyperref[subsec:Specification]{empirical specification} section. The results do not show any statistically significant effect of gender diversity on newcomer behavior by newcomer gender. In addition to this, I consider newcomers by race in table 11 using the tobit model specified in equation 8c above in the empirical specification section. The results show that non-white newcomers are most affected by the diversity of the team they join. Non-white newcomers cooperate less than whites when they join a homogeneous racial team ($P-value<0.1$) but cooperate even less in the diverse racial teams as compared to whites in the racially homogeneous teams ($P-value<0.05$). This shows that the racial diversity of the team does not affect cooperation of the white newcomers ($P-value=0.45$) overall except for cases where a white newcomer joins a homogeneous gender team, in-congruent with their gender. This is particularly important given that whites are the majority in the labor force of the United States\citep{b21}. It is noteworthy that the demographic distribution of our participants mean racially homogeneous groups are mostly groups of homogeneous all-white teams. Interestingly, non-white newcomers who join racially diverse teams cooperate at level that is 28.9\% of their endowment lower than white newcomers in homogeneous racial teams ($P-value<0.01$). The lower cooperation of non-white newcomers is robust to additional controls and the chat controls of participants in columns 2 and 3 respectively in table 11. The results presented here and further analysis \footnote{Further analysis of behavior of newcomers is presented in Appendix A.} lead to my fourth finding.   

\textbf{Result 4a: Newcomers are sensitive to the existing team diversity but the nature of cooperation is dependent on racial identity of the newcomer.}


\begin{table}[H]
 \captionsetup{justification=raggedright,singlelinecheck=false}
\caption{Incumbent Team Diversity and Newcomer Cooperation } \label{tab:table6}
    \begin{center}
        \begin{table}[htbp]
    \begin{tabular}{c c c c}
    \toprule
    \textbf{Variables} & \textbf{(1)} & \textbf{(2)} & \textbf{(3)}     \\ 
\midrule
Moderate$-$Diverse  &  17.71\sym{*} &  11.71            &  12.97     \\
                    &  (9.42)       &  (8.95)           &  (8.94)    \\
\addlinespace
Most$-$Diverse      &  5.86       &   3.05                &  3.95     \\
                    &  (11.21)      & (10.80)             &  (10.77)     \ \\

\addlinespace
Positive                 &           &                     &  -2.56     \\
                         &           &                     &  (7.81)   \\
\addlinespace
Negative                 &           &                      &  -1.19  \\
                          &          &                      &  (4.68)    \\
\addlinespace
Engagement               &          &                        & 5.29    \\
                         &          &                          &  (4.55)   \\

\midrule
Other Controls   &   No &  Yes &    Yes    \\
Status Controls &   No  &    No    &    Yes    \\
Chat          &    No    &    No   &    Yes          \\
\midrule
Number of Participants & 111   &    111 &    106    \\
\midrule
Observations          &   555   &  555  &  530         \\
\bottomrule

\end{tabular}
\begin{footnotesize}
\newline
*P$<$0.1, **P$<$0.05, ***P$<$0.01
\newline
Note: Robust standard errors clustered at the group level. Marginal effects of Tobit Model reported. 
\newline
The dependent variable is the contributions toward the group in public goods provision in Part II.
\end{footnotesize}
\end{table}

    \end{center}
\end{table}

\begin{table}[H]
\caption{Incumbent Team Diversity and Newcomer Cooperation by Gender } \label{tab:table6}
    \begin{center}
        \begin{table}[htbp]
    \begin{tabular}{c c c c}
    \toprule
    \textbf{Variables} & \textbf{(1)} & \textbf{(2)} & \textbf{(3)}       \\ 
\midrule
Diverse$-$Gender            &     0.55    &    0.34                               &  1.68  \\
                            &     (8.80)  &    (8.45)                             &  (8.63) \\

Notmale                  &      -2.39  &    -6.32                            &  -6.19  \\
                            &      (8.41)  &    (8.06)                        &  (8.63) \\
Diverse*Notmale         &      -11.37   &  -14.08                            &  -16.90 \\
                            &      (14.58)  &    (13.66)                     &  (15.59) \\


Positive Chat                        &                     &                       &  -0.32  \\
                                 &                     &                           &  (7.78)  \\
Negative  Chat                       &                     &                       &  -2.46  \\
                                 &                     &                           &  (4.81)  \\
Engagement Chat                      &                     &                       &  6.68  \\
                                 &                     &                           &  (4.33)  \\
\midrule
Other Controls   &   No &  Yes &    Yes    \\
Status Controls &   No  &    No    &    Yes    \\
Chat          &    No    &    No   &    Yes          \\
\midrule
Number of Participants & 111   &    111 &    106    \\
\midrule
Observations          &   555   &  555  &  530         \\
\bottomrule

\end{tabular}
\begin{footnotesize}
\newline
*P$<$0.1, **P$<$0.05, ***P$<$0.01
\newline
Note: Robust standard errors clustered at the group level. Marginal effects of Tobit Model reported.
\newline
The dependent variable is the contributions toward the group in public goods provision in Part II.\end{footnotesize}
\end{table}

    \end{center}
\end{table}


\begin{table}[H]
 \captionsetup{justification=raggedright,singlelinecheck=false}
\caption{Incumbent Team Diversity and Newcomer Cooperation by Race } \label{tab:table6}
    \begin{center}
        \begin{table}[htbp]
    \begin{tabular}{c c c c}
    \toprule
    \textbf{Variables} & \textbf{(1)} & \textbf{(2)} & \textbf{(3)}      \\ 
\midrule
Diverse$-$Race            &     5.03     &    0.12                               &  -0.12  \\
                            &   (8.71)   &    (8.56)                             &  (8.34) \\

Notwhite                  &      -15.63\sym{*}  &    -20.61\sym{**}                 &  -20.27\sym{**}   \\
                            &      (8.38)          &    (9.35)                        &  (9.43) \\
Diverse*Notwhite         &      -18.07\sym{*}   &  -23.64\sym{**}                 &  -22.79\sym{*}  \\
                            &      (10.07)         &    (11.21)                     &  (12.09) \\


Positive Chat                        &                     &                       &  -4.24  \\
                                 &                     &                           &  (7.93)  \\
Negative  Chat                       &                     &                       &  -2.53  \\
                                 &                     &                           &  (4.54)  \\
Engagement Chat                      &                     &                       &  6.43  \\
                                 &                     &                           &  (4.34)  \\
\midrule
Other Controls   &   No &  Yes &    Yes    \\
Status Controls &   No  &    No    &    Yes    \\
Chat          &    No    &    No   &    Yes          \\
\midrule
Number of Participants & 111   &    111 &    106    \\
\midrule
Observations          &   555   &  555  &  530         \\
\bottomrule
\end{tabular}

\begin{footnotesize}
\newline
*P$<$0.1, **P$<$0.05, ***P$<$0.01
\newline
Note: Robust standard errors clustered at the group level. Marginal effects of Tobit Model reported.
\newline
The dependent variable is the contributions toward the group in public goods provision in Part II. \end{footnotesize}
\end{table}

    \end{center}
\end{table}





\hspace  *{0mm} Finally, I look at how incumbents’ contributions are impacted conditional on the identity of the incumbent and the newcomer. First, participants are classified by whether they are whites or minorities. Consequently, incumbents are grouped by whether they share identities with the newcomers. Incumbents of the same gender as the newcomers are considered as sharing gender identity. Non-white Minorities are considered to share the same racial identity as other minorities and whites are considered as sharing the same racial identity. This results in 4 groups of incumbents – those that share race and gender with the newcomer, those that share race but not gender with the newcomer, incumbents that gender but not race with the newcomer and incumbents that do not share race and gender with the newcomer. The tobit regression model specified in equation 9 above in the \hyperref[subsec:Specification]{empirical specification} section is used to estimate cooperation among incumbents in the public goods provision. Table 12 presents the results of the regression estimates. Column 1 presents the base model without controls and considers incumbents that do not share race and gender with the newcomer as the omitted category. It is evident that incumbents do not vary their cooperation conditional on the identities of the newcomer. The evidence shows that while incumbent individuals are on average more cooperative when they share the same race with a newcomer but not gender as compared to incumbents that do not share both race and gender, the coefficient is not statistically significant ($P-value=0.72$).

On the contrary, the average cooperation in the public goods provision activity is negative among incumbents that share the same gender as the newcomer but not the same race. However, the effect is also not statistically significant as compared to the omitted group ($P-value=0.57$). Finally, there is no statistically significant difference in the cooperation of incumbents that do not share the same race and gender and incumbents that do share both identities with the newcomer ($P-value=0.99$). These findings lead me to my fifth primary result:

\textbf{Result 5a: The identity of the newcomer does not impact the cooperation of incumbent members of the team.}
\begin{table}[H]
 \captionsetup{justification=raggedright,singlelinecheck=false}
\caption{Incumbent Cooperation and Newcomer Identity } \label{tab:table7}
    \begin{center}
        \begin{table}[htbp]
    \begin{tabular}{c c c c c}
    \toprule
    \textbf{Variables} & \textbf{(1)} & \textbf{(2)} & \textbf{(3)}      & \textbf{(4)}     \\ 
\midrule
Congruentgender           &     -4.20              &     -3.61          &  -2.88            &  -3.24 \\
                                            &     (7.41)             &     (7.57)         &  (7.55)           &  (8.33) \\
\addlinespace
Congruentrace           &      2.68              &     4.17           &   3.30            &  3.35   \\
                                            &      (6.44)            &     (6.76)         &   (6.80)          &  (7.27) \\
\addlinespace
Congruentboth                 &      -0.06              &   -0.22           &  0.47             &  0.17  \\
                                            &      (6.92)            &    (7.07)          &  (7.03)           &  (7.24) \\
\addlinespace
Positive                         &                     &                     &                    &  1.02  \\
                                 &                     &                     &                    &  (4.67)  \\
\addlinespace
Negative                         &                     &                     &                    &  -5.41  \\
                                 &                     &                     &                    &  (3.53)  \\
\addlinespace
Engagement                       &                     &                     &                    &  9.68\sym{**}  \\
                                 &                     &                     &                    &  (3.70)  \\

\midrule
Other Controls                   &    No                &    Yes              &    Yes          &    Yes        \\
Status Controls                    &    No                &    No               &    Yes          &    Yes        \\
Chat                             &    No                &    No               &    No           &    Yes        \\
\midrule
Number of Participants           &    333               &    333              &    333          &    318       \\
\midrule
Observations                     &       1665           &       1665          &  1665           &  1590    \\
\bottomrule


\end{tabular}

\begin{footnotesize}
\newline
*P$<$0.1, **P$<$0.05, ***P$<$0.01
\newline
Note: Robust standard errors clustered at the group level. Marginal effects of Tobit Model reported. 
\newline
The dependent variable is the contributions toward the group in public goods provision in Part II. \end{footnotesize}
\end{table}
    \end{center}
\end{table}

\subsection{Coordination in the Minimum Effort Activity}

\noindent To explore coordination choices in the study, I analyze the participants' hours contributed to the group activity in the minimum effort game. At the individual level, participants are tasked with selecting their preferred coordination choice in a group activity. The choices range from 10 hours to 70 hours. There are multiple equilibria in this activity. Team members coordinating at a higher level is pareto optimal and is beneficial to the team. The choices are analyzed as coordination behavior in the team. In each round, participants must determine the number of hours they are willing to dedicate to the group activity. Importantly, participants make their decisions with the knowledge that their individual payoffs are contingent upon the choices made by other team members. The determination of individual payoffs is calculated using equation 2 specified above in the \hyperref[subsec:Design]{experimental design} section.

\noindent\textbf{\textit{Gender Diversity}} 

\noindent Similar to the analysis of the cooperation of individuals in the teams, I initiate my analysis of team coordination by first examining incumbent teams. Incumbent teams (three-person teams) are classified by levels of diversity. Teams are classified based on gender diversity, racial diversity or both. As stated previously, gender diverse teams are non-homogeneous gender teams that have at least two individuals with different gender identities. I begin the analysis by looking at the impact of the level of gender diversity of the incumbent team on behavior across gender using the tobit regression model specified in equation 3. The results of the analyses are presented in table 13 below. Column 1 shows the base model and the omitted group is a man in a homogeneous gender team. The findings show that coordination choices of men do not vary by the level of gender diversity in the team ($P-value=0.90$). The situation is different for non-male identifying individuals. They coordinate at choices that are on average 12.2\% of the total hours possible lower than men in the homogeneous gender team ($P-value<0.05$). Furthermore, non-male identifying individuals in the diverse gender teams coordinate at choices that are 30.5\% lower than men in the homogeneous gender teams ($P-value<0.01$) and even lower compared to men in the gender diverse teams ($P-value<0.01$). The results are robust to the addition of controls in column 2. Additional controls for interactions among participants show a decrease in the coefficients but the findings are still robust. It is also evident that positive interactions with the team increase coordination choices ($P-value<0.1$). 

\begin{table}[H]
 \captionsetup{justification=raggedright,singlelinecheck=false}
\caption{Incumbent Team Diversity and Coordination by Gender } \label{tab:table8}
    \begin{center}
        \begin{table}[htbp]
    \begin{tabular}{c c c c}
    \toprule
    \textbf{Variables} & \textbf{(1)} & \textbf{(2)} & \textbf{(3)}         \\ 
\midrule
Diverse$-$Gender            &     0.42    &    0.35                               &  -0.25  \\
                            &     (3.39)  &    (3.32)                             &  (3.68) \\

Notmale                  &      -8.55\sym{**} &    -8.86\sym{**}             &  -6.87\sym{*}  \\
                            &      (3.69)         &    (3.67)                     &  (3.77) \\
Diverse*Notmale         &      -13.20\sym{***}   &  -13.93\sym{***}            &  -12.00\sym{**}\\
                            &      (4.90)         &    (4.90)                     &  (5.10) \\


Positive Chat                        &                     &                       &  7.19\sym{*}  \\
                                 &                     &                           &  (4.08)  \\
Negative  Chat                       &                     &                       &  -0.82  \\
                                 &                     &                           &  (2.05)  \\
Engagement Chat                      &                     &                       &  -3.48  \\
                                 &                     &                           &  (2.69)  \\
\midrule
Other Controls                   &    No               &    Yes                    &    Yes        \\
Status Controls                    &    No               &    No                   &    Yes        \\
Chat                             &    No               &    No                     &    Yes        \\
\midrule
Number of Participants           &    333               &    333                   &    324        \\
\midrule
Observations                     &       1665          &       1665                &  1620    \\
\bottomrule
\end{tabular}
\begin{footnotesize}
\newline
*P$<$0.1, **P$<$0.05, ***P$<$0.01
\newline
Note: Robust standard errors clustered at the group level. Marginal effects of Tobit Model reported.
\newline
The dependent variable is the hours toward the group activity in the minimum effort activity in Part I. 
\end{footnotesize}
\end{table}
    \end{center}
\end{table}

\noindent\textbf{\textit{Racial Diversity}} 

\noindent Now, I consider how the level of racial diversity of the incumbent team impact behavior across race using equation 4 above in the \hyperref[subsec:Specification]{empirical specification} section. Individuals are grouped as white or non-white (minorities) based on their self-identified race. The results of the analysis are presented in table 14 below. Column 1 shows the base model and the omitted group is a white individual in the homogeneous racial team. The findings show that coordination choices of whites in the racially diverse team are higher than whites in the racially homogeneous team ($P-value<0.05$). The addition of controls in column 2 do not affect the significance of the result. Furthermore, non-white individuals coordinate at lower choices in  the homogeneous racial teams than whites in the homogeneous racial teams ($P-value<0.01$). Similarly, non-white individuals in racially diverse teams coordinate at a level that is 41.4\% of the total hours possible lower than whites in the homogeneous racial teams ($P-value<0.01$). Further controls for age, major, education level and political affiliation do not affect the statistical significance of the results. Further controls for interactions among teams members show that positive chat is important for higher coordination choices ($P-value<0.1$). However, the difference in coordination between whites in the diverse racial groups and whites in the homogeneous racial groups becomes statistically insignificant ($P-value=0.26$). 

\begin{table}[H]
 \captionsetup{justification=raggedright,singlelinecheck=false}
\caption{Incumbent Team Diversity and Coordination by Race } \label{tab:table8}
    \begin{center}
        \begin{table}[htbp]
    \begin{tabular}{c c c c}
    \toprule
    \textbf{Variables} & \textbf{(1)} & \textbf{(2)} & \textbf{(3)}       \\ 
\midrule
Diverse$-$Race        &     6.44\sym{**}                   &    5.62\sym{*}                   &  4.13 \\
                      &     (3.13)                 &     (3.01)                &  (3.65) \\
Notwhite           &      -9.80\sym{***}     &    -8.13\sym{***}      &  -8.52\sym{***}     \\
                      &      (2.77)              &    (3.06)                   &  (3.45) \\
Diverse$-$Race*Notwhite  &      -12.73\sym{***}   &  -12.36\sym{***}      &  -12.90\sym{***}\\
                            &      (3.20)               &    (3.47)              &  (3.92) \\

Positive Chat                        &                     &                       &  7.42\sym{*}  \\
                                 &                     &                           &  (4.33)  \\
Negative  Chat                       &                     &                       &  -0.64  \\
                                 &                     &                           &  (2.14)  \\
Engagement Chat                      &                     &                       &  -3.69  \\
                                 &                     &                           &  (2.74)  \\
\midrule
Other Controls                   &    No               &    Yes                    &    Yes        \\
Status Controls                    &    No               &    No                   &    Yes        \\
Chat                             &    No               &    No                     &    Yes        \\
\midrule
Number of Participants           &    333               &    333                   &    324        \\
\midrule
Observations                     &       1665          &       1665                &  1620    \\
\bottomrule

\end{tabular}
\begin{footnotesize}
\newline
*P$<$0.1, **P$<$0.05, ***P$<$0.01
\newline
Note: Robust standard errors clustered at the group level. Marginal effects of Tobit Model reported. 
\newline
The dependent variable is the hours toward the group activity in the minimum effort activity in Part I.\end{footnotesize}
\end{table}
    \end{center}
\end{table}

 \noindent\textbf{\textit{Gender and Racial Diversity}}

\noindent Finally, I consider the impact of the level of racial and gender diversity of the incumbent team on behavior across race and gender identities of the individual. Teams are classified based on whether there is moderate diversity - at least one dimension of diversity on gender or race, least diverse - no heterogeneity in gender and race and most diverse - there is heterogeneity in race and gender. Individuals are also grouped into four types - white males, non-male identifying whites, non-white males and individuals who are not white and non-male identifying. The regression model specified in equation 5 from above in the \hyperref[subsec:Specification]{empirical specification} section is estimated to analyze the coordination choices of participants. The results of the analysis are presented in table 15 below. Column 1 shows the base model. The omitted group is a white man in the least diverse incumbent team. There is no statistically significant difference between the coordination choices of white men in the least diverse team and white men in the more diverse teams ($P-value=0.70$ for moderately diverse and $P-value=0.32$ for the most diverse). However, other individuals in the least diverse team do coordinate at choices different from those of white men. Non-male identifying whites coordinate at 10.9\% of the total hours possible lower than white men in the least teams ($P-value<0.1$), while non-white males and non-white individuals who are not men coordinate at choices that is 13.7\% ($P-value<0.1$) and 20.0\%($P-value<0.01$) respectively lower than white men in the least diverse teams. Additionally, non-male identifying whites in the most diverse teams coordinate at choices that are 22.8\% of the total hours possible lower than white men in the least diverse team ($P-value<0.1$). The effect is robust to the addition of controls in columns 2 and 3 in table 15. However, positive chat among incumbents increases average coordination choices. These findings collectively lead to my next result:

\textbf{Result 1b: Individuals who are racial minorities coordinate at lower levels in the more diverse teams than whites in the least diverse teams.}

% \begin{landscape}
\begin{table}[H]
 \captionsetup{justification=raggedright,singlelinecheck=false}
\caption{Incumbent Team Diversity and Coordination by Individual Identity } \label{tab:table8}
    \begin{center}
        \begin{table}[htbp]
    \begin{tabular}{c c c c}
    \toprule
    \textbf{Variables} & \textbf{(1)} & \textbf{(2)} & \textbf{(3)}       \\ 
\midrule
Moderately$-$Diverse        &     1.42            &    0.80                 &  -2.28     \\
                            &     (3.63)          &     (3.71)             &  (4.23) \\
Most$-$Diverse              &     4.26             &     3.95                & 2.37   \\
                            &    (4.25)         &     (4.38)           &  (4.74)  \\
WhiteNotmale          &    -7.65\sym{*}        &    -7.75\sym{*}    &  -5.35  \\
                            &      (4.15)         &    (4.21)             & (4.18)  \\
NonwhiteMale         &       -9.59\sym{*}   &  -7.05                &  -6.79   \\
                            &      (5.55)         &    (5.77)             &  (6.32) \\
NonwhiteNotmale    &  -14.05\sym{***}    &  -13.75\sym{***}      &  -13.29\sym{***} \\
                            &      (3.97)         &    (4.43)              &   (4.54) \\
Moderate*WhiteNotmale    &   -5.88          & -5.94               &  -3.86  \\
                                 &  (7.40)        &   (7.38)             &  (7.18)  \\
Moderate*NonwhiteMale      &   -2.00       &  -4.01                &  -3.81  \\
                                 &  (10.20)       &     (9.92)             &  (10.77)   \\
Moderate*NonwhiteNotmale  &  -19.30\sym{***} & -18.77\sym{***}     &  -18.93\sym{**} \\
                                 &  (6.98)        &     (6.94)             &  (7.31)   \\
Most*WhiteNotmale          &  -12.57\sym{*} &     -13.96\sym{***}     &  -13.13\sym{*}   \\
                                 &  (7.86)        &     (7.96)             &  (8.22)  \\
Most*NonwhiteMale         &   -9.40        &     -12.96\sym{***}     &   -13.79  \\
                                 &  (10.47)        &     (10.30)             &  (12.35) \\
Most*NonwhiteNotmale    &   -28.31\sym{***} &   -28.05\sym{***}     &  -32.38\sym{***}  \\
                                 &  (7.75)        &     (7.90)             &  (8.86)   \\
Positive Chat                        &                     &                       &  7.71\sym{*}  \\
                                 &                     &                           &  (4.20)  \\
Negative  Chat                       &                     &                       &  -1.42  \\
                                 &                     &                           &  (2.25)  \\
Engagement Chat                      &                     &                       &  -3.97  \\
                                 &                     &                           &  (2.64)  \\
\midrule
Other Controls                   &    No               &    Yes                    &    Yes        \\
Status Controls                    &    No               &    No                   &    Yes        \\
Chat                             &    No               &    No                     &    Yes        \\
\midrule
Number of Participants           &    333               &    333                   &    324        \\
\midrule
Observations                     &       1665          &       1665                &  1620    \\
\bottomrule

\end{tabular}
\begin{footnotesize}
\newline
*P$<$0.1, **P$<$0.05, ***P$<$0.01
\newline
Note: Robust standard errors clustered at the group level. Marginal effects of Tobit Model reported. 
\newline
The dependent variable is the hours toward the group activity in the minimum effort activity in Part I.\end{footnotesize}
\end{table}
    \end{center}
\end{table}
% \end{landscape}

\hspace  *{0mm} Next, I compare the overall coordination choices between incumbents and newcomers in the entire study. As previously, I plot coordination choices by the levels of racial and ethnic diversity in figure 3 below. The least diverse teams are homogeneous racial and gender teams. Most diverse teams are teams that have heterogeneous race and gender and moderately diverse teams have heterogeneous racial or heterogeneous gender compositions. Notably, newcomers demonstrate a significantly lower level of coordination compared to incumbents during the study in all the teams. The results of the estimation of the tobit model specified in equation 6a in table 16 further emphasizes this. The base model shows that male newcomers coordinate at choices that is 15.9\% of the total hours possible lower than incumbent men - the omitted group ($P-value<0.01$). However, there is no statistically significant difference in coordination choices of men and non-male identifying individuals in the homogeneous gender team ($P-value=0.23$). Furthermore, the lower coordination choices of newcomers are robust to the inclusion of additional controls in columns 2 and 3 of table 16 below.  

Using the specification stated in equation 6b in the specification section above, I present the results of coordination choices by racial identity of the individual and racial composition of the group in table 17. The base model is a white incumbent. The results in column 1 show that not only do newcomers coordinate at lower choices but also choices vary by race. Non-white individuals in the incumbent teams coordinate at choices that is 8.3\% less than white incumbents ($P-value<0.01$). Furthermore, white newcomers coordinate at choices that is 15.3\% of the total possible hours lower than incumbent whites. However, non-white newcomers do not coordinate at a level different from white newcomers ($P-value=0.45$). The results of these findings lead to my next result: 

\textbf{Result 2b: Incumbents make higher coordination choices than newcomers but choices vary by race.}

\begin{figure}[H]
\captionsetup{justification=raggedright,singlelinecheck=false}
\caption{Incumbent Versus Newcomer Overall Coordination Choices}
\includegraphics[scale=0.2]{Figures/Overall_me_new_inc.png} 
\end{figure}

\begin{table}[H]
 \captionsetup{justification=raggedright,singlelinecheck=false}
\caption{Incumbent Versus Newcomer Overall Coordination by Gender} \label{tab:table9}
    \begin{center}
        \begin{table}[htbp]
    \begin{tabular}{c c c c}
    \toprule
    \textbf{Variables} & \textbf{(1)} & \textbf{(2)} & \textbf{(3)}       \\ 
\midrule
Newcomer                         &     -11.10\sym{***} &     -10.69\sym{***}  &  -3.26   \\
                                 &     (2.45)          &     (2.44)          &  (3.06)             \\
Notmale                         &    -3.34            &     -3.88   &  -6.50     \\
                                 &     (2.75)          &     (2.73)          &  (3.13)             \\
Newcomer*Notmale               &     5.71            &  4.34               &  3.80    \\
                                 &     (3.76)          &   (3.84)          &  (5.30)             \\


\midrule
Other Controls                   &    No               &    Yes              &    Yes            \\
Status Controls                    &    No               &    No               &    Yes           \\
Chat                             &    No               &    No               &    No             \\
\midrule
Number of Participants           &    444               &    444              &    444               \\
\midrule
Observations                     &       4440          &       4440          &  4440            \\
\bottomrule

\end{tabular}
\begin{footnotesize}
\newline
*P$<$0.1, **P$<$0.05, ***P$<$0.01
\newline
Note: Robust standard errors clustered at the group level. Marginal effects of Tobit Model reported. 
\newline
The dependent variable is the hours toward the group activity in the minimum effort activity in both Parts. \end{footnotesize}
\end{table}

    \end{center}
\end{table}


\begin{table}[H]
 \captionsetup{justification=raggedright,singlelinecheck=false}
\caption{Incumbent Versus Newcomer Overall Coordination by Race} \label{tab:table9}
    \begin{center}
        \begin{table}[htbp]
    \begin{tabular}{c c c c}
    \toprule
    \textbf{Variables} & \textbf{(1)} & \textbf{(2)} & \textbf{(3)}     \\ 
\midrule
Newcomer                         &     -10.74\sym{***} &     -10.50\sym{***}  &  -1.27       \\
                                 &     (2.54)          &     (2.49)          &  (3.19)             \\
Notwhite                         &  -5.84\sym{***} &     -4.56\sym{*}  &  -6.60\sym{**}   \\
                                 &     (2.16)          &     (2.37)          &  (2.77)             \\
Newcomer*Notwhite              &     -2.79              &     -1.42   &  -4.68   \\
                                 &     (3.71)          &     (3.90)          &  (4.95)             \\


\midrule
Other Controls                   &    No               &    Yes              &    Yes            \\
Status Controls                    &    No               &    No               &    Yes           \\
Chat                             &    No               &    No               &    No             \\
\midrule
Number of Participants           &    444               &    444              &    444               \\
\midrule
Observations                     &       4440          &       4440          &  4440            \\
\bottomrule

\end{tabular}
\begin{footnotesize}
\newline
*P$<$0.1, **P$<$0.05, ***P$<$0.01
\newline
Note: Robust standard errors clustered at the group level. Marginal effects of Tobit Model reported. 
\newline
The dependent variable is the hours toward the group activity in the minimum effort activity in both Parts. 
\end{footnotesize}
\end{table}

    \end{center}
\end{table}


\hspace  *{0mm} Building on the aforementioned findings, I  delve deeper into the coordination choices of incumbents and newcomers. Table 18 compares coordination before and after the newcomer joins the team based on the model specified in equation 7 above in the \hyperref[subsec:Specification]{empirical specification} section. The model considers incumbents in rounds 1 - 5 as the omitted group in the analyses. Interestingly, the coordination choices after group composition changes are lower for newcomers as compared to incumbents before changes in group composition ($P-value<0.01$). Similarly, incumbents decrease their coordination choices after group composition changes ($P-value<0.05$). However, similar to cooperation choices before joining the team, newcomers coordinate on lower levels as compared to the incumbents in rounds 1 to 5 ($P-value<0.01$). This pattern persists when controls for age, political affiliation and educational background are included in column 2. This consequential finding leads to my next result:

\textbf{Result 3b: Incumbents’ and newcomers’ coordination choices change after newcomers join the team.}

\begin{table}[H]
 \captionsetup{justification=raggedright,singlelinecheck=false}
\caption{Incumbent and Newcomer Coordination Before and After Group Composition Changes} \label{tab:table10}
    \begin{center}
        \begin{table}[htbp]
    \begin{tabular}{c c c c}
    \toprule
    \textbf{Variables} & \textbf{(1)} & \textbf{(2)} & \textbf{(3)}      \\ 
\midrule
Newcomer(Before=1)               &     -17.91\sym{***}    &    -17.33\sym{***}  &  -16.94\sym{***}   \\
                                 &     (2.43)             &     (2.38)          &  (2.79)         \\
\addlinespace
Newcomer(After=1)                &     -8.79\sym{***}     &     -8.18\sym{***}   &  -7.73          \\
                                 &     (2.88)             &     (2.86)           &  (3.26)         \\
\addlinespace
Incumbent(After=1)               &     -4.26\sym{**}      &     -4.27\sym{**}  &  -4.30\sym{**}     \\
                                 &     (1.70)             &     (1.70)         &  (1.70)         \\
\midrule
Other Controls                   &    No                  &    Yes              &    Yes             \\
Status Controls                    &    No                  &    No               &    Yes             \\
Chat                             &    No                  &    No               &    No               \\
\midrule
Number of Participants           &    444               &    444              &    444               \\
\midrule
Observations                     &       4440          &       4440          &  4440          \\
\bottomrule

\end{tabular}
\begin{footnotesize}
\newline
*P$<$0.1, **P$<$0.05, ***P$<$0.01
\newline
Note: Robust standard errors clustered at the group level. Marginal effects of Tobit Model reported. 
\newline
The dependent variable is the hours toward the group activity in the minimum effort activity in both Parts. 
\end{footnotesize}
\end{table}
    \end{center}
\end{table}

\hspace  *{0mm} My focus shifts to understanding the impact of the incumbent team diversity on the newcomer coordination choices after newcomers join the team. Similar to the analysis of cooperation in the public goods provision, incumbent teams are classified by the level of diversity existing in the team before changes to the group composition. I first consider how the level of gender and racial diversity of the incumbent teams affect newcomer coordination choices. As before, the most diverse teams are the ones that have heterogeneous gender and racial compositions. Moderately diverse teams have diverse racial composition or diverse gender composition. The findings presented in table 19 reflects the regression model presented in equation 8a above in the \hyperref[subsec:Specification]{empirical specification} section with the base group being newcomers that join teams with homogeneous racial and homogeneous gender compositions - the least diverse. The results shown in column 1 show that newcomers that join moderately diverse incumbent teams do not coordinate at levels different from newcomers in the least diverse teams ($P-value=0.89$). Similarly, the coordination choices of newcomers in the most diverse team is on average positive but is not statistically significant ($P-value=0.53$). Controlling for other covariates in columns 2 and 3 does not affect the magnitude and direction of the effects. 

Next, I consider how the gender diversity of a team impacts newcomer coordination choices by newcomer gender identity. The findings of the base model of the estimation of equation 8b is presented in table 20. The model considers male newcomers to the homogeneous gender team as the omitted group. The results show that men in diverse gender teams coordinate at higher choices but the effect is not statistically significant ($P-value=0.14$). Similarly, the coefficients for non-male identifying individuals in the homogeneous gender groups and non-male identifying individuals in the gender diverse teams are also not statistically significant despite non-male identifying individuals in both teams coordinating at levels that are higher than men in the homogeneous gender teams. 

Finally, I consider the effect of the team racial diversity on newcomer coordination by newcomer race in table 21. I utilize equation 8c specified above in the \hyperref[subsec:Specification]{empirical specification} section to examine how the level of racial diversity in the incumbent team affects newcomer coordination choices by newcomer race. Intriguingly, among white newcomers, there is no statistically significant difference between coordination choices in homogeneous racial groups and diverse racial groups ($P-value=0.58$). Furthermore, non-white newcomers in the homogeneous racial teams tend to coordinate at lower choices despite the coefficients being statistically insignificant ($P-value=0.41$). Addition of controls in columns 2 and 3 does not affect the direction and statistical significance of the coefficients. These findings collectively lead to the following result:

\textbf{Result 4b: The diversity of the existing team does not have a discernible impact on the coordination choices of newcomers after they join the team.}

\begin{table}[H]
 \captionsetup{justification=raggedright,singlelinecheck=false}
\caption{Incumbent Team Diversity and Newcomer Coordination } \label{tab:table11}
    \begin{center}
        \begin{table}[htbp]
    \begin{tabular}{c c c c}
    \toprule
    \textbf{Variables} & \textbf{(1)} & \textbf{(2)} & \textbf{(3)}    \\ 
\midrule
Moderate$-$Diverse  &  0.72         &  0.82            &  0.02     \\
                    &  (5.32)       &  (5.33)           &  (5.27)    \\
\addlinespace
Most$-$Diverse      &  3.93       &   4.94                &  4.90     \\
                    &  (6.30)      & (6.18)             &  (6.20)     \ \\

\addlinespace
Positive                 &           &                     &  6.13     \\
                         &           &                     &  (4.36)   \\
\addlinespace
Negative                 &           &                      &  -2.63  \\
                          &          &                      &  (2.38)    \\
\addlinespace
Engagement               &          &                        & -0.70    \\
                         &          &                          &  (3.52)   \\

\midrule
Other Controls   &   No &  Yes &    Yes    \\
Status Controls &   No  &    No    &    Yes    \\
Chat          &    No    &    No   &    Yes          \\
\midrule
Number of Participants & 111   &    111 &    106    \\
\midrule
Observations          &   555   &  555  &  530         \\
\bottomrule

\end{tabular}

\begin{footnotesize}
\newline
*P$<$0.1, **P$<$0.05, ***P$<$0.01
\newline
Note: Robust standard errors clustered at the group level. Marginal effects of Tobit Model reported. 
\newline
The dependent variable is the hours toward the group activity in the minimum effort activity in Part II. \end{footnotesize}
\end{table}

    \end{center}
\end{table}

\begin{table}[H]
 \captionsetup{justification=raggedright,singlelinecheck=false}
\caption{Incumbent Team Diversity and Newcomer Coordination by Gender } \label{tab:table11}
    \begin{center}
        \begin{table}[htbp]
    \begin{tabular}{c c c c}
    \toprule
    \textbf{Variables} & \textbf{(1)} & \textbf{(2)} & \textbf{(3)}       \\ 
\midrule
Diverse$-$Gender            &     7.12    &    6.98                               &  6.54  \\
                            &     (4.80)  &    (4.71)                             &  (4.74) \\

Notmale                  &      1.95  &    -0.34                            &  -8.08  \\
                            &      (4.63)  &    (4.85)                        &  (4.93) \\
Diverse*Notmale         &      4.47      &  5.61                            &  -4.70 \\
                            &      (7.76)  &    (7.63)                     &  (7.48) \\


Positive Chat                        &                     &                       &  7.49  \\
                                 &                     &                           &  (4.14)  \\
Negative  Chat                       &                     &                       &  -3.54  \\
                                 &                     &                           &  (2.46)  \\
Engagement Chat                      &                     &                       &  -1.09  \\
                                 &                     &                           &  (3.46)  \\
\midrule
Other Controls   &   No &  Yes &    Yes    \\
Status Controls &   No  &    No    &    Yes    \\
Chat          &    No    &    No   &    Yes          \\
\midrule
Number of Participants & 111   &    111 &    106    \\
\midrule
Observations          &   555   &  555  &  530         \\
\bottomrule

\end{tabular}
\begin{footnotesize}
\newline
*P$<$0.1, **P$<$0.05, ***P$<$0.01
\newline
Note: Robust standard errors clustered at the group level. Marginal effects of Tobit Model reported. 
\newline
The dependent variable is the hours toward the group activity in the minimum effort activity in Part II. 
\end{footnotesize}
\end{table}

    \end{center}
\end{table}

\begin{table}[H]
 \captionsetup{justification=raggedright,singlelinecheck=false}
\caption{Incumbent Team Diversity and Newcomer Coordination by Race } \label{tab:table11}
    \begin{center}
        \begin{table}[htbp]
    \begin{tabular}{c c c c}
    \toprule
    \textbf{Variables} & \textbf{(1)} & \textbf{(2)} & \textbf{(3)}      \\ 
\midrule
Diverse$-$Race              &     -2.62          &    -1.97                               &  -0.49  \\
                            &   (4.76)           &    (4.72)                             &  (4.67) \\

Notwhite                 &      -3.79       &    -4.09                       &  -4.78   \\
                            &      (4.56)          &    (5.07)                        &  (4.91) \\
Diverse*Notwhite         &      -6.73          &  -6.81                       &  -5.31  \\
                            &      (5.75)         &  (6.28)                     &  (6.43) \\


Positive Chat                        &                     &                       &  5.91  \\
                                 &                     &                           &  (4.46)  \\
Negative  Chat                       &                     &                       &  -2.92  \\
                                 &                     &                           &  (2.39)  \\
Engagement Chat                      &                     &                       &  -0.66  \\
                                 &                     &                           &  (3.48)  \\
\midrule
Other Controls   &   No &  Yes &    Yes    \\
Status Controls &   No  &    No    &    Yes    \\
Chat          &    No    &    No   &    Yes          \\
\midrule
Number of Participants & 111   &    111 &    106    \\
\midrule
Observations          &   555   &  555  &  530         \\
\bottomrule

\end{tabular}

\begin{footnotesize}
\newline
*P$<$0.1, **P$<$0.05, ***P$<$0.01
\newline
Note: Robust standard errors clustered at the group level. Marginal effects of Tobit Model reported.
\newline
The dependent variable is the hours toward the group activity in the minimum effort activity in Part II. 
\end{footnotesize}
\end{table}
    \end{center}
\end{table}



\hspace  *{0mm} Finally, I look at how the identity of the newcomers affects incumbents’ coordination choices. I employ the procedure previously used in the analysis of the cooperation of participants. Shared identity is defined based on congruence between the identity of an incumbent and the newcomer. Racial identity is categorized into white and non-whites (minorities). Incumbents are then classified based on similarity to the newcomer regarding racial identity and gender identity. In table 22, I report the results of the tobit regression of the coordination choices of the incumbents. The base treatment group consists of incumbents that do not share both racial and gender identities with the newcomer. The results are based on the estimation of equation 9 in the \hyperref[subsec:Specification]{empirical specification} section. The findings in column 1 show the base model. It is evident that there is no statistically significant difference in the incumbents’ coordination choices by identities of the newcomer. There is no difference in coordination between incumbents that share gender identity but not race with the newcomer and the incumbents that do not share identities with the newcomer ($P-value=0.74$). Similarly, incumbents that share race and gender identities with the newcomer do not coordinate differently from incumbents that do not share both identities ($P-value=0.75$). Furthermore, the difference between the coordination choices of incumbents that do not share both racial and gender identities with the newcomer and the incumbents that share only racial identity with the newcomer is not statistically significant ($P-value=0.84$). It is evident that regardless of the identity of the newcomer, the average coordination choices of the existing team members are not impacted. These findings lead to my final result: 

\textbf{Result 5b: The incumbent members of the team do not vary their coordination choices in the minimum effort activity regardless of the newcomer's identity.}

\begin{table}[H]
 \captionsetup{justification=raggedright,singlelinecheck=false}
\caption{ Incumbent Coordination and Newcomer Identity } \label{tab:table12}
    \begin{center}
        \begin{table}[htbp]
    \begin{tabular}{c c c c c}
    \toprule
    \textbf{Variables} & \textbf{(1)} & \textbf{(2)} & \textbf{(3)}      & \textbf{(4)}     \\ 
\midrule
Congruentgender            &     -0.15              &     -0.71          &  -0.35            &  -2.02 \\
                                            &     (3.80)             &     (3.61)         &  (3.55)           &  (3.62) \\
\addlinespace
CongruentRace           &      2.06              &     2.05           &   1.99            &  0.52   \\
                                            &      (3.29)            &     (3.38)         &   (3.46)          &  (3.57) \\
\addlinespace
Congruentboth                &      1.21              &     -0.03           &  0.25                & -1.45  \\
                                            &      (3.71)             &    (3.54)          &  (3.70)            &  (3.65) \\
\addlinespace
Positive                         &                     &                     &              & -0.86  \\
                                 &                     &                     &              &  (2.14)  \\
\addlinespace
Negative                         &                     &                     &            &  -3.86\sym{**}  \\
                                 &           &                     &                    &  (1.83)  \\
\addlinespace
Engagement                       &                     &     &                    &  2.38\sym{*}  \\
                                 &                     &   &                    &  (1.45)  \\


\midrule
Other Controls                   &    No                &    Yes              &    Yes          &    Yes        \\
Status Controls                    &    No                &    No               &    Yes          &    Yes        \\
Chat                             &    No                &    No               &    No           &    Yes        \\
\midrule
Number of Participants           &    333               &    333              &    333          &    318       \\
\midrule
Observations                     &       1665           &       1665          &  1665           &  1590    \\
\bottomrule
 
\end{tabular}

\begin{footnotesize}
\newline
*P$<$0.1, **P$<$0.05, ***P$<$0.01
\newline
Note: Robust standard errors clustered at the group level. Marginal effects of Tobit Model reported. 
\newline
The dependent variable is the hours toward the group activity in the minimum effort activity in Part II.\end{footnotesize}
\end{table}
    \end{center}
\end{table}

\section{Conclusion} \label{sec:Conclusion}

The changing demographic composition of the US labor force signals an increasing diversity within teams, promising novel experiences for individuals across various team settings. This paper significantly contributes to the literature by investigating the impact of race and gender identity on team cooperation and coordination in dynamic settings, particularly when introducing newcomers to incumbent teams to examine economic decision making.  This study adopts an experimental economics approach to uncover compelling results with significant implications for team decisions in cooperation and coordination decisions. 

Newcomers are exogenously assigned to teams with varying degrees of diversity, emphasizing the salience of race and gender identities in the experimental environment. The analysis centers on incumbent teams, distinguishing between gender and racial diversity. Notably, in gender-homogeneous teams, men exhibit higher cooperation than non-male identifying individuals, who contribute less. Racially diverse teams witness increased cooperation from white individuals but decreased contributions from non-white individuals. When considering the gender and racial identities of individuals, white men in more diverse teams cooperate more, while non-male identifying and non-white individuals in less diverse teams contribute less.

A key finding is that incumbents consistently demonstrate higher cooperation than newcomers, with newcomers exhibiting lower cooperation rates equivalent to 15\% less than incumbents in the initial phase of the study. However, cooperation levels between incumbents and newcomers converge after team composition changes, achieving parity. The study identifies that newcomers' cooperation varies based on team diversity, with increased cooperation observed in moderately diverse teams. Importantly, the identity of newcomers does not significantly impact the cooperation choices of incumbent team members.

In the realm of coordination, men in gender-diverse teams show no significant difference compared to those in homogeneous gender teams, while non-male identifying individuals coordinate less. In racially diverse teams, whites exhibit higher coordination choices than non-white individuals. The combined analysis of race and gender diversity reveals that non-white, non-male identifying individuals coordinate at significantly lower levels in more diverse teams. However, no significant variation is found in incumbent coordination choices based on shared racial or gender identity with the newcomer.  

In conclusion, this study sheds light on the complexity of cooperation and coordination choices in diverse teams, with implications for both incumbents and newcomers. These results, while complementing other findings in economic literature, underscore the importance of fostering team identity, especially in diverse teams. Through the application of social dilemma activities, valuable insights are provided into the dynamics of team cooperation and coordination within distinct contexts, emphasizing the impact of dynamic team composition, a frequent occurrence in various organizations.

Importantly, the study recognizes the limitations of its findings, applicable within an environment where specific demographics are a majority, and diverse individuals are expected to work in teams. Furthermore, the findings raise an intriguing, unresolved question, prompting a forthcoming study. Given the discovered influence of team composition on individual behaviors, it becomes imperative to consider environmental factors. The next phase of exploration will delve into how the environment shapes the cooperative and coordination choices of individuals, particularly examining the influence of physical interactions among team members. These future investigations aim to enhance our understanding of the nuanced dynamics of team decision-making in diverse settings.

\bibliography{mybiblio.bib}

\section{Online Appendix}
 
\subsection{Appendix A : Additional Analysis of Newcomer } \label{sec:App}
I provide further analysis of what is driving the behavior of newcomers once they join a team. As shown earlier in the analysis, newcomers to moderately diverse teams are more cooperative than newcomers in the least diverse team. Here, I consider a different specification to understand whether newcomers' behavior is driven by having individuals in the existing team that look like them. I devise a classification based on whether there exists similar individuals in the team like them in the new team. In this specification, newcomers that share gender with at least 1 member of the incumbent team are considered to share gender with the team. A similar approach is taken for race.

First, I utilize equation 10 below to examine how sharing gender with at least 1 incumbent member of the team impacts actions of a newcomer. 

\begin{center}
 $Y_{i}=\beta_1Gen$-$Match_{i}+\beta_2Notmale_{i}+\beta_3Gender$-$Match_{i}*Notmale_{i}+\theta X_{i}+\epsilon_{i}$\space (10)
\end{center}


\noindent where $Y_{i}$ is the contributions of newcomer $i$ in either the public goods provision or the minimum effort activity. $Gen$-$Match_{i}$ is a dummy variable indicating whether newcomer $i$, the newcomer is in a group where they share gender with at least 1 incumbent member of the team. $Notmale_{i}$ is an indicator for newcomer gender. $Gen$-$Match_{i}*Notmale_{i}$ is an interaction of non-male identifying newcomer who has a gender match in the team. $X_{i}$ is a set of individual characteristics such as age, major, income, individual interactions with the group, parents' socioeconomic background and other personal characteristics. $\epsilon_{i}$ is the residual term. 

The results are presented in table 23 in the appendix. The first 3 columns show the results of newcomers' actions in the cooperation activity while the last 3 columns show the results of newcomers' actions in the coordination activity. Similar to earlier analysis, the first column presents the base model with male newcomers without gender match in the team being the omitted group.  The results in column 1 show that having homogeneity in terms of gender with at least 1 member of the incumbent team does not affect the cooperation of the newcomer overall. In fact, newcomers who join teams where an existing member shares gender with them on average cooperate less despite the effect being statistically insignificant among men ($P-value=0.41$). Additionally, the effect does not vary by gender of the newcomer ($P-value=0.92$). This result emphasizes further that the overall diversity of the team matters more than specific gender matches between team members. Additional controls in column 2 and column 3 do not affect the direction and the significance of the effects. The results of the analysis of the base model of coordination choices presented in column 4 shows coordination choices of newcomers are not affected by having an incumbent member of the team the newcomer shares gender identity with among men ($P-value=0.95$) in the base model. However, additional controls in the third column of coordination choices show that at the 10\% significance, positive conversation with the incumbent team increases coordination choices of the newcomer. Furthermore, non-male individuals coordinate at lower levels than men in teams where they do not share gender with any incumbent ($P-value<0.1$). Non-male identifying individuals who join teams where they share gender with at least 1 member of the team also coordinate at choices lower than men in teams where they do not have a gender match ($P-value<0.1$).

\begin{table}[H]
 \captionsetup{justification=raggedright,singlelinecheck=false}
\caption{Newcomer Decisions: Gender Match} \label{tab:tablea1}
    \begin{center}
        \begin{table}[htbp]
    \begin{tabular}{c c c c c c c}
    \toprule
          & \multicolumn{3}{c}{Cooperation}   &\multicolumn{3}{c}{Coordination}     \\
\midrule
Gender$-$Match      &  -7.31     &  -9.97   &  -8.34     &    -0.28    &  -2.16         &  0.58   \\
                    &  (8.88)    &  (8.75)  &  (9.17)   &     (4.72)   &  (5.01)        &  (5.18)\\
\addlinespace
Notmale        &  -0.81      &   -4.82   &  -5.05    &   2.02   &   0.35             &  -8.50\sym{*}    \\
                  &  (8.48)     &    (8.18) &  (8.87)   &   (4.71) &   (4.96)           &  (5.07) \ \\
\addlinespace
Gender$-$Match*Notmale &  -5.32 & -10.12   & -12.31    &   2.30   &   0.84           &  -9.73\sym{*}    \\
                  &  (10.42)     &    (10.11) &  (11.00)   &  (5.93) & (6.03)           &  (6.19) \ \\
                        
\addlinespace
Positive  Chat    &           &           &  -1.22     &             &               &  7.91\sym{*} \\
                  &           &           &  (7.88)    &             &                &  (4.23)   \\
\addlinespace
Negative   Chat  &            &           &  -2.89    &               &                &  -3.80   \\
                 &            &           &  (4.86)   &               &                 &  (2.57)  \\
\addlinespace
Engagement  Chat   &         &            &   5.98     &              &                &  -0.92 \\
                   &         &            &  (4.33)   &               &                &  (3.31) \\

\midrule
Other Controls      &   No   &  Yes     &    Yes    &    No    &    Yes  &    Yes \\
Status Controls     &   No   &    No    &    Yes    &    No    &   No    &    Yes \\
Chat                &    No  &    No    &    Yes     &    No    &    No   &    Yes             \\
\midrule
Number of Participants & 111   &    111 &    106  &   62   &    62 &    58     \\
\midrule
Observations          &   555   &  555  &  530   &  310     &  310   &  290             \\
\bottomrule

\end{tabular}

\begin{footnotesize}
\newline
*P$<$0.1, **P$<$0.05, ***P$<$0.01
\newline
Note: Robust standard errors clustered at the group level. Marginal effects of Tobit Model reported. 
\newline
Columns 1 - 3: The dependent variable is the contributions toward the group in the public goods provision in Part II. 
\newline
Columns 4 - 6: The dependent variable is the hours toward the group activity in the minimum effort activity in Part II.
\end{footnotesize}
\end{table}
    \end{center}
\end{table}

Next, I consider the impact of having an individual that shares a newcomer's race on the newcomer's economic decisions using equation 11 below: 

\begin{center}
 $Y_{i} = \beta_1 Race$-$Match_{i} + \beta_2 Notwhite_{i} + \beta_3 Race$-$Match_{i}*Notwhite_{i} + \theta X_{i} + \epsilon_{i}$  \space (11)
\end{center}

\noindent where $Y_{i}$ is the contributions of participant $i$ in either the public goods provision or the minimum effort activity. $Race$-$Match_{i}$ is a dummy variable indicating whether participant $i$, the newcomer is in a group where they share race with at least 1 incumbent member of the team. $Notwhite_{i}$ is an indicator for participant race. $Race$-$Match_{i}*Notwhite_{i}$ is an interaction of non-white newcomer who joins a team where he/she shares race with a member of the team. $X_{i}$ is a set of individual characteristics such as age, major, income, individual interactions with the group, parents' socioeconomic background and other personal characteristics. $\epsilon_{i}$ is the residual term. 

The results of the effect of sharing race with at least 1 member of the incumbent team on newcomer economic decision$-$making are presented in table 24. The first three columns show cooperation choices and the last 3 columns show the results of newcomers in the coordination activity. Overall, cooperation is lower among non-white newcomers who join teams where there is no incumbent they share the same race with as compared to omitted group of white newcomers in teams where they do not share race with any member of the team ($P-value=0.29$).  However, the effect is statistically insignificant. The coefficient of newcomers in teams where they share race with at least an incumbent member is even lower though statistically insignificant ($P-value=0.17$). The coefficient attenuates and is statistically significant at the 5\% significance level once controls for major, age, income and minority status are included ($P-value=0.04$). The effect stays statistically significant and consistent when additional controls of risk aversion and interactions with the incumbent members of the team are controlled for ($P-value<0.05$). Interestingly, coordination choices are lower when they share race with incumbents in the base model in column 4 of table 24. Newcomers who share race with at least 1 member of the incumbent team they join actually coordinate at lower choices than those who join teams where they do not share race with anyone . Among white newcomers, the effect is 13.6\% of the maximum possible less than whites in teams where they do not share race with with anyone ($P-value<0.05$) while among non-whites, coordination is 48.4\% of the maximum possible less than whites in teams where they do not share race with with anyone ($P-value<0.1$). The coefficients are robust to the inclusion of additional controls in column 5.  


\begin{table}[H]
 \captionsetup{justification=raggedright,singlelinecheck=false}
\caption{Newcomer Decisions: Race Match} \label{tab:tablea1}
    \begin{center}
        \begin{table}[htbp]
    \begin{tabular}{c c c c c c c}
    \toprule
          & \multicolumn{3}{c}{Cooperation}   &\multicolumn{3}{c}{Coordination}     \\
\midrule
Race$-$Match      &  6.64                   &   8.80             &  4.95     &  -9.50\sym{**}  &  -12.35\sym{**}   &  -3.94   \\
                  &  (12.37)                &  (13.11)           &  (14.13)   &  (3.94)        &  (6.35)           &  (7.48) \\
\addlinespace
Notwhite        &  -12.23                &   -16.96         &  -18.07  &  -11.36\sym{**}  &   -12.80\sym{**}   &  -8.42     \\
                  &   (11.46)                &    (11.82)         &  (12.04)   &  (4.57)        &    (5.57)     &  (6.11) \\
\addlinespace
Race$-$Match*Notwhite & -19.39   & -29.19\sym{**}  & -26.98\sym{**} & -13.03\sym{*}  & -15.15\sym{**} & -17.06 \sym{**}     \\
                        & (14.38)            &  (14.00)      &  (13.58)   & (6.66)       &    (6.97)        &  (6.51) \\
                        
\addlinespace
Positive  Chat    &                        &           &  -3.01     &             &               &  6.63  \\
                  &                        &           &  (7.85)    &             &                &  (4.33)   \\
\addlinespace
Negative   Chat  &                        &           &  -2.42    &               &                &   -2.79    \\
                 &                        &           &  (4.66)   &               &                 &  (2.42)  \\
\addlinespace
Engagement  Chat   &                      &            &   6.29     &              &                &  -0.53 \\
                   &                     &             &  (4.30)   &               &                &  (3.28) \\

\midrule
Other Controls      &   No   &  Yes     &    Yes    &    No    &    Yes  &    Yes \\
Status Controls     &   No   &    No    &    Yes    &    No    &   No    &    Yes \\
Chat                &    No  &    No    &    Yes     &    No    &    No   &    Yes             \\
\midrule
Number of Participants & 111   &    111 &    106  &   62   &    62 &    58     \\
\midrule
Observations          &   555   &  555  &  530   &  310     &  310   &  290             \\
\bottomrule

\end{tabular}

\begin{footnotesize}
\newline
*P$<$0.1, **P$<$0.05, ***P$<$0.01
\newline
Note: Robust standard errors clustered at the group level. Marginal effects of Tobit Model reported.
\newline
Columns 1 - 3: The dependent variable is the contributions toward the group in the public goods provision in Part II. 
\newline
Columns 4 - 6: The dependent variable is the hours toward the group activity in the minimum effort activity in Part II. 
\end{footnotesize}
\end{table}
    \end{center}
\end{table}

Combining the findings from tables 23 and 24, it is evident that the effect driving the behavior of newcomers in moderately diverse teams is not based on matching identities with an individual. In fact, individuals sharing the same race cooperate less when they join homogeneous teams that they do not share gender with them. This is particularly true for white newcomers as shown in figure 11 of the appendix. White men and women cooperate at a statistically significantly lower level when they join all-white women groups and all-white men groups respectively. Their cooperation choices is however highest when they join moderately diverse teams where there is heterogeneous gender or heterogeneous racial compositions. In terms of coordination choices, individuals coordinate less when they share race with members of the incumbent, again emphasizing the fact the effect driven by indentity matches. 


\subsection{Appendix B : Tables} \label{sec:App}



% \begin{table}[H]
% \caption{Summary Statistics [Cooperation] - \texttt{Mean and Standard Deviation} } \label{tab:table3}
%     \begin{center}
%         \begin{table}
    \begin{tabular}{ccccccc}
    \toprule
        & Incumbents & Newcomer & Newcomer(1-5) & Newcomer(6-10) &  Incumbent(1-5) & Incumbent(6-10) \\
        \midrule
        All-Men      & 60.59 & 49.25  & 39.76    & 58.73   &    57.71 & 63.46\\
                    & (39.67) & (31.49) & (27.96)  & (32.08)  &  (37.60) & (41.50)\\
        \addlinespace
        All-Women &   55.84 &   47.52 & 42.50    &  52.55  &    54.92 &   56.76\\
                        & (35.24) & (33.51) & (32.52) & (33.81) &  (34.87) & (35.61)\\
        \addlinespace
        Mixed Gender     & 60.39 &  47.73 & 40.55 & 54.91  &    61.42 &    59.35 \\
                        & (37.08) &   (36.46) & (35.19)  & (36.36) &  (35.87) &  (38.25)\\
        \addlinespace
        Combined &    58.53 &  47.92 & 41.21   &  54.62  &    58.06&    59.02 \\
                        & (36.88) & (34.37) & (32.85) & (34.58) &  (35.88) &  (37.86) \\
    \bottomrule
    \end{tabular}
    \caption{Summary Statistics [Cooperation] - \texttt{Mean and Standard Deviation} }
    \label{tab:my_label}
\end{table}

%     \end{center}
% \end{table}

%  

\begin{table}[H]
 \captionsetup{justification=raggedright,singlelinecheck=false}
\caption{Categories of Chat Content } \label{tab:table4}
        \begin{table}[h]
\begin{tabular}{|l|p{8cm}|}
\hline
\textbf{Category} & \textbf{Description} \\
\hline
Frustration & Displayed frustration during the puzzle-solving process (1 = yes, 0 = no) \\
\hline
Confusion & Expressions of confusion related to the puzzle-solving process (1 = yes, 0 = no) \\
\hline
Talk in agreement & Conversation related to the puzzle-solving process - agreement with another participant (1 = yes, 0 = no) \\
\hline
Talk in disagreement & Conversation related to the puzzle-solving process - disagreement with another participant (1 = yes, 0 = no) \\
\hline
Confident & Was confident in their abilities to solve the puzzle (1 = not confident at all...5 = very confident) \\
\hline
Excitement & Expressions of excitement or satisfaction related to the puzzle-solving process (1 = yes, 0 = no) \\
\hline
Assertive & Was assertive in their communication with other participant(s) (1 = not assertive at all...5 = very assertive) \\
\hline
Comfortable & How did the participant's language manifest or what nonverbal cues did the participant exhibit? (1 = not showing this at all...5 = shows very clear signs of this) - Comfortability \\
\hline
\end{tabular}
\label{tab:categories}
\end{table}
\end{table}

\begin{table}[H]
 \captionsetup{justification=raggedright,singlelinecheck=false}
\caption{Summary Statistics  } \label{tab:table4}
    \begin{center}
        \begin{tabular}{l*{5}{c}}
\toprule
                & Incumbent & Newcomer & Newcomer(M) & Newcomer(W) &    Combined \\
\midrule
All-Men      & 54.45  & 44.2  &    41.45&    47.56&    51.89\\
                &(22.11) &(20.85)&  (22.09)&  (18.81)&  (22.24)\\
\addlinespace
All-Women&   53.25 &    45.96 &    45.32&    46.54&    51.43\\
                & (18.17) & (20.03)&  (20.05)&  (20.02)&  (18.91)\\
\addlinespace
Mixed Gender     &    53.49&      49.22&    45.36&    52.91&    52.42\\
                &   (17.95)&   (20.22)&  (22.19)&  (17.41)&  (20.06)\\
\addlinespace
Combined &    53.56&     46.96&    44.56&    49.32&    51.91\\
                &  (17.97)&  (20.33)&  (21.36)&  (19.00)&  (20.01)\\

\bottomrule
\end{tabular}

    \end{center}
\end{table}

 

% \begin{table}[H]
% \caption{Summary Statistics [Coordination] - \texttt{Mean and Standard Deviation} } \label{tab:table5}
%     \begin{center}
%         
\begin{tabular}{l*{6}{c}}
\toprule
                & Incumbent & Newcomer & Newcomer(1-5) & Newcomer(6-10) &  Incumbent(1-5) & Incumbent(6-10)\\
\midrule
All-Men      & 54.45  & 44.2  & 43.3   &   45.1  & 57.73 & 51.6 \\
                &(22.11) &(20.85)& (20.94) & (20.82)  &(19.82)&(23.87)\\
\addlinespace
All-Women&   53.25 &    45.96 &  43.39  &  48.52  &  54.26&   52.25  \\
                & (18.17) & (20.03)&  (19.84)&  (19.92)&  (17.05)& (19.19)\\
\addlinespace
Mixed Gender     &    53.49&      49.22  & 46.27  & 52.18  &    56.12&    50.86 \\
                &   (17.95)&   (20.22)&  (20.58)&  (19.46) &   (17.95)&   (21.37) \\
\addlinespace
Combined &    53.56&     46.96 &   44.54  & 49.39  &  55.56 &    51.57\\
                &  (17.97)&  (20.33)&  (20.36)&  (20.04)&  (17.97)&  (20.99)\\

\bottomrule
\end{tabular}

%     \end{center}
% \end{table}

% \vfill



\subsection{Appendix C : Figures}

\begin{figure}[H]
 \captionsetup{justification=raggedright,singlelinecheck=false}
\caption{Part I Puzzle Pieces}
\includegraphics[scale=0.6]{Figures/S1.png} 
\end{figure}

 

\begin{figure}[H]
 \captionsetup{justification=raggedright,singlelinecheck=false}
\caption{Part I Puzzle Solution}
\includegraphics[scale=0.6]{Figures/Completed1.png} 
\end{figure}

\begin{figure}[H]
 \captionsetup{justification=raggedright,singlelinecheck=false}
\caption{Part II Puzzle Pieces}
\includegraphics[scale=0.6]{Figures/S2.png} 
\end{figure}


\begin{figure}[H]
 \captionsetup{justification=raggedright,singlelinecheck=false}
\caption{Part II Puzzle Solution}
\includegraphics[scale=0.6]{Figures/Completed2.png} 
\end{figure}

\begin{figure}[H]
 \captionsetup{justification=raggedright,singlelinecheck=false}
\caption{Environment of Lab}
\includegraphics[scale=0.5]{Figures/E.png} 
\end{figure}

\begin{figure}[H]
 \captionsetup{justification=raggedright,singlelinecheck=false}
\caption{Incumbent Team Diversity and Newcomer Cooperation }
\includegraphics[scale=0.2]{Figures/Newcomer_types_in_diversity_4_ppg.png} 
\end{figure}



\begin{figure}[H]
 \captionsetup{justification=raggedright,singlelinecheck=false}
\caption{Incumbent Team Diversity and Newcomer Coordination }
\includegraphics[scale=0.2]{Figures/Newcomer_types_in_diversity_4_me.png} 
\end{figure}

\begin{figure}[H]
 \captionsetup{justification=raggedright,singlelinecheck=false}
\caption{Incumbent Team Diversity and Newcomer Cooperation }
\includegraphics[scale=0.2]{Figures/Newcomer_types_in_diversity_ppg.png} 
\end{figure}


\begin{figure}[H]
 \captionsetup{justification=raggedright,singlelinecheck=false}
\caption{Incumbent Team Diversity and Newcomer Coordination }
\includegraphics[scale=0.2]{Figures/Newcomer_types_in_diversity_me.png} 
\end{figure}


\subsection{Appendix D : Instructions}
\subsubsection{General Instructions}
Welcome! Thank you for coming to this experiment on group behavior. Please power off your cell phone. This study has some stages that allow communication and other stages that don’t. So, we ask there be no talking among the participants, unless you are allowed to.  Violations will disqualify you for this study. If you have a question, please raise your hand. An experimenter will come to help you.
This experiment has different parts. You will be given instructions at the beginning of each part. The parts that involve group members’ interactions will be video recorded. The parts with decision making on the computers will not be video recorded. 
At the end of the experiment, you will be paid individually and privately in cash based on an exchange rate: 
100 Experimental Currency Units (ECUs) = \$1. 
No one else but the experimenters will know a participant’s decisions and earnings. So, you are under no obligations to share this information with other participants. 
There are 4 participants in this session. Each of you has received a sticker with your ID (A to D) on the back. Please don’t reveal it to anyone else except the experimenters. Now an experimenter will come. Please show your ID privately to the experimenter and he will lead each of you to your assigned lab. 

\subsubsection{Part I Instructions (Incumbents)}
\subsubsection{\textbf{Stage 1}}
\newline
Part I of this study has two stages. In the first stage, you and two other participants in your group will play a puzzle game. There are 3 envelopes on the table, one for every group member. Each envelope contains FOUR pieces of cardboard. Your task is to use these four pieces to form a Triangle like the one on the table, with one right angle of 90° and two 45° angles. To complete the task, each group member will need to complete their Triangle. 
All the Triangles were cut in the same way, and the pieces were shuffled so that the four pieces in your envelope cannot form a Triangle. You need to find the right piece or pieces from your group members. Group members are encouraged to share ideas and talk to each other with some rules to follow: 
\begin{itemize}
    \item Group members may give pieces to other group members but may not take pieces from other group members. 
    \item You must give the piece or pieces directly to one another instead of throwing the piece or pieces in the middle for others to take. 
    \item When making your Triangle, the pieces cannot overlap each other.
\end{itemize}
Each participant will receive payoffs based on the number of the pieces all the group members correctly place at a rate of 10 ECUs per correctly placed piece. For example, correctly placing all four pieces by each group member will earn everyone 120 ECUs (12*10 ECUs). 
You have 10 minutes to work on this task. When you finish, please raise your hand, the experimenter will check. You will find out about your payoffs at the end of the experiment. Just a gentle reminder. This puzzle game will be video recorded. 
\subsubsection{\textbf{Stage 2}}
Please find the computer that matches your ID. We’re now starting Stage 2 of part I. This stage of decision making on computers will not be video recorded. Please do not talk to each other in this stage. Raise your hand if you have any questions. 
\newline
\textbf{\textit{Instructions on Computer Screen}}
\newline
\textit{Screen 1}
\newline
In Stage 2, you and the 2 other group members will play 2 decision-making games on the computers. You will receive instructions at the beginning of each game. Your payoffs from each game will depend on your decisions and the other two group members’ decisions in that game. You will be paid based on one randomly chosen game. Since each game has the equal probability of being chosen, it is important for you to make decisions in each game as if it is the one that will be chosen to compute earnings. You will not see your payoffs in each game or which game is chosen for earnings until the end of this stage after all the games are played.

Now please click ‘Next’ to go to the first game. 
\newline
\textit{Game 1}
\newline
Game 1 Instructions
\newline
Game 1 consists of 5 rounds. In each round, you and the other 2 group members will receive an initial endowment of 100 ECUs and then decide how much of the initial endowment to keep and how much to allocate towards a group account.

For each ECU allocated to the group account, every group member will earn 0.5 ECUs. That is, every ECU you allocate to the group account will generate 0.5 ECUs for you and 0.5 ECUs for everyone else in your group, therefore this leads to a total of 0.5*3 = 1.5 ECUs for the group per ECU allocated. Similarly, if someone else allocates 1 ECU to the group account, you and everyone else in your group will earn 0.5 ECUs per person.

Therefore, your round payoffs are calculated as follows:
\begin{itemize}
    \item Your Round Payoffs = ECUs you keep for yourself + 0.5*(total amount allocated of all members in your group).
\end{itemize}

For example, if you allocate 50 ECUs to the group account and keep 50 ECUs for yourself. If the total amount allocated to the group account is 150 ECUs. Then, your round payoff will be the ECUs you kept for yourself + 0.5 times the total amount allocated to the group account making 125 ECUs (i.e. 50 ECUs + 75 ECUs).

Now please click ‘Next’, and you will be redirected to a page with questions to check your comprehension of the instructions.
\newline
\textit{Game 2}
\newline
Game 2 Instructions
\newline
Game 2 has 5 rounds too. Consider yourself and the other 2 participants work in a group at a firm. You can think of a round of this game as being a workweek.

In each week, you and your group members can each choose to spend up to 70 hours on an activity at work. The payoff that each group member receives in a round depends on the number of hours she/he chooses to spend on the activity and the number of hours the two other group members choose to spend. The formula below determines your round payoff, which is the minimum hours of all the group members minus 0.75 times your hours plus 85 ECUs. The payoff table is provided for you below, so you do not need to memorize this formula. This payoff table will be available at any point where you need to make a decision. Your round payoffs are calculated as follows:
\begin{itemize}
    \item Your Round Payoff (ECUs) = Minimum Hours of Your Group - 0.75*Your Hours + 85 ECUs (Please note that ECUs are rounded up to the nearest integer)
\end{itemize}


For example, if you spend 70 hours on the activity, and the other two group members spend 70 hours and 60 hours respectively: then your round payoff will be 60 – 0.75*70 +85 ECUs making 93 ECUs (highlighted in blue in the table below). Your round payoff will be determined by tracing row 70 for "your hours" and column 60 for "group minimum hours" in the payoff table. The payoff of the group member who spent 70 will be identical. However, the payoff of the group member who spent 60 will be 60 - 0.75*60 + 85 ECUs making 100 ECUs (highlighted in green in the table below).

Now please click ‘Next’, and you will be redirected to a page with questions to check your comprehension of the instructions.

\subsubsection{Part I Instructions (Newcomer)}
\subsubsection{\textbf{Stage 1}}
\newline
Part I of this study has two stages. In the first stage, you will wait for 10 minutes.
\subsubsection{\textbf{Stage 2}}
Please find the computer that matches your ID. We’re now starting Stage 2. This stage of decision making on computers will not be video recorded. Please raise your hand if you have any questions. 
\newline
\textbf{\textit{Instructions on Computer Screen}}
\newline
\textit{Screen 1}
\newline
In Stage 2, you and two computer robots will play 2 decision-making games on the computer. You will receive instructions at the beginning of each game. Your payoffs from each game will depend on your decisions and the computer randomly generated decisions of the robots. You will be paid based on one randomly chosen game. Since each game has the equal probability of being chosen, it is important for you to make decisions in each game as if it is the one that will be chosen to compute earnings. You will not see your payoffs in each game or which game is chosen for earnings until the end of this stage after all the games are played.

Now please click ‘Next’ to go to the first game.
\newline
\textit{Game 1}
\newline
Game 1 Instructions
\newline
Game 1 consists of 5 rounds. In each round, you and 2 computer robots will receive an initial endowment of 100 ECUs and then decide how much of the initial endowment to keep and how much to allocate towards a group account.

For each ECU allocated to the group account, every group member will earn 0.5 ECUs. That is, every ECU you allocate to the group account will generate 0.5 ECUs for you and 0.5 ECUs each for the two computer robots in your group, therefore this leads to a total of 0.5*3 = 1.5 ECUs for the group per ECU allocated. Similarly, if a computer robot allocates 1 ECU to the group account, you and the other computer robot in your group will earn 0.5 ECUs each. Please note that the decisions of the two robots are randomly generated by the computer. Therefore, your round payoffs are calculated as follows:
\begin{itemize}
    \item Your Round Payoffs = ECUs you keep for yourself + 0.5*(total amount allocated of all members in your group).
\end{itemize}

For example, if you allocate 50 ECUs to the group account and keep 50 ECUs for yourself. If the total amount allocated to the group account is 150 ECUs. Then, your round payoff will be the ECUs you kept for yourself + 0.5 times the total amount allocated to the group account making 125 ECUs (i.e., 50 ECUs + 75 ECUs).

Now please click ‘Next’, and you will be redirected to a page with questions to check your comprehension of the instructions.
\textit{Game 2}
Game 2 Instructions
Game 2 has 5 rounds too. Consider yourself and the other 2 computer robots work in a group at a firm. You can think of a round of this game as being a workweek.

In each week, you and your group members can each choose to spend up to 70 hours on an activity at work. The payoff that you receive in a round depends on the number of hours you choose to spend on the activity and the number of hours chosen by the two computer robots in your group. Please note that the decisions of the two robots are randomly generated by the computer. The formula below determines your round payoff, which is the minimum hours of all the group members minus 0.75 times your hours plus 85 ECUs. The payoff table is provided for you below, so you do not need to memorize this formula. This payoff table will be available at any point where you need to make a decision. Your round payoffs are calculated as follows:
\begin{itemize}
    \item Your Round Payoff (ECUs) = Minimum Hours of Your Group - 0.75*Your Hours + 85 ECUs (Please note that ECUs are rounded up to the nearest integer)
\end{itemize}

For example, if you spend 70 hours on the activity, and the other two computer robots spend 70 hours and 60 hours respectively: then your round payoff will be 60 – 0.75*70 +85 ECUs making 93 ECUs (highlighted in blue in the table below). Your round payoff will be determined by tracing row 70 for "your hours" and column 60 for "group minimum hours" in the payoff table. The payoff of the group member who spent 70 will be identical. However, the payoff of the group member who spent 60 will be 60 - 0.75*60 + 85 ECUs making 100 ECUs (highlighted in green in the table below).

Now please click ‘Next’, and you will be redirected to a page with questions to check your comprehension of the instructions.

\subsubsection{Part II Instructions (Incumbents and Newcomers)}
\newline
\subsubsection{\textbf{Stage 1}}
\newline
We are starting Part II of this study. A new participant has joined the group so there are 4 group members now including the 3 old members and one new member. The old 3-person group played a triangle-puzzle game and the 2 decision-making games among themselves. The new member played the same 2 decision-making games with two computer robots. 
Similar to Part I, Part II has two stages. In the first stage, everyone in the 4-person group will be given new envelopes and play a different puzzle game.   
There are 4 envelopes on the table, one for every group member. Each envelope contains SIX pieces of cardboard. Your task is to use these six pieces to form a Triangle like the one on the table, with one right angle of 90° and two 45° angles. To complete the task, each group member will need to complete their Triangle. 
Similar to the previous triangle puzzle, all the Triangles were cut in the same way, and the pieces were shuffled so that the six pieces in your envelope cannot form a Triangle. You need to find the right piece or pieces from your group members. Group members are encouraged to share ideas and talk to each other with the same rules as previously: 
\begin{itemize}
    \item Group members may give pieces to other group members but may not take pieces from other group members.
    \item You must give the piece or pieces directly to one another instead of throwing the piece or pieces in the middle for others to take. 
    \item When making your Triangle, the pieces cannot overlap each other.
\end{itemize}
Each participant will receive payoffs based on the number of the pieces all the group members correctly place at a rate of 10 ECUs per correctly placed piece. For example, correctly placing all six pieces by each group member will earn everyone 240 ECUs (24*10 ECUs). 
You have 10 minutes to work on this task. When you finish, please raise your hand, the experimenter will check. You will find out about your payoffs at the end of the experiment. Note this puzzle game will be video recorded.
\newline
\subsubsection{\textbf{Stage 2}}
\newline
Please find the computer that matches your ID. We’re now starting Stage 2 of Part II. This stage of decision making on computers will not be video recorded.  Please do not talk to each other in this stage. Raise your hand if you have any questions. 
\newline
\textbf{\textit{Instructions on Computer Screen}}
\newline
\textit{Screen 1}
\newline
In Stage 2, you and the 3 other group members will play 3 decision-making games on the computers. Your payoff from game 3 will be added to the payoff from the randomly chosen game from this part. Otherwise, the rules are the same as before. Since a new member joined, let’s recap these rules. You will receive instructions at the beginning of each game. Your payoffs from each game will depend on your decisions and the other 3 group members’ decisions in that game. You will be paid based on one randomly chosen game from games 1 and 2, and game 3. Since each of the first two games has the equal probability of being chosen, it is important for you to make decisions in each game as if it is the one that will be chosen to compute earnings. You will not see your payoffs in each game or which game is chosen for earnings until the end of this stage after all the games are played.

Now please click ‘Next’ to go to the first game.
\newline
\textit{Game 1}
\newline
Game 1 Instructions
\newline
Game 1 consists of 5 rounds. In each round, you and the other 3 group members will receive an initial endowment of 100 ECUs and then decide how much of the initial endowment to keep and how much to allocate towards a group account.

For each ECU allocated to the group account, every group member will earn 0.438 ECUs. That is, every ECU you allocate to the group account will generate 0.438 ECUs for you and 0.438 ECUs for everyone else in your group, therefore this leads to a total of 0.438*4 = 1.752 ECUs for the group per ECU allocated. Similarly, if someone else allocates 1 ECU to the group account, you and everyone else in your group will earn 0.438 ECUs per person.

Therefore, your round payoffs are calculated as follows:
\begin{itemize}
    \item Your Round Payoffs = ECUs you keep for yourself + 0.438*(total amount allocated of all members in your group).
\end{itemize}

For example, if you allocate 50 ECUs to the group account and keep 50 ECUs for yourself. If the total amount allocated to the group account is 200 ECUs. Then, your round payoff will be the ECUs you kept for yourself + 0.438 times the total amount allocated to the group account making 137.6 ECUs (i.e. 50 ECUs + 87.6 ECUs).

Now please click ‘Next’, and you will be redirected to a page with questions to check your comprehension of the instructions.
\newline
\textit{Game 2}
\newline
Game 2 Instructions
\newline
Game 2 has 5 rounds too. Consider yourself and the other 3 participants work in a group at a firm. You can think of a round of this game as being a workweek.

In each week, you and your group members can each choose to spend up to 70 hours on an activity at work. The payoff that each group member receives in a round depends on the number of hours she/he chooses to spend on the activity and the number of hours the 3 other group members choose to spend. The formula below determines your round payoff, which is the minimum hours of all the group members minus 0.75 times your hours plus 85 ECUs. The payoff table is provided for you below, so you do not need to memorize this formula. This payoff table will be available at any point where you need to make a decision. Your round payoffs are calculated as follows:
\begin{itemize}
    \item Your Round Payoff (ECUs) = Minimum Hours of Your Group - 0.75*Your Hours + 85 ECUs (Please note that ECUs are rounded up to the nearest integer)
\end{itemize}


For example, if you spend 70 hours on the activity, and the other 3 group members spend 70 hours, 10 hours and 60 hours respectively: then your round payoff will be 10 – 0.75*70 +85 ECUs making 43 ECUs (highlighted in blue in the table below). Your round payoff will be determined by tracing row 70 for "your hours" and column 10 for "group minimum hours" in the payoff table. The payoff of the group member who spent 70 will be identical. However, the payoff of the group member who spent 60 will be 10 - 0.75*60 + 85 ECUs making 50 ECUs (highlighted in green in the table below). Finally, the payoff of the group member who spent 10 will be 10 - 0.75*10 + 85 ECUs making 88 ECUs (highlighted in pink in the table below).

Now please click ‘Next’, and you will be redirected to a page with questions to check your comprehension of the instructions.

 
% \end{document}